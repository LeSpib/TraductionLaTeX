% \iffalse meta-comment
%
% Copyright 1993-2014
%
% The LaTeX3 Project and any individual authors listed elsewhere
% in this file.
%
% This file is part of the Standard LaTeX `Tools Bundle'.
% -------------------------------------------------------
%
% It may be distributed and/or modified under the
% conditions of the LaTeX Project Public License, either version 1.3c
% of this license or (at your option) any later version.
% The latest version of this license is in
%    http://www.latex-project.org/lppl.txt
% and version 1.3c or later is part of all distributions of LaTeX
% version 2005/12/01 or later.
%
% The list of all files belonging to the LaTeX `Tools Bundle' is
% given in the file `manifest.txt'.
%
% \fi
% \iffalse
%% File: xspace.dtx Copyright (C) 1991-1997 David Carlisle
%% File: xspace.dtx Copyright (C) 2004-2006 David Carlisle,
%%                                          Morten H\o gholm
%
%<*dtx>
          \ProvidesFile{xspace.dtx}
%</dtx>
%<package>\NeedsTeXFormat{LaTeX2e}
%<package>\ProvidesPackage{xspace}
%<driver>\ProvidesFile{xspace.drv}
% \fi
%         \ProvidesFile{xspace.dtx}
          [28/10/2014 v1.13 Space after command names (DPC,MH)]
%
% \iffalse
%<*driver>
\documentclass{ltxdoc}
\usepackage[T1]{fontenc}
\usepackage[utf8]{inputenc}
\usepackage[french]{babel}
\makeatletter
\providecommand*\eTeX{{%
  \if b\expandafter\@car\f@series\@nil\boldmath\fi$\m@th
  \varepsilon$-\TeX}}
\def\MaintainedByLaTeXTeam#1{%
{\gdef\@maintainedby{%
Ce fichier est maintenu par l'équipe du \og \LaTeX{} Project \fg{}.\\%
Les rapports d'anomalie peuvent être envoyés en anglais à \\%
\url{http://latex-project.org/bugs.html} (catégorie \texttt{#1}).}}}
\makeatother
% Bloc pour traduction
\usepackage{xcolor}
\usepackage{pifont}
\definecolor{orange5}{RGB}{255,153,0} 
\newcommand{\trad}[1]{\textbf{\textcolor{orange5}{\noindent\ding{54} #1 \ding{54}}}}
\newcommand{\tradini}{\color{orange5}\ding{54}}
\newcommand{\tradfin}{\ding{54}\color{black}}
% Fin du bloc qui doit être retiré une fois le travail achevé
\usepackage{xspace}[2006/05/08]
\begin{document}
\DocInput{xspace-fr.dtx}
\end{document}
%</driver>
% \fi
%
% \GetFileInfo{xspace.dtx}
% \title{L'extension \textsf{xspace}\thanks{Ce fichier a pour numéro de 
%        version \fileversion\ et a été revu le \filedate. La première 
%        traduction, basée sur la version 1.06 a été publiée par 
%        Jean-Drucbert en 2001.}}
% \author{David Carlisle \and Morten H\o gholm}
% \date{\filedate}
% \MaintainedByLaTeXTeam{tools}
% \maketitle
%
% %%%%%%%%%%%%%%%%%%%%%%%%%%%%%%%%%%%%%%%%%%%%%%%%%%%%%%%%%%%%%%%%%%%%
%
%
% \changes{v1.00}{1991/08/30}{Initial version}
% \changes{v1.01}{1992/06/26}{Re-issue for the new doc and docstrip}
% \changes{v1.02}{1994/01/31}{Re-issue for LaTeX2e (no change to
%   code)}
% \changes{v1.07}{2004/12/07}{Make extensible. tools/3712}
% \changes{v1.07}{2004/12/07}{Fix active characters. tools/3747}
% \changes{v1.07}{2004/12/07}{Update documentation}
% \changes{v1.08}{2005/05/07}{Better fix for active characters}
% \changes{v1.09}{2005/07/26}{Improve test by exiting if
%   \cs{@let@token} is a letter}
% \changes{v1.10}{2005/10/04}{Use higher level functions for
% conditional processing}
% \changes{v1.10}{2005/10/04}{Improve expansion method}
%

% \begin{abstract}
% La commande |\xspace| s'utilise à la fin de commandes pensées pour être
% intégrée principalement dans du texte. Elle ajoute une espace à moins que 
% la commande ne soit suivie par certains signes de ponctuation. 
% \end{abstract}
%
% \section{Introduction}
% \newcommand{\gb}{La Grande Bretagne\xspace}\DescribeMacro{\xspace}
% Après avoir défini |\newcommand{\gb}{Great Britain\xspace}|, la commande 
% |\gb| détermine si une espace doit être insérée après elle ou pas. Ainsi, 
% la saisie
% \begin{quote}
% |\gb est un très bel endroit pour vivre.\\| \\
% |\gb, une petite île au large des côtes françaises.\\|\\
% |\gb\footnote{La petite île au large des côtes françaises.}|\\
% |est un très bel endroit pour vivre.|
% \end{quote}
% conduit au résultat suivant 
% \begin{quote}
% \gb est un très bel endroit pour vivre. \\
% \gb, une petite île au large des côtes françaises.\\
% \gb\footnote{La petite île au large des côtes françaises.}
% est un très bel endroit pour vivre.
% \end{quote}
% |\xspace| évite à l'utilisateur de saisir \verb*+\ + ou |{}| après la plupart
% des occurrences d'une commande dans du texte. Cependant, si l'une de ces
% constructions suit |\xspace|, ce dernier n'ajoute pas d'espace. Ceci implique
% qu'il est d'ajouter |\xspace| à la fin de commandes déjà présentes sans faire
% trop de changements dans votre document. En particulier, |\xspace| insèrera
% toujours une espace si l'élément qui le suit est une lettre normale, ce qui
% est cas classique.
%
% Parfois, |\xspace| peut prendre une mauvaise décision et ajouter alors une
% espace non souhaitée. Il y a différentes raisons à ce comportement mais ce 
% problème peut toujours être géré en faisant suivre la commande d'un |{}|, 
% dans la mesure où ceci a pour effet de supprimer l'espace.
%
%
% \subsection{Ajout de nouvelles exceptions}
%
% Une des raisons les plus courantes pour |\xspace| d'insérer une espace non
% souhaitée se produit quand il est suivi d'une commande qui n'est pas dans sa
% liste des exceptions. 
% \DescribeMacro{\xspaceaddexceptions}%
% Avec |\xspaceaddexceptions|, vous pouvez ajouter de nouveaux caractères ou 
% commandes pour qu'ils soient reconnus par la mécanique d'analyse de 
% |\xspace|. Les utilisateurs d'extensions avancées sur les notes en bas de
% page comme \textsf{manyfoot} vont souvent définir de nouvelles commandes de
% notes de bas de page qui ne devraient pas impliquer l'ajout d'une espace à la
% suite d'une commande \og améliorée \fg{} par |\xspace|. Si vous définissez 
% les commandes de note de page |\footnoteA| et |\footnoteB|, ajoutez la ligne
% suivante à votre préambule.
% \begin{quote}
% |\xspaceaddexceptions{\footnoteA \footnoteB}|
% \end{quote}
%
%
% \subsection{Support des caractères actifs}
%
%
% L'autre exemple courant où |\xspace| ne traite pas bien la situation se
% produit en présence de caractères actifs. Généralement, cette extension doit
% être chargée \emph{après} toute extension linguistique (ou autre) qui rend la
% ponctuation \og active \fg{}. Ceci complexifie la tâche pour \textsf{xspace} 
% lors de travaux avec l'extension populaire \textsf{babel} tout 
% particulièrement parce que les signes de ponctuation peuvent basculer du 
% statut \og actif \fg{} à \og autre \fg{} et inversement. À partir de la 
% version~1.08 de \textsf{xspace}, deux manières de gérer ce point sont 
% disponibles en fonction du moteur utilisé par votre format \LaTeX : 

% \begin{description}
%   \item[\TeX] Les signes de ponctuation sont ajoutés à la liste d'exceptions 
%   à la fois dans leur version normale et active, assurant ainsi qu'ils seront
%   toujours reconnus.
%   \item[\eTeX] Les caractères sont lus à nouveau lors du passage dans la
%   liste des exceptions, ce qui signifie que la comparaison interne se fait
%   sur l'état courant du caractère. Ceci fonctionne quelque soit l'astuce de
%   code de catégorie utilisée.
% \end{description}
%
% Au moment de la rédaction de ce document, toutes les grandes distributions
% \TeX\ utilisent \eTeX\ comme moteur pour \LaTeX, ce qui fait que tout devrait
% bien se passer. S'il se trouve que vous utilisez le \TeX\ standard et que
% |\xspace| semble prendre la mauvaise décision, alors vous pouvez soit 
% utiliser |{}| comme décrit ci-dessus pour le corriger, soit ajouter le
% caractère à la liste mais avec le code de catégorie souhaité. Voir 
% l'implémentation pour un exemple de ce qu'il faut alors faire.
%
% \subsection{Toujours pas satisfait ?}
%
% Certaines personnes n'aiment pas la liste des exceptions, aussi peuvent-ils
% retrancher un élément à la fois de cette liste avec une commande dédiée 
% \DescribeMacro{\xspaceremoveexception}^^A
% \cs{xspaceremoveexception}\marg{unité-lexicale}.  De plus, la commande 
% \DescribeMacro{\@xspace@hook}\cs{@xspace@hook} peut être définie pour 
% analyser plus avant la suite du texte dans le cas où vous souhaiteriez
% vérifier plus d'unités lexicales. Elle est appelée une fois que \cs{xspace}
% a déterminé s'il faut insérer une espace ou si une exception a été trouvée 
% (la définition par défaut de \cs{@xspace@hook} est vide). Par conséquent, 
% vous pouvez utiliser \cs{unskip} pour retirer l'espace insérée si 
% \cs{@let@token} rencontre quelque chose de spécial. L'exemple ci-dessous 
% montre comment garantir qu'un tiret demi-quadratin\footnote{Ce tiret est codé
% par |-{}-| en \LaTeX.} obtienne une espace tandis qu'un trait d'union non.
% \begin{verbatim}
% \xspaceremoveexception{-}
% \makeatletter
% \renewcommand*\@xspace@hook{%
%   \ifx\@let@token-%
%     \expandafter\@xspace@dash@i
%   \fi
% }
% \def\@xspace@dash@i-{\futurelet\@let@token\@xspace@dash@ii}
% \def\@xspace@dash@ii{%
%   \ifx\@let@token-%
%   \else
%     \unskip
%   \fi
%   -%
% }
% \makeatother
% \end{verbatim}
%
%
% \StopEventually{}
%
% \section{Les commandes}
%
% |\xspace| jette un coup d'\oe il à l'unité lexicale suivante. Si elle 
% appartient à notre liste d'exception, |\xspace| sort de la boucle d'analyse
% et ne fait rien; sinon il essaye de développer l'unité lexicale et 
% recommence son analyse. Si ceci conduit à une unité lexicale non développable
% sans qu'une exception ait été trouvée, une espace est insérée.
%
%    \begin{macrocode}
%<*package>
%    \end{macrocode}
%
% \begin{macro}{\xspace}
% |\xspace| regarde juste un cran en avant et appelle alors |\@xspace|.
% \changes{v1.03}{1994/11/15}{Make robust}
%    \begin{macrocode}
\DeclareRobustCommand\xspace{\@xspace@firsttrue
  \futurelet\@let@token\@xspace}
%    \end{macrocode}
% \end{macro}
%
% \begin{macro}{\if@xspace@first}
% \changes{v1.11}{2006/02/12}{Added macro}
% \begin{macro}{\@xspace@simple}
% \changes{v1.11}{2006/02/12}{Added macro}
% Quelques aides pour éviter des appels multiples de |\@xspace@eTeX@setup|.
%    \begin{macrocode}
\newif\if@xspace@first
\def\@xspace@simple{\futurelet\@let@token\@xspace}
%    \end{macrocode}
% \end{macro}
% \end{macro}
%
% \begin{macro}{\@xspace@exceptions@tlp}
% \changes{v1.07}{2004/12/07}{Added macro}
% La liste d'exception. Si la mécanique d'analyse trouve une de ces exceptions,
% il n'y a pas insertion d'une espace après la commande. Le \texttt{tlp} dans 
% le nom signifie \og token list pointer \fg{} (pointeur de liste d'unité
% lexicale).
%    \begin{macrocode}
\def\@xspace@exceptions@tlp{%
  ,.'/?;:!~-)\ \/\bgroup\egroup\@sptoken\space\@xobeysp
  \footnote\footnotemark
  \xspace@check@icr
}
%    \end{macrocode}
% Et ici nous avons la définition non vide de \cs{check@icr}. 
% \changes{v1.13}{2009/10/20}{fix for "tools/3895": `text font commands fool xspace'}
%    \begin{macrocode}
\begingroup
  \text@command\relax
  \global\let\xspace@check@icr\check@icr
\endgroup
%    \end{macrocode}
% \end{macro}
%
% \tradini
% \begin{macro}{\xspaceaddexceptions}
% \changes{v1.07}{2004/12/07}{Added macro}
% The user command, which just adds tokens to the list.
%    \begin{macrocode}
\newcommand*\xspaceaddexceptions{%
  \g@addto@macro\@xspace@exceptions@tlp
}
%    \end{macrocode}
% \end{macro}
% \begin{macro}{\xspaceremoveexception}
% \changes{v1.10}{2005/10/04}{Use higher level functions for
% conditional processing}
% This command removes an exception globally.
%    \begin{macrocode}
\newcommand*\xspaceremoveexception[1]{%
%    \end{macrocode}
% First check that it is in the list at all.
%    \begin{macrocode}
  \def\reserved@a##1#1##2##3\@@{%
    \@xspace@if@q@nil@NF##2{%
%    \end{macrocode}
% It's in the list, remove it.
%    \begin{macrocode}
      \def\reserved@a####1#1####2\@@{%
        \gdef\@xspace@exceptions@tlp{####1####2}}%
      \expandafter\reserved@a\@xspace@exceptions@tlp\@@
    }%
  }%
  \expandafter\reserved@a\@xspace@exceptions@tlp#1\@xspace@q@nil\@@
}
%    \end{macrocode}
% \end{macro}
%
% \begin{macro}{\@xspace@break@loop}
% \changes{v1.08}{2005/04/28}{Added macro}
% \changes{v1.10}{2005/10/04}{Use quark instead}
% To stop the loop.
%    \begin{macrocode}
\def\@xspace@break@loop#1\@nil{}
%    \end{macrocode}
% \end{macro}
%
% \begin{macro}{\@xspace@hook}
% \changes{v1.09}{2005/07/26}{Added macro}
% A hook for users with special needs.
%    \begin{macrocode}
\providecommand*\@xspace@hook{}
%    \end{macrocode}
% \end{macro}
%
% Now we check if we're running \eTeX. We can't use \cs{@ifundefined}
% as that will lock catcodes and we need to change some of those.
% As there is a small risk that someone already set \cs{eTeXversion}
% to \cs{relax} by accident we make sure we check for that case but
% without setting it to \cs{relax} if it wasn't already.
%    \begin{macrocode}
\begingroup\expandafter\expandafter\expandafter\endgroup
  \expandafter\ifx\csname eTeXversion\endcsname\relax
%    \end{macrocode}
% If we are running normal \TeX\ we add the most common cases of active
% punctuation characters. First we make them active.
%    \begin{macrocode}
  \begingroup
    \catcode`\;=\active  \catcode`\:=\active
    \catcode`\?=\active  \catcode`\!=\active
%    \end{macrocode}
%  The \texttt{alltt} environment also makes |,|, |'|, and |-| active
%  so we add them as well.
%    \begin{macrocode}
    \catcode`\,=\active  \catcode`\'=\active  \catcode`\-=\active
    \xspaceaddexceptions{;:?!,'-}
  \endgroup
  \let\@xspace@eTeX@setup\relax
%    \end{macrocode}
% \begin{macro}{\@xspace@eTeX@setup}
% \changes{v1.10}{2005/10/04}{Added macro}
% \changes{v1.12}{2006/05/08}{Bug fix for verbatim in output routine}
% When we're running \eTeX, we have the advantage of \cs{scantokens}
% which will rescan tokens with current catcodes. This little
% expansion trick makes sure that the exception list is redefined to
% itself but with the contents of it exposed to the current catcode
% regime. That is why we must make sure the catcode of space is 10,
% since we have a \verb*|\ | inside the list.
%    \begin{macrocode}
\else
  \def\@xspace@eTeX@setup{%
    \begingroup
      \everyeof{}%
      \endlinechar=-1\relax
      \catcode`\ =10\relax
      \makeatletter
%    \end{macrocode}
% We may also be so unfortunate that the re-reading of the list takes
% place when the catcodes of |\|, |{| and |}| are ``other,'' e.g., if
% it takes place in a header and the output routine was called in the
% middle of a \texttt{verbatim} environment.
%    \begin{macrocode}
      \catcode`\\\z@
      \catcode`\{\@ne
      \catcode`\}\tw@
      \scantokens\expandafter{\expandafter\gdef
        \expandafter\@xspace@exceptions@tlp
        \expandafter{\@xspace@exceptions@tlp}}%
    \endgroup
  }
\fi
%    \end{macrocode}
% \end{macro}
% \begin{macro}{\@xspace}
% \changes{v1.03}{1994/11/15}{Add exclamation mark}
% \changes{v1.04}{1996/05/17}{Add slash}
% \changes{v1.05}{1996/12/06}{Add space for alltt etc. tools/2322}
% \changes{v1.06}{1997/10/13}{Add normal space. tools/2632}
% \changes{v1.07}{2004/12/07}{Now runs through a list of exceptions}
% \changes{v1.07}{2004/12/07}{Added \cs{footnote} and
% \cs{footnotemark}}
% \changes{v1.08}{2005/05/07}{Use recursive loop instead of \cs{@tfor}}
% \changes{v1.09}{2005/07/26}{Only check non-letters and add hook}
% \changes{v1.10}{2005/10/04}{Use higher level functions for
% conditional processing}
% \changes{v1.10}{2005/10/04}{Improve expansion method}
% If the next token is one of a specified list of characters,
% do nothing, otherwise add a space. With version~1.07 the
% approach was altered dramatically to run through the
% exception list |\@xspace@exceptions@tlp| and check each
% token one at a time.
%    \begin{macrocode}
\def\@xspace{%
%    \end{macrocode}
% Before we start checking the exception list it makes sense to
% perform a quick check on the token in question. Most of the time
% \cs{xspace} is used in regular text so \cs{@let@token} is set equal
% to a letter. In that case there is no point in checking the list
% because it will definitely not contain any tokens with catcode~11.
%
% You may wonder why there are special functions here instead of
% simpler \cs{ifx} conditionals. The reason is that a)~this way we
% don't have to add many, many \cs{expandafter}s to get the nesting
% right and b)~we don't get into trouble when \cs{@let@token} has been
% let equal to \cs{if} etc.
%    \begin{macrocode}
  \@xspace@lettoken@if@letter@TF \space{%
%    \end{macrocode}
% Otherwise we start testing after setting up a few things. If running
% \eTeX{} we rescan the catcodes but only the first time around.
%    \begin{macrocode}
    \if@xspace@first
      \@xspace@firstfalse
      \let\@xspace@maybespace\space
      \@xspace@eTeX@setup
    \fi
    \expandafter\@xspace@check@token
      \@xspace@exceptions@tlp\@xspace@q@nil\@nil
%    \end{macrocode}
% If an exception was found \cs{@xspace@maybespace} is let to
% \cs{relax} and we do nothing.
%    \begin{macrocode}
    \@xspace@token@if@equal@NNT \space \@xspace@maybespace
%    \end{macrocode}
% Otherwise we check to see if we found something expandable and try
% again with that token one level expanded. If no expandable token is
% found we insert a space and then execute the hook.
%    \begin{macrocode}
    {%
      \@xspace@lettoken@if@expandable@TF
      {\expandafter\@xspace@simple}%
      {\@xspace@maybespace\@xspace@hook}%
    }%
  }%
}
%    \end{macrocode}
% \end{macro}
%
% \begin{macro}{\@xspace@check@token}
% \changes{v1.07}{2004/12/07}{Added macro}
% \changes{v1.08}{2005/05/07}{Made function recursive}
% \changes{v1.10}{2005/10/04}{Use higher level functions for
% conditional processing}
% \changes{v1.11}{2006/02/12}{Modified so the \cs{@let@token} can be
%   \cs{outer}}
% This macro just checks the current item in the exception list
% against the \cs{@let@token}. If they are equal we make sure that no
% space is inserted and break the loop.
%    \begin{macrocode}
\def\@xspace@check@token #1{%
  \ifx\@xspace@q@nil#1%
    \expandafter\@xspace@break@loop
  \fi
  \expandafter\ifx\csname @let@token\endcsname#1%
    \let\@xspace@maybespace\relax
    \expandafter\@xspace@break@loop
  \fi
  \@xspace@check@token
}
%    \end{macrocode}
% \end{macro}
%
% That's all, folks! That is, if we were running \LaTeX3. In that case
% we would have had nice functions for all the conditionals but here
% we must define them ourselves. We also optimize them here as
% \cs{@let@token} will always be the argument in some cases.
%
% \begin{macro}{\@xspace@if@lettoken@letter@TF}
% \begin{macro}{\@xspace@if@lettoken@expandable@TF}
% \begin{macro}{\@xspace@cs@if@equal@NNF}
% First a few comparisons.
%    \begin{macrocode}
\def\@xspace@lettoken@if@letter@TF{%
  \ifcat\noexpand\@let@token @% letter
    \expandafter\@firstoftwo
  \else
    \expandafter\@secondoftwo
  \fi}
\def\@xspace@lettoken@if@expandable@TF{%
  \expandafter\ifx\noexpand\@let@token\@let@token%
    \expandafter\@secondoftwo
  \else
    \expandafter\@firstoftwo
  \fi
}
\def\@xspace@token@if@equal@NNT#1#2{%
  \ifx#1#2%
    \expandafter\@firstofone
  \else
    \expandafter\@gobble
  \fi}
%    \end{macrocode}
% \end{macro}
% \end{macro}
% \end{macro}
% \begin{macro}{\@xspace@q@nil}
% \begin{macro}{\@xspace@if@q@nil@NF}
% Some macros dealing with quarks.
%    \begin{macrocode}
\def\@xspace@q@nil{\@xspace@q@nil}
\def\@xspace@if@q@nil@NF#1{%
  \ifx\@xspace@q@nil#1%
    \expandafter\@gobble
  \else
    \expandafter\@firstofone
  \fi}
%    \end{macrocode}
% \end{macro}
% \end{macro}
%    \begin{macrocode}
%</package>
%    \end{macrocode}
%
% \Finale
