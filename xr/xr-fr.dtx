% \iffalse meta-comment
%
% Copyright (c) 1993-2016
% Le LaTeX3 Project et tout auteur listé dans ce fichier.
%
% Ce fichier est la traduction en français du fichier indentfirst.dtx
% -------------------------------------------------------------------
%
% Ce fichier appartient à la "boîte à outils" du LaTeX standard.
% 
% Il peut être distribué et/ou modifié sous les conditions définies par la
% LaTeX Project Public License, soit en version 1.3c, soit (à votre choix) en
% toute version ultérieure.
% La dernière version de cette license est à l'adresse :
%   http://www.latex-project.org/lppl.txt
% et la version 1.3c et les suivantes se retrouve dans toutes les distributions
% de LaTeX dès la version 2005/12/01.
%
% La liste de tous les fichiers appartenant à la "Boîte à outils" est donnée
% dans le fichier "manifest.txt".
%
% \fi
% \iffalse
%% File: xr.dtx Copyright (C) 1994-1994 David Carlisle
%
%<package>\NeedsTeXFormat{LaTeX2e}
%<package>\ProvidesPackage{xr}
%<package>         [28/05/1994 v5.02 Références externes (DPC)]
%
%<*driver>
\documentclass{ltxdoc}
\usepackage[utf8]{inputenc}
\usepackage[T1]{fontenc}
\usepackage[ltxdoc,babel]{translatex-fr}
\usepackage{xr}
\GetFileInfo{xr.sty}
\begin{document}
\title{L'extension \texttt{xr}\thanks{Ce fichier a pour numéro de
        version \fileversion\ et a été mis à jour le \filedatefr. La
        première traduction, basée sur la version 5.02, a été publiée par 
        Jean-Pierre Drucbert en 2000.}}
\author{David Carlisle\thanks{%
  L'auteur des versions 1 à 4 est Jean-Pierre Drucbert.}}
\date{\filedatefr}
\MaintainedByLaTeXTeam{tools}
\maketitle
\DocInput{xr-fr.dtx}
\end{document}
%</driver>
% \fi
%
% %%%%%%%%%%%%%%%%%%%%%%%%%%%%%%%%%%%%%%%%%%%%%%%%%%%%%%%%%%%%%%%%%%%%
%
%
% \changes{v5.00}{07/07/1993}
%         {Première version par DPC (avec l'accord de J-PD).  Nouveau mécanisme
%         (\cmd{\read} au lieu de \cmd\input).}
%
% \changes{v5.01}{20/07/1993}{Correction d'erreur par DPC : la v5.00 
%         n'importait par les fichiers .aux des fichiers inclus avec
%         \cmd\include (signalé par J-PD).}
%
% \changes{v5.02}{28/05/1994}{Mise à jour pour LaTeX2e}
%
%
% Cette extension permet de gérer des références externes\footnote{NDT : soit 
% en anglais des \emph{eXternal References}, ce qui donne son nom à 
% l'extension.}. 
%
% Pour qu'un document se réfère à certaines sections d'un autre, par exemple
% |aaa.tex|, il convient de charger cette extension dans le fichier principal 
% et placer la commande |\externaldocument{aaa}| dans le préambule.
%
% Vous pouvez alors utiliser |\ref| et |\pageref| pour vous référer à tout ce 
% qui a été défini par |\label| dans chaque |aaa.tex| ou dans le document 
% principal. Il est possible de déclarer autant de documents externes que 
% souhaité.
%
% Si l'un des documents externes ou le document principal utilise le même 
% |\label|, une erreur survient car l'étiquette est définie plusieurs fois.
% Afin de contourner ce problème, la commande |\externaldocument| a un 
% argument optionnel. Si vous déclarez |\externaldocument[A-]{aaa}| alors 
% alors toutes les références dans le document |aaa| sont préfixées par |A-|.
% Par exemple, si une section de |aaa| contient |\label{intro}|, alors cette 
% étiquette pourra être appelée par |\ref{A-intro}|. Le préfixe peut être 
% différent de |A-| et être en fait toute chaîne de caractères telle que toutes
% les étiquettes importées depuis des fichiers externes soient uniques. Notez
% cependant que si vos extensions déclarent certains caractères comme actifs 
% (|:| en français, |"| en allemand), alors ils ne peuvent pas en général être
% utilisés dans l'argument de |\label| ni dans l'argument optionnel de 
% |\externaldocument|.
%
% \StopEventually{}
%
% \section{Les commandes}
%
%    \begin{macrocode}
%<*package>
%    \end{macrocode}
%
% Vérification de la présence d'un argument optionnel.
%    \begin{macrocode}
\def\externaldocument{\@ifnextchar[\XR@{\XR@[]}}
%    \end{macrocode}
%
% Sauvegarde du préfixe optionnel et début du traitement du premier fichier
% |aux|.
%    \begin{macrocode}
\def\XR@[#1]#2{{%
  \makeatletter
  \def\XR@prefix{#1}%
  \XR@next#2.aux\relax\\}}
%    \end{macrocode}
%
% Traitement du fichier |aux| suivant dans la liste et retrait de celui-ci
% de la tête de la liste des fichiers à traiter. 
%    \begin{macrocode}
\def\XR@next#1\relax#2\\{%
  \edef\XR@list{#2}%
  \XR@loop{#1}}
%    \end{macrocode}
%
% Vérification si la liste des fichiers |aux| est vide.
%    \begin{macrocode}
\def\XR@aux{%
  \ifx\XR@list\@empty\else\expandafter\XR@explist\fi}
%    \end{macrocode}
%
% Développement de la liste des fichiers |aux| et appel de |\XR@next| pour
% traiter le premier fichier.
%    \begin{macrocode}
\def\XR@explist{\expandafter\XR@next\XR@list\\}
%    \end{macrocode}
%
% Si le fichier |aux| existe, repérage ligne après ligne des |\newlabel| et 
% |\@input|. Sinon, passage au fichier suivant dans la liste.
%    \begin{macrocode}
\def\XR@loop#1{\openin\@inputcheck#1\relax
  \ifeof\@inputcheck
    \PackageWarning{xr}{^^JNo file #1^^JLABELS NOT IMPORTED.^^J}%
    \expandafter\XR@aux
  \else
    \PackageInfo{xr}{IMPORTING LABELS FROM #1}%
    \expandafter\XR@read\fi}
%    \end{macrocode}
%
% Lecture de la ligne suivante du fichier |aux|
%    \begin{macrocode}
\def\XR@read{%
  \read\@inputcheck to\XR@line
%    \end{macrocode}
% Le |...| assure que |\XR@test| a toujours suffisamment d'arguments.
%    \begin{macrocode}
  \expandafter\XR@test\XR@line...\XR@}
%    \end{macrocode}
%
% Observation de la première unité lexicale de la ligne. Si c'est un 
%|\newlabel|, exécution du |\newlabel|. Si c'est un |\@input|, ajout du
% nom du fichier dans la liste des fichiers à traiter. Sinon, pas 
% de traitement.
% Parcours en boucle tant que la fin du fichier n'est pas atteinte. 
% Ensuite passage au fichier suivant dans la liste.
%    \begin{macrocode}
\long\def\XR@test#1#2#3#4\XR@{%
  \ifx#1\newlabel
    \newlabel{\XR@prefix#2}{#3}%
  \else\ifx#1\@input
     \edef\XR@list{\XR@list#2\relax}%
  \fi\fi
  \ifeof\@inputcheck\expandafter\XR@aux
  \else\expandafter\XR@read\fi}
%    \end{macrocode}
%
%    \begin{macrocode}
%</package>
%    \end{macrocode}
%
% \Finale
%

