% \iffalse
%% File: xspace.dtx Copyright (C) 1991-1996 David Carlisle
%
%<*dtx>
          \ProvidesFile{f-xspace.dtx}
%</dtx>
%<package>\NeedsTeXFormat{LaTeX2e}
%<package>\ProvidesPackage{xspace}
%<driver>\ProvidesFile{xspace.drv}
% \fi
%         \ProvidesFile{f-xspace.dtx}
          [1997/10/13 v1.06 Space after command names (DPC)]
%
% \iffalse
%<*driver>
\def\traductionchanges{oui}
\documentclass{ltxdoc}

\usepackage[french]{babel}
\usepackage[latin1]{inputenc}

\usepackage{xspace}
\begin{document}
\DocInput{f-xspace.dtx}
\end{document}
%</driver>
% \fi
%
% \GetFileInfo{f-xspace.dtx}
% \title{Le package \textsf{xspace}\thanks{Cette version
% a le num�ro \fileversion. La derni�re r�vision date du \filedate.}}
% \author{David Carlisle \\
%	(traduction fran�aise: Jean-Pierre Drucbert\\
%	Jean-Pierre.Drucbert@onecert.fr)}
% \date{\filedate}
% \maketitle
%
% %%%%%%%%%%%%%%%%%%%%%%%%%%%%%%%%%%%%%%%%%%%%%%%%%%%%%%%%%%%%%%%%%%%%
%
% \CheckSum{87}
%
% \changes{v1.00}{1991/08/30}{Premi�re version}
% \changes{v1.01}{1992/06/26}{Adaptation pour new doc et docstrip}
% \changes{v1.02}{1994/01/31}{Adaptation pour LaTeX2e (aucun changement 
% dans le code)}
%
% \begin{abstract}
% |\xspace| se place � la fin d'une macro utilis\'ee g�n�ralement pour
% du texte. Elle ajoute une espace \`a moins que la macro ne soit 
% suivie d'un caract\`ere de ponctuation particulier.
% \end{abstract}
%
% \section{Introduction}
% Quand on tape \verb|\newcommand{\gb}{Great Britain\xspace}|, on 
% obtient:\\
% \newcommand{\gb}{Great Britain\xspace}
% |\gb is a very nice place to live.|\\
% \gb is a very nice place to live.\\
% |\gb, a small island off the coast of France.|\\
% \gb, a small island off the coast of France.
%
% \verb|\xspace| \'evite \`a l'utilisateur d'avoir \`a taper
% \verb*+\ + ou \verb|{}| apr\`es la plupart des occurrences d'un nom de
% macro dans le texte. Cependant % si l'une de ces constructions suit
% \verb|\xspace|, celui-ci n'ajoute pas une espace. Il est ainsi facile
% d'ajouter \verb|\xspace| \`a la fin d'une macro existante sans devoir
% faire trop de modifications dans le document.
%
% |\xspace| peut prendre quelquefois une mauvaise d�cision et ajouter
% une espace alors qu'elle n'est pas n�cessaire. Dans ce cas, faire 
% suivre la macro d'un |{}| aura pour effet de supprimer l'espace.
%
% Notez que ce package doit \^etre charg� \emph{apr�s} tout package de
% langue (ou autre) qui rend 'actifs' les caract�res de ponctuation.
% \StopEventually{}
%
% \section{Les macros}
%
%    \begin{macrocode}
%<*package>
%    \end{macrocode}
%
% \begin{macro}{\xspace}
% |\xspace| effectue juste une v�rification et appelle |\@xspace|.
% \changes{v1.03}{1994/11/15}{Make robust}
%    \begin{macrocode}
\DeclareRobustCommand\xspace{\futurelet\@let@token\@xspace}
%    \end{macrocode}
% \end{macro}
% \begin{macro}{\@xspace}
% Si le caract�re qui suit est un de ceux que l'on trouve dans une liste
% sp�ciale, ne fait rien, sinon ajoute une espace. Si vous utilisez 
% souvent un caract�re de ponctuation particulier, ajoutez une ligne
% appropri�e (ne pas oublier d'ajouter un |\fi| � la fin!).
% \changes{v1.03}{1994/11/15}{Add exclamation mark}
% \changes{v1.04}{1996/05/17}{Add slash}
% \changes{v1.05}{1996/12/06}{Add space for alltt etc. tools/2322}
% \changes{v1.06}{1997/10/13}{Add normal space. tools/2632}
%    \begin{macrocode}
\def\@xspace{%
  \ifx\@let@token\bgroup\else
  \ifx\@let@token\egroup\else
  \ifx\@let@token\/\else
  \ifx\@let@token\ \else
  \ifx\@let@token~\else
  \ifx\@let@token.\else
  \ifx\@let@token!\else
  \ifx\@let@token,\else
  \ifx\@let@token:\else
  \ifx\@let@token;\else
  \ifx\@let@token?\else
  \ifx\@let@token/\else
  \ifx\@let@token'\else
  \ifx\@let@token)\else
  \ifx\@let@token-\else
  \ifx\@let@token\@xobeysp\else
  \ifx\@let@token\space\else
  \ifx\@let@token\@sptoken\else
   \space
%    \end{macrocode}
% La premi�re fois que j'ai �crit cela, il y avait une demi-douzaine de
% |\fi| � cet endroit, et j'ai pens� que c'�tait suffisamment moche 
% comme �a.
%    \begin{macrocode}
   \fi\fi\fi\fi\fi\fi\fi\fi\fi\fi\fi\fi\fi\fi\fi\fi\fi\fi}
%</package>
%    \end{macrocode}
% \end{macro}
%
% \Finale
%

