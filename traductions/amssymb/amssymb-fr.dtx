% \def\filename{amssymb-fr.dtx}
% \def\fileversion{3.01}
% \def\filedate{2013/01/14}
%
% \iffalse meta-comment
%
% American Mathematical Society
% Technical Support
% Publications Technical Group
% 201 Charles Street
% Providence, RI 02904
% USA
% tel: (401) 455-4080
%      (800) 321-4267 (USA and Canada only)
% fax: (401) 331-3842
% email: tech-support@ams.org
%
% Copyright 2001, 2009, 2013 American Mathematical Society.
%
% Unlimited copying and redistribution of this file are permitted as
% long as this file is not modified.  Modifications, and distribution
% of modified versions, are permitted, but only if the resulting file
% is renamed.
%
% \fi
%
% \iffalse
%<*driver>
\documentclass{ltxdoc}
\usepackage[inputenc,fontenc,babel]{translatex-fr}
% On retraite certaines commandes qu'avait à l'origine "amsdtx"
\newcommand{\fn}[1]{\texttt{#1}}
\newcommand{\pkg}[1]{\textsf{#1}}
\newcommand{\opt}[1]{\texttt{#1}}
\begin{document}
\def\filedatefr{14/01/2013}
\title{L'extension \textsf{amssymb}\thanks{Ce fichier a pour numéro de
        version \fileversion\ et a été mis à jour le \filedatefr. Son 
        titre original est \og \emph{The \textsf{amssymb} package} \fg.}}
\author{American Mathematical Society}
\date{\filedatefr}
\raggedbottom % not much available stretch for filling pages
\DocInput{amssymb-fr.dtx}
\end{document}
%</driver>
% \fi
%
% \maketitle
%
% \MakeShortVerb\|
%
% \section{Introduction}
%    Ce fichier définit tous les symboles trouvés dans les polices de symboles
%    de l'AMS : \fn{msam} et \fn{msbm}.
%
% \StopEventually{}
%
% \section{Implémentation}
%    Le fichier fournit tout d'abord l'indentification de l'extension.
%    \begin{macrocode}
\NeedsTeXFormat{LaTeX2e}% LaTeX 2.09 can't be used (nor non-LaTeX)
[1994/12/01]% LaTeX date must be December 1994 or later
\ProvidesPackage{amssymb}[2013/01/14 v3.01 AMS font symbols]
%    \end{macrocode}
%
%    Voir la documentation de l'extension \pkg{amsfonts} pour une discussion de
%    l'obsolescence de l'option \opt{psamfonts}.
%    \begin{macrocode}
\DeclareOption{psamsfonts}{\PassOptionsToPackage{psamsfonts}{amsfonts}}
%    \end{macrocode}
%
%    Traitement des options.
%    \begin{macrocode}
\ProcessOptions\relax
%    \end{macrocode}
%
%    Nous faisons appel à l'extension \pkg{amsfonts} pour faire toute la 
%    configuration des polices dont nous avons besoin.
%    \begin{macrocode}
\RequirePackage{amsfonts}[1995/01/01]
%    \end{macrocode}
%
%    Un message d'avertissement en cas de chargement de l'extension 
%    \pkg{stix}\footnote{N.D.T. : ce message indique : \og L'extension 
%    `amssymb' est redondante quand vous utilisez l'extension `stix', aussi
%    je ne vais pas charger `amssymb' \fg.}.
%    \begin{macrocode}
\@ifpackageloaded{stix}{%
    \PackageWarningNoLine{amssymb}{The 'amssymb' package is redundant when
      you are using the 'stix' package, so I'm not going to load amssymb}
    \endinput
}{}
%    \end{macrocode}
%
%    Nous annulons quelques définitions de symboles qui ont pu être définis
%    par l'extension \pkg{amsfonts} \trad{(q.v.)}; sinon |\DeclareMathSymbol|
%    génère quelques messages d'erreur (tous ces noms de symboles sont 
%    redéfinis avec |\let| par rapport au premier défini ; de cette manière, 
%    si les codes sous-jacents changent, seul un changement est à faire ici.
%    \begin{macrocode}
\let\square\relax \let\rightsquigarrow\square \let\lozenge\square
\let\vartriangleright\square \let\vartriangleleft\square
\let\trianglerighteq\square \let\trianglelefteq\square
%    \end{macrocode}
%
%    Changement de code de catégorie de la \trad{double quote} pour 
%    faire en sorte qu'elle ne soit pas active (ce qui était un problème
%    quand des extensions comme \pkg{german} étaient utilisées). Ceci
%    implique que les affectations avec |\let| doivent être rendues 
%    globales.
%    \begin{macrocode}
\begingroup \catcode`\"=12
%    \end{macrocode}
%    Maintenant, nous définissons l'ensemble de tous les noms de symboles
%    standards pour les polices \fn{msam} et \fn{msbm}. Les redéfinitions
%    de symboles ou les commandes qui ne peuvent être définies par le biais
%    de |\DeclareMathSymbol| sont déjà traitées dans l'extension 
%    \pkg{amsfonts} (par exemple |\yen| ou |\widehat|).
%    \begin{macrocode}
\DeclareMathSymbol{\boxdot}       {\mathbin}{AMSa}{"00}
\DeclareMathSymbol{\boxplus}      {\mathbin}{AMSa}{"01}
\DeclareMathSymbol{\boxtimes}     {\mathbin}{AMSa}{"02}
\DeclareMathSymbol{\square}       {\mathord}{AMSa}{"03}
\DeclareMathSymbol{\blacksquare}  {\mathord}{AMSa}{"04}
\DeclareMathSymbol{\centerdot}    {\mathbin}{AMSa}{"05}
\DeclareMathSymbol{\lozenge}      {\mathord}{AMSa}{"06}
\DeclareMathSymbol{\blacklozenge} {\mathord}{AMSa}{"07}
\DeclareMathSymbol{\circlearrowright}   {\mathrel}{AMSa}{"08}
\DeclareMathSymbol{\circlearrowleft}    {\mathrel}{AMSa}{"09}
%% In amsfonts.sty:
%%\DeclareMathSymbol{\rightleftharpoons}{\mathrel}{AMSa}{"0A}
\DeclareMathSymbol{\leftrightharpoons}  {\mathrel}{AMSa}{"0B}
\DeclareMathSymbol{\boxminus}     {\mathbin}{AMSa}{"0C}
\DeclareMathSymbol{\Vdash}        {\mathrel}{AMSa}{"0D}
\DeclareMathSymbol{\Vvdash}       {\mathrel}{AMSa}{"0E}
\DeclareMathSymbol{\vDash}        {\mathrel}{AMSa}{"0F}
\DeclareMathSymbol{\twoheadrightarrow}  {\mathrel}{AMSa}{"10}
\DeclareMathSymbol{\twoheadleftarrow}   {\mathrel}{AMSa}{"11}
\DeclareMathSymbol{\leftleftarrows}     {\mathrel}{AMSa}{"12}
\DeclareMathSymbol{\rightrightarrows}   {\mathrel}{AMSa}{"13}
\DeclareMathSymbol{\upuparrows}         {\mathrel}{AMSa}{"14}
\DeclareMathSymbol{\downdownarrows} {\mathrel}{AMSa}{"15}
\DeclareMathSymbol{\upharpoonright} {\mathrel}{AMSa}{"16}
 \global\let\restriction\upharpoonright
\DeclareMathSymbol{\downharpoonright}   {\mathrel}{AMSa}{"17}
\DeclareMathSymbol{\upharpoonleft}  {\mathrel}{AMSa}{"18}
\DeclareMathSymbol{\downharpoonleft}{\mathrel}{AMSa}{"19}
\DeclareMathSymbol{\rightarrowtail} {\mathrel}{AMSa}{"1A}
\DeclareMathSymbol{\leftarrowtail}  {\mathrel}{AMSa}{"1B}
\DeclareMathSymbol{\leftrightarrows}{\mathrel}{AMSa}{"1C}
\DeclareMathSymbol{\rightleftarrows}{\mathrel}{AMSa}{"1D}
\DeclareMathSymbol{\Lsh}            {\mathrel}{AMSa}{"1E}
\DeclareMathSymbol{\Rsh}            {\mathrel}{AMSa}{"1F}
\DeclareMathSymbol{\rightsquigarrow}  {\mathrel}{AMSa}{"20}
\DeclareMathSymbol{\leftrightsquigarrow}{\mathrel}{AMSa}{"21}
\DeclareMathSymbol{\looparrowleft}  {\mathrel}{AMSa}{"22}
\DeclareMathSymbol{\looparrowright} {\mathrel}{AMSa}{"23}
\DeclareMathSymbol{\circeq}       {\mathrel}{AMSa}{"24}
\DeclareMathSymbol{\succsim}      {\mathrel}{AMSa}{"25}
\DeclareMathSymbol{\gtrsim}       {\mathrel}{AMSa}{"26}
\DeclareMathSymbol{\gtrapprox}    {\mathrel}{AMSa}{"27}
\DeclareMathSymbol{\multimap}     {\mathrel}{AMSa}{"28}
\DeclareMathSymbol{\therefore}    {\mathrel}{AMSa}{"29}
\DeclareMathSymbol{\because}      {\mathrel}{AMSa}{"2A}
\DeclareMathSymbol{\doteqdot}     {\mathrel}{AMSa}{"2B}
 \global\let\Doteq\doteqdot
\DeclareMathSymbol{\triangleq}    {\mathrel}{AMSa}{"2C}
\DeclareMathSymbol{\precsim}      {\mathrel}{AMSa}{"2D}
\DeclareMathSymbol{\lesssim}      {\mathrel}{AMSa}{"2E}
\DeclareMathSymbol{\lessapprox}   {\mathrel}{AMSa}{"2F}
\DeclareMathSymbol{\eqslantless}  {\mathrel}{AMSa}{"30}
\DeclareMathSymbol{\eqslantgtr}   {\mathrel}{AMSa}{"31}
\DeclareMathSymbol{\curlyeqprec}  {\mathrel}{AMSa}{"32}
\DeclareMathSymbol{\curlyeqsucc}  {\mathrel}{AMSa}{"33}
\DeclareMathSymbol{\preccurlyeq}  {\mathrel}{AMSa}{"34}
\DeclareMathSymbol{\leqq}         {\mathrel}{AMSa}{"35}
\DeclareMathSymbol{\leqslant}     {\mathrel}{AMSa}{"36}
\DeclareMathSymbol{\lessgtr}      {\mathrel}{AMSa}{"37}
\DeclareMathSymbol{\backprime}    {\mathord}{AMSa}{"38}
\DeclareMathSymbol{\risingdotseq} {\mathrel}{AMSa}{"3A}
\DeclareMathSymbol{\fallingdotseq}{\mathrel}{AMSa}{"3B}
\DeclareMathSymbol{\succcurlyeq}  {\mathrel}{AMSa}{"3C}
\DeclareMathSymbol{\geqq}         {\mathrel}{AMSa}{"3D}
\DeclareMathSymbol{\geqslant}     {\mathrel}{AMSa}{"3E}
\DeclareMathSymbol{\gtrless}      {\mathrel}{AMSa}{"3F}
%% in amsfonts.sty
%% \DeclareMathSymbol{\sqsubset}    {\mathrel}{AMSa}{"40}
%% \DeclareMathSymbol{\sqsupset}    {\mathrel}{AMSa}{"41}
\DeclareMathSymbol{\vartriangleright}{\mathrel}{AMSa}{"42}
\DeclareMathSymbol{\vartriangleleft} {\mathrel}{AMSa}{"43}
\DeclareMathSymbol{\trianglerighteq} {\mathrel}{AMSa}{"44}
\DeclareMathSymbol{\trianglelefteq}  {\mathrel}{AMSa}{"45}
\DeclareMathSymbol{\bigstar}    {\mathord}{AMSa}{"46}
\DeclareMathSymbol{\between}    {\mathrel}{AMSa}{"47}
\DeclareMathSymbol{\blacktriangledown}  {\mathord}{AMSa}{"48}
\DeclareMathSymbol{\blacktriangleright} {\mathrel}{AMSa}{"49}
\DeclareMathSymbol{\blacktriangleleft}  {\mathrel}{AMSa}{"4A}
\DeclareMathSymbol{\vartriangle}        {\mathrel}{AMSa}{"4D}
\DeclareMathSymbol{\blacktriangle}      {\mathord}{AMSa}{"4E}
\DeclareMathSymbol{\triangledown}       {\mathord}{AMSa}{"4F}
\DeclareMathSymbol{\eqcirc}       {\mathrel}{AMSa}{"50}
\DeclareMathSymbol{\lesseqgtr}    {\mathrel}{AMSa}{"51}
\DeclareMathSymbol{\gtreqless}    {\mathrel}{AMSa}{"52}
\DeclareMathSymbol{\lesseqqgtr}   {\mathrel}{AMSa}{"53}
\DeclareMathSymbol{\gtreqqless}   {\mathrel}{AMSa}{"54}
\DeclareMathSymbol{\Rrightarrow}  {\mathrel}{AMSa}{"56}
\DeclareMathSymbol{\Lleftarrow}   {\mathrel}{AMSa}{"57}
\DeclareMathSymbol{\veebar}       {\mathbin}{AMSa}{"59}
\DeclareMathSymbol{\barwedge}     {\mathbin}{AMSa}{"5A}
\DeclareMathSymbol{\doublebarwedge} {\mathbin}{AMSa}{"5B}
%% In amsfonts.sty
%%\DeclareMathSymbol{\angle}        {\mathord}{AMSa}{"5C}
\DeclareMathSymbol{\measuredangle}  {\mathord}{AMSa}{"5D}
\DeclareMathSymbol{\sphericalangle} {\mathord}{AMSa}{"5E}
\DeclareMathSymbol{\varpropto}    {\mathrel}{AMSa}{"5F}
\DeclareMathSymbol{\smallsmile}   {\mathrel}{AMSa}{"60}
\DeclareMathSymbol{\smallfrown}   {\mathrel}{AMSa}{"61}
\DeclareMathSymbol{\Subset}       {\mathrel}{AMSa}{"62}
\DeclareMathSymbol{\Supset}       {\mathrel}{AMSa}{"63}
\DeclareMathSymbol{\Cup}          {\mathbin}{AMSa}{"64}
 \global\let\doublecup\Cup
\DeclareMathSymbol{\Cap}          {\mathbin}{AMSa}{"65}
 \global\let\doublecap\Cap
\DeclareMathSymbol{\curlywedge}   {\mathbin}{AMSa}{"66}
\DeclareMathSymbol{\curlyvee}     {\mathbin}{AMSa}{"67}
\DeclareMathSymbol{\leftthreetimes} {\mathbin}{AMSa}{"68}
\DeclareMathSymbol{\rightthreetimes}{\mathbin}{AMSa}{"69}
\DeclareMathSymbol{\subseteqq}    {\mathrel}{AMSa}{"6A}
\DeclareMathSymbol{\supseteqq}    {\mathrel}{AMSa}{"6B}
\DeclareMathSymbol{\bumpeq}       {\mathrel}{AMSa}{"6C}
\DeclareMathSymbol{\Bumpeq}       {\mathrel}{AMSa}{"6D}
\DeclareMathSymbol{\lll}          {\mathrel}{AMSa}{"6E}
 \global\let\llless\lll
\DeclareMathSymbol{\ggg}          {\mathrel}{AMSa}{"6F}
 \global\let\gggtr\ggg
\DeclareMathSymbol{\circledS}     {\mathord}{AMSa}{"73}
\DeclareMathSymbol{\pitchfork}    {\mathrel}{AMSa}{"74}
\DeclareMathSymbol{\dotplus}      {\mathbin}{AMSa}{"75}
\DeclareMathSymbol{\backsim}      {\mathrel}{AMSa}{"76}
\DeclareMathSymbol{\backsimeq}    {\mathrel}{AMSa}{"77}
\DeclareMathSymbol{\complement}   {\mathord}{AMSa}{"7B}
\DeclareMathSymbol{\intercal}     {\mathbin}{AMSa}{"7C}
\DeclareMathSymbol{\circledcirc}  {\mathbin}{AMSa}{"7D}
\DeclareMathSymbol{\circledast}   {\mathbin}{AMSa}{"7E}
\DeclareMathSymbol{\circleddash}  {\mathbin}{AMSa}{"7F}
%%   Begin AMSb declarations
\DeclareMathSymbol{\lvertneqq}    {\mathrel}{AMSb}{"00}
\DeclareMathSymbol{\gvertneqq}    {\mathrel}{AMSb}{"01}
\DeclareMathSymbol{\nleq}         {\mathrel}{AMSb}{"02}
\DeclareMathSymbol{\ngeq}         {\mathrel}{AMSb}{"03}
\DeclareMathSymbol{\nless}        {\mathrel}{AMSb}{"04}
\DeclareMathSymbol{\ngtr}         {\mathrel}{AMSb}{"05}
\DeclareMathSymbol{\nprec}        {\mathrel}{AMSb}{"06}
\DeclareMathSymbol{\nsucc}        {\mathrel}{AMSb}{"07}
\DeclareMathSymbol{\lneqq}        {\mathrel}{AMSb}{"08}
\DeclareMathSymbol{\gneqq}        {\mathrel}{AMSb}{"09}
\DeclareMathSymbol{\nleqslant}    {\mathrel}{AMSb}{"0A}
\DeclareMathSymbol{\ngeqslant}    {\mathrel}{AMSb}{"0B}
\DeclareMathSymbol{\lneq}         {\mathrel}{AMSb}{"0C}
\DeclareMathSymbol{\gneq}         {\mathrel}{AMSb}{"0D}
\DeclareMathSymbol{\npreceq}      {\mathrel}{AMSb}{"0E}
\DeclareMathSymbol{\nsucceq}      {\mathrel}{AMSb}{"0F}
\DeclareMathSymbol{\precnsim}     {\mathrel}{AMSb}{"10}
\DeclareMathSymbol{\succnsim}     {\mathrel}{AMSb}{"11}
\DeclareMathSymbol{\lnsim}        {\mathrel}{AMSb}{"12}
\DeclareMathSymbol{\gnsim}        {\mathrel}{AMSb}{"13}
\DeclareMathSymbol{\nleqq}        {\mathrel}{AMSb}{"14}
\DeclareMathSymbol{\ngeqq}        {\mathrel}{AMSb}{"15}
\DeclareMathSymbol{\precneqq}     {\mathrel}{AMSb}{"16}
\DeclareMathSymbol{\succneqq}     {\mathrel}{AMSb}{"17}
\DeclareMathSymbol{\precnapprox}  {\mathrel}{AMSb}{"18}
\DeclareMathSymbol{\succnapprox}  {\mathrel}{AMSb}{"19}
\DeclareMathSymbol{\lnapprox}     {\mathrel}{AMSb}{"1A}
\DeclareMathSymbol{\gnapprox}     {\mathrel}{AMSb}{"1B}
\DeclareMathSymbol{\nsim}         {\mathrel}{AMSb}{"1C}
\DeclareMathSymbol{\ncong}        {\mathrel}{AMSb}{"1D}
\DeclareMathSymbol{\diagup}       {\mathord}{AMSb}{"1E}
\DeclareMathSymbol{\diagdown}     {\mathord}{AMSb}{"1F}
\DeclareMathSymbol{\varsubsetneq}   {\mathrel}{AMSb}{"20}
\DeclareMathSymbol{\varsupsetneq}   {\mathrel}{AMSb}{"21}
\DeclareMathSymbol{\nsubseteqq}     {\mathrel}{AMSb}{"22}
\DeclareMathSymbol{\nsupseteqq}     {\mathrel}{AMSb}{"23}
\DeclareMathSymbol{\subsetneqq}     {\mathrel}{AMSb}{"24}
\DeclareMathSymbol{\supsetneqq}     {\mathrel}{AMSb}{"25}
\DeclareMathSymbol{\varsubsetneqq}  {\mathrel}{AMSb}{"26}
\DeclareMathSymbol{\varsupsetneqq}  {\mathrel}{AMSb}{"27}
\DeclareMathSymbol{\subsetneq}      {\mathrel}{AMSb}{"28}
\DeclareMathSymbol{\supsetneq}      {\mathrel}{AMSb}{"29}
\DeclareMathSymbol{\nsubseteq}      {\mathrel}{AMSb}{"2A}
\DeclareMathSymbol{\nsupseteq}      {\mathrel}{AMSb}{"2B}
\DeclareMathSymbol{\nparallel}      {\mathrel}{AMSb}{"2C}
\DeclareMathSymbol{\nmid}           {\mathrel}{AMSb}{"2D}
\DeclareMathSymbol{\nshortmid}      {\mathrel}{AMSb}{"2E}
\DeclareMathSymbol{\nshortparallel} {\mathrel}{AMSb}{"2F}
\DeclareMathSymbol{\nvdash}         {\mathrel}{AMSb}{"30}
\DeclareMathSymbol{\nVdash}         {\mathrel}{AMSb}{"31}
\DeclareMathSymbol{\nvDash}         {\mathrel}{AMSb}{"32}
\DeclareMathSymbol{\nVDash}         {\mathrel}{AMSb}{"33}
\DeclareMathSymbol{\ntrianglerighteq}{\mathrel}{AMSb}{"34}
\DeclareMathSymbol{\ntrianglelefteq}{\mathrel}{AMSb}{"35}
\DeclareMathSymbol{\ntriangleleft}  {\mathrel}{AMSb}{"36}
\DeclareMathSymbol{\ntriangleright} {\mathrel}{AMSb}{"37}
\DeclareMathSymbol{\nleftarrow}     {\mathrel}{AMSb}{"38}
\DeclareMathSymbol{\nrightarrow}    {\mathrel}{AMSb}{"39}
\DeclareMathSymbol{\nLeftarrow}     {\mathrel}{AMSb}{"3A}
\DeclareMathSymbol{\nRightarrow}    {\mathrel}{AMSb}{"3B}
\DeclareMathSymbol{\nLeftrightarrow}{\mathrel}{AMSb}{"3C}
\DeclareMathSymbol{\nleftrightarrow}{\mathrel}{AMSb}{"3D}
\DeclareMathSymbol{\divideontimes}  {\mathbin}{AMSb}{"3E}
\DeclareMathSymbol{\varnothing}     {\mathord}{AMSb}{"3F}
\DeclareMathSymbol{\nexists}        {\mathord}{AMSb}{"40}
\DeclareMathSymbol{\Finv}           {\mathord}{AMSb}{"60}
\DeclareMathSymbol{\Game}           {\mathord}{AMSb}{"61}
%% In amsfonts.sty:
%%\DeclareMathSymbol{\mho}          {\mathord}{AMSb}{"66}
\DeclareMathSymbol{\eth}            {\mathord}{AMSb}{"67}
\DeclareMathSymbol{\eqsim}          {\mathrel}{AMSb}{"68}
\DeclareMathSymbol{\beth}           {\mathord}{AMSb}{"69}
\DeclareMathSymbol{\gimel}          {\mathord}{AMSb}{"6A}
\DeclareMathSymbol{\daleth}         {\mathord}{AMSb}{"6B}
\DeclareMathSymbol{\lessdot}        {\mathbin}{AMSb}{"6C}
\DeclareMathSymbol{\gtrdot}         {\mathbin}{AMSb}{"6D}
\DeclareMathSymbol{\ltimes}         {\mathbin}{AMSb}{"6E}
\DeclareMathSymbol{\rtimes}         {\mathbin}{AMSb}{"6F}
\DeclareMathSymbol{\shortmid}       {\mathrel}{AMSb}{"70}
\DeclareMathSymbol{\shortparallel}  {\mathrel}{AMSb}{"71}
\DeclareMathSymbol{\smallsetminus}  {\mathbin}{AMSb}{"72}
\DeclareMathSymbol{\thicksim}       {\mathrel}{AMSb}{"73}
\DeclareMathSymbol{\thickapprox}    {\mathrel}{AMSb}{"74}
\DeclareMathSymbol{\approxeq}       {\mathrel}{AMSb}{"75}
\DeclareMathSymbol{\succapprox}     {\mathrel}{AMSb}{"76}
\DeclareMathSymbol{\precapprox}     {\mathrel}{AMSb}{"77}
\DeclareMathSymbol{\curvearrowleft} {\mathrel}{AMSb}{"78}
\DeclareMathSymbol{\curvearrowright}{\mathrel}{AMSb}{"79}
\DeclareMathSymbol{\digamma}        {\mathord}{AMSb}{"7A}
\DeclareMathSymbol{\varkappa}       {\mathord}{AMSb}{"7B}
\DeclareMathSymbol{\Bbbk}           {\mathord}{AMSb}{"7C}
\DeclareMathSymbol{\hslash}         {\mathord}{AMSb}{"7D}
%% In amsfonts.sty:
%%\DeclareMathSymbol{\hbar}         {\mathord}{AMSb}{"7E}
\DeclareMathSymbol{\backepsilon}    {\mathrel}{AMSb}{"7F}
%    \end{macrocode}
%    Maintenant, nous fermons le groupe afin que |"| récupère son ancien
%    code de catégorie.
%    \begin{macrocode}
\endgroup
%    \end{macrocode}
%
%    Enfin, est mis le traditionnel |\endinput| pour garantir que des
%    éléments inutiles en fin de fichier ne soient copiés par \fn{docstrip}.
%    \begin{macrocode}
\endinput
%    \end{macrocode}
%
% \Finale
