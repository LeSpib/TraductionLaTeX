% \CheckSum{831}
%\iffalse
%    This file is built for \LaTeXe, so we make sure an error is
%    generated when it is used with another format
%<*dtx>
\ProvidesFile{f-layout.dtx}
%</dtx>
%<+package>\NeedsTeXFormat{LaTeX2e}
%<+package>\ProvidesPackage{layout}
%<+driver>\ProvidesFile{layout.drv}
%\ProvidesFile{f-layout.dtx}
                [1995/11/23 v1.1k Show layout parameters]
%
%    A short driver is provided that can be extracted if necessary by
%    the \textsf{DocStrip} program provided with \LaTeXe.
%    \begin{macrocode}
%<*driver>

\def\traductionchanges{oui}

\documentclass{ltxdoc}

\newcommand\Lopt[1]{\textsf{#1}}
\usepackage[T1]{fontenc}
\usepackage[latin1]{inputenc}
\usepackage{afterpage,ifthen}
\usepackage[french]{layout}
\usepackage[frenchb]{babel}

\pagestyle{myheadings}
\DisableCrossrefs
\begin{document}
\RecordChanges
\DocInput{f-layout.dtx}
\PrintChanges
\end{document}
%</driver>
%    \end{macrocode}
%\fi
% \changes{v1.1c}{1994/07/13}{D\'eplacement des informations d'identification
%    en t\^ete de fichier et suppression de l'usage permanent de
%    \cs{filedate} et ses comparses}
% \changes{v1.1d}{1994/09/08}{Les textes sont stock\'es dans des commandes
%    pour permettre l'internationalisation.}
%
% \GetFileInfo{f-layout.dtx}
% \title{Afficher les variables de mise en page}
% \author{Kent McPherson a.o.\thanks{Adapt\'e \`a \LaTeXe\ par
%        Johannes Braams}\and Traduit en Fran\c{c}ais par\\
%        Jean-Pierre Drucbert\\et Benjamin Bayart}
% \date{\filedate}
%
% \markboth{package layout version \fileversion\space au \filedate}
%          {package layout version \fileversion\space au \filedate}
%
% \maketitle
%
% \section{Introduction}
%
%
%Ce package \LaTeXe\ est une nouvelle impl�mentation de %
%\texttt{layout.sty} par Kent McPherson. Il d�finit la commande |\layout|
%qui produit un aper\c{c}u de la mise en page du document courant. La commande
%|\layout*| recalcule les valeurs qu'elle utilise pour produire le
%sch�ma. \textit{NdT}~: Les figures sur les pages suivantes montrent le r�sultat de la commande
%|\layout*| pour ce document (avec les options \texttt{verbose}, \texttt{reals} 
%et \texttt{french}).
%
%   \afterpage{\clearpage%
%        \ifthenelse{\isodd{\value{page}}}%
%            {\afterpage{{\centering\layout*}}}%
%            {{\centering\layout*}}}
%
%
%\subsection{Utilisation}
%Dans le dessin, divers libell�s apparaissent: le langage utilis� peut �tre 
%choisi par une option du package: \texttt{dutch}, \texttt{english} (d�faut), 
%\texttt{french} (ou \texttt{francais}), \texttt{spanish}, \texttt{portuguese}, 
%\texttt{brazilian}.
%
%L'option \texttt{verbose} fait s'afficher certains param�tres sur 
%votre terminal par |\layout|. Par d�faut, l'option \texttt{silent} ne les 
%affiche pas.
%
%Le comportement normal de ce package est d'imprimer les param�tres sous 
%forme enti�re (donc tronqu�e): c'est l'option \texttt{integers}. L'option 
%\texttt{reals} permet d'imprimer les param�tres avec leur partie 
%fractionnaire.
%\clearpage
%
% \StopEventually{}
%
% \section{L'impl\'ementation}
%
% \changes{v1.1b}{1994/03/23}{supprim\'e le \cs{wlog} puisqu'on a
%    maintenant \cs{ProvidesPackage} pour faire \c{c}a}
% \changes{v1.1d}{1994/09/08}{Ajout des options dutch et english}
%
%    Ce package imprime un dessin qui illustre le format de page
%    qui est impl\'ement\'e par la classe de document. Dans le
%    dessin, plusieurs mots apparaissent. Ils sont stock\'es dans
%    des commandes pour \^etre en mesure de s\'electionner une
%    langue diff\'erente.
% \changes{v1.1e}{1994/10/29}{Les textes n\'eerlandais devraient \^etre en
%    un seul mot.}
% \changes{v1.1f}{1995/03/14}{Ajout\'e \cs{notshown}}
% \changes{v1.1f}{1995/03/14}{Ajout\'e l'option french}
% \changes{v1.1j}{1995/10/29}{Ajout\'e les options spanish, brazilian et
%    portuguese} 
% \changes{v1.1k}{1995/11/23}{Correction de la documentation}
%    \begin{macrocode}
%<*package>
\DeclareOption{dutch}{%
  \def\Headertext{Kopregel}
  \def\Bodytext{Broodtekst}
  \def\Footertext{Voetregel}
  \def\MarginNotestext{Marge\\Notities}
  \def\oneinchtext{een inch}
  \def\notshown{niet getoond}
  }
\DeclareOption{english}{%
  \def\Headertext{Header}
  \def\Bodytext{Body}
  \def\Footertext{Footer}
  \def\MarginNotestext{Margin\\Notes}
  \def\oneinchtext{one inch}
  \def\notshown{not shown}
  }
\DeclareOption{french}{%
  \def\Headertext{Ent\^{e}te}
  \def\Bodytext{Corps}
  \def\Footertext{Pied de page}
  \def\MarginNotestext{Marge\\Notes}
  \def\oneinchtext{un pouce}
  \def\notshown{non affich\'{e}}
  }
\DeclareOption{francais}{\ExecuteOptions{french}}
\DeclareOption{spanish}{%
  \def\Headertext{Encabezamiento}
  \def\Bodytext{Cuerpo}
  \def\Footertext{Pie de p\'agina}
  \def\MarginNotestext{Notas\\ Marginales}
  \def\oneinchtext{una pulgada}
  \def\notshown{no mostradas}
  }
\DeclareOption{portuguese}{%
  \def\Headertext{Cabe\c{c}alho}
  \def\Bodytext{Corpo}
  \def\Footertext{Rodap\'e}
  \def\MarginNotestext{Notas\\ Marginais}
  \def\oneinchtext{uma polegada}
  \def\notshown{n\~ao mostradas}
  }
\DeclareOption{brazilian}{%
  \def\Headertext{Cabe\c{c}alho}
  \def\Bodytext{Corpo}
  \def\Footertext{Rodap\'e}
  \def\MarginNotestext{Notas\\ Marginais}
  \def\oneinchtext{uma polegada}
  \def\notshown{n\~ao mostradas}
  }
%    \end{macrocode}
%
%    Ce package a une option \Lopt{verbose}. En l'utilisant, cela
%    affichera plusieurs des param\`etres de mise en page sur le terminal.
%    \begin{macrocode}
\DeclareOption{verbose}{\let\type\typeout}
\DeclareOption{silent}{\let\type\@gobble}
%    \end{macrocode}
%
% \changes{v1.1h}{1995/04/07}{Ajout\'e les options \Lopt{integers}
%    (par d\'efaut) et \Lopt{reals}}
%
%    Le comportement normal de ce package quand il montre les valeurs
%    des param\`etres de mise en page est de les tronquer. Cependant,
%    si vous souhaitez voir les v\'eritables param\`etres (valeurs
%    r\'eelles), vous pouvez utiliser l'option \Lopt{reals} qui permet
%    ce comportement.
% \changes{v1.1i}{1995/06/25}{La version 1995/06/01 de \LaTeX\ n'a plus
%    besoin de doubles di\`eses dans l'argument de \cs{DeclareOption}}
%    \begin{macrocode}
\def\lay@value{}
\DeclareOption{integers}{%
  \renewcommand*{\lay@value}[2]{%
    \expandafter\number\csname #1@#2\endcsname pt}}
\DeclareOption{reals}{%
  \renewcommand*{\lay@value}[2]{\the\csname #2\endcsname}}
%    \end{macrocode}
%
%    Par d\'efaut, la langue est l'Anglais, le mode est
%    \Lopt{silent} et les param\`etres sont affich\'es en tant qu'entiers.
%    \begin{macrocode}
\ExecuteOptions{english,silent,integers}
\ProcessOptions
%    \end{macrocode}
%
% \begin{macro}{\bs}
%    D\'efinit |\bs| si elle n'est pas d\'efinie, la red\'efinit si
%    elle l'est.
%    \begin{macrocode}
\@ifundefined{bs}{\newcommand\bs{\char '134 }}%
  {\renewcommand\bs{\char '134 }}
%    \end{macrocode}
% \end{macro}
%
% \begin{macro}{\ConvertToCount}
%    Cette commande m\'emorise la valeur d'un registre de type
%    \emph{length} dans un registre de type \emph{count}.
%    \begin{macrocode}
\def\ConvertToCount#1#2{%
%    \end{macrocode}
%    D'abord copier la valeur
%    \begin{macrocode}
  #1=#2
%    \end{macrocode}
%    Puis diviser par 65536.
%    \begin{macrocode}
  \divide #1 by 65536}
%    \end{macrocode}
%    Le r\'esultat est que le compteur contient la valeur de la longueur
%    en point (la partie enti\`ere, pour \^etre pr\'ecis).
% \end{macro}
%
% \begin{macro}{\SetToHalf}
% \begin{macro}{\SetToQuart}
%    De petites macros utilis\'ees pour calculer des positions.
%    \begin{macrocode}
\def\SetToHalf#1#2{#1=#2\relax\divide#1by\tw@}
\def\SetToQuart#1#2{#1=#2\relax\divide#1by4}
%    \end{macrocode}
% \end{macro}
%
% \begin{macro}{\Identify}
%    Une petite macro pour identifier les dimensions.
%    \begin{macrocode}
\def\Identify#1{%
  \put(\PositionX,\PositionY){\circle{20}}
  \put(\PositionX,\PositionY){\makebox(0,0){\tiny #1}}
}
%    \end{macrocode}
% \end{macro}
% \end{macro}
%
% \begin{macro}{\InsideHArrow}
%    Cette macro est utilis\'ee pour produire deux fl\`eches horizontales
%    dans une bo\^ite. L'argument donne la largeur de la bo\^ite.
% \changes{v1.1c}{1994/07/14}{Ajout\'e un test pour d\'etecter les fl\`eches de longueur n\'egative}
%    \begin{macrocode}
\def\InsideHArrow#1{{%
  \ArrowLength = #1
  \divide\ArrowLength by \tw@
  \advance\ArrowLength by -10
  \advance\PositionX by -10
  \ifnum\ArrowLength<\z@
    \put(\PositionX,\PositionY){\vector(1,0){-\ArrowLength}}
    \advance\PositionX by 20
    \put(\PositionX,\PositionY){\vector(-1,0){-\ArrowLength}}
  \else
    \put(\PositionX,\PositionY){\vector(-1,0){\ArrowLength}}
    \advance\PositionX by 20
    \put(\PositionX,\PositionY){\vector(+1,0){\ArrowLength}}
  \fi
}}
%    \end{macrocode}
% \end{macro}
%
% \begin{macro}{\InsideVArrow}
%    Cette macro est utilis\'ee pour produire deux fl\`eches
%    verticales dans une bo\^ite, l'argument est la hauteur de
%    la bo\^ite.
%    \begin{macrocode}
\def\InsideVArrow#1{{%
  \ArrowLength = #1
  \divide\ArrowLength by \tw@
  \advance\ArrowLength by -10
  \advance\PositionY by -10
  \put(\PositionX,\PositionY){\vector(0,-1){\ArrowLength}}
  \advance\PositionY by 20
  \put(\PositionX,\PositionY){\vector(0,+1){\ArrowLength}}
}}
%    \end{macrocode}
% \end{macro}
%
% \begin{macro}{\OutsideHArrow}
%    Cette macro est utilis\'ee pour produire deux fl\`eches
%    horizontales pour d\'elimiter une longueur. Le premier
%    argument est la position de la fl\`eche de droite, le
%    deuxi\`eme est la longueur, et le troisi\`eme sp\'ecifie
%    la longueur des fl\`eches.
%    \begin{macrocode}
\def\OutsideHArrow#1#2#3{{%
  \PositionX = #1
  \advance\PositionX by #3
  \put(\PositionX,\PositionY){\vector(-1,0){#3}}
  \PositionX = #1 \advance\PositionX-#2
  \advance\PositionX by -#3
  \put(\PositionX,\PositionY){\vector(+1,0){#3}}
}}
%    \end{macrocode}
% \end{macro}
%
% \begin{macro}{\OutsideVArrow}
%    Cette macro est utilis\'ee pour produire deux fl\`eches
%    verticales pour d\'elimiter une longueur. Le premier argument
%    est la position de la fl\`eche du bas, le deuxi\`eme la longueur,
%    et le troisi\`eme la longueur des deux fl\`eches.
%    \begin{macrocode}
\def\OutsideVArrow#1#2#3#4{{%
  \PositionY = #1
  \advance\PositionY by -#3
  \put(\PositionX,\PositionY){\vector(0,+1){#3}}
  \PositionY = #1
  \advance\PositionY#2
  \advance\PositionY#4
  \put(\PositionX,\PositionY){\vector(0,-1){#4}}
}}
%    \end{macrocode}
% \end{macro}
%
% \begin{macro}{\Show}
% \changes{v1.1g}{1995/04/03}{Simplifi\'e la d\'efinition, montre maintenant
%    des valeurs compl\`etes}
% \changes{v1.1h}{1995/04/07}{Utilise \cs{lay@value} pour montrer la valeur
%    \`a cause de la pr\'esence d'une option qui d\'ecide laquelle de deux
%    mani\`eres doit \^etre utilis\'ee pour le faire}
%
%    Macro utilis\'ee dans la table qui montre les r\`eglages des
%    param\`etres.
%    \begin{macrocode}
\def\Show#1#2{\bs #2 = \lay@value{#1}{#2}}
%    \end{macrocode}
% \end{macro}
%
% \begin{macro}{\Type}
% \changes{v1.1g}{1995/04/03}{Simplifi\'e la d\'efinition, montre maintenant
%    la valeur compl\`ete}
% \changes{v1.1h}{1995/04/07}{Utilise \cs{lay@value} pour montrer la valeur
%    puisqu'il y a une option qui indique laquelle des deux mani\`eres de le
%    faire doit \^etre utilis\'e}
%
%    Macro utilis\'ee pour montrer les r\`eglages sur le terminal.
%    \begin{macrocode}
\def\Type#1#2{%
  \type{#2 = \lay@value{#1}{#2}}}
%    \end{macrocode}
% \end{macro}
%
% \begin{macro}{\oneinch}
%    Une constante donnant la longueur d'un pouce en points (approximativement).
%    \begin{macrocode}
\newcount\oneinch
\oneinch=72
%    \end{macrocode}
% \end{macro}
%
%    Parce que la vue d'ensemble de la mise en page est produite dans
%    un environnement figure,  on a besoin d'allouer un certain nombre
%    de compteurs qui sont utilis\'es pour m\'emoriser les valeurs de
%    diverses dimensions.
%
% \begin{macro}{\cnt@paperwidth}
% \begin{macro}{\cnt@paperheight}
%    Les dimensions du papier,
%    \begin{macrocode}
\newcount\cnt@paperwidth
\newcount\cnt@paperheight
\ConvertToCount\cnt@paperwidth\paperwidth
\ConvertToCount\cnt@paperheight\paperheight
%    \end{macrocode}
% \end{macro}
% \end{macro}
%
% \begin{macro}{\cnt@hoffset}
% \begin{macro}{\cnt@voffset}
%    les offsets,
%    \begin{macrocode}
\newcount\cnt@hoffset
\newcount\cnt@voffset
\ConvertToCount\cnt@hoffset\hoffset
\ConvertToCount\cnt@voffset\voffset
%    \end{macrocode}
% \end{macro}
% \end{macro}
%
% \begin{macro}{\cnt@textheight}
% \begin{macro}{\cnt@textwidth}
%    dimensions de la zone de texte,
%    \begin{macrocode}
\newcount\cnt@textheight
\newcount\cnt@textwidth
%    \end{macrocode}
% \end{macro}
% \end{macro}
%
% \begin{macro}{\cnt@topmargin}
% \begin{macro}{\cnt@oddsidemargin}
% \begin{macro}{\cnt@evensidemargin}
%    les marges,
%    \begin{macrocode}
\newcount\cnt@topmargin
\newcount\cnt@oddsidemargin
\newcount\cnt@evensidemargin
%    \end{macrocode}
% \end{macro}
% \end{macro}
% \end{macro}
%
% \begin{macro}{\cnt@headheight}
% \begin{macro}{\cnt@headsep}
%    dimensions des textes courants (en-t\^etes),
%    \begin{macrocode}
\newcount\cnt@headheight
\newcount\cnt@headsep
%    \end{macrocode}
% \end{macro}
% \end{macro}
%
% \begin{macro}{\cnt@marginparsep}
% \begin{macro}{\cnt@marginparwidth}
% \begin{macro}{\cnt@marginparpush}
%    paragraphes en marge,
%    \begin{macrocode}
\newcount\cnt@marginparsep
\newcount\cnt@marginparwidth
\newcount\cnt@marginparpush
%    \end{macrocode}
% \end{macro}
% \end{macro}
% \end{macro}
%
% \begin{macro}{\cnt@footskip}
%    la distance entre le pied de page courant et le texte,
%    \begin{macrocode}
\newcount\cnt@footskip
%    \end{macrocode}
% \end{macro}
%    et la hauteur du pied de page, qui est utilis\'ee ici pour sortir
%    une bo\^ite, mais qui n'est pas utilis\'ee par \LaTeX.
% \begin{macro}{\fheight}
%    \begin{macrocode}
\newcount\fheight
\fheight=12
%    \end{macrocode}
% \end{macro}
%
%    En plus de la repr\'esentation enti\`ere des param\`etres de mise
%    en page, on a aussi besoin de registres pour m\'emoriser des
%    valeurs de r\'ef\'erence.
%
% \begin{macro}{\ref@top}
%    La position du haut de la zone imprimable est un pouce sous le
%    haut de la page, par d\'efaut. La valeur de |\ref@top| est relative
%    au coin en bas \`a gauche de l'environnement picture qui sera
%    utilis\'e.
%
%    \begin{macrocode}
\newcount\ref@top
\ref@top=\cnt@paperheight \advance\ref@top by -\oneinch
%    \end{macrocode}
% \end{macro}
%
% \begin{macro}{\ref@hoffset}
% \begin{macro}{\ref@voffset}
%    Pour les offsets,
%    \begin{macrocode}
\newcount\ref@hoffset
\newcount\ref@voffset
%    \end{macrocode}
%    les valeurs |\hoffset| et |\voffset| sont ajout\'ees \`a la
%    valeur par d\'efaut d'un pouce.
%    \begin{macrocode}
\ref@hoffset=\cnt@hoffset  \advance\cnt@hoffset by \oneinch
\ref@voffset=\cnt@voffset  
%    \end{macrocode}
%
%    |\cnt@voffset| est convertie pour \^etre relative \`a l'origine
%    de l'environnement picture.
%    \begin{macrocode}
\cnt@voffset=\ref@top 
\advance\cnt@voffset by -\ref@voffset
%    \end{macrocode}
% \end{macro}
% \end{macro}
%
% \begin{macro}{\ref@head}
%    et la zone de texte, les en-t\^etes courants,
%    \begin{macrocode}
\newcount\ref@head
%    \end{macrocode}
% \end{macro}
%
% \begin{macro}{\ref@body}
%    le corps du texte
%    \begin{macrocode}
\newcount\ref@body
%    \end{macrocode}
% \end{macro}
%
% \begin{macro}{\ref@foot}
%    et le pied de page courant.
%    \begin{macrocode}
\newcount\ref@foot
%    \end{macrocode}
% \end{macro}
%
% \begin{macro}{\ref@margin}
% \begin{macro}{\ref@marginwidth}
% \begin{macro}{\ref@marginpar}
%    Ces valeurs sont diff\'erentes pour les pages paires et impaires,
%    donc elles sont calcul\'ees par |\layout|.
%    \begin{macrocode}
\newcount\ref@margin
\newcount\ref@marginwidth
\newcount\ref@marginpar
%    \end{macrocode}
% \end{macro}
% \end{macro}
% \end{macro}
%
%    Ces valeurs sont les registres temporaires utilis\'es dans le
%    positionnement des diff\'erentes parties de l'environnement
%    picture.
%    \begin{macrocode}
\newcount\Interval
\newcount\ExtraYPos
\newcount\PositionX
\newcount\PositionY
\newcount\ArrowLength
%    \end{macrocode}
%
%  \begin{macro}{\lay@getvalues}
% \changes{v1.1j}{1995/10/30}{Ajout\'e une macro pour calculer les valeurs
%    plus tard}
%    Toutes les valeurs qui peuvent changer pendant la vie du document
%    sont calcul\'ees par la macro |\lay@getvalues|. Par d\'efaut cette
%    macro est ex\'ecut\'ee pendant le |\begin{document}|.
%    \begin{macrocode}
\def\lay@getvalues{%
  \ConvertToCount\cnt@textheight\textheight
  \ConvertToCount\cnt@textwidth\textwidth
  \ConvertToCount\cnt@topmargin\topmargin
  \ConvertToCount\cnt@oddsidemargin\oddsidemargin
  \ConvertToCount\cnt@evensidemargin\evensidemargin
  \ConvertToCount\cnt@headheight\headheight
  \ConvertToCount\cnt@headsep\headsep
  \ConvertToCount\cnt@marginparsep\marginparsep
  \ConvertToCount\cnt@marginparwidth\marginparwidth
  \ConvertToCount\cnt@marginparpush\marginparpush
  \ConvertToCount\cnt@footskip\footskip
  \ref@head=\ref@top
    \advance\ref@head by -\ref@voffset
    \advance\ref@head by -\cnt@topmargin  
    \advance\ref@head by -\cnt@headheight
  \ref@body=\ref@head  
    \advance\ref@body by -\cnt@headsep   
    \advance\ref@body by -\cnt@textheight
  \ref@foot=\ref@body
    \advance\ref@foot by -\cnt@footskip
  }
\AtBeginDocument{\lay@getvalues}
%    \end{macrocode}
%  \end{macro}
%  \begin{macro}{\computevalues}
%
% \begin{macro}{\layout}
% \begin{macro}{\layout*}
%    La commande |\layout| fait une image (environnement picture) et
%    un tableau qui repr\'esentent les param\`etres de mise en page.
%
% \changes{v1.1b}{1994/03/23}{Produit deux pages en mode recto-verso}
% \changes{v1.1j}{1995/10/30}{Ajout\'e \cs{layout*}}
%    \begin{macrocode}
\newcommand\layout{%
  \@ifstar{\lay@getvalues\lay@xlayout}{\lay@xlayout}}
\def\lay@xlayout{%
  \lay@layout
  \if@twoside
    \lay@layout
  \fi}
%    \end{macrocode}
%
% \begin{macro}{\lay@layout}
%    La macro interne |\lay@layout| fait toutes les basses \oe{uvres}.
%    \begin{macrocode}
\newcommand\lay@layout{%
  \thispagestyle{empty}
%    \end{macrocode}
%
%    Les actions de |\layout| d\'ependent du <<~pagetyle~>> en cours.
%    \begin{macrocode}
  \if@twoside
    \ifodd\count\z@
%    \end{macrocode}
%
%    Ici, nous nous occupons d'une page impaire dans un document
%    recto-verso.
%
%    \begin{macrocode}
      \typeout{Two-sided document style, odd page.}
%    \end{macrocode}
%
%    Ainsi, nous calculons |\ref@marginwidth|, |\ref@marginpar| et
%    |\ref@margin|.
% \changes{v1.1}{1994/02/23}{Ajout\'e une v\'erification pour reversemargin}
%    \begin{macrocode}
      \ref@marginwidth=\cnt@oddsidemargin
      \ref@marginpar=\oneinch
      \advance\ref@marginpar by \ref@hoffset
      \advance\ref@marginpar by \cnt@oddsidemargin
      \ref@margin\ref@marginpar
      \if@reversemargin
        \advance\ref@marginpar by -\cnt@marginparsep
        \advance\ref@marginpar by -\cnt@marginparwidth
      \else
        \advance\ref@marginpar by \cnt@textwidth
        \advance\ref@marginpar by \cnt@marginparsep
      \fi
%    \end{macrocode}
% 
%    \begin{macrocode}
    \else
%    \end{macrocode}
%
%    Ici, nous traitons le cas d'une page paire dans un document
%    recto-verso.
%
%    \begin{macrocode}
  \typeout{Two-sided document style, even page.}
%    \end{macrocode}
%
%    On calcule donc |\ref@marginwidth|, |\ref@marginpar| et
%    |\ref@margin|.
% \changes{v1.1}{1994/02/23}{Ajout\'e une verification pour reversemargin}
%    \begin{macrocode}
      \ref@marginwidth=\cnt@evensidemargin
      \ref@marginpar=\oneinch
      \advance\ref@marginpar by \ref@hoffset
      \advance\ref@marginpar by \cnt@evensidemargin
      \ref@margin\ref@marginpar
      \if@reversemargin
        \advance\ref@marginpar by \cnt@textwidth
        \advance\ref@marginpar by \cnt@marginparsep
      \else
        \advance\ref@marginpar by -\cnt@marginparsep
        \advance\ref@marginpar by -\cnt@marginparwidth
      \fi
%    \end{macrocode}
% 
%    \begin{macrocode}
    \fi
  \else
%    \end{macrocode}
%
%    Enfin, le cas d'un document recto.
%
% \changes{v1.1}{1994/02/23}{Ajout\'e une v\'erification pour reversemargin}
%    \begin{macrocode}
    \typeout{One-sided document style.}
    \ref@marginwidth=\cnt@oddsidemargin
    \ref@marginpar=\oneinch
    \advance\ref@marginpar by \ref@hoffset
    \advance\ref@marginpar by \cnt@oddsidemargin
    \ref@margin\ref@marginpar
    \if@reversemargin
      \advance\ref@marginpar by -\cnt@marginparsep
      \advance\ref@marginpar by -\cnt@marginparwidth
    \else
      \advance\ref@marginpar by \cnt@textwidth
      \advance\ref@marginpar by \cnt@marginparsep
    \fi
  \fi
%    \end{macrocode}
%
%
%  Maintenant, nous commen\c{c}ons l'environnement picture, la
%  division par deux de tous les param\`etres est faite en r\'eglant
%  |\unitlength| \`a \texttt{0.5pt}.
%    \begin{macrocode}
  \setlength{\unitlength}{.5pt}
  \begin{picture}(\cnt@paperwidth,\cnt@paperheight)
    \centering
    \thicklines
%    \end{macrocode}
%
%    En premier lieu, nous positionnons la bo\^ite de la page et
%    les lignes de r\'ef\'erence.
%
%    \begin{macrocode}
    \put(0,0){\framebox(\cnt@paperwidth,\cnt@paperheight){\mbox{}}}
    \put(0,\cnt@voffset){\dashbox{10}(\cnt@paperwidth,0){\mbox{}}}
    \put(\cnt@hoffset,0){\dashbox{10}(0,\cnt@paperheight){\mbox{}}}
%    \end{macrocode}
%
%    ensuite les en-t\^etes,
%
%    \begin{macrocode}
    \put(\ref@margin,\ref@head){%
      \framebox(\cnt@textwidth,\cnt@headheight)%
        {\footnotesize\Headertext}}
%    \end{macrocode}
%
%    le corps du texte,
%
%    \begin{macrocode}
    \put(\ref@margin,\ref@body){%
      \framebox(\cnt@textwidth,\cnt@textheight){\Bodytext}}
%    \end{macrocode}
%
%     les pieds de page
%
%    \begin{macrocode}
    \put(\ref@margin,\ref@foot){%
      \framebox(\cnt@textwidth,\fheight){\footnotesize\Footertext}}
%    \end{macrocode}
%
%    et l'emplacement pour les notes en marge.
%
%    \begin{macrocode}
    \put(\ref@marginpar,\ref@body){%
      \framebox(\cnt@marginparwidth,\cnt@textheight)%
               {\footnotesize\shortstack{\MarginNotestext}}}
%    \end{macrocode}
%
%    Maintenant, nous commen\c{c}ons \`a positionner les fl\`eches pour
%    marquer les diff\'erents param\`etres. \`A partir de maintenant,
%    on utilise |\thinlines|.
%    \begin{macrocode}
    \thinlines
%    \end{macrocode}
%
%    |\PositionX| et |\PositionY| seront les coordonn\'ees du centre des
%    fl\`eches indiquant |\textwidth|.
%    \begin{macrocode}
    \SetToHalf\PositionX\cnt@textwidth 
    \advance\PositionX by \ref@margin
%    \end{macrocode}
%    Les fl\`eches devraient \^etre un peu au-dessus du bas de la
%    bo\^ite indiquant le corps du texte.
%    \begin{macrocode}
    \PositionY = \ref@body
    \advance\PositionY by 50
%    \end{macrocode}
%    Un identifiant est mis ici, dans un cercle.
%    \begin{macrocode}
    \Identify{8}
%    \end{macrocode}
%    Ensuite on dessine la fl\`eche.
%    \begin{macrocode}
    \InsideHArrow\cnt@textwidth
%    \end{macrocode}
%
% Maintenant |\textheight|
%    \begin{macrocode}
    \SetToHalf\PositionY\cnt@textheight
    \advance\PositionY by \ref@body
%    \end{macrocode}
% La position horizontale de la fl\`eches est au $4/5$ de la largeur
% du corps de texte.
%    \begin{macrocode}
    \PositionX = \cnt@textwidth
    \divide\PositionX by 5
    \multiply \PositionX by 4
    \advance\PositionX by \ref@margin
%    \end{macrocode}
%
% On positionne un identifiant, dans un cercle.
%    \begin{macrocode}
    \Identify{7}
    \InsideVArrow\cnt@textheight
%    \end{macrocode}
% L'offset horizontal |\hoffset|,
%    \begin{macrocode}
    \PositionY = \ref@foot
    \SetToHalf\PositionX\cnt@hoffset
    \Identify{1}
    \InsideHArrow\cnt@hoffset
%    \end{macrocode}
% la largeur de la marge,
%    \begin{macrocode}
    \SetToQuart\PositionY\cnt@textheight
    \advance\PositionY by \ref@body
    \OutsideHArrow\ref@margin\ref@marginwidth{20}
    \PositionX = \cnt@hoffset
    \advance\PositionX by -30
    \Identify{3}
%    \end{macrocode}
% la largeur des paragraphes en marge |\marginparwidth|,
%    \begin{macrocode}
    \SetToQuart\PositionY\cnt@textheight
    \advance\PositionY by \ref@body
%    \end{macrocode}
% cette fl\`eche doit \^etre un peu sous celle pour |\oddsidemargin| ou
% |\evensidemargin|.
%    \begin{macrocode}
    \advance\PositionY by 30
    \SetToHalf\PositionX\cnt@marginparwidth
    \advance\PositionX by \ref@marginpar
    \Identify{10}
    \InsideHArrow\cnt@marginparwidth
%    \end{macrocode}
%
%    La distance entre paragraphes en marge et corps de texte,
%    |\marginparsep| variant selon que l'on soit en recto ou en
%    recto-verso.
%
%    \begin{macrocode}
    \advance\PositionY by 30
    \if@twoside
      \ifodd\count\z@
%    \end{macrocode}
%
%    Document recto, page paire:
% \changes{v1.1b}{1994/03/23}{\cs{OutSideHArrow} aurait d\^u \^etre
%    \cs{OutsideHArrow}}
%
%    \begin{macrocode}
        \OutsideHArrow\ref@marginpar\cnt@marginparsep{20}
        \PositionX = \ref@marginpar
      \else
%    \end{macrocode}
%    et page impaire.
% \changes{v1.1b}{1994/03/23}{\cs{OutSideHArrow} aurait d\^u \^etre
%    \cs{OutsideHArrow}}
%
%    \begin{macrocode}
        \OutsideHArrow\ref@margin\cnt@marginparsep{20}
        \PositionX = \ref@margin
      \fi
    \else
%    \end{macrocode}
%
%    Document recto simple.
%
%    \begin{macrocode}
      \OutsideHArrow\ref@marginpar\cnt@marginparsep{20}
      \PositionX = \ref@marginpar
    \fi
    \advance\PositionX by 30
    \Identify{9}
%    \end{macrocode}
%
%    Identifions |\footskip|. La fl\`eche sera \`a $1/8$ de la largeur
%    du texte (|\textwidth|).
%    \begin{macrocode}
    \PositionX = \cnt@textwidth
    \divide\PositionX by 8
    \advance\PositionX by \ref@margin
    \OutsideVArrow\ref@foot\cnt@footskip{20}{20}
    \PositionY = \ref@foot
    \advance\PositionY by -30
    \Identify{11}
%    \end{macrocode}
%
%    Identifions |\voffset|. La fl\`eche sera un peu \`a gauche du bord
%    du papier.
%    \begin{macrocode}
    \PositionX = \cnt@paperwidth
    \advance\PositionX by -50
    \PositionY = \cnt@paperheight
    \ExtraYPos = \PositionY
    \advance\ExtraYPos by -\cnt@voffset
    \advance\PositionY by \cnt@voffset
    \divide\PositionY by \tw@
    \Identify{2}
    \InsideVArrow\ExtraYPos
%    \end{macrocode}
%
%    Identifions |\topmargin|, |\headheight| et |\headsep|.
%
%    Les fl\`eches seront situ\'ees au $1/8$ de |\textwidth|,
%    s\'epar\'ees l'une de l'autre par la m\^eme distance, m\'emoris\'ee
%    dans |\Interval|.
%    \begin{macrocode}
    \Interval = \cnt@textwidth
    \divide\Interval by 8
    \PositionX = \ref@margin
    \advance\PositionX by \Interval
%    \end{macrocode}
%    Commen\c{c}ons par |\topmargin|. Les longueurs des fl\`eches du
%    haut pour ces trois param\`etres sont diff\'erentes. Dans le calcul
%    de ces longueurs, on doit compenser le diam\`etre du cercle.
%    \begin{macrocode}
    \Identify{4}
    \ExtraYPos = \ref@head
    \advance\ExtraYPos\cnt@headheight
    \ArrowLength = \PositionY
    \advance\ArrowLength-\ExtraYPos
    \advance\ArrowLength-\cnt@topmargin
    \advance\ArrowLength-10%
    \OutsideVArrow\ExtraYPos\cnt@topmargin{20}{\ArrowLength}
    \advance\PositionX by \Interval
%    \end{macrocode}
%    Ensuite |\headheight|
%    \begin{macrocode}
    \Identify{5}
    \advance\ArrowLength\cnt@topmargin
    \OutsideVArrow\ref@head\cnt@headheight{20}{\ArrowLength}
    \advance\PositionX by \Interval
%    \end{macrocode}
%    Enfin |\headsep|
%    \begin{macrocode}
    \Identify{6}
    \advance\ArrowLength\cnt@headheight
    \ExtraYPos=\ref@body
    \advance\ExtraYPos\cnt@textheight
    \OutsideVArrow\ExtraYPos\cnt@headsep{20}{\ArrowLength}
%    \end{macrocode}
%
%    Maintenant, on peut terminer l'environnement picture et
%    ajouter un peu de blanc.
%    \begin{macrocode}
  \end{picture}

  \medskip 
%    \end{macrocode}
%
%    Sous la figure, on place un tableau pour montrer la v\'eritable valeur
%    des param\`etres. Notez que la partie d\'ecimale des longueurs
%    est tronqu\'ee, \emph{i.e.} \texttt{72.27pt} est annonc\'e comme
%    \texttt{72pt}.
%
%    Le tableau est plac\'e dans une bo\^ite de profondeur nulle pour
%    qu'il reste sur la m\^eme page que l'environnement picture.
% \changes{v1.1b}{1994/03/23}{L'affichage des marges droite et gauche
%    ne fonctionnait pas}
% \changes{v1.1c}{1994/07/14}{en mode compatibilit\'e, \cs{footnotesize}
%    appel \cs{normalfont}; on a donc besoin de passer en fonte tt plus
%    tard}
% \changes{v1.1f}{1995/03/14}{ajout\'e \cs{notshown}}
%    \begin{macrocode}
  \vtop to 0pt{%
    \@minipagerestore\footnotesize\ttfamily
    \begin{tabular}{@{}rl@{\hspace{20pt}}rl}
      1 & \oneinchtext\ + \bs\texttt{hoffset}
        & 2 & \oneinchtext\ + \bs\texttt{voffset} \\
      3 & \if@twoside
            \ifodd\count\z@ \Show{cnt}{oddsidemargin}
            \else \Show{cnt}{evensidemargin}
            \fi
          \else
            \Show{cnt}{oddsidemargin}
          \fi                    & 4 & \Show{cnt}{topmargin} \\
      5 & \Show{cnt}{headheight} & 6 & \Show{cnt}{headsep} \\
      7 & \Show{cnt}{textheight} & 8 & \Show{cnt}{textwidth} \\
      9 & \Show{cnt}{marginparsep}&10& \Show{cnt}{marginparwidth} \\
      11& \Show{cnt}{footskip}   &   & \Show{cnt}{marginparpush}
       \rlap{(\notshown)}\\
        & \Show{ref}{hoffset}    &   & \Show{ref}{voffset} \\
        & \Show{cnt}{paperwidth} &   & \Show{cnt}{paperheight} \\
      
  \end{tabular}\vss}
%    \end{macrocode}
%    Quand l'option \Lopt{verbose} est utilis\'ee, les lignes suivantes affichent
%    les param\`etres sur le terminal.
%    \begin{macrocode}
  \Type{ref}{hoffset}
  \Type{ref}{voffset}
  \Type{cnt}{textheight}
  \Type{cnt}{textwidth}
%    \end{macrocode}
%    Pour finir, on commence une nouvelle page.
%    \begin{macrocode}
  \newpage
}
%</package>
%    \end{macrocode}
% \end{macro}
% \end{macro}
% \end{macro}
% \end{macro}
%
% \Finale
\endinput
