% \iffalse meta-comment
%
% Copyright 1993 1994 1995 1996 1997 1998 1999
% The LaTeX3 Project and any individual authors listed elsewhere
% in this file.
%
% This file is part of the Standard LaTeX `Tools Bundle'.
% -------------------------------------------------------
%
% It may be distributed and/or modified under the
% conditions of the LaTeX Project Public License, either version 1.2
% of this license or (at your option) any later version.
% The latest version of this license is in
%    http://www.latex-project.org/lppl.txt
% and version 1.2 or later is part of all distributions of LaTeX
% version 1999/12/01 or later.
%
% The list of all files belonging to the LaTeX `Tools Bundle' is
% given in the file `manifest.txt'.
%
% \fi
%
% \iffalse    This is a METACOMMENT
%
%% Package `multicol' to use with LaTeX2e
%% Copyright 1989-1999 Frank Mittelbach
%%
%%  In addition to the terms of LPPL any distributed version
%%  (unchanged or modified) of multicol has to keep the statement
%%  about the moral obligation for using multicol. In case of major
%%  changes where this would not be appropriate the author of the
%%  changed version should contact the copyright holder.
%%
%%
%%  Moral obligation for using multicol:
%%  ------------------------------------
%%
%%  Users of multicol who wish to include or use multicol or a modified
%%  version in a proprietary and commercially market product are asked
%%  under certain conditions (see below) for the payment of a license
%%  fee.  The size of this fee is to be determined, in each instance,
%%  by the commercial user, depending on his/her judgment of the value of
%%  multicol for his/her product.
%%
%%
%%  The conditions for this are as follows:
%%
%%   The producer of a proprietary and commercially market product
%%   that involves typesetting using multicol is asked to determine
%%   the value of a license fee for using multicol if
%%
%%   - the product is a document and the producer has decided to
%%     include multicol to typeset (parts of) the document or has
%%     directed the author of the document to include multicol (for
%%     example, by providing a class file to be used by the author)
%%
%%   - the product is a LaTeX class or package that includes multicol
%%
%%
%%   There is no moral obligation in case
%%
%%   - the product is a document but producer has not directed
%%     the author to include multicol (in that case the moral obligation
%%     lies with the author of the document)
%%
%%   - the product does not involve typesetting, e.g., consists, for
%%     example, of distributing multicol and its documentation.
%%
%%   - the product is not proprietary, i.e., is made available as free
%%     software itself (which doesn't prohibit its commercial marketing)
%%
%%   - multicol is used for non-commercial purposes
%%
%%
%% Determinating a license fee might result in a license fee of zero
%% (i.e., no payment) in case a producer has determined that the use
%% of multicol has no enhancing effect on the product. This is a
%% plausible scenario, i.e., in the above two cases the producer is
%% only asked to evaluate the value of multicol for the product
%% not for the payment of a license fee per se (which might or might
%% not follow from this evaluation).
%%
%% The license fee, if any, can be payed either to the LaTeX3 fund
%% (see ltx3info.txt in the base LaTeX distribution) or to the author of
%% the program who can be contacted at
%%
%%     Frank.Mittelbach@latex-project.org
%%
%<*dtx>
          \ProvidesFile{f-multicol.dtx}
%</dtx>
%<package>\NeedsTeXFormat{LaTeX2e}[1997/12/01]
%<package>\ProvidesPackage{multicol}
%<driver> \ProvidesFile{multicol.drv}
% \fi
%         \ProvidesFile{f-multicol.dtx}
          [1999/10/21 v1.5w  multicolumn formatting (FMi)]
%
%
%% \CheckSum{1439}
%% \CharacterTable
%%  {Upper-case    \A\B\C\D\E\F\G\H\I\J\K\L\M\N\O\P\Q\R\S\T\U\V\W\X\Y\Z
%%   Lower-case    \a\b\c\d\e\f\g\h\i\j\k\l\m\n\o\p\q\r\s\t\u\v\w\x\y\z
%%   Digits        \0\1\2\3\4\5\6\7\8\9
%%   Exclamation   \!     Double quote  \"     Hash (number) \#
%%   Dollar        \$     Percent       \%     Ampersand     \&
%%   Acute accent  \'     Left paren    \(     Right paren   \)
%%   Asterisk      \*     Plus          \+     Comma         \,
%%   Minus         \-     Point         \.     Solidus       \/
%%   Colon         \:     Semicolon     \;     Less than     \<
%%   Equals        \=     Greater than  \>     Question mark \?
%%   Commercial at \@     Left bracket  \[     Backslash     \\
%%   Right bracket \]     Circumflex    \^     Underscore    \_
%%   Grave accent  \`     Left brace    \{     Vertical bar  \|
%%   Right brace   \}     Tilde         \~}
%%
%
%
% \changes{v1.5n}{1997/06/05}{Int\'egration de
%                            l'am\'elioration de la documentation,
%                            gr\^acieusement fournie par Robin~\bsc{Fairbairns}}
% \changes{v1.4h}{1992/06/04}{Ajout des marques de
%                            tra\c{c}age par tracingmulticols$\ge2$}
% \changes{v1.4a}{1992/02/11}{Ajout du traitement des
%                            multi-colonnes en mode imbriqu\'e}
% \changes{v1.0d}{1989/05/17}{Toutes les lignes sont
%                            raccourcies \`a 72 ou moins}
% \changes{v1.0e}{1989/06/21}{Red\'efinition de l'env.
%                            description pour utiliser
%                            \cs{descriptionmargin}\quotechar=5pt dans la
%                             documentation}
% \changes{v1.0f}{1989/07/11}{\cs{z@} fix\'e \`a 0~pt pour
%                             la red\'efinition de description}
% \changes{v1.1a}{1989/09/20}{\cs{multicolssep} chang\'e en \cs{multicolsep}.}
%
% \def\description{\list{}{\labelwidth 0pt \leftmargin\descriptionmargin
%    \itemindent-\leftmargin \let\makelabel\descriptionlabel}}
% \newdimen\descriptionmargin \descriptionmargin=5pt
%
% \DoNotIndex{\@M,\@Mi,\@bsphack,\@cclv,\@colht,\@currlist,\@deferlist}
% \DoNotIndex{\@elt,\@esphack,\@floatplacement}
% \DoNotIndex{\@ifundefined,\@ifnextchar,\@makecol}
% \DoNotIndex{\@ne,\@outputpage,\@scolelt,\@spaces,\@tempb,\@tempcnta}
% \DoNotIndex{\@width}
% \DoNotIndex{\addvspace,\advance,\allowbreak}
% \DoNotIndex{\baselineskip,\begingroup,\box,\columnsep,\copy,\count}
% \DoNotIndex{\count@,\def,\dimen@,\divide,\docdate}
% \DoNotIndex{\edef,\eject,\egroup,\else,\endgroup,\endinput}
% \DoNotIndex{\fi,\fileversion,\filedate}
% \DoNotIndex{\gdef,\global}
% \DoNotIndex{\hbox,\hfil,\hrule,\ht,\hss}
% \DoNotIndex{\ifdim,\ifnum,\ifvoid,\ignorespaces,\insert,\immediate}
% \DoNotIndex{\let,\loop}
% \DoNotIndex{\maxdepth,\message,\multiply}
% \DoNotIndex{\newbox,\newcount,\newdimen,\newskip,\number,\newpage}
% \DoNotIndex{\outputpenalty,\p@,\penalty}
% \DoNotIndex{\relax,\repeat,\setbox,\skip,\space,\splitmaxdepth}
% \DoNotIndex{\splittopskip,\string,\sixt@@n}
% \DoNotIndex{\the,\thepage,\thr@@,\topskip,\tw@,\typeout}
% \DoNotIndex{\unvbox,\vbox,\vfill,\vsplit,\voidb@x,\vrule}
% \DoNotIndex{\write,\wd}
% \DoNotIndex{\z@}
%
% \MakeShortVerb{\|}
% \newcommand{\mc}{{\sf multicols}}
% \newcommand{\TUB}{{\sl TUGboat\/}}
% \newcommand{\TB}{{\sl\TeX book\/}}
%
%
% \setcounter{StandardModuleDepth}{2}
% \setcounter{collectmore}{3}
%
% \GetFileInfo{f-multicol.dtx}
% \title{Un environnement pour composer du texte \\ sur
% plusieurs colonnes.\thanks{Version~:~\fileversion, derni\`ere
% r\'evision~:~\filedate}%
% \thanks{Note~:~Ce package est distribu\'e sous des
% conditions qui influent sur son utilisation dans un
% contexte commercial. Cf. les d\'etails au d\'ebut du
% fichier source.}%
% \thanks{Titre original~: An environment for multicolumn
% output}}
%
% \author{Frank~\bsc{Mittelbach}\thanks{Traduction fran\c{c}aise par Jean-Pierre~\bsc{Drucbert}
% et Fran\c{c}oise~\bsc{Marre}-\bsc{Fournier}, du 06/03/2000.}\\
%       {\rm m\'el~:~} cf. au d\'ebut du fichier source}
% \date{Imprim\'e le \today}
% \maketitle
%
% \begin{abstract}
% Ce document d\'ecrit l'utilisation et l'impl\'ementation de
% l'environnement \mc{}. Cet environnement permet de passer
% d'un format sur une colonne \`a un format sur plusieurs
% colonnes et inversement, sur la m\^eme page. Les notes de bas
% de page sont trait\'ees correctement (dans la plupart des
% cas), mais seront plac\'ees en bas de la page et non en bas
% de chaque colonne. Le m\'ecanisme des flottants de
% \LaTeX\ est cependant partiellement d\'esactiv\'e dans la
% r\'ealisation actuelle. et sera ajout\'e dans une version
% future. Pour le moment, seuls les flottants portant
% sur toute la largeur de la page (c'est-\`a-dire les formes~$*$)
% peuvent \^etre utilis\'es \`a l'int\'erieur de la port\'ee
% de l'environnement.
% \end{abstract}
%
%
% \begin{multicols}{3}[\section*{Pr\'eface pour la version 1.5}]
%   \hbadness=10000
% Cette nouvelle version contient deux modifications
% principales~:~\mc{} prendra maintenant en compte jusqu'\`a 10
% colonnes et deux possibilit\'es suppl\'ementaires de
% r\'eglage ont \'et\'e ajout\'ees \`a la routine
% d'\'equilibrage. Cette routine v\'erifie maintenant la
% mauvaise qualit\'e des colonnes r\'esultantes et rejette
% les r\'esultats des calculs qui sont sup\'erieurs \`a un
% seuil donn\'e.
%
% En m\^eme temps, \mc{} a \'et\'e mis \`a jour pour
% fonctionner avec \LaTeXe{}.
%
% \columnbreak
%
% Je\footnote{Frank~\bsc{Mittelbach}} vous pr\'esente mes
% excuses pour le niveau de documentation du code, mais mes
% responsabilit\'es envers \LaTeXe{} m'occupent trop pour faire
% du bon travail. J'esp\`ere que cela sera corrig\'e dans un
% proche avenir.
% \end{multicols}
%
%
% \setcounter{collectmore}{2}
% \begin{multicols}{3}[\section{Introduction}]
%   \hbadness=10000
% Passer d'une \`a deux colonnes et inversement est possible
% en \LaTeX, mais chaque utilisation de |\twocolumn|
% ou de |\onecolumn| d\'ebute une nouvelle page. De plus, la
% toute derni\`ere page compos\'ee en deux colonnes n'est pas
% \'equilibr\'ee et cela donne souvent une colonne de droite vide
% ou presque vide. Quand j'ai commenc\'e \`a \'ecrire des
% macros pour {\sf doc.sty} (cf. <<~The {\tt
% doc}--Option~>>, \TUB\ volume 10~\#2, pp.~245--273), je
% pensais qu'il serait mieux de placer l'index sur la m\^eme
% page que la bibliographie. L'\'equilibrage de cette
% derni\`ere page n'est pas seulement une simple question
% d'esth\'etique, cela gagne aussi de la place, \`a condition,
% bien s\^ur, qu'il soit possible de commencer la section
% suivante sur la m\^eme page. Comparativement, r\'e\'ecrire
% l'environnement d'index est facile, mais le but \'etant de
% construire un environnement qui prenne en compte les notes
% de bas de page, les flottants..., c'\'etait une t\^ache plus
% ardue. Cela m'a pris un week-end entier\,\footnote{J'avais
% commenc\'e avec l'algorithme pr\'esent\'e page~417 du \TeX
% book. Sans cette aide, un week-end n'aurait pas \'et\'e
% suffisant.} pour r\'eunir les quelques lignes de code
% ci-dessous et il y a encore de grandes chances que j'aie
% malgr\'e tout oubli\'e quelque chose.
%
%
% Essayez-le et adoptez-le et, \emph{s'il
% vous pla\^it}, envoyez vos suggestions et signalez
% les bugs directement \`a Mainz.
% \end{multicols}
%
%
% \setcounter{collectmore}{0}
% \begin{multicols}{3}[\section{Interface utilisateur}]
%   \hbadness=10000
% Pour utiliser cet environnement, il suffit de dire
% simplement\\*[2mm]
% \hspace*{2mm}|\begin{multicols}{|\meta{number}|}|
% \hspace*{12mm}\meta{multicolumn text}\\
% \hspace*{2mm}|\end{multicols}|\\[2mm]
% o\`u \meta{number} est le nombre demand\'e de colonnes et
% \meta{multicolumn text} peut contenir des commandes
% \LaTeX\ arbitraires, sauf que les flottants et les
% notes marginales ne sont pas autoris\'es dans la version
% actuelle\,\footnote{Ceci est impos\'e par un manque de temps.
% Pour autoriser les flottants, il faudra r\'e\'ecrire toute
% la routine de sortie de \LaTeX.}.
%
%
% \DescribeMacro\premulticols
% Comme premi\`ere action, l'environnement {\sf multicols}
% mesure la page courante pour d\'eterminer s'il y a assez de
% place pour une certaine portion de sortie sur plusieurs
% colonnes. Ceci est contr\^ol\'e par la variable \meta{dimen}
% |\premulticols| dont la valeur peut \^etre modifi\'ee par les
% commandes \LaTeX.
%
% \DescribeMacro\multicolsep
% Si l'espace est inf\'erieur \`a |\premulticols|, une nouvelle
% page est commenc\'ee. Sinon, un saut vertical |\vskip| de
% hauteur |\multicolsep| est ajout\'e\,\footnote{En fait
% l'espace ajout\'e peut \^etre plus petit car la commande
% \cs{addvspace} est utilis\'ee. (cf. le manuel \LaTeX{}
% pour plus d'informations sur cette commande)}.
%
% \DescribeMacro\postmulticols
% Lorsque la fin de l'environnement \mc{} est atteint, un
% m\'ecanisme analogue est employ\'e, mais nous testons
% alors si un espace sup\'erieur \`a |\postmulticols| est
% disponible. De nouveau nous ajoutons l'espace
% |\multicolsep| ou commen�ons une nouvelle page.
%
% Il est souvent pratique de placer du texte en chapeau au-dessus
% de toutes les colonnes, avant la sortie en
% multi-colonnes, sans qu'une rupture de page ne puisse se
% produire entre ce chapeau et ces colonnes. Pour r\'ealiser
% cela, l'environnement \mc{} a un second argument optionnel
% qui peut \^etre utilis\'e dans ce but. Par exemple, le texte
% que vous lisez actuellement a commenc\'e par :
% \begin{verbatim}
% \begin{multicols}{3}
% [\section{Interface
%         utilisateur}]...
% \end{verbatim}
%
% Si un tel texte est inhabituellement long (ou court) la
% valeur de |\premulticols| peut avoir besoin d'\^etre ajust\'ee
% pour \'eviter une mauvaise rupture de page. Nous offrons donc
% un troisi\`eme argument qui peut \^etre utilis\'e pour
% outrepasser la valeur par d\'efaut de |\premulticols| seulement
% pour cette occasion. Donc si vous voulez combiner un long
% texte sur une seule colonne avec un environnement
% multicols, vous pourriez \'ecrire~:
% \begin{verbatim}
% \begin{multicols}{3}
%     [\section{Index}
%      Cet index contient ...]
%     [6cm]
%  ...
% \end{verbatim}
%
% \DescribeMacro\columnsep
% L'espace entre les colonnes est contr\^ol\'e par le param\`etre
% de longueur |\columnsep|. La largeur des colonnes
% individuelles est calcul\'ee automatiquement \`a partir de ce
% param\`etre et de la valeur courante de |\linewidth|. Ici une
% valeur de \textsf{\the\columnsep} a \'et\'e utilis\'ee.
%
% \DescribeMacro\columnseprule
% La s\'eparation des colonnes par des filets verticaux est
% obtenue en for�ant le param\`etre |\columnseprule| \`a une
% longueur positive. Si cette longueur est nulle ou n\'egative,
% le filet est invisible. Ici elle vaut {\sf 0.4~pt}.
%
% \DescribeMacro\multicolbaselineskip
% Puisque l'espacement des lignes dans des colonnes \'etroites
% a tendance \`a devoir \^etre ajust\'e, nous fournissons aussi un
% param\`etre \meta{skip} (longueur \'elastique) nomm\'e
% |\multicolbaselineskip| qui sera ajout\'e au param\`etre
% |\baselineskip| \`a l'int\'erieur de chaque environnement
% \mc{}. Il est prudent d'utiliser ce param\`etre avec
% pr\'ecaution ou de ne pas y toucher~;~il n'est destin\'e qu'aux
% concepteurs de packages car m\^eme de toutes petites
% modifications peuvent produire des bouleversements
% inattendus dans votre document.
%
%
%    \subsection{\'Equilibrage des colonnes}
%
% Outre les param\`etres mentionn\'es ci-dessus, d'autres sont
% disponibles pour influencer la mise en forme des colonnes
% produites.
%
% Le traitement des paragraphes en \TeX{} est contr\^ol\'e par
% plusieurs param\`etres. L'un des plus importants est appel\'e
% |\tolerance|~:~il contr\^ole la �~\emph{looseness}~� (c'est-\`a-dire
% la quantit\'e d'espace blanc entre les mots). Sa valeur par
% d\'efaut est 200 (la commande |\fussy| de \LaTeX), qui est
% trop faible pour des colonnes \'etroites. D'autre part, la
% d\'eclaration |\sloppy| (qui force |\tolerance| \`a
% $10000=\infty$) est trop grande, permettant des espacements
% vraiment trop laids\,\footnote{Regardez le paragraphe
% suivant, qui a \'et\'e compos\'e avec la d\'eclaration
% \cs{sloppy}}.
%
% \begin{sloppypar}
% \DescribeMacro\multicoltolerance \DescribeMacro\multicolpretolerance
% Nous utilisons donc un autre param\`etre
% |\multicolstolerance| pour la valeur de |\tolerance| \`a
% l'int\'erieur de l'environnement \mc{}. Sa valeur par d\'efaut
% est 9999, qui est inf\'erieure \`a l'infini mais assez
% �~mauvaise~� pour la plupart des paragraphes dans un
% environnement multicolonne. Changer sa valeur doit \^etre
% fait \emph{en dehors} de l'environnement \mc{}. Puisque
% |\tolerance| est forc\'ee \`a |\multicolstolerance| au d\'ebut de
% chaque environnement {\sf multicols}, on peut localement
% outrepasser cette valeur par d\'efaut en imposant
% \verb*+\tolerance = +\meta{valeur d\'esir\'ee}. Il existe
% \'egalement un param\`etre |\multicolpretolerance| conservant la
% valeur de |\pretolerance| \`a l'int\'erieur d'un environnement
% \mc{}. Ces deux param\`etres ne sont habituellement utilis\'es
% que par les concepteurs de packages.
% \end{sloppypar}
%
% La g\'en\'eration de la sortie sur plusieurs colonnes peut \^etre
% divis\'ee en deux parties. Dans la premi\`ere partie, nous
% rassemblons du mat\'eriel pour la page, l'envoyons vers la
% sortie, collectons du mat\'eriel pour la page suivante, et
% ainsi de suite. Dans une seconde \'etape, l'\'equilibrage sera
% fait lorsque la fin de l'environnement \mc{} sera atteint.
% Dans la premi\`ere \'etape \TeX{} pourrait consid\'erer davantage
% de mat\'eriel en trouvant les colonnes finales qu'il n'en
% utilise effectivement lors de l'\'emission de la page. Ceci
% peut provoquer un probl\`eme si une note en bas de page est
% trouv\'ee dans la partie consid\'er\'ee, mais non utilis\'ee
% sur la page courante.
% Dans ce cas la note peut appara\^\i{}tre sur la page courante,
% tandis que l'appel de note correspondant peut se trouver
% sur la page suivante\,\footnote{La raison qui se cache
% derri\`ere ce comportement est le caract\`ere asynchrone du
% \textit{page\_builder} de \TeX. Cependant ceci
% pourrait \^etre \'evit\'e en d\'efinissant des routines de sortie
% tr\`es compliqu\'ees qui n'utilisent pas des primitives \TeX{},
% telles que \cs{insert}, mais font tout \`a la main. Ceci n'a h\'elas
% pas  \'et\'e fait par manque de temps. Cela d\'epasse
% vraiment les limites d'un week-end.}
% Donc l'environnement \mc{} donne un message
% d'avertissement\,\footnote{et ce message sera \'emis m\^eme s'il
% n'y a pas de notes dans cette partie du texte} chaque fois
% qu'il est incapable d'utiliser en totalit\'e le mat\'eriel
% consid\'er\'e jusque l\`a.
%
% Si vous n'utilisez pas de notes trop souvent, les risques
% que quelque chose se passe mal sont tr\`es faibles, mais si
% cela arrive, vous pouvez aider \TeX{} en pla�ant une
% commande |\pagebreak| dans le document final. Un autre
% moyen pour influencer le comportement de \TeX{} \`a ce sujet
% est offert par la variable compteur �~{\sf collectmore}~�.
% Si vous utilisez la d\'eclaration |\setcounter| pour forcer
% ce compteur \`a un \meta{nombre}, \TeX{} consid\`erera
% \meta{nombre} lignes en plus (ou en moins) avant de prendre
% sa d\'ecision finale. La valeur $-1$ pourra donc r\'esoudre
% tous vos probl\`emes au prix de colonnes un peu moins
% optimales.
%
% Dans la seconde \'etape (\'equilibrage des colonnes) nous avons
% d'autres gadgets. D'abord, vous pouvez dire |\raggedcolumns|
% si vous n'avez pas besoin que les lignes du bas des
% colonnes soient align\'ees. L'option |\flushcolumns| est
% prise par d\'efaut, donc \TeX{} essaiera d'aligner les lignes
% de base du haut et du bas de toutes les colonnes.
%
% De plus, vous pouvez utiliser un autre compteur, <<~{\sf
% unbalance}~>>, en lui donnant une valeur positive
% \meta{nombre}. Ceci rendra toutes les colonnes, sauf celle
% la plus \`a droite, plus longues de \meta{nombre} lignes
% qu'elles ne l'auraient \'et\'e normalement. Les �~lignes~� dans
% ce contexte sont des lignes de texte normal (c'est-\`a-dire
% \'ecart\'ees de |\baselineskip|)~;~ainsi, si votre colonne
% contient des hors-texte, par exemple, vous aurez besoin
% d'un \meta{nombre} plus grand pour d\'ecaler quelque chose
% d'une colonne vers une autre.
%
% Contrairement \`a �~{\sf collectmore}~�, le compteur �~{\sf
% unbalance}~� est remis \`a z\'ero \`a la fin de l'environnement
% afin qu'il ne s'applique qu'\`a un seul environnement \mc{}.
%
% Les deux m\'ethodes peuvent \^etre combin\'ees mais je conseille
% de n'utiliser ces possibilit\'es que pour mettre la derni\`ere
% touche \`a d'importantes publications.
%
% Deux nouvelles possibilit\'es de r\'eglage ont \'et\'e ajout\'ees
% dans la version~1.5. \TeX{} permet de mesurer la
% mauvaise qualit\'e~--~la laideur~--~d'une colonne sous forme d'une valeur
% enti\`ere, o\`u 0 signifie optimal et toute autre valeur plus
% haute d\'enote la pr\'esence d'une certaine quantit\'e d'espace
% blanc en trop. 10000 est consid\'er\'e comme infiniment mauvais
% (\TeX{} ne fait plus de distinction au-dessus de cette
% valeur). De plus, la valeur sp\'eciale 100000 signifie un
% d\'ebordement (c'est-\`a-dire que la colonne contient plus de
% texte qu'il n'est possible d'y mettre).
%
% La nouvelle version mesure d\'esormais chaque colonne
% engendr\'ee et ignore les solutions pour lesquelles au moins
% une colonne a une laideur sup\'erieure \`a la valeur du compteur
% {\sf columnbadness}. La valeur par d\'efaut pour ce compteur
% est 10000, donc \TeX{} acceptera toutes les solutions sauf
% celles avec d\'ebordement.
% En donnant \`a ce ce compteur une valeur plus petite, vous
% pouvez forcer l'algorithme \`a rechercher des solutions qui
% n'auront pas de colonnes contenant beaucoup d'espace blanc.
%
% Cependant, si la valeur choisie est trop basse,
% l'algorithme peut ne trouver aucune solution acceptable et
% finalement il choisira la solution extr\^eme qui consiste \`a
% placer tout le texte dans la premi\`ere colonne.
%
% Souvent, lorsque les colonnes sont \'equilibr\'ees, il est
% impossible de trouver une solution qui distribue le texte
% \'equitablement entre toutes les colonnes. Si c'est le cas,
% habituellement la derni\`ere colonne a moins de texte que les
% autres. Dans les premi\`eres versions, ce texte \'etait \'etir\'e
% pour produire une colonne de m\^eme hauteur que toutes les
% autres, donnant parfois des colonnes vraiment horribles \`a
% voir.
%
% Dans la nouvelle version, cet \'etirement n'est effectu\'e que
% si la laideur de la derni\`ere colonne n'est pas plus grande
% que la valeur du compteur {\sf finalcolumnbadness}. Sa
% valeur par d\'efaut est 9999, ce qui emp\^eche l'\'etirement pour
% toutes les colonnes que \TeX{} consid\'ererait comme
% infiniment laides. Dans ce cas, la colonne finale
% peut \^etre plus courte, ce qui donne un bien meilleur
% r\'esultat.
%
% Il y a deux autres param\`etres qui sont encore un peu
% exp\'erimentaux~:~l'un s'appelle |\multicolovershoot| et
% l'autre |\multicolundershoot|. Ils contr\^olent la quantit\'e
% d'espace dont une colonne est autoris\'ee \`a \^etre �~trop
% pleine~� ou �~trop courte~� sans affecter la laideur de la
% colonne. Ils valent 2pt par d\'efaut.
%
% \subsection{Non \'equilibrage des colonnes}
%
% Bien que ce package ait \'et\'e \'ecrit pour r\'esoudre le probl\`eme
% de l'\'equilibrage des colonnes, il m'a souvent \'et\'e demand\'e
% une version dans laquelle tout l'espace blanc est
% automatiquement plac\'e dans la (les) derni\`ere(s) colonne(s).
% Depuis la version v1.5q ceci existe d\'esormais~: si vous
% utilisez \texttt{multicols*} au lieu de l'environnement
% habituel, les colonnes sur la derni\`ere page ne sont pas
% \'equilibr\'ees. Bien s\^ur, cet environnement ne fonctionne
% qu'au niveau le plus haut, c'est-\`a-dire \`a l'int\'erieur d'une
% bo\^\i{}te qu'il faut \'equilibrer afin de d\'eterminer une hauteur
% de colonne en l'absence de valeur fix\'ee.
%
% \subsection{Coupure manuelle des colonnes}
%
% Un autre demande souvent entendue \'etait: �~Comment
% puis-je dire \`a \LaTeX{} qu'il doit couper la premi\`ere
% colonne apr\`es cette ligne-l\`a~?~�. La commande
% |\pagebreak| (qui fonctionne avec l'option
% \texttt{twocolumn} de \LaTeX) ne sert \`a rien ici car elle
% terminerait la phase de collecte de \mc{} et donc toutes
% les colonnes sur cette page. Dans la version 1.5u la
% commande |\columnbreak| a donc \'et\'e ajout\'ee. Si elle est
% utilis\'ee \`a l'int\'erieur d'un paragraphe, elle marque
% la fin de la ligne courante comme point de coupure de
% colonne souhait\'e. Vous pouvez observer son effet sur la
% premi\`ere page\,\footnote{Dans le fichier original, la
% rupture de colonne \'etait r\'ealis\'ee \`a la fin de la
% premi\`ere colonne de la page 3. Cependant, pour respecter
% l'esprit du package \mc{}, cette rupture de page \`a
% \'et\'e d\'eplac\'ee dans le � de pr\'eface. NdT} la page
% o\`u quelques lignes de texte ont \'et\'e artificiellement
% forc\'ees dans la troisi\`eme colonne (ce qui a donn\'e un
% peu d'espace blanc entre les paragraphes dans la deuxi\`eme
% colonne).
%
% \subsection{\'El\'ements flottants \`a l'int\'erieur d'un environnement \mc{}}
%
% \`A l'int\'erieur de l'environnement \mc{}, les commandes
% habituelles d'\'el\'ements flottants sont disponibles (sous
% leur forme $*$) mais leur fonction est assez diff\'erente de
% celle qu'elles avaient dans le mode deux colonnes du
% \LaTeX{} standard. Les flottants $*$, comme {\tt figure*},
% indiquent des \'el\'ements flottants sur toute la largeur de la
% page qui sont trait\'es d'une mani\`ere analogue \`a celle des
% flottants normaux en dehors de l'environnement \mc{}.
% Cependant, il ne seront jamais imprim\'es sur la page o\`u ils
% sont rencontr\'es. En d'autres termes, on peut influencer
% leur positionnement en sp\'ecifiant une combinaison de {\tt
% t}, {\tt b} et/ou {\tt p} dans leur argument optionnel,
% mais {\tt h} ne fonctionne pas car le premier endroit
% possible est le haut de la page suivante. Il faut aussi
% noter que ceci signifie que le comportement de leur
% placement est d\'etermin\'e par les valeurs de |\topfraction|,
% etc. et non par |\dbl|...
%
% \subsection{Avertissements}
%
% Dans certaines circonstances, l'utilisation de
% l'environnement \mc{} peut provoquer quelques
% avertissements de la part de \TeX{} ou \LaTeX{}. Voici une
% liste des plus importants et des causes possibles~
% \begin{description}
%
% \item[] {\hspace*{-\labelsep}\tt Underfull \string\hbox\space
%  (badness ...)}
%
% Comme les colonnes sont souvent tr\`es \'etroites, \TeX{} n'a
% pas \'et\'e capable de trouver un bonne fa�on de couper le
% paragraphe en lignes. �~Underfull~� signale une ligne
% creuse mais tant que la laideur est en dessous de $10000$
% le r\'esultat est probablement acceptable.
%
% \item[]
% {\hspace*{-\labelsep}\tt Underfull \string\vbox\space ...  while
%  \string\output\space is active}
%
% Si une colonne contient un caract\`ere ayant une profondeur
% inhabituelle, par exemple une �~(~�, dans la ligne du bas,
% alors ce message peut appara\^\i{}tre. Il n'est en g\'en\'eral pas
% significatif tant que la valeur ne d\'epasse pas quelques
% points.
%
% \item[] {\hspace*{-\labelsep}\tt LaTeX Warning: I moved some lines
%  to the next page}
%
% Comme cela a \'et\'e dit plus haut, \mc{} perturbe parfois la
% num\'erotation des notes infrapaginales. Par pr\'ecaution,
% chaque fois qu'il y a une note sur une page pour laquelle
% \mc{} a d\^u laisser un reste pour la page suivante, cet
% avertissement appara\^\i{}t. V\'erifiez la num\'erotation des notes
% infrapaginales sur cette page. S'il se trouve qu'elle est
% incorrecte, il vous faudra couper manuellement la page en
% utilisant |\newpage| ou |\pagebreak[...]|.
%
% \item[] {\hspace*{-\labelsep}\tt Floats and marginpars not allowed
%  inside `multicols' environment!}
%
% Ce message appara\^\i{}t si vous essayez d'utiliser une commande
% |\marginpar| ou une forme sans \'etoile d'un environnement
% {\sf figure} ou {\sf table}. De tels \'el\'ements flottants
% dispara\^\i{}tront~!
%
% \end{description}
%
%
% \subsection{Tracer la sortie}
%
% Pour comprendre le raisonnement qu'il y a derri\`ere les
% d\'ecisions que prend \TeX{} lors du traitement d'un
% environnement \mc{}, un m\'ecanisme de tra\c{c}age est offert. Si
% vous donnez une valeur positive \meta{nombre} au compteur
% <<~\mc{}~>>, vous obtiendrez des informations de tra\c{c}age \`a la
% fois sur votre terminal et dans le fichier \texttt{.log}~
% \begin{description}
% \item [$\meta{nombre}=1$.] \TeX{} vous dira, chaque fois qu'il
% entre ou sort de l'environnement \mc{}, le nombre de
% colonnes sur lequel il travaille et sa d\'ecision s'il faut
% faire une nouvelle page avant ou apr\`es l'environnement~;
% \item [$\meta{nombre}=2$.] Dans ce cas, vous aurez en plus des
% informations issues de la routine d'\'equilibrage~:~les
% hauteurs essay\'ees pour les colonnes les plus \`a gauche et \`a
% droite, des informations sur le r\'etr\'ecissement si la
% d\'eclaration |\raggedcolumns| est effective ainsi que la
% valeur du compteur �~{\sf unbalance}~� si elle est
% positive.
% \item [$\meta{nombre}= 3$.] En donnant cette valeur au
% \meta{nombre}, l'algorithme de gestion de marques sera aussi
% trac\'e. Cela montrera quelles marques sont trouv\'ees, quelles
% marques sont prises en consid\'eration, etc. Pour comprendre
% pleinement ces informations, vous aurez probablement besoin
% de lire avec soin l'impl\'ementation.
% \item [$\meta{number}\geq 4$.] Donner une valeur aussi \'elev\'ee
% fera qu'en plus une |\hrule| sera plac\'ee dans la sortie,
% pour s\'eparer la partie de texte qui a d\'ej\`a \'et\'e consid\'er\'ee
% sur la page pr\'ec\'edente du reste. \'Evidemment cette valeur
% \emph{ne doit pas} \^etre utilis\'ee pour l'impression
% d\'efinitive. Elle activera aussi beaucoup plus de code de
% correction des erreurs pour la prise en compte des marques.
% \end{description}
%
%
% \end{multicols}
%
% \begin{multicols}{3}[\section{Pr\'eface des anciennes versions}
%                      \subsection{Preface de la version 1.4}]
%   \hbadness=10000
% \`A c\^ot\'e de la correction de bugs, comme indiqu\'e dans
% le fichier {\sf multicol.bug}, cette nouvelle version
% am\'eliore l'environnement en autorisant l'\'equilibrage
% dans des contextes arbitraires. Il est maintenant possible,
% par exemple, d'\'equilibrer un texte \`a l'int\'erieur d'un
% \mc{} ou d'une {\sf minipage} comme montr\'e en
% \ref{tab:newcmds}, o\`u un environnement \mc{} est
% utilis\'e \`a l'int\'erieur d'un environnement {\sf quote}.
% Maintenant, il est m\^eme possible d'imbriquer des
% environnements \mc{}.
%
% La seule restriction pour un tel environnement \mc{} int\'erieur
% (imbriqu\'e ou dans le mode vertical interne de \TeX{}) est
% que de telles variations produiront une bo\^\i{}te dont le
% mat\'eriel y sera \'equilibr\'e, et donc ne pourront \^etre
% \`a cheval sur plusieurs pages ou colonnes.
%
% De plus, j'ai r\'e\'ecrit l'algorithme d'\'equilibrage pour
% qu'il produise maintenant des r\'esultats l\'eg\`erement
% meilleurs.
%
% J'ai mis \`a jour la documentation mais excusez-moi par
% avance pour les parties <<~laiss\'ees de c\^ot\'e~>> qui
% sont pass\'ees \`a travers les corrections.
%
% Une note pour les utilisateurs qui d\'esirent am\'eliorer
% l'algorithme d'\'equilibrage de \mc{}~: la routine
% d'\'equilibrage est maintenant constitu\'ee par une seule
% macro appel\'ee |\balance@columns|. Cela signifie que l'on
% peut ais\'ement essayer diff\'erentes routine
% d'\'equilibrage en r\'e\'ecrivant cette macro. L'interface
% d'utilisation est expliqu\'ee dans le tableau
% \ref{tab:balance}. Il y a plusieurs am\'eliorations
% possibles, on peut penser \`a l'int\'egration de la
% fonction |\badness| de \TeX3, d\'efinir un algorithme plus
% rapide de recherche de la bonne hauteur de colonne, etc. Si
% quelqu'un pense qu'il/elle a un solution d'am\'elioration,
% je serais heureux de le savoir. Mais, s'il vous plait,
% respectez le copyright et de changez pas {\sf multicol.dtx}
% directement~!
%   \begin{table*}
%    \begin{quote}
%    \begin{multicols}{2}
%     \raggedcolumns
% La macro |\balance@columns| qui contient le code pour
% \'equilibrer le mat\'eriel rassembl\'e, est une macro sans
% param\`etres. Elle suppose que le mat\`eriel \`a \'equilibrer
% est stock\'e dans la bo\^\i{}te |\mult@box| qui est une
% |\vbox|. Elle <<~connait~>> aussi tous les param\`etres
% fix\'es par l'environnement \mc{}, comme |\col@number|,
% etc. Elle peut aussi supposer que |\@colroom| est
% l'espace encore disponible dans la page en cours.
%
% Quand elle a fini, elle doit retourner les colonnes
% individuelles dans des bo\^\i{}tes appropri\'ees pour la
% suite du traitement par |\page@sofar|. Cela signifie que la
% colonne gauche devra \^etre stock\'ee dans la bo\^\i{}te
% registre |\mult@gfirstbox|, la suivante dans le registre
% |\mult@firstbox|${}+2$, \ldots, seule la derni\`ere, par
% exception, dans le registre |\mult@grightbox|. En outre,
% elle doit fixer les deux macros |\kept@firstmark| et
% |\kept@botmark| pour r\'ecup\'erer les valeurs des marques
% initiales et du bas comme trouv\'ees dans les colonnes
% individuelles. Il y a quelques fonctions d'aide dans la
% section~\ref{sec:v14} qui peuvent \^etre utilis\'ees pour
% cela. Donner les marques correctes <<~\`a la main~>> n'est
% pas trivial et il peut \^etre n\'ecessaire de jeter un coup
% d'\oe{}il \`a la documentation et \`a l'impl\'ementation de
% |\balance@columns| avant d'essayer \`a nouveau.
%    \end{multicols}
%    \end{quote}
%    \vspace*{-3ex}
%    \vspace*{-0ex}
%    \caption{Description de l'interface de \cs{balance@columns}}
%    \label{tab:balance}
%   \end{table*}
% \end{multicols}
%
% \begin{multicols}{3}[\subsection{Pr\'eface de la version 1.2}]
%   \hbadness=10000
% Apr\`es que l'article sur l'environnement \mc{} a \'et\'e
% publi\'e dans \TUB\ 10\#3, j'ai re\c{c}u de nombreuses
% demandes pour ces macros. Quoi qu'il en soit, j'ai aussi
% re\c{c}u une version modifi\'ee de mon fichier style, en
% m\^eme temps qu'une lettre me demandant si je voulais
% inclure les changements pour donner de meilleurs
% r\'esultats de mise en forme des paragraphes dans les cas
% de lignes \'etroites. Les principales diff\'erences avec
% mon fichier style original \'etaient des param\`etres
% additionnels (comme |\multicoladjdemerits| \`a utiliser pour
% |\adjdemerits|, etc) qui pouvaient influencer l'algorithme
% de rupture de lignes.
%
% Mais, en fait, remettre \`a z\'ero de tels param\`etres, ou
% m\^eme pire leur donner une valeur n\'egative, ne donnera
% pas de meilleures coupures de lignes dans l'environnement
% \mc{}. L'algorithme de rupture de ligne de \TeX{} n'envisagera
% que les coupures de lignes potentielles qui peuvent
% \^etre r\'ealis\'ees sans induire une laideur sup\'erieure \`a la
% valeur courante de |\tolerance| (ou |\pretolerance| pour la
% premi\`ere passe). Si ce n'est pas possible, alors, en
% dernier ressort, \TeX{} produira des bo\^\i{}tes
% d\'ebordantes. Tous ces points de rupture possibles (et
% seulement ceux-la) seront pris en compte et, finalement,
% la s\'equence qui sera la moins p\'enalisante sera choisie.
% Cela signifie qu'une valeur de $-1000$ pour |\adjdemerits|
% enjoint \`a \TeX{} de pr\'ef\'erer des lignes visiblement
% incompatibles plut\^ot que de produire de meilleures
% ruptures de lignes.
%
% Cependant, avec \TeX{}~3.0, il est possible de produire des
% ruptures de lignes convenables m\^eme dans de petites
% colonnes, en fixant |\emergencystretch| \`a une valeur
% appropri\'ee. J'ai impl\'ement\'e une version qui est
% capable de fonctionner \`a la fois pour l'ancien et le
% nouveau \TeX{} (en fait, elle ignorera simplement la
% nouvelle caract\'eristique si elle n'est pas disponible).
% Le calcul de |\emergencystretch| est probablement
% incorrect. J'ai fait quelques tests, mais, bien s\^ur, on
% aura beaucoup plus d'exp\'erience avec les nouvelles
% possibilit\'es pour atteindre la qualit\'e maximum.
%
% La version~1.1a a une charmante <<~caract\'eristique~>>~:
% la punition pour l'utilisation des flottants interdits
% \'etait leur suppression d\'efinitive \`a partir de la
% |\@freelist| de \LaTeX{}, de telle sorte qu'apr\`es quelques
% |\marginpar| dans l'environnement \mc{}, les
% flottants \'etaient d\'esactiv\'es pour toujours (Merci \`a
% Chris~\bsc{Rowley} pour l'avoir remarqu\'e). J'ai
% enlev\'e ce dysfonctionnement et, en m\^eme temps,
% d\'ecid\'e de permettre qu'au moins les flottants
% puissent \^etre \`a cheval sur toutes les colonnes,
% c'est-\`a-dire, produits par l'environnement
% |figure*|. Vous pouvez voir cette nouvelle fonctionnalit\'e
% dans le tableau~\ref{tab:newcmds} qui a \'et\'e ins\'er\'e
% ici m\^eme.
%    \begin{table*}
%     \small
%     \setlength{\multicolsep}{0pt}
%     \begin{quote}
%      \begin{multicols}{2}
%         |\setemergencystretch|~: C'est un point d'entr\'ee pour
%         les utilisateurs qui aiment jouer finement avec les r\'eglages.
%         Il est suppos\'e
%         fixer le registre |\emergencystretch| \meta{dimen}
%         fourni par \TeX{}~3.0. Le premier argument est le
%         nombre de colonnes et le second la |\hsize|
%         courante. Pour l'instant, la d\'efinition par
%         d\'efaut est $4~\mbox{\tt pt} \times |#1|$,
%         c'est-\`a-dire que |\hsize| n'est pas utilis\'e
%         du tout. Mais, il y a peut-\^etre de meilleures
%         formules.
%
%         \setlength{\emergencystretch}{20pt}
%         |\set@floatcmds|~: C'est le point d'entr\'ee pour les experts
%         qui aiment impl\'ementer un m\'ecanisme complet de
%         traitement des flottants pour l'environnement
%         \mc{}. Le |@| dans le nom devrait signaler que ce
%         ne sera pas tr\`es facile.
%     \end{multicols}
%    \end{quote}
%    \vspace*{-1ex}
%    \vspace*{-0ex}
%    \caption[]{Les nouvelles commandes pour {\sf
%               multicol.sty}, version~1.2.
%               Les deux commandes pourraient \^etre
%               supprim\'ees si de bonnes solutions \`a ces
%               probl\`emes non r\'esolus sont trouv\'ees.
%               J'esp\`ere que ces commandes emp\^echeront que
%               des styles semblables d\'eriv\'es de celui-l\`a
%               ne se prom\`enent dans la nature.}
%    \label{tab:newcmds}
%    \end{table*}
% Cependant, des flottants dans les colonnes uniques
% sont toujours interdits et je ne pense pas que j'aurai le
% temps de m'attaquer \`a ce probl\`eme dans un proche
% avenir. \`A tous ceux qui veulent essayer~: attendez
% \TeX{}~3.0. Il a quelques caract\'eristiques qui rendront
% la vie plus facile dans un environnement multi-colonne.
% Toutefois, nous travaillons ici \`a la pointe des
% possibilit\'es de \TeX{}, et
% des solutions vraiment parfaites n\'ecessiteraient une
% approche diff\'erente de celle qui \'etait r\'ealis\'ee dans le
% {\it page\_builder}\,\footnote{c'est la partie de \TeX{} qui
% d\'ecoupe le texte courant en pages, se charge de
% l'insertion des flottants, des notes, de la composition des
% en-t\^etes, ...} de \TeX{}.
%
% Le texte ci-dessous est presque inchang\'e, j'ai seulement
% ajout\'e de la documentation aux endroits o\`u le nouveau code
% a \'et\'e ins\'er\'e.
% \end{multicols}
%
% \changes{v1.5l}{1996/01/13}{Fait de gros efforts pour
% expliquer les r\'ef\'erences irr\'esolues si
% \cs{OnlyDescription} est utilis\'e}
%
% \StopEventually{\PrintIndex \PrintChanges
%   \ifx\Finale\relax
%    \typeout{**********************************}
%    \typeout{* Info: Typesetting this document with
%                     \protect\OnlyDescription\space will}
%    \typeout{* Info: result in one unresolved
%                     reference to `sec:v14'.}
%    \typeout{* Info: --- tough, it's just not there in this case!}
%    \typeout{**********************************}
%   \fi
%  }
%
%
% \begin{multicols}{2}[\section{L'impl\'ementation}
%         Nous sortons maintenant du mode multi-colonne pour
%         montrer les capacit\'es de cet environnement (et
%         les mauvaises d\'ecisions de mise en page)
%      \subsection{Le fichier pilote de documentation}
%         ][10\baselineskip]
%
%   \hbadness=10000
%
%
% Le petit bout de code suivant contient le fichier pilote de
% documentation de \TeX{}, \emph{i.e.} le fichier qui
% produira la documentation que vous \^etes en train de lire.
% Il sera extrait de ce fichier par le programme {\tt
% docstrip}.
% Puisque c'est le premier code dans ce fichier, on peut
% produire la documentation simplement en ex\'ecutant
% \LaTeX{} sur le fichier \texttt{.dtx}.
%    \begin{macrocode}
%<*driver>
\documentclass{ltxdoc}
%    \end{macrocode}
% Nous utilisons l'option \texttt{balancingshow} quand nous
% chargeons \mc{} pour que le tra\c{c}age complet soit
% produit. Cela doit \^etre fait avant que le package \texttt{doc} ne
% soit charg\'e, puisque \texttt{doc} exige \mc{} sans aucune
% option.
%    \begin{macrocode}
\usepackage{multicol}[1999/05/25]
\usepackage{doc}
%    \end{macrocode}
% Tout d'abord, nous fixons la mise en page convenable pour
% cet article.
%    \begin{macrocode}
\setlength{\textwidth}{39pc}
\setlength{\textheight}{54pc}
\setlength{\parindent}{1em}
\setlength{\parskip}{0pt plus 1pt}
\setlength{\oddsidemargin}{0pc}
\setlength{\marginparwidth}{0pc}
\setlength{\topmargin}{-2.5pc}
\setlength{\headsep}{20pt}
\setlength{\columnsep}{1.5pc}
%    \end{macrocode}
% Nous voulons un filet entre les colonnes.
%    \begin{macrocode}
\setlength\columnseprule{.4pt}
%    \end{macrocode}
% Nous voulons aussi nous assurer que le nouvel environnement
% \mc{} trouve assez de place \`a la fin de la page.
%    \begin{macrocode}
\setlength\premulticols{6\baselineskip}
%    \end{macrocode}
% Lors de l'\'equilibrage des colonnes, nous ne tenons aucun compte
% des solutions trop mauvaises. De plus, si la derni\`ere
% colonne est trop mauvaise, nous la composons sans
% \'etirement.
%    \begin{macrocode}
\setcounter{columnbadness}{7000}
\setcounter{finalcolumnbadness}{7000}
%    \end{macrocode}
% L'index est cens\'e appara\^\i{}tre sur quatre colonnes. Et
% nous ne montrons pas les noms des macros dans la marges.
%    \begin{macrocode}
\setcounter{IndexColumns}{4}
\let\DescribeMacro\SpecialUsageIndex
\let\DescribeEnv\SpecialEnvIndex
\renewcommand\PrintMacroName[1]{}
\CodelineIndex
%\DisableCrossrefs           % Partial index
\RecordChanges               % Change log
%    \end{macrocode}
% Les num\'eros de lignes sont tr\`es petites pour cet
% article.
%    \begin{macrocode}
\renewcommand{\theCodelineNo}
  {\scriptsize\rm\arabic{CodelineNo}}
\settowidth\MacroIndent{\scriptsize\rm 00\ }
%</driver>
%    \end{macrocode}
% \iffalse
% Ajout pour la traduction:
%    \begin{macrocode}
%<*driver>
\usepackage[T1]{fontenc}
\usepackage[latin1]{inputenc}
\usepackage[frenchb]{babel}
%</driver>
%    \end{macrocode}
% \fi
%    \begin{macrocode}
%<*driver>
\begin{document}
  \typeout
   {****************************************
 ^^J* Expect some Under- and overfull boxes.
 ^^J****************************************}
   \DocInput{f-multicol.dtx}
\end{document}
%</driver>
%    \end{macrocode}
%
%
% \end{multicols}
%
% \begin{multicols}{2}[\subsection{Identification et
%                      ex\'ecution des options}]
%
%
% Nous commen\c{c}ons par identifier le package. Puisque son
% utilisation requiert des caract\'eristiques seulement
% disponibles dans \LaTeXe{}, nous nous assurons que ce format
% est disponible (maintenant, cela est fait plus t\^ot dans
% le fichier).
%    \begin{macrocode}
%<*package>
% \NeedsTeXFormat{LaTeX2e}
% \ProvidesPackage{multicol}[..../../..
%    v... multicolum formatting]
%    \end{macrocode}
%
%^^A \subsection{Ex\'ecution des options}
%
% Ensuite, nous d\'eclarons les options prises en charge par
% \mc{}. Le mode \texttt{twocolumn} et \mc{} ne fonctionnent
% pas ensemble, donc nous mettons en garde contre des
% probl\`emes possibles. Quoi qu'il en soit, puisque nous
% pouvons revenir au mode |\onecolumn| dans lequel \mc{}
% fonctionne, nous n'entra\^\i{}nons pas d'erreur.
%    \begin{macrocode}
\DeclareOption{twocolumn}
   {\PackageWarning{multicol}{May not work
             with the twocolumn option}}
%    \end{macrocode}
% Le tra\c{c}age est fait grace \`a un compteur. Cependant,
% il est \'egalement possible d'appeler le tra\c{c}age en
% utilisant les options d\'eclar\'ees ci-dessous.
%    \begin{macrocode}
\newcount\c@tracingmulticols
\DeclareOption{errorshow}
    {\c@tracingmulticols\z@}
\DeclareOption{infoshow}
    {\c@tracingmulticols\@ne}
\DeclareOption{balancingshow}
    {\c@tracingmulticols\tw@}
\DeclareOption{markshow}
    {\c@tracingmulticols\thr@@}
\DeclareOption{debugshow}
    {\c@tracingmulticols5\relax}
\ProcessOptions
%    \end{macrocode}
%
%
% \end{multicols}
%
% \begin{multicols}{2}[\subsection{Ouvrir et
%                      fermer l'environnement \mc{} }]
%
% \begin{macro}{\multicols}
% Comme pr\'ecis\'e pr\'ec\'edemment, l'environnement \mc{} a
% un argument obligatoire (le nombre de colonnes) et jusqu'\`a
% deux arguments optionnels. Nous commen\c{c}ons par lire le
% nombre de colonnes dans le registre |\col@number|.
%    \begin{macrocode}
\def\multicols#1{\col@number#1\relax
%    \end{macrocode}
% Si l'utilisateur oublie l'argument, \TeX{} se plaindra de
% l'absence d'un nombre \`a ce point. Le m\'ecanisme de
% r\'eparation d'erreur utilisera ensuite z\'ero, qui n'est
% pas un bon choix dans ce cas. Donc, nous devons maintenant
% tester si tout est correct. Le minimum est de deux colonnes
% pour le moment.
% \changes{v1.3b}{1990/10/09}{Minimum de deux colonnes}
%    \begin{macrocode}
  \ifnum\col@number<\tw@
     \PackageWarning{multicol}%
      {Using `\number\col@number'
       columns doesn't seem a good idea.^^J
       I therefore use two columns instead}%
     \col@number\tw@ \fi
%    \end{macrocode}
% Nous avons seulement assez de registres de bo\^\i{}te pour
% dix colonnes, donc nous avons besoin de v\'erifier que
% l'utilisateur n'en demande pas plus.
% \changes{v1.4k}{1992/06/27}{Maximum de 5 colonnes (temporaire)}
% \changes{v1.5a}{1992/11/04}{Permet \`a nouveau 10 colonnes}
%    \begin{macrocode}
  \ifnum\col@number>10
     \PackageError{multicol}%
      {Too many columns}%
      {Current implementation doesn't
       support more than 10 columns.%
       \MessageBreak
       I therefore use 10 columns instead}%
     \col@number10 \fi
%    \end{macrocode}
% \`A l'int\'erieur de l'environnement, nous avons besoin
% d'une version sp\'eciale de la commande fondamentale |\@footnotetext|
% puisque l'original d\'efinit |\hsize| pour |\columnwidth|
% ce qui n'est pas correct dans l'environnement \mc.
% Ici, |\columnwidth| se r\'ef\`ere \`a la largeur d'une
% colonne unique et la note de bas de page doit \^etre de
% longueur |\textwidth|. Puisque |\@footnotetext| a une
% d\'efinition diff\'erente dans l'environnement minipage,
% nous ne le red\'efinissons pas directement. \`A la
% place, nous fixons localement |\columnwidth| \`a |\textwidth|
% et appelons la d\'efinition originale (courante) stock\'ee
% dans |\orig@footnotetext|.
% \changes{v1.5p}{1997/12/14}{Red\'efinition de \cs{@footnotetext}
%                           seulement \`a l'int\'erieur de l'env. rapport pr/2689.}
% \changes{v1.5r}{1998/08/17}{Utilisation de \cs{@footnotetext}
%                            mais avec un changement local
%                            de \cs{columnwidth}.}
% \changes{v1.5w}{1999/10/21}{\cs{@footnotetext} rendu plus
%                             long pour permettre des notes de bas de page
%                             multi-paragraphe.}
%    \begin{macrocode}
     \let\orig@footnotetext\@footnotetext
     \long\def\@footnotetext##1{\begingroup
         \columnwidth\textwidth
         \orig@footnotetext{##1}\endgroup}%
%    \end{macrocode}
% Maintenant, nous pouvons rechercher, en toute s\'ecurit\'e,
% les arguments optionnels.
%    \begin{macrocode}
  \@ifnextchar[\mult@cols{\mult@cols[]}}
%    \end{macrocode}
% \end{macro}
%
%
% \begin{macro}{\mult@cols}
% La macro |\mult@cols| saisit le premier argument optionnel
% (s'il existe) et recherche le second.
%    \begin{macrocode}
\def\mult@cols[#1]{\@ifnextchar[%
%    \end{macrocode}
% Cet argument doit \^etre une \meta{dimen} indiquant
% l'espace libre minimum n\'ecessaire sur la page courante
% pour commencer l'environnement. Si l'utilisateur n'en
% fournit pas, nous utilisons |\premulticols| par d\'efaut.
%    \begin{macrocode}
  {\mult@@cols{#1}}%
  {\mult@@cols{#1}[\premulticols]}}
%    \end{macrocode}
% \end{macro}
%
%
% \begin{macro}{\mult@@cols}
% Apr\`es avoir enlev\'e tous les arguments de l'entr\'ee, nous
% sommes capables de commencer avec |\mult@@cols|.
%    \begin{macrocode}
\def\mult@@cols#1[#2]{%
%    \end{macrocode}
% La premi\`ere chose que nous faisons est de d\'ecider si,
% oui ou non, il y a un environnement multicol non li\'e,
% \emph{i.e.} \`a cheval sur plusieurs pages ou compos\'e
% dans une bo\^\i{}te. Si nous sommes en mode
% <<~int\'erieur~>> de \TeX{} (par exemple, d\'ej\`a dans une
% bo\^\i{}te), alors nous avons une version embo\^\i{}t\'ee
% de multicol et par cons\'equent, nous fixons
% |@boxedmulticols| \`a vrai. \mc{} doit d\'emarrer en mode
% vertical. Si nous n'y sommes pas d\'ej\`a, nous le
% for\c{c}ons maintenant avec |\par| puisque, autrement, le
% test de mode <<~int\'erieur~>> ne sera pas fait pas si nous
% sommes dans une bo\^\i{}te.
% \changes{v1.4f}{1992/04/28}{\cs{par} ajout\'e pour
%                            autoriser correctement les tests internes}
%    \begin{macrocode}
  \par
  \ifinner \@boxedmulticolstrue
%    \end{macrocode}
% Autrement, nous v\'erifions |\doublecol@number|. Ce compteur est
% \`a z\'ero \`a l'ext\'erieur de l'environnement
% multicol mais positif \`a l'int\'erieur (cela arrive un
% petit peu plus tard). Dans le second cas, nous avons besoin
% de composer les multicolonnes courantes aussi en <<~mode
% embo\^\i{}t\'e~>> et donc de changer le commutateur en
% cons\'equence.
%    \begin{macrocode}
  \else
    \ifnum \doublecol@number>\z@
       \@boxedmulticolstrue
    \fi
  \fi
%    \end{macrocode}
% Puis nous regardons si des statistiques sont demand\'ees~:
%    \begin{macrocode}
  \mult@info\z@
      {Starting environment with
       \the\col@number\space columns%
%    \end{macrocode}
% En mode embo\^\i{}t\'e, nous ajoutons quelques info.
% suppl\'ementaires.
% \changes{v1.4f}{1992/04/28}{\cs{on@line} ajout\'e pour les info. de tra\c{c}age}
%    \begin{macrocode}
        \if@boxedmulticols\MessageBreak
           (boxed mode)\fi
      }%
%    \end{macrocode}
% Puis nous mesurons la page courante pour voir si une partie
% utilisable de l'environnement multicol peut \^etre
% compos\'ee. Cette routine peut \'ventuellement commencer
% une nouvelle page.
% \changes{v1.4a}{1992/02/14}{Ajout des accolades oubli\'ees}
%    \begin{macrocode}
   \enough@room{#2}%
%    \end{macrocode}
% Maintenant nous sortons le premier argument et produisons
% un espace vertical au-dessus des colonnes (Notez que cet
% argument correspond au premier argument optionnel de
% l'environnement {\sf multicols}). Dans plusieurs versions,
% cet argument \'etait compos\'e dans un groupe pour donner
% un effet similaire \`a celui de |\twocolumn[..]| dans
% lequel l'argument est aussi implicitement entour\'e de
% crochets. Cependant, ceci entre en conflit avec des changements
% locaux r\'ealis\'es par des instructions comme les commandes de
% sectionnement (qui rendent compte de la plupart des commandes
% utilis\'ees dans cet argument), g\^achant l'espacement
% vertical, etc., plus tard dans le document, de telle sorte qu'\`a partir
% de la version v1.5q, cet argument est de nouveau compos\'e dans un
% niveau ext\`erieur.
% \changes{v1.4e}{1992/03/16}{Arg. optionnel compos\'e dans le groupe}
% \changes{v1.5q}{1998/01/19}{Et, de nouveau, suppression du groupe six ans plus tard}
%    \begin{macrocode}
   #1\par\addvspace\multicolsep
%    \end{macrocode}
% Nous commen\c{c}ons un nouveau niveau de groupement pour
% cacher tous les changements suivants (r\'ealis\'es dans |\prepare@multicols|
% par exemple).
%    \begin{macrocode}
   \begingroup
     \prepare@multicols
%    \end{macrocode}
% Si nous sommes en mode embo\^\i{}t\'e, nous ouvrons
% maintenant une bo\^\i{}te pour y composer tout le mat\'eriel
% \`a partir du corps multicolonne, sinon nous poursuivons
% simplement.
% \changes{v1.4g}{1992/05/07}{\cs{global} est probablement mauvais mais au
%                             moins inutile}
%    \begin{macrocode}
     \if@boxedmulticols
       \setbox\mult@box\vbox\bgroup
%    \end{macrocode}
% \changes{v1.5?}{1994/?/?}{Penalit\'e d\'eplac\'ee plus loin}
% Ici, nous avons peut-\^etre \`a r\'einitialiser des
% param\`etres ; peut-\^etre |\@parboxrestore| serait la bonne
% action, mais nous le laissons, pour le moment.
% \changes{v1.4l}{1992/08/17}{\cs{@totalleftmargin} maintenant dans \cs{prepare@multicols}}
%    \begin{macrocode}
     \fi
%    \end{macrocode}
% Nous finissons par supprimer les espaces initiaux.
%    \begin{macrocode}
     \ignorespaces}
%    \end{macrocode}
% \end{macro}
%
%  \begin{macro}{\if@boxedmulticols}
% Ici est le commutateur et la bo\^\i{}te pour le code des
% multicolonnes <<~embo\^\i{}t\'ees~>>.
%    \begin{macrocode}
\newif\if@boxedmulticols
\@boxedmulticolsfalse
\newbox\mult@box
%    \end{macrocode}
%  \end{macro}
%
% \begin{macro}{\enough@room}
% \changes{v1.0c}{1989/05/12}{Penalit\'e 0 ajout\'ee pour vider la liste de
%                            contribution.}
% La macro |\enough@room| utilis\'ee ci-dessous n'est pas
% parfaite mais fonctionne raisonnablement bien dans ce
% contexte. Nous mesurons l'espace libre sur la page courante
% en soustrayant |\pagetotal| de |\pagegoal|. Ce n'est pas
% totalement correct puisque il ne prend pas en compte le
% <<~retr\'ecissement~>> (\emph{i.e.} |\pageshrink|). La
% <<~r\'ecente liste de contribution~>> pourrait \^etre
% non-vide, donc nous commen\c{c}ons avec |\par| et une
% p\'enalit\'e implicite\,\footnote{cf. la documentation
% de \cs{endmulticols} pour plus de d\'etails}.
% En fait, nous utilisons |\addpenalty| pour nous assurer
% qu'un  |\addvspace| suivant <<~verra~>> l'espace vertical
% qui pourrait \^etre pr\'esent.
% L'utilisation de |\addpenalty| aura pour cons\'equence que
% tous les items, provenant des contributions r\'ecentes,
% seront d\'eplac\'es dans la liste verticale principale et
% la valeur de |\pagetotal| sera correctement mise \`a jour.
% Cependant, la p\'enalit\'e sera plac\'ee devant tout morceau
% de glue avec pour r\'esultat que la liste verticale
% principale sera d\'ej\`a d\'ebordante m\^eme si \TeX{}
% n'appelle pas la routine de sortie.
% \changes{v1.3b}{1990/10/09}{R\'ealise \cs{penalty} avec \cs{addpenalty}}
% \changes{v1.4e}{1992/03/16}{Mais ignore \cs{@nobreak} dans \cs{addpenalty}}
%    \begin{macrocode}
\def\enough@room#1{%
%    \end{macrocode}
% Mesurer a uniquement un sens quand nous ne sommes pas en
% <<~mode embo\^\i{}t\'e~>>, donc la routine ne fait rien si
% le commutateur est \`a vrai.
%    \begin{macrocode}
   \if@boxedmulticols\else
   \par
%    \end{macrocode}
%    \label{mac:enoughroom}
% Pour vider la liste de contribution, la premi\`ere version
% contenait une p\'enalit\'e z\'ero, mais cela avait pour
% cons\'equence que |\addvspace| ne pouvait d\'etecter la
% glue pr\'ec\'edente. Donc, ceci a \'et\'e chang\'e en |\addpenalty|.
% Cela s'est av\'er\'e cependant insuffisant puisque |\addpenalty|
% n'ajoutera pas de p\'enalit\'e quand |@nobreak| est vrai.
% Donc, nous for\c{c}ons localement ce commutateur \`a faux.
% Il peut en r\'esulter une rupture entre le texte
% pr\'ec\'edent et le d\'ebut de l'environnement
% multicol, mais cela semble acceptable puisqu'il y a un
% argument optionnel pr\'ecis\'ement pour cette raison.
%    \begin{macrocode}
   \bgroup\@nobreakfalse\addpenalty\z@\egroup
   \page@free \pagegoal
   \advance \page@free -\pagetotal
%    \end{macrocode}
% Pour \^etre capable de sortir la valeur, nous avons besoin
% de l'assigner pr\'ealablement \`a un registre puisqu'elle
% pourra\^\i{}t \^etre un registre (par d\'efaut), auquel cas
% il nous faudrait utiliser |\the|, ou elle
% pourra\^\i{}t \^etre une valeur \`a part enti\`ere et alors
% utiliser |\the| serait mauvais.
% \changes{v1.5e}{1994/05/26}{Assigne arg. pour rendre le registre
%                             capable de sortir la valeur}
%    \begin{macrocode}
     \@tempskipa#1\relax
%    \end{macrocode}
% Maintenant nous testons si les informations de tra\c{c}age
% sont demand\'ees~:
%    \begin{macrocode}
   \mult@info\z@
       {Current page:\MessageBreak
        height=%
        \the\pagegoal: used \the\pagetotal
        \space -> free=\the\page@free
        \MessageBreak
        needed \the\@tempskipa
              \space(for #1)}%
%    \end{macrocode}
% Notre derni\`ere action est de forcer un saut de page s'il
% ne reste pas assez de place.
%    \begin{macrocode}
   \ifdim \page@free <#1\newpage \fi
  \fi}
%    \end{macrocode}
% \end{macro}
%
% \begin{macro}{\prepare@multicols}
% Lors de la pr\'eparation de la sortie multicolonne,
% plusieurs choses sont \`a faire.
%    \begin{macrocode}
\def\prepare@multicols{%
%    \end{macrocode}
% Nous commen\c{c}ons par sauver la valeur courante de
% |\@totalleftmargin| et puis remettons \`a z\'ero |\parshape|
% dans le cas o\`u nous sommes \`a l'int\'erieur de quelque
% environnement de liste. L'indentation convenable pour
% l'environnement \mc{} dans un tel cas sera produite par le
% d\'eplacement du r\'esultat vers la droite par la commande
%  |\multicol@leftmargin|. Si nous voulions
%  utiliser directement la valeur de |\@totalleftmargin|
%  alors les listes \`a l'int\'erieur d'un environnement
%  \mc{} pourra\^\i{}ent provoquer un d\'ecalage de la sortie.
% \changes{v1.4l}{1992/08/17}{sauvegarde de \cs{@totalleftmargin}}
%    \begin{macrocode}
  \multicol@leftmargin\@totalleftmargin
  \@totalleftmargin\z@
  \parshape\z@
%    \end{macrocode}
% Nous d\'efinissons \'egalement le registre |\doublecol@number|
% pour un usage ult\'erieur. Ce registre doit contenir $2\times |\col@number|$.
% Il indique aussi que nous sommes dans un environnement
% \mc{} comme pr\'ecis\'e ci-dessus.
% \changes{v1.5a}{1992/11/04}{Ajout d'offset \`a \cs{doublecolnumber}}
%    \begin{macrocode}
  \doublecol@number\col@number
  \multiply\doublecol@number\tw@
  \advance\doublecol@number\mult@rightbox
%    \end{macrocode}
%
%    \begin{macrocode}
  \if@boxedmulticols
    \let\l@kept@firstmark\kept@firstmark
    \let\l@kept@botmark\kept@botmark
    \global\let\kept@firstmark\@empty
    \global\let\kept@botmark\@empty
  \else
%    \end{macrocode}
% Nous ajoutons une bo\^\i{}te vide \`a la liste verticale
% principale pour nous assurer que nous prenons toutes les
% insertions (remises plus tard ou ins\'er\'ees en haut de la
% page). Sinon, il pourrait arriver que |\eject| soit
% supprim\'e sans appel de la routine de sortie. Dans cette
% routine de sortie, nous enlevons de nouveau cette
% bo\^\i{}te. De nouveau, le code est appliqu\'e seulement si
% nous sommes dans la liste verticale principale et non dans
% une bo\^\i{}te.
% Cependant, ce n'est pas suffisant pour annuler
% l'interligne, nous devons aussi effacer |\topskip| avant
% d'ajouter cette bo\^\i{}te, puisque |\topskip| est toujours
% ins\'er\'e avant la premi\`ere bo\^\i{}te d'une page, ce qui
% pourrait nous laisser avec un espace suppl\'ementaire de |\topskip|
% si \mc{} commence sur une nouvelle feuille.
% \changes{v1.3c}{1991/03/03}{\cs{null} ins\'er\'e et enlev\'e pour la sortie}
% \changes{v1.4a}{1992/02/11}{Code conditionnel pour l'ajout du mode d'embo\^\i{}tage.}
% \changes{v1.4o}{1992/11/22}{\cs{topskip} remis localement \`a z\'ero.}
%    \begin{macrocode}
    \nointerlineskip {\topskip\z@\null}%
    \output{%
      \global\setbox\partial@page\vbox
        {%
%    \end{macrocode}
% Maintenant, nous devons nous assurer que nous prenons en
% compte une situation particuli\`ere qui pourrait causer de
% la perte de texte~! Si l'utilisateur a une \'enorme
% quantit\'e de mat\'eriel vertical dans le premier argument
% optionnel, qui est plus grand que |\premulticols| et que
% nous sommes pr\`es du bas de page, alors il peut arriver
% que ce ne soit pas |\eject| qui d\'eclenche cette routine
% sp\'eciale de sortie, mais plut\^ot la liste verticale
% principale d\'ebordante. Dans ce cas, nous donnons un autre
% point de rupture \`a travers la p\'enalit\'e de |\eject|.
% Par cons\'equent, cette routine sp\'eciale de sortie peut
% \^etre appel\'ee deux fois et le contenu de |\partial@page|,
% \emph{i.e.} le mat\'eriel avant l'environnement \mc{},
% est perdu. Il y a plusieurs solutions pour
% contourner ce probl\`eme, mais pour l'instant, nous
% le d\'etecterons simplement et informerons l'utilisateur
% qu'il/elle doit \'elargir |\premulticols| en utilisant une
% valeur appropri\'ee pour le second argument.
% \changes{v1.4a}{1992/02/11}{V\'erification des pertes de texte.}
%    \begin{macrocode}
%<*check>
         \ifvoid\partial@page\else
           \PackageError{multicol}%
            {Error saving partial page}%
            {The part of the page before
             the multicols environment was
             nearly full with^^Jthe result
             that starting the environment
             will produce an overfull
             page.  Some^^Jtext may be lost!
             Please increase \premulticols
             either generally or for this%
             ^^Jenvironment by specifying a
             suitable value in the second
             optional argument to^^Jthe
             multicols environment.}
           \unvbox\partial@page
           \box\last@line
         \fi
%</check>
         \unvbox\@cclv
         \global\setbox\last@line\lastbox
        }%
%    \end{macrocode}
% Finalement, nous avons besoin d'enregistrer les balises qui
% sont toujours pr\'esentes \`a l'int\'erieur de |\partial@page|
% afin de pouvoir construire plus tard des balises
% correctes de d\'ebut et de fin. Cela est r\'ealis\'e par
% le code suivant.
% \changes{v1.4a}{1992/02/14}{Initialisation des marques conserv\'ees}
%    \begin{macrocode}
           \prep@keptmarks
%    \end{macrocode}
% Finalement, nous devons initialiser |\kept@topmark| qui,
% dans l'id\'eal, devrait \^etre initialis\'e avec la balise
% qui est en cours en <<~haut~>> de la page. Malheureusement,
% nous ne pouvons utiliser |\topmark| parce que ce registre
% ne contiendra pas toujours ce que son nom veut dire, parce
% que, parfois, \LaTeX{} appelle la routine de sortie pour la
% gestion des flottants\,\footnote{pendant un tel appel,
% \cs{botmark} est copi\'e globalement dans \cs{topmark} par
% le programme \TeX{}}. Donc, nous utilisons le deuxi\`eme
% meilleure solution, l'initialisant avec |\firstmark|.
% En fait, dans notre optique, cela n'a vraiment pas
% d'importance, puisque nous utilisons |\kept@topmark|
% uniquement pour initialiser |\firstmark| et |\botmark| pour
% la page suivante, si nous n'avons pas trouv\'e de balise
% pour la page en cours.
% \changes{v1.4i}{1992/06/18}{\cs{kept@topmark} initialis\'e.}
%    \begin{macrocode}
           \global\let\kept@topmark\firstmark
          }\eject
%    \end{macrocode}
% La chose suivante \`a faire est d'assigner une nouvelle
% valeur \`a |\vsize|. \LaTeX{} conserve la place libre de la
% page (\emph{i.e.} la hauteur de page sans l'espace pour les
% flottants d\'ej\`a pris en compte) dans le registre |\@colroom|.
% Nous devons soustraire la hauteur de |\partial@page| pour
% mettre la v\'eritable place libre dans cette variable.
%    \begin{macrocode}
    \advance\@colroom-\ht\partial@page
%    \end{macrocode}
% Puis, nous devons calculer la valeur de |\vsize| \`a
% utiliser pendant l'assemblage des colonnes. |\set@mult@vsize|
% prend un argument qui permet les d\'efinitions locales (|\relax|)
% ou globales (|\global|). La derni\`ere possibilit\'e est
% utilis\'ee \`a l'int\'erieur de la routine de sortie
% ci-dessous. \`A cet instant, nous devons r\'ealiser
% un changement local pour |\vsize| parce que nous voulons
% restaurer la valeur originale de |\vsize| dans le
% cas o\`u l'environnement \mc{} finit sur la m\^eme page que
% celle sur laquelle il a commenc\'e.
% \changes{v1.4p}{1992/11/26}{Utilisation de diff\'erentes valeurs de \cs{vsize}}
% \changes{v1.4p}{1992/11/26}{Code d\'eplac\'e pour \cs{set@mult@vsize}}
%    \begin{macrocode}
    \set@mult@vsize\relax
%    \end{macrocode}
% Maintenant nous passons \`a une nouvelle routine |\output|
% qui sera utilis\'ee pour rassembler le contenu des
% colonnes.
%    \begin{macrocode}
    \output{\multi@column@out}%
%    \end{macrocode}
% Finalement, nous g\'erons les insertions des notes de bas
% de page. Nous devons multiplier le facteur
% d'agrandissement et le saut suppl\'ementaire par le
% nombre de colonnes puisque chaque note de bas de page
% r\'eduit l'espace de chaque colonne (rappelez-vous que
% les notes de bas de page s'\'etendent sur toute la largeur
% de la page).
% Si, d'autre part, les notes sont compos\'ees tout \`a la
% fin du document, notre sch\'ema fonctionnera encore,
% puisque |\count\footins| est \'egal \`a z\'ero et, donc, il
% ne changera pas.
% Pour permettre une personnalisation avanc\'ee, la
% d\'efinition des param\`etres |\footins| est r\'ealis\'ee
% dans une macro s\'epar\'ee.
% \changes{v1.5?}{1994/?/?}{Utilisation de \cs{init@mult@footins}}
%    \begin{macrocode}
    \init@mult@footins
%    \end{macrocode}
% Pour la m\^eme raison (notes de la largeur de la page), le
% registre \meta{dimen} contr\^olant l'espace maximum
% utilis\'e pour les notes de bas de page n'est pas modifi\'e.
% Ayant fait cela, nous
% devons r\'e-ins\'erer toutes les notes de bas de page qui
% sont d\'ej\`a pr\'esentes (\emph{i.e.} celles rencontr\'ees
% quand le mat\'eriel sauv\'e dans |\partial@page| a \'et\'e
% trait\'e en premier). Cela r\'eduira l'espace libre
% (\emph{i.e.} |\pagetotal|) de la quantit\'e appropri\'ee
% puisque nous avons chang\'e le facteur d'agrandissement,
% etc., ci-dessus.
%    \begin{macrocode}
    \reinsert@footnotes
%    \end{macrocode}
% Tout le code ci-dessus \'etait seulement n\'ecessaire pour
% la version de \mc{} non-restreinte, \emph{i.e.} celle qui
% permet les ruptures de page. Si nous sommes \`a
% l'int\'erieur d'une bo\^\i{}te, il n'y a pas de raison de
% d\'efinir les routines sp\'eciales de sortie, ou  |\vsize|, etc.
%    \begin{macrocode}
  \fi
%    \end{macrocode}
% Mais maintenant, nous en venons au code qui est
% n\'ecessaire dans tous les cas. Nous assignons de nouvelles
% valeurs \`a |\vbadness|, |\hbadness| et |\tolerance|
% puisque il est assez dur pour \TeX{} de produire de
% <<~bons~>> paragraphes \`a l'int\'erieur de colonnes
% \'etroites.
% \changes{v1.2a}{1990/02/05}{\cs{vbadness} 10001 maintenant.}
%    \begin{macrocode}
  \vbadness\@Mi \hbadness5000
  \tolerance\multicoltolerance
%    \end{macrocode}
% Puisque la premi\`ere passe \'echouera presque toujours,
% nous l'ignorons et demandons \`a
% \TeX{} de r\'ealiser les c\'esures directement. En fait,
% maintenant nous utilisons un autre registre pour garder la
% valeur pour la pr\'e-tol\'erance de multicol, afin qu'un
% concepteur puisse utiliser |\pretolerance|.
% \changes{v1.2a}{1990/02/05}{\cs{pretolerance} -1 car \c{c}a ne r\'eussit
%                            presque jamais.}
% \changes{v1.4e}{1992/03/20}{Using}
%    \begin{macrocode}
  \pretolerance\multicolpretolerance
%    \end{macrocode}
% Pour l'utilisation avec le nouveau \TeX{}, nous
% fixons |\emergencystretch| \`a  $|\col@number| \times$ 4~pt.
% Cependant, c'est seulement une estimation, donc pour
% l'instant cela est r\'ealis\'e dans une macro,
% |\setemergencystretch| qui prend la valeur courante de
% |\hsize| et le nombre de colonnes comme argument. Donc, les
% utilisateurs sont capables de construire leur propre
% formule.
% \changes{v1.2a}{1990/02/05}{\cs{setemergencystretch} ajout\'e.}
%    \begin{macrocode}
  \setemergencystretch\col@number\hsize
%    \end{macrocode}
% Un autre point d'entr\'ee, pour permettre aux utilisateurs
% d'ajouter leurs propres extensions sans faire un nouveau
% package, est |\set@floatcmds| qui g\`erent toutes les
% red\'efinitions des commandes de flottant internes \`a
% \LaTeX{} pour fonctionner avec l'environnement \mc{}. Pour
% le moment, c'est seulement utilis\'e pour red\'efinir |\@dblfloat|
% et |\end@dblfloat|.
% \changes{v1.2a}{1990/02/05}{\cs{set@floatcmds} ajout\'e.}
%    \begin{macrocode}
  \set@floatcmds
%    \end{macrocode}
% De plus, nous augmentons |\baselineskip| de |\multicolbaselineskip|
% pour permettre les corrections des colonnes \'etroites.
%    \begin{macrocode}
  \advance\baselineskip\multicolbaselineskip
%    \end{macrocode}
% \changes{v1.0e}{1989/06/21}{\cs{textwidth} chang\'e en \cs{linewidth}.}
% \changes{v1.0e}{1989/06/21}{Pour que ce fichier puisse fonctionner avec la
%                              commande <<~twocolumn~>>.}
%    La valeur |\hsize| des colonnes est donn\'ee par la formule~:
%    \[
%       { |\linewidth| - (|\col@number|-1)
%                             \times
%                             |\columnsep|
%        \over
%        |\col@number|}
%    \]
% La formule ci-dessus a chang\'e de version en version.
% Maintenant, nous commen\c{c}ons avec la valeur courante de |\linewidth|
% afin que la largeur de la colonne soit correctement
% calcul\'ee quand nous sommes dans une minipage, dans une
% liste ou dans tout autre environnement.
% Cela sera accompli par~:
%    \begin{macrocode}
  \hsize\linewidth \advance\hsize\columnsep
  \advance\hsize-\col@number\columnsep
  \divide\hsize\col@number
%    \end{macrocode}
% Nous fixons \'egalement |\linewidth| et |\columnwidth| \`a
% |\hsize|. Dans le pass\'e,  |\columnwidth| avait \'et\'e
% laiss\'e tel quel. C'est incoh�rent mais |\columnwidth| est utilis\'e
% seulement par les flottants (qui ne sont pas permis dans
% leur impl\'ementation courante) et par la macro
% |\footnote|. Puisque nous voulons des notes de bas de page
% de la largeur de la page\,\footnote{Je ne suis pas s\^ur de
% vouloir des notes de la largeur de la page, mais
% l'\'equilibrage de la derni\`ere page ne peut \^etre
% accompli qu'avec cette approche ou avec un algorithme
% <<~multi-path~>> qui est compliqu\'e et lent. C'est toutefois un
% d\'efi pour qui veut me d\'emontrer que c'est mauvais~!
% Une autre possibilit\'e est de r\'eimpl\'ementer une petite partie
% de la proc\'edure {\it fire\_up\/} de \TeX{} (le
% programme). Je pense que c'est la meilleure solution si
% vous \^etes int\'eress\'es par la composition complexe de pages,
% mais cela a pour inconv\'enient que le programme
% r\'esultant ne peut plus \^etre appel\'e \TeX.}, cette
% simple astuce nous \'evite de r\'e-\'ecrire les macros
% |\footnote|. Cependant, certaines applications se
% r\'ef\`erent \`a |\columnwidth| comme <<~la largeur de la
% colonne courante~>> pour composer les affichages (le
% package \texttt{amsmath}, par exemple), et pour permettre
% l'utilisation de telles applications en m\^eme temps que
% \texttt{multicol}, cela a \'et\'e maintenant modifi\'e.
%
% Avant de remplacer |\linewidth| par sa nouvelle valeur, nous
% enregistrons son ancienne valeur dans un registre
% appel\'e |\full@width|. Cette valeur est utilis\'ee plus
% tard, quand nous rassemblons toutes les colonnes.
% \changes{v1.0e}{1989/06/21}{Fixation de \cs{columnwidth} supprim\'ee.}
% \changes{v1.5o}{1997/11/16}{Fixation de \cs{columnwidth}
%                             ajout\'ee de nouveau. Rapport pr/2664.}
%    \begin{macrocode}
  \full@width\linewidth
  \linewidth\hsize
  \columnwidth\hsize
}
%    \end{macrocode}
% \end{macro}
%
%  \begin{macro}{\init@mult@footins}
% Cette macro est utilis\'ee pour fixer les param\`etres
% associ\'es avec les notes (de bas de page) flottantes. Elle
% peut \^etre red\'efinie par des applications qui demandent
% des quantit\'es d'espace diff\'erentes lors de la composition
% des notes.
%    \begin{macrocode}
\def\init@mult@footins{%
    \multiply\count\footins\col@number
    \multiply\skip \footins\col@number
}
%    \end{macrocode}
% \end{macro}
%
%  \begin{macro}{\set@mult@vsize}
%
% Puisque nous avons fix\'e les colonnes |\col@umber| sur une
% page, chacune avec une hauteur de |\@colroom|, nous devons
% assigner
%    $
%      |\vsize|  = |\col@number| \times |\@colroom|
%    $
%     pour rassembler assez de mat\'eriel avant d'entrer de
%     nouveau dans la routine |\output|. En fait, nous devons
%     encore ajouter
%    $
%       (|\col@number|-1) \times (|\baselineskip|-|\topskip|)
%    $
%     si vous y r\'efl\'echissez s\'erieusement.
% \changes{v1.4p}{1992/11/26}{Macro ajout\'ee.}
%    \begin{macrocode}
\def\set@mult@vsize#1{%
    \vsize\@colroom
    \@tempdima\baselineskip
    \advance\@tempdima-\topskip
    \advance\vsize\@tempdima
    \vsize\col@number\vsize
    \advance\vsize-\@tempdima
%    \end{macrocode}
% Mais cela pourrait \^etre insuffisant puisque nous
% utilisons plus tard |\vsplit| pour extraire les colonnes du
% mat\'eriel r\'euni. C'est pourquoi nous ajoutons des <<~lignes
% suppl\'ementaires~>>, leur nombre d\'ependant de la valeur
% du compteur <<~\mc{}~>>. La valeur finale est assign\'ee
% globalement si |#1| est |\global| parce ce que nous voulons
% utiliser cette macro, plus tard, aussi \`a l'int\'erieur de
% la routine de sortie.
%    \begin{macrocode}
    #1\advance\vsize
        \c@collectmore\baselineskip}
%    \end{macrocode}
%  \end{macro}
%
%
%  \begin{macro}{\multicol@leftmargin}
% Voici le registre \meta{dimen} dont nous avons besoin pour
% sauvegarder la valeur externe de |\@totalleftmargin|.
%    \begin{macrocode}
\newdimen\multicol@leftmargin
%    \end{macrocode}
%  \end{macro}
%
% \begin{macro}{\endmulticols}
% Quand la fin de l'environnement \mc{} est d\'etect\'ee,
% nous devons \'equilibrer le mat\'eriel r\'euni. Que
% nous soyons ou non \`a l'int\'erieur d'une multicol
% embo\^\i{}t\'ee, diff\'erentes choses peuvent survenir.
% Mais, d'abord, nous terminons le paragraphe courant avec
% une commande |\par|.
%    \begin{macrocode}
\def\endmulticols{\par
  \if@boxedmulticols
%    \end{macrocode}
% En mode embo\^\i{}t\'e, nous devons fermer la bo\^\i{}te
% dans laquelle nous avons rassembl\'e tout le mat\'eriel
% pour les colonnes.
%    \begin{macrocode}
    \egroup
%    \end{macrocode}
% Maintenant, nous appelons |\balance@columns|, la routine
% qui \'equilibre le mat\'eriel stock\'e dans la bo\^\i{}te |\mult@box|.
% \changes{v1.5?}{1994/?/?}{S\'eparation des bo\^\i{}tes z\'ero d\'eplac\'ee
%                           \`a \cs{balance@columns}}
%    \begin{macrocode}
    \balance@columns
%    \end{macrocode}
% Apr\`es \'equilibrage, le r\'esultat doit \^etre retourn\'e
% par la commande |\page@sofar|. Mais avant de le faire, nous
% r\'eins\'erons toutes les balises trouv\'ees dans la
% bo\^\i{}te |\mult@box|.
%    \begin{macrocode}
    \return@nonemptymark{first}%
                 \kept@firstmark
    \return@nonemptymark{bot}%
                  \kept@botmark
    \page@sofar
%    \end{macrocode}
%
%    \begin{macrocode}
    \global\let\kept@firstmark
               \l@kept@firstmark
    \global\let\kept@botmark
            \l@kept@botmark
%<*marktrace>
    \mult@info\tw@
      {Restore kept marks to\MessageBreak
       first: \meaning\kept@firstmark
       \MessageBreak bot\space\space:
                     \meaning\kept@botmark }%
%</marktrace>
%    \end{macrocode}
% Cela finit le code pour les cas <<~embo\^\i{}t\'es~>>.
%    \begin{macrocode}
  \else
%    \end{macrocode}
% Si nous sommes dans un environnement \mc{} non-restreint,
% nous fermons le paragraphe courant par |\par|, mais ce
% n'est pas suffisant puisque le {\it page\_builder}
% \TeX{} ne videra pas totalement la liste de
% contribution\,\footnote{Cela causait auparavant un bug
% fort intriguant dans lequel du mat\'eriel \'etait \'equilibr\'e deux
% fois, ce qui causait des superpositions. La raison \'etait
% que \cs{eject} \'etait plac\'e \`a la fin de la liste de
% contribution. Le {\it page\_builder} \'etait alors appel\'e
% (un \cs{penalty} explicite videra la liste de
% contribution), mais la ligne avec \cs{eject} ne tenait pas
% sur la page courante. Il \'etait ensuite repris en
% compte apr\'es la fin de la routine de sortie,
% entra\^\i{}nant une seconde rupture apr\`es une ligne}.
% Donc, nous devons aussi ajouter un |\penalty| explicite.
% La liste de contribution sera d\'esormais vid\'ee et, si son
% mat\'eriel ne tient pas enti\`erement dans la page
% courante, alors la routine de sortie sera appel\'ee avant
% que nous la changions.
% \changes{v1.3b}{1990/10/09}{R\'ealise \cs{penalty} avec \cs{addpenalty}}
% \changes{v1.4e}{1992/03/16}{Mais ignore \cs{@nobreak} dans \cs{addpenalty}}
% \changes{v1.5c}{1993/04/18}{De nouveau utilisation de \cs{penalty}}
% \`A ce moment, nous avons besoin d'utiliser |\penalty| et non
% |\addpenalty| pour nous assurer que a) les r\'ecentes
% contributions sont vid\'ees et b) le tout dernier item de
% la liste verticale principale est un point de rupture
% valide et donc que \TeX{} coupera la page dans le cas o\`u
% elle est d\'ebordante.
%    \begin{macrocode}
   \penalty\z@
%    \end{macrocode}
% Maintenant, il n'est pas dangereux de modifier la routine de
% sortie pour \'equilibrer les colonnes.
%    \begin{macrocode}
    \output{\balance@columns@out}\eject
%    \end{macrocode}
% Si le corps de l'environnement \mc{} est compl\`etement
% vide, ou si un \mc{} multi-page finit juste \`a une liaison
% entre page, nous avons le cas inhabituel o\`u |\eject| n'aura
% aucun effet (puisque la liste verticale principale est vide)
% -- donc, aucune routine de sortie ne sera jamais appel\'ee.
% En cons\'equence, le mat\'eriel pr\'ec\'edant \mc{}
% (stock\'e dans |\partial@page|) sera perdu si on ne g\`ere pas
% \c{c}a \`a la main.
% \changes{v1.4m}{1992/09/04}{V\'erification du vidage de \cs{partial@page}}
%    \begin{macrocode}
    \ifvbox\partial@page
         \unvbox\partial@page\fi
%    \end{macrocode}
% Apr\`es l'action de la routine de sortie, nous restaurons
% les balises sauvegard\'ees \`a leur valeur initiale.
%    \begin{macrocode}
     \global\let\kept@firstmark\@empty
     \global\let\kept@botmark\@empty
%<*marktrace>
     \mult@info\tw@
       {Make kept marks empty}%
%</marktrace>
  \fi
%    \end{macrocode}
% La routine de sortie ci-dessus s'occupera de |\vsize| et
% r\'eins\`erera les colonnes \'equilibr\'ees, etc. Mais
% elle ne pourra r\'eins\'erer |\footnotes| parce que nous
% devons d'abord restaurer le param\`etre |\footins|,
% puisque nous sommes revenus en mode une colonne. Cela sera
% r\'ealis\'e dans la prochaine ligne de code~; nous fermons
% simplement le groupe commenc\'e dans |\multicols|.
%
% Pour corriger un sombre bug qui est le r\'esultat de la
% d\'efinition courante des macros |\begin| \ldots\ |\end|,
% nous v\'erifions que nous sommes encore (logiquement
% parlant) dans l'environnement \mc{}. Si, par exemple, nous
% oublions de fermer quelque environnement \`a l'int\'erieur
% de l'environnement \mc{}, le |\endgroup| suivant devrait
% \^etre consid\'er\'e de fa\c{c}on incorrecte pour la
% fermeture de cet environnement.
% \changes{v1.3c}{1991/03/14}{V\'erification de la fermeture de l'env.}
%    \begin{macrocode}
  \@checkend{multicols}%
  \endgroup
%    \end{macrocode}
% Maintenant, il est temps de retourner toutes les notes de
% bas de page, si nous sommes en mode non-restreint~:
%    \begin{macrocode}
  \if@boxedmulticols\else
    \reinsert@footnotes
  \fi
%    \end{macrocode}
% Nous fixons \'egalement le compteur <<~{\sf unbalance}~>>
% \`a sa valeur par d\'efaut. Cela est r\'ealis\'e
% globalement puisque les compteurs \LaTeX{} sont toujours
% modifi\'es de cette mani\`ere\,\footnote{En fait, nous sommes
% toujours dans un groupe commenc\'e par la macro \cs{begin}
% et donc \cs{global} doit \^etre utilis\'e de toute
% fa\c{c}on}.
%    \begin{macrocode}
  \global\c@unbalance\z@
%    \end{macrocode}
% Nous jetons aussi un coup d'\oe{}il \`a la quantit\'e
% d'espace libre dans la page courante, pour voir s'il est
% temps de faire un saut de page. L'espace vertical ajout\'e
% par la suite dispara\^\i{}tra si |\enough@room| commence
% une nouvelle page.
%    \begin{macrocode}
  \enough@room\postmulticols
  \addvspace\multicolsep
%    \end{macrocode}
% Si des statistiques sont demand\'ees, nous annon\c{c}ons
% finalement que nous avons tout termin\'e.
%    \begin{macrocode}
  \mult@info\z@
     {Ending environment
             \if@boxedmulticols
                \space(boxed mode)\fi
              }}
%    \end{macrocode}
% \end{macro}
%
%
% \begin{macro}{\c@unbalance}
% \SpecialMainIndex{\c@collectmore}
% \SpecialMainIndex{\col@number}
% \SpecialMainIndex{\doublecol@number}
% \SpecialMainIndex{\premulticols}
% \SpecialMainIndex{\multicoltolerance}
% \SpecialMainIndex{\multicolpretolerance}
% \SpecialMainIndex{\page@free}
% \SpecialMainIndex{\premulticols}
% \SpecialMainIndex{\postmulticols}
% \SpecialMainIndex{\multicolsep}
% \SpecialMainIndex{\multicolbaselineskip}
% \SpecialMainIndex{\partial@page}
% Finissons cette section en allouant tous les registres
% d\'ej\`a utilis\'es.
%    \begin{macrocode}
\newcount\c@unbalance
\newcount\c@collectmore
%    \end{macrocode}
% Dans la nouvelle version de \LaTeX{}, |\col@number| est
% d\'ej\`a allou\'e par le noyau, donc nous ne l'allouons pas
% de nouveau.
%    \begin{macrocode}
%\newcount\col@number
\newcount\doublecol@number
\newcount\multicoltolerance
\newcount\multicolpretolerance
\newdimen\full@width
\newdimen\page@free
\newdimen\premulticols
\newdimen\postmulticols
\newskip\multicolsep
\newskip\multicolbaselineskip
\newbox\partial@page
\newbox\last@line
%    \end{macrocode}
% Et voil\`a leurs valeurs par d\'efaut~:
%    \begin{macrocode}
\c@unbalance   = 0
\c@collectmore = 0
%    \end{macrocode}
% Pour permettre de v\'erifier si quelque macro est
% utilis\'ee \`a l'int\'erieur de l'environnement \mc{}, le
% compteur |\col@number| prend par d\'efaut la valeur |1| \`a
% l'ext\'erieur de l'environnement.
% \changes{v1.3d}{1991/10/23}{\cs{col@number} fix\'e \`a un}
%    \begin{macrocode}
\col@number = 1
\multicoltolerance = 9999
\multicolpretolerance = -1
\premulticols = 50pt
\postmulticols= 20pt
\multicolsep = 12pt plus 4pt minus 3pt
\multicolbaselineskip=0pt
%    \end{macrocode}
% \end{macro}
%
% \end{multicols}
%
% \begin{multicols}{2}[\subsection{Les routines de sortie}]
%
% Tout d'abord, nous commen\c{c}ons par quelques macros
% simples. Lors de la composition de la page, nous
% sauvegardons les colonnes soit dans les registres de
% bo\^\i{}te 0, 2, 4,... (localement) soit 1, 3, 5,...
% (globalement). C'est la r\`egle de \PlainTeX{} pour
% \'eviter un d\'ebordement des piles stock\'ees.
%
% \begin{macro}{\process@cols}
% Donc, nous d\'efinissons une macro |\process@cols| pour
% nous aider \`a utiliser ces registres dans les routines de
% sortie ci-dessous. Elle a deux arguments~: le premier est
% un nombre~; le second est l'information d'ex\'ecution. Elle
% commence la boucle avec |\count@=#1| (|\count@| est un
% registre temporaire d\'efini en \PlainTeX), ex\'ecute
% l'argument |#2|, ajoute deux \`a |\count@|, ex\'ecute de
% nouveau l'argument |#2|, etc. jusqu'\`a ce que |\count@|
% soit plus grand que |\doublecol@number|. Il pourrait \^etre
% plus facile de comprendre gr\^ace \`a un exemple, donc
% nous la d\'efinissons maintenant et expliquons son
% utilisation plus tard.
%    \begin{macrocode}
\def\process@cols#1#2{\count@#1\relax
     \loop
%<*debug>
     \typeout{Looking at box \the\count@}
%</debug>
      #2%
     \advance\count@\tw@
     \ifnum\count@<\doublecol@number
   \repeat}
%    \end{macrocode}
% \end{macro}
%
%
% \begin{macro}{\page@sofar}
% Nous d\'efinissons maintenant |\page@sofar| pour donner un
% exemple de la macro |\process@cols|. |\page@sofar| devrait
% sortir tout ce qui est pr\'epar\'e par la routine
% d'\'equilibrage, |\balance@columns|.
%    \begin{macrocode}
\def\page@sofar{%
%    \end{macrocode}
% |\balance@columns| pr\'epare sa sortie dans les
% registres temporaires de bo\^\i{}te num\'erot\'es pairs.
% Maintenant, nous sortons les colonnes r\'eunies en
% supposant qu'elles sont sauv\'ees dans les registres de
% bo\^\i{}te 2 (colonne de gauche), 4 (seconde colonne)
% ... Cependant, la derni\`ere colonne (\emph{i.e.} la
% plus \`a droite) devrait \^etre sauv\'ee dans le registre
% de bo\^\i{}te 0\,\footnote{Vous verrez la raison ce cette
% num\'erotation, quand nous regarderons les routines de
% sortie \cs{multi@column@out} et \cs{balance@columns@out}.}.
% D'abord, nous nous assurons que les colonnes ont des
% largeurs identiques. Nous utilisons |\process@cols| dans ce
% but, commen\c{c}ant avec $|\count@|=|\mult@rightbox|$. Donc
% |\count@| boucle suivant |\mult@rightbox|, $|\mult@rightbox| + 2$,...
% (jusqu'\`a |\doublecol@number|).
% \changes{v1.5a}{1992/11/04}{Nouveau m\'ecanisme d'embo\^\i{}tement}
%    \begin{macrocode}
   \process@cols\mult@rightbox
%    \end{macrocode}
% Nous devons v\'erifier si la bo\^\i{}te en question est
% vide, parce que l'op\'eration |\wd|\meta{number} sur une
% bo\^\i{}te vide \emph{ne} changera \emph{pas} ses
% dimensions (*soupir*).
% \changes{v1.5h}{1994/08/26}{V\'erification des bo\^\i{}tes vides}
% \changes{v1.5i}{1994/10/02}{Mais n'enl\`eve pas le code original.}
%    \begin{macrocode}
       {\ifvoid\count@
          \setbox\count@\hbox to\hsize{}%
        \else
          \wd\count@\hsize
        \fi}%
%    \end{macrocode}
% Maintenant, nous donnons des informations de tra\c{c}age.
% \changes{v1.4l}{1992/08/17}{Utilisation de  \cs{multicol@leftmargin}
%                            \`a la place de \cs{@totalleftmargin}}
%    \begin{macrocode}
   \mult@info\z@
    {Column spec:\MessageBreak
       (\the\multicol@leftmargin\space -->
        \the\full@width\space = \the\hsize
        \space x \the\col@number)%
     }%
%    \end{macrocode}
% Pour le moment, nous devons toujours \^etre en mode
% vertical.
%    \begin{macrocode}
\ifvmode\else\errmessage{Multicol Error}\fi
%    \end{macrocode}
% Maintenant, nous mettons ensemble toutes les colonnes dans |\hbox|
% de largeur |\full@width| (la modifiant par |\multicol@leftmargin|
% \`a droite afin qu'elle soit plac\'ee correctement si nous
% sommes dans un environnement de liste).
% \changes{v1.4l}{1992/08/17}{Utilisation de \cs{multicol@leftmargin}
%                             \`a la place de \cs{@totalleftmargin}}
%    \begin{macrocode}
   \moveright\multicol@leftmargin
    \hbox to\full@width{%
%    \end{macrocode}
% et s\'eparons les colonnes par un filet, si on le d\'esire.
% \changes{v1.5a}{1992/11/04}{Nouveau m\'ecanisme d'embo\^\i{}tement}
%    \begin{macrocode}
     \process@cols\mult@gfirstbox{\box\count@
       \hss\vrule\@width\columnseprule\hss}%
%    \end{macrocode}
% Comme vous le remarquerez, nous commen\c{c}ons avec le
% registre de bo\^\i{}te |\mult@gfirstbox| (\emph{i.e.} la
% colonne de gauche). Alors, cette fois, |\count@| boucle \`a
% travers 2, 4, ... (plus le d\'ecalage appropri\'e).
% Finalement, nous ajoutons la bo\^\i{}te 0 et fermons |\hbox|.
% \changes{v1.5a}{1992/11/04}{Nouveau m\'ecanisme d'embo\^\i{}tement}
%    \begin{macrocode}
     \box\mult@rightbox
%    \end{macrocode}
% Les profondeurs des colonnes d\'ependent de leurs derni\`eres
% lignes. Pour nous assurer que nous aurons toujours la
% m\^eme apparence en ce qui concerne les filets, nous
% for\c{c}ons la profondeur \`a \^etre au moins celle de la
% lettre~<<~p~>>.
%    \begin{macrocode}
%     \strut
      \rlap{\phantom p}%
}%
}
%    \end{macrocode}
% \end{macro}
%
% \begin{macro}{\reinsert@footnotes}
% Avant d'attaquer les plus grosses routines de sortie, nous
% d\'efinissons juste une macro de plus qui nous aidera \`a
% trouver notre chemin \`a travers les myst\`eres plus tard.
% |\reinsert@footnotes| fera ce que son nom indique~: elle
% r\'eins\`ere les notes de bas de page pr\'esentes dans |\footinbox|
% et donc elles seront r\'e-ex\'ecut\'ees par le {\it
% page\_builder} de \TeX.
%
% Au lieu de vraiment r\'eins\'erer les notes de bas de
% page, nous ins\'erons une note vide. Cela d\'eclenchera le
% m\'ecanisme d'insertion et, puisque les anciennes notes
% sont encore dans leur bo\^\i{}te et que nous sommes sur une
% nouvelle page, |\skip| |footins| sera correctement pris en
% compte.
% \changes{v1.3c}{1990/03/03}{\cs{unbox}ing \'evit\'e.}
%    \begin{macrocode}
\def\reinsert@footnotes{\ifvoid\footins\else
         \insert\footins{}\fi}
%    \end{macrocode}
% \end{macro}
%
%
% \begin{macro}{\multi@column@out}
% Maintenant, nous ne pouvons plus repousser les difficult\'es
% plus loin. La routine |\multi@column@out| sera appel\'ee
% dans deux situations. Soit la page est pleine (\emph{i.e.}
% nous avons accumul\'e assez de mat\'eriel pour construire
% toutes les colonnes d\'esir\'ees) ou un flottant ou la
% note de marge (ou bien encore un |\clearpage|) est d\'etect\'e. Dans le
% dernier cas, |\outputpenalty| est inf\'erieur \`a $-10000$,
% sinon la p\'enalit\'e qui d\'eclenche la routine de sortie
% est plus grande. Donc, il est facile de faire la
% distinction entre les deux cas~: nous testons simplement ce
% registre.
% \changes{v1.5c}{1993/04/18}{Prise en charge de \cs{clearpage}}
%    \begin{macrocode}
\def\multi@column@out{%
   \ifnum\outputpenalty <-\@M
%    \end{macrocode}
% Si c'\'etait un |\clearpage|, un flottant ou une note de
% marge, nous appelons |\speci@ls|.
%    \begin{macrocode}
   \speci@ls \else
%    \end{macrocode}
% sinon, nous construisons la page d\'efinitive.
% \changes{v1.5u}{1999/05/25}{Prise en charge de \cs{columnbreak}}
% \changes{v1.5v}{1999/07/18}{Ajout des rapports de
%                             d\'ebuggage pour la prise en compte
%                             des ruptures de colonne}
%
% Pour la partie du code qui suit, voir les commentaires dans
% la section~\ref{sec:colbreak}.
%    \begin{macrocode}
%<*colbreak>
   \ifvoid\colbreak@box\else
     \mult@info\@ne{Re-adding forced
               break(s) for splitting}%
     \setbox\@cclv\vbox{%
        \unvbox\colbreak@box
        \penalty-\@Mv\unvbox\@cclv}%
   \fi
%</colbreak>
%    \end{macrocode}
% Consid\'erons le cas normal. Nous avons \`a |\vsplit|er les
% colonnes \`a partir du mat\'eriel accumul\'e dans la
% bo\^\i{}te 255. Donc, nous assignons d'abord les valeurs
% appropri\'ees \`a |\splittopskip| et |\splitmaxdepth|.
%    \begin{macrocode}
   \splittopskip\topskip
   \splitmaxdepth\maxdepth
%    \end{macrocode}
% Puis nous calculons la hauteur de la colonne courante (dans
% |\dimen@|). Notez que la hauteur de |\partial@page| est
% d\'ej\`a soustraite de |\@colroom|, donc, nous pouvons
% utiliser sa valeur comme point de d\'epart.
%    \begin{macrocode}
   \dimen@\@colroom
%    \end{macrocode}
% Mais nous devons aussi soustraire l'espace occup\'e par les
% notes de bas de page sur la page courante. Notez que,
% d'abord, nous avons \`a remettre le registre de saut \`a sa
% valeur initiale.
% Encore une fois, la v\'eritable action est de transposer
% dans une macro utilitaire, afin que d'autres applications
% puissent la modifier.
% \changes{v1.5?}{1994/?/?}{Utilisation de \cs{leave@mult@footins}}
%    \begin{macrocode}
   \divide\skip\footins\col@number
   \ifvoid\footins \else
      \leave@mult@footins
   \fi
%    \end{macrocode}
% Maintenant, nous sommes capables de |\vsplit|er toutes les
% colonnes, sauf la derni\`ere. Rappelez-vous que ces
% colonnes doivent \^etre sauv\'ees dans les registres de
% bo\^\i{}te 2, 4, ... (plus le d\'ecalage).
% \changes{v1.5a}{1992/11/04}{Nouveau m\'ecanisme d'embo\^\i{}tement}
%    \begin{macrocode}
   \process@cols\mult@gfirstbox{%
        \setbox\count@
            \vsplit\@cclv to\dimen@
%    \end{macrocode}
% Apr\`es s\'eparation, nous mettons \`a jour les balises
% conserv\'ees.
%    \begin{macrocode}
            \set@keptmarks
%    \end{macrocode}
% Si |\raggedcolumns| est forc\'e, nous ajoutons |vfill| au
% bas en d\'esembo\^\i{}tant la bo\^\i{}te s\'epar\'ee.
% \changes{v1.3c}{1990/03/03}{\cs{unbox}ing \'evit\'e.}
%    \begin{macrocode}
            \ifshr@nking
              \setbox\count@
                 \vbox to\dimen@
                  {\unvbox\count@\vfill}%
            \fi
           }%
%    \end{macrocode}
% Puis la derni\`ere colonne suit.
% \changes{v1.5a}{1992/11/04}{Nouveau m\'ecanisme d'embo\^\i{}tement}
%    \begin{macrocode}
   \setbox\mult@rightbox
       \vsplit\@cclv to\dimen@
   \set@keptmarks
   \ifshr@nking
      \setbox\mult@rightbox\vbox to\dimen@
          {\unvbox\mult@rightbox\vfill}%
   \fi
%    \end{macrocode}
% Ayant fait cela, nous esp\'erons que la bo\^\i{}te 255 est
% vid\'ee. Si elle ne l'est pas, nous r\'eins\'erons son
% contenu.
%    \begin{macrocode}
   \ifvoid\@cclv \else
       \unvbox\@cclv
       \penalty\outputpenalty
%    \end{macrocode}
% Dans ce cas, une note de bas de page, qui peut \^etre
% ins\'er\'ee dans le petit espace restant, sera compos\'ee
% sur la mauvaise page. Donc, nous avertissons l'utilisateur
% si la page courante contient des notes. Les anciennes
% versions de \mc{} produisaient cet avertissement qu'il y
% ait ou non des notes, ce qui entra\^\i{}nait beaucoup
% d'avertissements inutiles.
% \changes{v1.3c}{1991/02/17}{V\'erification de la
%                             pr\'esence effective des notes de
%                             bas de page avant d'envoyer un message d'alerte}
%    \begin{macrocode}
       \ifvoid\footins\else
         \PackageWarning{multicol}%
          {I moved some lines to
           the next page.\MessageBreak
           Footnotes on page
           \thepage\space might be wrong}%
       \fi
%    \end{macrocode}
% Si le compteur <<~{\sf tracingmulticols}~>> est \`a 4 ou
% plus, nous ajoutons \'egalement un filet.
%    \begin{macrocode}
       \ifnum \c@tracingmulticols>\thr@@
                    \hrule\allowbreak \fi
   \fi
%    \end{macrocode}
% Pour donner des balises correctes pour la page courante,
% nous devons \`red\'efinir (localement) |\firstmark| and |\botmark|.
% Si |\kept@firstmark| n'est pas vide, alors |\kept@botmark|
% ne doit pas \^etre vide non plus, donc nous pouvons utiliser
% leurs valeurs, Sinon, nous utilisons la valeur de |\kept@topmark|
% qui a d'abord \'et\'e initialis\'e quand nous avons r\'euni
% |\partical@page| et plus tard, a \'et\'e mise \`a jour au
% |\botmark| pour la page pr\'ec\'edente.
%
% \changes{v1.4a}{1992/02/14}{\cs{botmark} fix\'e \`a \cs{splitbotmark}}
%    \begin{macrocode}
   \ifx\@empty\kept@firstmark
      \let\firstmark\kept@topmark
      \let\botmark\kept@topmark
   \else
      \let\firstmark\kept@firstmark
      \let\botmark\kept@botmark
   \fi
%    \end{macrocode}
% Nous initialisons \'egalement |\topmark| avec
% |\kept@topmark|. Cela rendra cette balise correcte pour
% toutes les pages m\'edianes de l'environnement \mc{}.
% \changes{v1.5d}{1993/09/15}{r\'einitialisation de \cs{topmark}}
%    \begin{macrocode}
   \let\topmark\kept@topmark
%<*marktrace>
   \mult@info\tw@
        {Use kept top mark:\MessageBreak
          \meaning\kept@topmark
         \MessageBreak
         Use kept first mark:\MessageBreak
          \meaning\kept@firstmark
        \MessageBreak
         Use kept bot mark:\MessageBreak
          \meaning\kept@botmark
        \MessageBreak
         Produce first mark:\MessageBreak
          \meaning\firstmark
        \MessageBreak
        Produce bot mark:\MessageBreak
          \meaning\botmark
         \@gobbletwo}%
%</marktrace>
%    \end{macrocode}
% Avec un petit peu plus d'effort, nous aurions pu faire
% mieux. Si nous avions, par exemple, enregistr\'e le
% retr\'ecissement du mat\'eriel dans |\partial@page|, il
% serait maintenant possible d'essayer les valeurs
% sup\'erieures pour |\dimen@| (\emph{i.e.} la hauteur des
% colonnes) pour passer outre le probl\`eme de la bo\^\i{}te
% 255 non vide. Mais cela aurait rendu le code beaucoup plus
% complexe, donc je l'ai ignor\'e (le probl\`eme) dans
% l'impl\'ementation actuelle.
%
% Maintenant, nous utilisons le m\'ecanisme standard de
% sortie de \LaTeX\,\footnote{Cela est extr\^emement
% co\^uteux car les deux routines de sortie sont gard\'ees
% en m\'emoire. La bonne solution aurait \'et\'e de refaire
% enti\`erement la routine de sortie utilis\'ee dans \LaTeX}.
% C'est vrai que c'est une fa\c{c}on de faire amusante.
%
%    \begin{macrocode}
   \setbox\@cclv\vbox{\unvbox\partial@page
                      \page@sofar}%
%    \end{macrocode}
% La macro |\@makecol| ajoute \`a la page courante tous les flottants
% assign\'es \`a cette page. |\@outputpage| d\'ebarque la
% bo\^\i{}te r\'esultante. Notez qu'il est seulement possible
% que de tels flottants soient pr\'esents m\^eme si nous n'en
% permettons aucun dans un environnement \mc{}.
%    \begin{macrocode}
   \@makecol\@outputpage
%    \end{macrocode}
% Apr\'es la sortie de la page, nous devons pr\'eparer les
% marques conserv\'ees pour la page suivante. |\kept@firstmark| et
% |\kept@botmark| sont r\'einitialis\'ees en les fixant \`a
% |\@empty|. La valeur de |\botmark| est ensuite assign\'ee
% \`a |\kept@topmark|.
% \changes{v1.4g}{1992/06/03}{Change seulement \cs{kept@topmark}
%                             si \cs{kept@botmark} n'est pas vide}
% \changes{v1.4i}{1992/06/18}{Fixation de \cs{kept@topmark} \`a \cs{botmark}}
%    \begin{macrocode}
     \global\let\kept@topmark\botmark
     \global\let\kept@firstmark\@empty
     \global\let\kept@botmark\@empty
%<*marktrace>
     \mult@info\tw@
        {(Re)Init top mark:\MessageBreak
         \meaning\kept@topmark
         \@gobbletwo}%
%</marktrace>
%    \end{macrocode}
% Maintenant nous fixons |\@colroom| \`a |\@colht|, qui est
% la valeur \LaTeX{} sauvegard\'ee pour |\textheight|.
%    \begin{macrocode}
   \global\@colroom\@colht
%    \end{macrocode}
% Puis, nous traitons les flottants en suspens attendant leur
% tour pour \^etre plac\'es sur la page suivante.
%    \begin{macrocode}
   \process@deferreds
   \@whilesw\if@fcolmade\fi{\@outputpage
      \global\@colroom\@colht
      \process@deferreds}%
%    \end{macrocode}
% Si l'utilisateur demande des statistiques, nous
% l'informons sur la quantit\'e d'espace r\'eserv\'ee aux
% flottants.
%    \begin{macrocode}
   \mult@info\@ne
     {Colroom:\MessageBreak
      \the\@colht\space
              after float space removed
              = \the\@colroom \@gobble}%
%    \end{macrocode}
% Ayant fait tout cela, nous devons pr\'eparer la prise en
% compte de la page suivante. Donc, nous assignons une
% nouvelle valeur \`a |\vsize|. Nouvelle, parce que |\partial@page|
% est maintenant vide et que |\@colroom| pourrait \^etre
% diminu\'e de l'espace r\'eserv\'e aux flottants.
% \changes{v1.4p}{1992/11/26}{Utilisation de diff\'erentes valeurs de\cs{vsize}}
%    \begin{macrocode}
    \set@mult@vsize \global
%    \end{macrocode}
% Le registre de saut |\footins| sera ajust\'e quand le groupe de sortie
% sera ferm\'e.
% \changes{v1.3c}{1991/03/03}{Suppression du code inutile}
%    \begin{macrocode}
  \fi}
%    \end{macrocode}
% \end{macro}
%
%
%
% \begin{macro}{\leave@mult@footins}
% Cette macro est utilis\'ee pour soustraire la quantit\'e
% d'espace, occup\'ee par les notes de bas de page, de
% l'espace courant disponible pour la colonne courante.
% L'espace de la colonne courante est stock\'e dans |\dimen@|.
% Voir ci-dessus la description de l'action par d\'efaut.
% \changes{v1.5?}{1994/?/?}{Macro ajout\'ee}
%    \begin{macrocode}
\def\leave@mult@footins{%
   \advance\dimen@-\skip\footins
   \advance\dimen@-\ht\footins
}
%    \end{macrocode}
% \end{macro}
%
%
%
% \begin{macro}{\speci@ls}
% Nous avons laiss\'e de c\^ot\'e deux macros~: |\process@deferreds|
% et |\speci@ls|.
% \changes{v1.5c}{1993/04/18}{Prise en charge de \cs{clearpage}}
%    \begin{macrocode}
\def\speci@ls{%
 \ifnum\outputpenalty <-\@Mi
%    \end{macrocode}
% Si le document finit au milieu d'un environnement
% multicolonne, par exemple si l'utilisateur a oubli\'e
% |\end{multicols}|, \TeX{} ajoute une p\'enalit\'e tr\`es
% n\'egative \`au texte en cours de traitement qui est cens\'ee signaler \`a la
% routine de sortie qu'il est temps de se pr\'eparer \`a
% exp\'edier tout ce qui reste. Puisqu'\`a l'int\'erieur de
% multicol, la routine de sortie de \LaTeX{} est
% quelquefois d\'esactiv\'ee, il est mieux pour nous de
% v\'erifier ce cas~: si nous trouvons une p\'enalit\'e tr\`es
% n\'egative, nous produisons un message d'erreur et
% ex\'ecutons la routine de sortie standard dans ce cas.
% \changes{v1.5s}{1998/09/10}{V\'erification de la p\'enalit\'e \cs{stop}. rapport pr/2873}
%    \begin{macrocode}
  \ifnum \outputpenalty<-\@MM
   \PackageError{multicol}{Document end
           inside multicols environment}\@ehd
   \@specialoutput
  \else
%    \end{macrocode}
% \changes{v1.5u}{1999/05/25}{Prise en charge de \cs{columnbreak}}
% \changes{v1.5v}{1999/07/18}{Ajout des rapports de
%                             d\'ebugage pour la prise en compte
%                             des ruptures de colonne}
% Pour la prochaine partie du code, voir les commentaires
% dans la section~\ref{sec:colbreak}.
%    \begin{macrocode}
%<*colbreak>
     \ifnum\outputpenalty = -\@Mv
         \mult@info\@ne{Forced column
                        break seen}%
         \global\advance\vsize-\pagetotal
         \global\setbox\colbreak@box
                \vbox{\ifvoid\colbreak@box
                      \else
                        \unvbox\colbreak@box
                        \penalty-\@Mv
                      \fi
                      \unvbox\@cclv}
         \reinsert@footnotes
     \else
%</colbreak>
%    \end{macrocode}
% Si nous rencontrons un flottant ou une note de marge dans
% l'impl\'ementation actuelle, nous avertissons simplement
% l'utilisateur que cela n'est pas permis. Puis, nous
% r\'eins\'erons la page et ses notes de bas de page.
%    \begin{macrocode}
        \PackageWarning{multicol}%
           {Floats and marginpars not
            allowed inside `multicols'
            environment!
           \@gobble}%
        \unvbox\@cclv\reinsert@footnotes
%    \end{macrocode}
% De plus, nous vidons |\@currlist| pour \'eviter des
% messages d'erreur futurs quand la routine de sortie de
% \LaTeX{} sera forc\'ee de nouveau. Mais d'abord, nous devons
% replacer les bo\^\i{}tes dans |\@freelist| (la valeur par d\'efaut de
% |\@elt| est |\relax| donc cela est possible avec |\xdef|).
% \changes{v1.2a}{1990/02/05}{Bo\^\i{}tes flottantes lib\'er\'ees.}
%    \begin{macrocode}
       \xdef\@freelist{\@freelist\@currlist}%
       \gdef\@currlist{}%
%<*colbreak>
     \fi
%</colbreak>
  \fi
%    \end{macrocode}
% Si la p\'enalit\'e est $-1000$, elle proviendra de |\clearpage|
% et nous ex\'ecuterons |\@doclearpage| pour nous d\'ebarrasser de tous
% les flottants en attente.
%    \begin{macrocode}
 \else \@doclearpage \fi
}
%    \end{macrocode}
% \end{macro}
%
% \begin{macro}{\process@deferreds}
% |\process@deferreds| est une version simplifi\'ee de la
% commande |\@startpage| de \LaTeX. Nous appelons d'abord |\@floatplacement|
% pour sauvegarder les param\`etres utilisateurs courants dans
% les registres internes. Puis nous commen\c{c}ons un nouveau
% groupe et sauvegardons temporairement |\@deferlist| dans la
% macro |\@tempb|.
%    \begin{macrocode}
\def\process@deferreds{%
   \@floatplacement
   \@tryfcolumn\@deferlist
   \if@fcolmade\else
     \begingroup
    \let\@tempb\@deferlist
%    \end{macrocode}
% Notre prochaine action est de vider (globalement) |\@deferlist|
% et assigner une nouvelle signification \`a |\@elt|. Ici, |\@scolelt|
% est une macro qui regarde les bo\^\i{}tes dans une liste
% pour d\'ecider si elles doivent \^etre plac\'ees sur une
% nouvelle page (\emph{i.e.} dans |\@toplist| ou |\@botlist|)
% ou si elles doivent attendre un autre traitement.
%    \begin{macrocode}
      \gdef\@deferlist{}%
      \let\@elt\@scolelt
%    \end{macrocode}
% Maintenant, nous appelons |\@tempb| qui est de la forme
%    \begin{center}
%      |\@elt|\meta{box register}|\@elt|^^A
%                        \meta{box register}\ldots{}
%    \end{center}
% Donc |\@elt| (\emph{i.e.} |\@scolelt|) distribuera les
% bo\^\i{}tes aux trois listes.
%    \begin{macrocode}
        \@tempb \endgroup
   \fi}
%    \end{macrocode}
% \end{macro}
%
%
%
% \begin{macro}{ifshr@nking}
% \begin{macro}{\raggedcolumns}
% \begin{macro}{\flushcolumns}
% \changes{v1.1a}{1989/09/20}{\cs{flushedcolumns} renomm\'e en \cs{flushcolumns}.}
% Les d\'eclarations de |\raggedcolumns| et |\flushcolumns|
% sont d\'efinies \`a l'aide d'une nouvelle macro |\if...|.
%    \begin{macrocode}
\newif\ifshr@nking
%    \end{macrocode}
% Les d\'efinitions r\'eelles sont simples~: nous commutons
% juste \`a {\sf true} ou {\sf false} en fonction de l'action
% d\'esir\'ee. Pour \'eviter des espaces suppl\'ementaires en
% sortie, nous encapsulons ces changements dans |\@bsphack|\ldots{}\allowbreak|\@esphack|.
%    \begin{macrocode}
\def\raggedcolumns{%
   \@bsphack\shr@nkingtrue\@esphack}
\def\flushcolumns{%
   \@bsphack\shr@nkingfalse\@esphack}
%    \end{macrocode}
% \end{macro}
% \end{macro}
% \end{macro}
%
%
% \begin{macro}{\balance@columns@out}
% Maintenant, la derni\`ere partie du spectacle~: la
% routine de sortie d'\'equilibrage des colonnes. Puisque ce
% code est appel\'e par une p\'enalit\'e explicite
% (|\eject|), il n'est pas besoin de v\'erifier des trucs
% sp\'eciaux (par exemple, les flottants). Nous commen\c{c}ons
% par \'equilibrer le mat\'eriel r\'euni.
%    \begin{macrocode}
\def\balance@columns@out{%
%    \end{macrocode}
% Pour cela, nous avons besoin de mettre de contenu de la
% bo\^\i{}te 255 dans |\mult@box|.
%    \begin{macrocode}
%<-colbreak> \setbox\mult@box
%<-colbreak>   \vbox{\unvbox\@cclv}%
%    \end{macrocode}
% \changes{v1.5u}{1999/05/25}{Prise en charge de \cs{columnbreak}}
% \changes{v1.5v}{1999/07/18}{Ajout des rapports de
%                             d\'ebugage pour la prise en compte
%                             des ruptures de colonne}
% Pour la partie suivante du code, voir les commentaires dans
% la section~\ref{sec:colbreak}.
%    \begin{macrocode}
%<*colbreak>
   \setbox\mult@box\vbox{%
       \ifvoid\colbreak@box\else
         \unvbox\colbreak@box\break
         \mult@info\@ne{Re-adding
           forced break(s) in balancing}%
       \fi
       \unvbox\@cclv}%
%</colbreak>
   \balance@columns
%    \end{macrocode}
% Cela nous am\`enera dans la position dans laquelle nous
% appliquerons |\page@sofar|. Mais d'abord, nous devons
% fixer |\vsize| \`a une valeur idoine pour la sortie d'une
% colonne.
%    \begin{macrocode}
   \global\vsize\@colroom
   \global\advance\vsize\ht\partial@page
%    \end{macrocode}
% Puis nous |\unvbox|ons |\partial@page| (qui peut \^etre
% vide si nous n'avons pas trait\'e la premi\`ere page de cet
% environnement \mc).
%    \begin{macrocode}
   \unvbox\partial@page
%    \end{macrocode}
% Puis nous retournons les marques, premi\`ere et du bas,
% et le mat\'eriel r\'euni, dans la liste verticale
% principale.
%    \begin{macrocode}
   \return@nonemptymark{first}\kept@firstmark
   \return@nonemptymark{bot}\kept@botmark
   \page@sofar
%    \end{macrocode}
% Nous avons besoin d'ajouter une p\'enalit\'e \`a cet
% endroit, ce qui nous permettra de couper en ce point
% puisque l'appel de la routine de sortie peut avoir
% supprim\'e l'unique point de rupture possible, <<~collant~>>
% ainsi tout saut suivant \`a la bo\^\i{}te \'equilibr\'ee.
% Dans le cas o\`u il y a des r\'eglages bizarres de
% |\multicolsep|, etc., cela pourrait produire des
% r\'esultats amusants.
%  \changes{v1.5c}{1993/04/18}{Ajout de p\'enalit\'e en sortie de routine de sortie}
%    \begin{macrocode}
   \penalty\z@
}
%    \end{macrocode}
% Comme nous le savons d\'ej\`a, la r\'einsertion des notes
% de bas de page se fait par la macro |\endmulticols|.
% \end{macro}
%
%
%  \begin{macro}{\balance@columns}
% La macro r\'ealise maintenant le v\'eritable \'equilibrage.
%    \begin{macrocode}
\def\balance@columns{%
%    \end{macrocode}
% Nous commen\c{c}ons par r\'egler les marques conserv\'ees
% en les mettant \`a jour avec les marques de cette
% bo\^\i{}te. Cela doit \^etre r\'ealis\'e \emph{avant} d'ajouter la
% p\'enalit\'e de $-10000$ en haut de la bo\^\i{}te, sinon
% seule une bo\^\i{}te vide sera prise en compte.
% \changes{v1.5h}{1994/08/26}{Prendre les marques sauvegard\'ees d'abord}
%    \begin{macrocode}
   \get@keptmarks\mult@box
%    \end{macrocode}
% Nous continuons ensuite par la remise \`a z\'ero des essais
% pour enlever tout le mat\'eriel effa\c{c}able \`a la fin de
% |\mult@box|. Ceci est plut\^ot exp\'erimental. Nous
% ajoutons aussi un point de rupture forc\'e au tout d\'ebut,
% et nous pouvons donc diviser la bo\^\i{}te de hauteur
% z\'ero plus tard, ajoutant ainsi une glue connue |\splittopskip|
% au d\'ebut.
%    \begin{macrocode}
   \setbox\mult@box\vbox{%
        \penalty-\@M
        \unvbox\mult@box
        \remove@discardable@items
        }%
%    \end{macrocode}
% Puis suivent des assignements de valeurs pour rendre
% |\vsplit| correct. Nous utilisons la partie naturelle de |\topskip|
% comme la partie naturelle de |\splittopskip| et permettons
% un peu de retrait et de d\'epassement par l'ajout
% d'\'elargissement et de retr\'ecissement.
% \changes{v1.5?}{1994/?/?}{Autorise les colonnes \`a
%                            \^etre un peu trop longues ou courtes}
%    \begin{macrocode}
   \@tempdima\topskip
   \splittopskip\@tempdima
       \@plus\multicolundershoot
       \@minus\multicolovershoot
   \splitmaxdepth\maxdepth
%    \end{macrocode}
% L'\'etape suivante est un peu d\'elicate~: quand \TeX{}
% assemble le mat\'eriel dans une bo\^\i{}te, la premi\`ere
% ligne n'est pas pr\'ec\'ed\'ee par la glue d'interligne,
% \emph{i.e.} il n'y a pas de parem\`etre tel que |\boxtopskip|
% dans \TeX{}. Cela signifie que la ligne de base de la
% premi\`ere ligne de notre bo\^\i{}te est situ\'ee en un
% point impr\'evisible qui d\'epend de la hauteur du
% caract\`ere le plus large de cette ligne. Mais, bien s\^ur,
% nous voulons que toutes les colonnes soient correctement
% align\'ees sur la ligne de base de leur premi\`ere ligne.
% Pour cette raison, nous avons ouvert |\mult@box| avec un
% |\penalty| de {\sf -10000}. Cela nous permettra alors
% de d\'etacher un petit morceau de |\mult@box| (en fait,
% rien, puisque le premier point de rupture possible est le
% premier item dans la bo\^\i{}te). En cons\'equence, |\splittopskip|
% est ins\'er\'e au d\'ebut de |\mult@box| ce qui est
% exactement ce que nous voulons faire.
% \changes{v1.5?}{1994/?/?}{Faire la s�paration � z�ro ici}
%    \begin{macrocode}
   \setbox\@tempboxa\vsplit\mult@box to\z@
%    \end{macrocode}
% Ensuite nous essayons de trouver un point de d\'epart
% convenable pour le calcul de hauteur de colonne. Il doit
% \^etre inf\'erieur \`a la hauteur finale d\'esir\'ee mais
% suffisant pour atteindre cette valeur finale en quelques
% it\'erations seulement. La formule qui est maintenant
% impl\'ement\'ee essaiera de commencer avec la valeur la
% plus proche d'un multiple de |\baselineskip|. Le codage est
% l\'eg\`erement subtil dans \TeX{} et il y a peut-\^etre
% un meilleur moyen...
% \changes{v1.4d}{1992/03/04}{Nouvel algorithme pour commencer la hauteur}
%    \begin{macrocode}
   \@tempdima\ht\mult@box
   \advance\@tempdima\dp\mult@box
   \divide\@tempdima\col@number
%    \end{macrocode}
% Le code ci-dessus r\`egle |\@tempdima| \`a la longueur
% d'une colonne si nous divisons simplement la bo\^\i{}te
% enti\`ere en parties \'egales. Pour donner le plus petit
% multiple suivant de |\baselineskip|, nous convertissons
% cette dimension en un nombre (le nombre de points
% d'\'echelle) et puis, nous multiplions ce r\'esultat par |\baselineskip|
% assignant cette valeur \`a |\dimen@|.
% Cela rend |\dimen@| $\leq$ \`a |\@tempdimena|.
%    \begin{macrocode}
   \count@\@tempdima
   \divide\count@\baselineskip
   \dimen@\count@\baselineskip
%    \end{macrocode}
% L'\'etape suivante est de corriger notre r\'esultat en
% prenant en compte la diff\'erence entre |\topskip| et
% |\baselineskip|. Nous commen\c{c}ons par ajouter
% |\topskip|~; si cela donne un r\'esultat trop grand, alors
% il nous faut soustraire une |\baselineskip|.
%    \begin{macrocode}
   \advance\dimen@\topskip
   \ifdim \dimen@ >\@tempdima
     \advance\dimen@-\baselineskip
   \fi
%    \end{macrocode}
% \`A la demande de l'utilisateur, nous commen\c{c}ons avec
% la valeur la plus haute (ou la plus faible, mais, d'habitude, cela
% augmente le nombre d'essais).
%    \begin{macrocode}
   \advance\dimen@\c@unbalance\baselineskip
%    \end{macrocode}
% Nous visualisons les statistiques si cela a \'et\'e
% demand\'e.
% \changes{v1.4f}{1992/04/28}{\cs{on@line} ajout\'e pour les info. de tra\c{c}age}
%    \begin{macrocode}
   \mult@info\@ne
      {Balance columns\on@line:
        \ifnum\c@unbalance=\z@\else
       (off balance=\number\c@unbalance)\fi
      \@gobbletwo}%
%    \end{macrocode}
% Mais nous ne permettons pas de valeurs absurdes pour commencer.
%    \begin{macrocode}
   \ifnum\dimen@<\topskip
     \mult@info\@ne
       {Start value
          \the\dimen@  \space ->
          \the\topskip \space (corrected)}%
     \dimen@\topskip
   \fi
%    \end{macrocode}
% Maintenant, nous essayons de trouver la hauteur finale des
% colonnes. Nous commen\c{c}ons par r\'egler |\vbadness| \`a
% l'infini (\emph{i.e.} $10000$) pour supprimer les
% bo\^\i{}tes insuffisamment remplies pendant que nous sommes
% en train de trouver une solution acceptable. Nous n'avons
% pas besoin de faire cela dans un groupe puisqu'\`a la fin
% de la routine de sortie, tout est restaur\'e. Le r\'eglage
% des colonnes finales produira toujours des bo\^\i{}tes peu
% remplies, avec une <<~laideur~>> de $10000$ et donc il n'y
% a pas lieu d'en avertir l'utilisateur.
% \changes{v1.2a}{1990/02/05}{Suppression du groupe autour
%                             de la boucle principale.}
%    \begin{macrocode}
   \vbadness\@M
%    \end{macrocode}
% Nous permettons aussi les bo\^\i{}tes d\'ebordantes pendant
% que nous essayons de s\'eparer les colonnes.
%    \begin{macrocode}
   \vfuzz \col@number\baselineskip
%    \end{macrocode}
% La variable |\last@try| prendra la dimension utilis\'ee
% dans le pr\'ec\'edent test de s\'eparation. Nous
% l'initialisons avec une valeur n\'egative.
% \changes{v1.5?}{1994/?/?}{Initialisation de \cs{last@try}}
%    \begin{macrocode}
   \last@try-\p@
   \loop
%    \end{macrocode}
% Afin de ne pas encombrer la pr\'ecieuse m\'emoire
% principale de \TeX{} avec des choses dont nous n'avons pas
% besoin, nous vidons tous les registres de bo\^\i{}te
% utilis\'es globalement. C'est n\'ecessaire si nous revenons
% \`a ce point apr\`es un essai infructueux. Nous utilisons
% |\process@cols| dans ce but, commen\c{c}ant avec
% |\mult@grightbox|. Notez les accolades
% suppl\'ementaires autour de l'appel de macro : elles sont
% n\'ecessaires puisque le m\'ecanisme |\loop|\ldots{}\allowbreak|\repeat|
% de \PlainTeX{} ne peut \^etre encapsul\'e au m\^eme niveau
% de groupement.
% \changes{v1.5a}{1992/11/04}{Nouveau m\'ecanisme d'embo\^\i{}tement}
%    \begin{macrocode}
    {\process@cols\mult@grightbox
           {\global\setbox\count@
                   \box\voidb@x}}%
%    \end{macrocode}
% Le contenu de la bo\^\i{}te |\mult@box| est maintenant
% copi\'e globalement dans la bo\^\i{}te |\mult@grightbox|
% (ce sera la colonne la plus \`a droite, comme nous devrions
% le voir plus loin).
%    \begin{macrocode}
    \global\setbox\mult@grightbox
           \copy\mult@box
%    \end{macrocode}
% Nous commen\c{c}ons par supposer que l'essai sera
% fructueux. Si nous finissons avec une trop mauvaise
% solution, nous r\'eglons |too@bad| \`a \texttt{true}.
% \changes{v1.5b}{1992/11/05}{Nouveau m\'ecanisme de <<~laideur~>>}
%    \begin{macrocode}
%<*badness>
   \global\too@badfalse
%</badness>
%    \end{macrocode}
% En utilisant |\vsplit|, nous extrayons les autres colonnes
% du registre de bo\^\i{}te |\mult@grightbox|. Cela laisse le
% registre de bo\^\i{}te |\mult@box| intact afin de pouvoir
% recommencer si cet essai a \'et\'e infructueux.
%    \begin{macrocode}
   {\process@cols\mult@firstbox{%
         \global\setbox\count@
         \vsplit\mult@grightbox to\dimen@
%    \end{macrocode}
% Apr\`es toutes les s\'eparations, nous v\'erifions la
% laideur des colonnes r\'esultantes : normalement, c'est la
% quantit\'e de blanc suppl\'ementaire dans la colonne.
%    \begin{macrocode}
%<*badness>
         \ifnum\c@tracingmulticols>\@ne
           \@tempcnta\count@
           \advance\@tempcnta-\mult@grightbox
           \divide\@tempcnta \tw@
           \message{^^JColumn
              \number\@tempcnta\space
               badness: \the\badness\space}%
         \fi
%    \end{macrocode}
% Si la laideur est plus grande que celle autoris\'ee par la
% laideur des colonnes, nous rejetons cette solution en
% fixant |too@bad| \`a \texttt{true}.
%    \begin{macrocode}
         \ifnum\badness>\c@columnbadness
           \ifnum\c@tracingmulticols>\@ne
             \message{too bad
                  (>\the\c@columnbadness)}%
           \fi
           \global\too@badtrue
         \fi
%</badness>
                        }}%
%    \end{macrocode}
% Il y a ici une subtilit\'e~: alors que toute les autres
% bo\^\i{}tes construites ont une profondeur d\'etermin\'ee
% par |\splitmaxdepth|, la derni\`ere bo\^\i{}te aura une
% profondeur naturelle ne tenant pas compte du r\'eglage
% original et de la valeur de |\splitmaxdepth| ou
% |\boxmaxdepth|. Cela signifie que nous pouvons terminer
% avec une profondeur tr\`es grande dans la bo\^\i{}te
% |\mult@grightbox|, ce qui pourrait donner un r\'esultat de
% test incorrect. Donc, nous modifions la valeur en vidant la
% bo\^\i{}te dans elle-m\^eme.
%    \begin{macrocode}
   \boxmaxdepth\maxdepth
   \global\setbox\mult@grightbox
      \vbox{\unvbox\mult@grightbox}%
%    \end{macrocode}
% Nous sauvegardons aussi une copie de la valeur <<~naturelle~>>
% |\mult@firstbox| pour un usage ult\'erieur.
%    \begin{macrocode}
   \setbox\mult@nat@firstbox
      \vbox{\unvcopy\mult@firstbox}%
%    \end{macrocode}
% Apr\`es que |\process@cols| a fait son travail, nous sommes
% dans la situation suivante~:
%    \begin{center}
%      \begin{tabular}{r@{$\:\:\longleftarrow\:\:$}l}
%        bo\^\i{}te |\mult@rightbox| & tout le mat\'eriel \\
%        bo\^\i{}te |\mult@gfirstbox| & premi\`ere colonne \\
%        bo\^\i{}te |\mult@gfirstbox|${}+2$ & seconde colonne \\
%        \multicolumn{1}{c}{$\vdots$} &
%        \multicolumn{1}{c}{$\vdots$} \\
%        bo\^\i{}te |\mult@grightbox| & derni\`ere colonne
%      \end{tabular}
%    \end{center}
% Nous rapportons la hauteur de la premi\`ere colonne, la
% taille naturelle \'etant donn\'ee entre crochets.
% \changes{v1.5?}{1994/?/?}{Montre la taille primitive}
%    \begin{macrocode}
    \ifnum\c@tracingmulticols>\@ne
       \message{^^JFirst column
           = \the\dimen@\space
           (\the\ht\mult@nat@firstbox)}\fi
%    \end{macrocode}
% Si |\raggedcolumns| est forc\'e, les anciennes versions de
% ce fichier retr\'ecissent aussi la premi\`ere colonne
% jusqu'\`a sa valeur naturelle \`a cet endroit. Cela \'etait
% r\'ealis\'e de telle sorte que la premi\`ere colonne ne soit pas plus
% petite compar\'ee \`a la derni\`ere, mais cela produisait en
% fait des r\'esultats incorrects (superposition de texte)
% dans les cas li\'es. C'est pourquoi depuis la v1.5q, |\raggedcolumns|
% permettent \`a toutes les colonnes d'\^etre l\'eg\`erement courtes.
% \changes{v1.5q}{1998/01/19}{Pas de remise \`a z\'ero de \cs{mult@firstbox} (rapport pr2739)}
%    \begin{macrocode}
%    \ifshr@nking
%      \global\setbox\mult@firstbox
%             \copy\mult@nat@firstbox
%    \fi
%    \end{macrocode}
% Puis, nous donnons des informations sur la derni\`ere
% colonne\,\footnote{Avec \TeX{} version~3.141, il est
% maintenant possible d'utiliser \cs{newlinechar}
% de \LaTeX{} dans la commande \cs{message}, mais les
% utilisateurs d'anciennes versions de \TeX{} obtiendront
% d\'esormais \texttt{\string^\string^J} plut\^ot
% qu'une nouvelle ligne \`a l'\'ecran}.
% \changes{v1.4a}{1992/02/12}{Modification du bon \cs{endlinechar} dans \cs{message}}
%    \begin{macrocode}
    \ifnum\c@tracingmulticols>\@ne
      \message{<> last column =
               \the\ht\mult@grightbox^^J}%
%    \end{macrocode}
% Du code de tra\c{c}age que nous ne compilons pas dans la
% version de production \`a moins qu'il ne soit demand\'e.
% Il produira d'\'enormes listes de bo\^\i{}tes impliqu\'ees
% dans l'\'equilibrage dans le fichier log.
%    \begin{macrocode}
%<*debug>
      \ifnum\c@tracingmulticols>4
         {\showoutput
          \batchmode
          \process@cols\@ne
           {\showbox\count@}}%
          \errorstopmode
      \fi
%</debug>
     \fi
%    \end{macrocode}
% Nous v\'erifions que notre essai est r\'eussi. Le test
% utilis\'e est tr\`es simple : nous comparons simplement la
% premi\`ere et la derni\`ere colonne. Ainsi, les colonnes
% interm\'ediaires peuvent \^etre plus longues que la
% premi\`ere si |\raggedcolumns| est utilis\'e. Si la colonne
% la plus \`a droite est plus longue que la prem\`ere, alors
% nous recommen\c{c}ons avec une plus grande valeur de |\dimen@|.
% \changes{v1.3c}{1991/03/03}{\cs{global}\cs{advance} remis \`a partir de l'ancien code}
%    \begin{macrocode}
    \ifdim\ht\mult@grightbox >\dimen@
%    \end{macrocode}
% Si la hauteur de la derni\`ere bo\^\i{}te est trop grande,
% nous signalons cet essai comme infructueux.
% \changes{v1.5v}{1999/07/18}{Ajout de rapport de
%                             tra\c{c}age pour les essais infructueux}
%    \begin{macrocode}
%<*badness>
      \too@badtrue
      \ifnum\c@tracingmulticols>\@ne
         \typeout{Rejected: last
                 column too large!}%
      \fi
    \else
%    \end{macrocode}
% \changes{v1.5v}{1999/07/18}{V\'erification de la
%                             derni\`ere colonne : si elle contient une
%                             rupture forc\'ee, l'essai est rejet\'e}
%
% Pour nous assurer qu'il n'y a pas de rupture forc\'ee dans
% la derni\`ere colonne, nous essayons d'extraire une
% bo\^\i{}te de dimension |\maxdimen| de |\mult@grightbox|
% (ou mieux, d'une de ses copies). Cela devrait produire une
% bo\^\i{}te vide apr\`es l'extraction, \`a moins qu'il n'y
% eut une rupture forc\'ee quelque part dans la colonne, dans
% la cas o\`u le mat\'eriel apr\`es la rupture serait rest\'e
% dans la bo\^\i{}te.
%    \begin{macrocode}
%<*colbreak>
      \setbox\@tempboxa
           \copy\mult@grightbox
      \setbox\z@\vsplit\@tempboxa to\maxdimen
      \ifvoid\@tempboxa
%</colbreak>
%    \end{macrocode}
% Donc, si |\@tempboxa| est vide nous avons une solution
% valide. Dans ce cas, nous pr\^etons un regard attentif \`a
% la derni\`ere colonne pour d\'ecider si cette colonne doit
% \^etre aussi longue que les autres ou si elle peut se
% permettre d\^etre plus courte.
% Pour cela, nous devons tout d'abord remettre en
% bo\^\i{}te la colonne dans une bo\^\i{}te de hauteur
% ad\'equate. Si le tra\c{c}age est activ\'e, nous montrons
% alors la laideur de cette bo\^\i{}te.
%    \begin{macrocode}
        \global\setbox\mult@grightbox
           \vbox to\dimen@
              {\unvbox\mult@grightbox}%
        \ifnum\c@tracingmulticols>\@ne
          \message{Final badness:
                   \the\badness}%
        \fi
%    \end{macrocode}
% Nous comparons ensuite cette laideur avec la m\'ediocrit\'e
% autoris\'ee pour la colonne finale. Si elle n'exc\`ede pas
% cette valeur, nous utilisons la bo\^\i{}te, sinon nous
% remettons en bo\^\i{}te une fois de plus et ajoutons de la
% glue au d\'ebut.
%    \begin{macrocode}
        \ifnum\badness>\c@finalcolumnbadness
          \global\setbox\mult@grightbox
           \vbox to\dimen@
              {\unvbox\mult@grightbox\vfill}%
           \ifnum\c@tracingmulticols>\@ne
             \message{ setting natural
              (> \the\c@finalcolumnbadness)}%
           \fi
        \fi
%    \end{macrocode}
% Si |\@tempboxa| ci-dessus n'est pas vide, notre essai est
% infructueux, nous rapportons donc ce fait et r\'eessayons.
%    \begin{macrocode}
%<*colbreak>
      \else
        \too@badtrue
        \ifnum\c@tracingmulticols>\@ne
          \typeout{Rejected: unprocessed
            forced break(s) in last column!}%
        \fi
      \fi
    \fi
%</colbreak>
%    \end{macrocode}
% Si la hauteur naturelle de la premi\`ere bo\^\i{}te est
% plus petite que la taille donn\'ee par l'essai en cours,
% mais plus grande que celle donn\'ee par l'essai
% pr\'ec\'edent, il est probable que nous avons oubli\'e une
% solution potentiellement meilleure (cela peut arriver si,
% pour quelque raison, la taille obtenue apr\`es notre
% premier essai \'etait trop grande). Dans ce cas, nous
% \'ecartons cet essai et red\'emarrons un nouvel essai avec la taille
% naturelle.
%    \begin{macrocode}
    \ifdim\ht\mult@nat@firstbox<\dimen@
      \ifdim\ht\mult@nat@firstbox>\last@try
        \too@badtrue
        \ifnum\c@tracingmulticols>\@ne
           \typeout{Retry: using natural
                    height of first column!}%
        \fi
        \dimen@\ht\mult@nat@firstbox
        \last@try\dimen@
        \advance\dimen@-\p@
      \fi
    \fi
%    \end{macrocode}
% Finalement, le commutateur |too@bad| est test\'e. S'il a
% \'et\'e fix\'e \`a vrai, soit plus t\^ot, soit \`a cause
% d'une colonne de droite trop grande, nous essayons de
% nouveau avec une valeur de |\dimen@| l\'eg\`erement plus
% grande.
%    \begin{macrocode}
    \iftoo@bad
%</badness>
      \advance\dimen@\p@
    \repeat
%    \end{macrocode}
% \`A ce moment, |\dimen@| prend la hauteur qui a \'et\'e
% d\'etermin\'ee par la boucle d'\'equilibrage.
% Si cette hauteur de colonne devient plus grande que
% l'espace disponible (qui est |\@colroom|) nous faisons
% entrer de force les colonnes dans l'espace, en supposant
% qu'elles auront assez de contractibilit\'e pour le
% faire\,\footnote{Ce sera peut \^etre mauvais, puisque la
% contractibilit\'e prise en compte pour la quantit\'e de
% mat\'eriel ne pourrait \^etre disponible que pour certaines
% colonnes. Mais il est mieux de le tester que de le faire
% directement}.
% \changes{v1.3c}{1991/03/03}{Limitation de la hauteur de colonne \`a \cs{@colroom}}
% \changes{v1.5q}{1998/01/19}{Suppression de la fixation de \cs{dimen@} (rapport pr2739)}
%    \begin{macrocode}
   \ifdim\dimen@>\@colroom
      \dimen@\@colroom
   \fi
%    \end{macrocode}
% Puis, nous d\'epla\c{c}ons le contenu des registres de
% bo\^\i{}te impairs vers les registres pairs, en
% l'\'etr\'ecissant s'il le faut. Nous devons utiliser |\vbox|
% et non |\vtop| (comme c'\'etait fait dans les premi\`eres
% versions) puisque, sinon, les bo\^\i{}tes r\'esultantes
% n'auront pas de hauteur (\TB\/ page 81). Cela devrait dire
% que le |\topskip| suppl\'ementaire est ajout\'e quand les
% bo\^\i{}tes sont rendues au {\it page\_builder} via |\page@sofar|.
% \changes{v1.3a}{1990/05/20}{Changement de \cs{vtop} en \cs{vbox}.}
%    \begin{macrocode}
   \process@cols\mult@rightbox
       {\@tempcnta\count@
        \advance\@tempcnta\@ne
        \setbox\count@\vbox to\dimen@
           {%
%    \end{macrocode}
%
%    \begin{macrocode}
            \vskip \z@
              \@plus-\multicolundershoot
              \@minus-\multicolovershoot
            \unvbox\@tempcnta
            \ifshr@nking\vfill\fi}}%
}
%    \end{macrocode}
%  \end{macro}
%
% \end{multicols}
%
% \begin{multicols}{2}[\subsection{Les allocations de bo\^\i{}tes}]
%
% \begin{macro}{\mult@rightbox}
% \begin{macro}{\mult@grightbox}
% \begin{macro}{\mult@firstbox}
% \begin{macro}{\mult@gfirstbox}
% Les pr\'ec\'edentes versions de ces macros utilisaient les
% premiers registres de bo\^\i{}te 0, 2, 4, ... pour les
% bo\^\i{}tes globales et 1, 3, 5, ... pour les
% bo\^\i{}tes locales correspondantes (Vous pourriez encore
% trouver des traces de ce setup dans la documentation.
% *soupir*). Cela pose un probl\`eme \`a partir du moment
% o\`u nous avons plus de 5 colonnes puisque les bo\^\i{}tes
% officiellement allou\'ees seront r\'e\'ecrites par
% l'algorithme. La nouvelle version utilise maintenant des
% registres de bo\^\i{}te priv\'es.
%    \begin{macrocode}
\newbox\mult@rightbox
\newbox\mult@grightbox
\newbox\mult@gfirstbox
\newbox\mult@firstbox
\newbox\@tempa\newbox\@tempa
\newbox\@tempa\newbox\@tempa
\newbox\@tempa\newbox\@tempa
\newbox\@tempa\newbox\@tempa
\newbox\@tempa\newbox\@tempa
\newbox\@tempa\newbox\@tempa
\newbox\@tempa\newbox\@tempa
\newbox\@tempa\newbox\@tempa
\newbox\@tempa
\let\@tempa\relax
%    \end{macrocode}
% \end{macro}
% \end{macro}
% \end{macro}
% \end{macro}
%
%
% \end{multicols}
%
%
% \begin{multicols}{2}[\section{Nouvelles macros et bidouilles pour la version 1.2}]
%
% \begin{macro}{\emergencystretch}
% \begin{macro}{\setemergencystretch}
% Si nous n'utilisons pas \TeX{}~3.0, |\emergencystretch|
% n'est pas d\'efini, donc, dans ce cas, nous ajoutons
% simplement un registre \meta{dimen} non utilis\'e.
% \changes{v1.4j}{1992/06/25}{Insertion de \cs{emergencystretch} en haut
%                  retir�.}
%    \begin{macrocode}
\@ifundefined{emergencystretch}
     {\newdimen\emergencystretch}{}
%    \end{macrocode}
% \changes{v1.2a}{1990/02/05}{Macro ajout\'ee.}
% Mes tests montrent que la formule suivante fonctionne assez
% bien. N\'eanmoins, la macro |\setemergencystretch| prend
% aussi |\hsize| en second argument pour permettre \`a
% l'utilisateur d'essayer diff\'erentes formules.
%    \begin{macrocode}
\def\setemergencystretch#1#2{%
   \emergencystretch 4pt
   \multiply\emergencystretch#1}
%    \end{macrocode}
% \end{macro}
% \end{macro}


% \begin{macro}{\set@floatcmds}
% \changes{v1.2a}{1990/02/05}{Macro ajout\'ee.}
% \changes{v1.5g}{1994/06/07}{Mise \`a jour, puisque les flottants ont chang\'e}
% \changes{v1.5j}{1994/06/07}{Mise \`a jour, puisque les flottants ont encore chang\'e}
% \changes{v1.5l}{1995/10/19}{Ajout de \cmd{@largefloatcheck}}
% M\^eme si cela doit \^etre utilis\'e comme point
% d'entr\'ee, nous utilisons |@| dans le nom, puisque c'est
% plus pour des experts.
%    \begin{macrocode}
\def\set@floatcmds{%
 \let\@dblfloat\@dbflt
 \def\end@dblfloat{\par
   \vskip\z@
   \egroup
   \color@endbox
   \@largefloatcheck
   \outer@nobreak
%    \end{macrocode}
% C'est facile (de diff\'erer les flottants jusqu'apr\`es la
% page courante) mais toute autre solution obligerait \`a
% s'enfoncer dans la routine de sortie de \LaTeX{}, et je
% n'aime pas y travailler tant que je ne sais pas quelle
% partie de la routine de sortie devra \^etre
% r\'eimpl\'ement\'ee pour \LaTeX3.
%    \begin{macrocode}
   \ifnum\@floatpenalty<\z@
%    \end{macrocode}
% Nous devons ajouter le flottant \`a |\@deferlist| parce que
% nous supposons qu'en dehors de l'environnement \mc{}, nous
% sommes en mode simple colonne. Ce n'est pas totalement
% correct, j'ai d\'ej\`a utilis\'e l'environnement \mc{} dans
% une d\'eclaration |\twocolumn| de \LaTeX{}, mais cela ira
% pour beaucoup d'applications.
%    \begin{macrocode}
     \@cons\@deferlist\@currbox
   \fi
   \ifnum\@floatpenalty=-\@Mii
     \@Esphack
   \fi}}
%    \end{macrocode}
% \end{macro}
%
% \end{multicols}
%
% \begin{multicols}{2}[\subsection{Maintenance des registres de marque}]
% \label{sec:v14}
%
% Cette section contient les routines qui fixent les marques
% pour qu'elles soient prises correctement en compte. Elles
% ont \'et\'e introduites avec la version~1.4.
%
%  \begin{macro}{\kept@topmark}
% \changes{v1.4h}{1992/06/04}{Initialisation des paires de doubles accolades}
%  \begin{macro}{\kept@firstmark}
%  \begin{macro}{\kept@botmark}
% La premi\`ere chose que nous faisons est de r\'eserver
% trois noms de macro, pour prendre le texte de remplacement,
% pour les primitives de \TeX{}, |\firstmark|, |\botmark| et
% |\topmark|. Nous initialisons les deux premi\`eres pour
% qu'elles soient vides, et |\kept@topmark| qui contient deux
% paires d'accolades vides. C'est n\'ecessaire puisque |\kept@topmark|
% est suppos\'e contenir la derni\`ere marque provenant d'une
% page pr\'ec\'edente, et, dans \LaTeX{}, toute marque
% <<~r\'eelle~>> doit contenir deux parties, repr\'esentant
% les informations des marques droites et gauches.
%    \begin{macrocode}
\def\kept@topmark{{}{}}
\let\kept@firstmark\@empty
\let\kept@botmark\@empty
%    \end{macrocode}
%  \end{macro}
%  \end{macro}
%  \end{macro}
%
%
%  \begin{macro}{\return@nonemptymark}
% Quelquefois, nous voulons retourner la valeur d'une marque
% <<~conserv\'ee~>> dans un n\oe{}ud |\mark| de la liste
% verticale principale. C'est r\'ealis\'e par la fonction
% |\return@nonemptymark|. Comme le nom le sugg\`ere, elle
% agit seulement si le texte de remplacement de la marque
% conserv\'ee n'est pas vide. C'est fait pour \'eviter
% d'ajouter une marque vide, quand aucune marque n'est
% effectivement pr\'esente. Si, n\'eanmoins, nous voulons
% ajouter une telle marque, elle doit \^etre consid\'er\'ee
% plus tard comme une |\firstmark| valide.
%    \begin{macrocode}
\def\return@nonemptymark#1#2{%
  \ifx#2\@empty
  \else
%    \end{macrocode}
% Dans un but de d\'ebuggage, nous jetons un coup d'\oe{}il
% \`a la valeur de la marque conserv\'ee que nous voulons
% retourner. Ce code le fera.
%    \begin{macrocode}
%<*marktrace>
    \mult@info\tw@
      {Returned #1 mark:\MessageBreak
       \meaning#2}%
%      \nobreak
%    \fi
%</marktrace>
%    \end{macrocode}
% Puisque le contenu de la marque peut \^etre un code
% \LaTeX{} arbitraire, nous nous assurons mieux de son
% incapacit\'e \`a entra\^\i{}ner une expansion quelconque (des
% expansions ont \'et\'e d\'ej\`a r\'ealis\'ees pendant
% l'ex\'ecution de |\markright| ou |\markboth|). Nous
% utilisons donc le m\'ecanisme habituel d'un registre de
% token pour interdire l'expansion\,\footnote{\`A cause de la
% d\'efinition courante de \cs{markright}, etc., ce ne serait
% pas utile de d\'efinir la commande \cs{protect} pour
% interdire l'expansion, puisque tout \cs{protect} a
% d\'ej\`a disparu \`a cause des expansions pr\'ec\'edentes}.
% \changes{v1.4n}{1992/09/10}{Consolidation des marques}
% \changes{v1.5t}{1999/03/22}{Rajout de la commande \cs{mark} qui avait
%                             \'et\'e comment\'ee ici par erreur en 1998
%                             (rapport pr/2978)}
%    \begin{macrocode}
    \toks@\expandafter{#2}%
    \mark{\the\toks@}%
%    \end{macrocode}
% Nous ne voulons aucun point de rupture entre une telle
% marque retourn\'ee et le mat\'eriel suivant (qui, habituellement, est
% juste la bo\^\i{}te d'o\`u venait la marque).
%    \begin{macrocode}
    \nobreak
  \fi}
%    \end{macrocode}
%  \end{macro}
%
%
%  \begin{macro}{\get@keptmarks}
% Si nous avons du mat\'eriel dans un registre de bo\^\i{}te,
% nous voulons, peut-\^etre, mettre la premi\`ere et la
% derni\`ere marques hors de cette bo\^\i{}te. Cela peut
% \^etre r\'ealis\'e par |\get@keptmarks| qui prend un
% argument : le num\'ero du registre de bo\^\i{}te ou son
% alias, d\'efini par |\newbox|.
%    \begin{macrocode}
\def\get@keptmarks#1{%
%    \end{macrocode}
% Dans un but de d\'ebuggage, nous regardons les dimensions
% courantes de la bo\^\i{}te car dans les anciennes
% versions du code, j'ai fait quelque erreurs dans ce
% domaine.
%    \begin{macrocode}
%<*debug>
      \typeout{Mark box #1 before:
               ht \the\ht#1, dp \the\dp#1}%
%</debug>
%    \end{macrocode}
% Maintenant, nous ouvrons un nouveau groupe et, localement,
% nous y copions la bo\^\i{}te. En cons\'equence, toute
% op\'eration, \emph{i. e.} |\vsplit|, auront seulement un
% effet local. Sans cette astuce, le contenu de la bo\^\i{}te
% perdrait le niveau auquel le dernier assignement du
% registre de bo\^\i{}te a \'et\'e r\'ealis\'e.
%    \begin{macrocode}
  \begingroup
   \vbadness\@M
   \setbox#1\copy#1%
%    \end{macrocode}
% Maintenant, nous \'etendons la bo\^\i{}te jusqu'\`a la
% dimension maximale possible. Cela doit s\'eparer le contenu
% entier de la bo\^\i{}te donc, tout est r\'eellement
% s\'epar\'e. Par cons\'equent, |\splitfirstmark| et |\splitbotmark|
% contiendront respectivement la premi\`ere et la derni\`ere
% marque.
%    \begin{macrocode}
   \setbox#1\vsplit#1to\maxdimen
%    \end{macrocode}
% Donc, nous pouvons maintenant fixer les marques
% conserv\'ees, ce qui est une op\'eration globale, et,
% ensuite, fermer le groupe. Cela restaurera le contenu
% original de la bo\^\i{}te.
%    \begin{macrocode}
   \set@keptmarks
 \endgroup
%    \end{macrocode}
% Pour d\'ebugger, nous regardons encore la dimension de la
% bo\^\i{}te qui ne doit pas avoir chang\'e.
%    \begin{macrocode}
%<*debug>
    \typeout{Mark box #1 \space after:
             ht \the\ht#1, dp \the\dp#1}%
%</debug>
}
%    \end{macrocode}
%  \end{macro}
%
%
%  \begin{macro}{\set@keptmarks}
% La macro |\set@keptmarks| est responsable du r\'eglage de
% |\kept@firstmark| et |\kept@botmark|, en v\'erifiant la
% valeur courante de |\splitfirstmark| et |\splitbotmark|.
%    \begin{macrocode}
\def\set@keptmarks{%
%    \end{macrocode}
% Si |\kept@firstmark| est vide, nous supposons qu'elle n'est pas
% configur\'ee. Strictement parlant, cela n'est pas correct
% puisque nous avons perdu la capacit\'e d'avoir des marques
% explicitement vides, mais pour les applications normales de
% \LaTeX{}, cela est suffisant. Si elle n'est pas vide, nous
% ne changeons pas la valeur -- ensuite, dans les routines de sortie,
% elle sera restaur\'ee comme |\@empty|.
%    \begin{macrocode}
   \ifx\kept@firstmark\@empty
%    \end{macrocode}
% Nous mettons maintenant le contenu de |\splitfirstmark|
% dans |\kept@firstmark|. Dans le cas o\`u il n'y avait pas du
% tout de marque, |\kept@firstmark| ne sera pas modifi\'ee
% par cette op\'eration.
%    \begin{macrocode}
     \expandafter\gdef\expandafter
        \kept@firstmark
        \expandafter{\splitfirstmark}%
%    \end{macrocode}
% Au cours du d\'ebuggage, nous montrons les assignements
% mais seulement si quelque chose se produit r\'eellement.
%    \begin{macrocode}
%<*marktrace>
     \ifx\kept@firstmark\@empty\else
       \mult@info\tw@
         {Set kept first mark:\MessageBreak
          \meaning\kept@firstmark%
          \@gobbletwo}%
     \fi
%</marktrace>
   \fi
%    \end{macrocode}
% Nous essayons toujours de fixer la marque du bas \`a |\splitbotmark|
% mais, bien s\^ur, seulement quand il y a
% |\splitbotmark|. De nouveau, nous supposons que si |\splitbotmark|
% est vide, cela signifie que la partie s\'epar\'ee de la
% bo\^\i{}te ne contient plus de marque.
%    \begin{macrocode}
   \expandafter\def\expandafter\@tempa
      \expandafter{\splitbotmark}%
   \ifx\@tempa\@empty\else
      \global\let\kept@botmark\@tempa
%<*marktrace>
      \mult@info\tw@
        {Set kept bot mark:\MessageBreak
         \meaning\kept@botmark%
         \@gobbletwo}%
%</marktrace>
   \fi}%
%    \end{macrocode}
%  \end{macro}
%
%
%  \begin{macro}{\prep@keptmarks}
% La fonction |\prep@keptmarks| est utilis\'ee pour
% initialiser les marques conserv\'ees \`a partir du contenu
% de |\partial@page|, \emph{i. e.} la bo\^\i{}te qui prend
% tout \`a partir du haut de la page courante avant de
% d\'emarrer l'environnement \mc{}. Cependant, une telle
% bo\^\i{}te est valable seulement si nous ne produisons pas
% de \mc{} embo\^\i{}t\'e.
%    \begin{macrocode}
\def\prep@keptmarks{%
   \if@boxedmulticols \else
     \get@keptmarks\partial@page
   \fi}
%    \end{macrocode}
%  \end{macro}
%
%
%  \begin{macro}{\remove@discardable@items}
%
%    \begin{macrocode}
\def\remove@discardable@items{%
%<*debug>
   \edef\@tempa{s=\the\lastskip,
                p=\the\lastpenalty,
                k=\the\lastkern}%
   \typeout\@tempa
%</debug>
        \unskip\unpenalty\unkern
%<*debug>
   \edef\@tempa{s=\the\lastskip,
                p=\the\lastpenalty,
                k=\the\lastkern}%
   \typeout\@tempa
%</debug>
        \unskip\unpenalty\unkern
%<*debug>
   \edef\@tempa{s=\the\lastskip,
                p=\the\lastpenalty,
                k=\the\lastkern}%
   \typeout\@tempa
%</debug>
        \unskip\unpenalty\unkern
%<*debug>
   \edef\@tempa{s=\the\lastskip,
                p=\the\lastpenalty,
                k=\the\lastkern}%
   \typeout\@tempa
%</debug>
        \unskip\unpenalty\unkern
}
%    \end{macrocode}
%  \end{macro}
%
%    \begin{macrocode}
%<*badness>
\newif\iftoo@bad
%    \end{macrocode}
%
% \begin{macro}{\c@columnbadness}
% \begin{macro}{\c@finalcolumnbadness}
%    \begin{macrocode}
\newcount\c@columnbadness
\c@columnbadness=10000
\newcount\c@finalcolumnbadness
\c@finalcolumnbadness=9999

\newdimen\last@try

\newdimen\multicolovershoot
\multicolovershoot=2pt
\newdimen\multicolundershoot
\multicolundershoot=2pt
\newbox\mult@nat@firstbox
%</badness>
%    \end{macrocode}
% \end{macro}
% \end{macro}
%
%  \begin{macro}{\mult@info}
% Une aide pour produire les messages d'information.
%    \begin{macrocode}
\def\mult@info#1#2{%
  \ifnum\c@tracingmulticols>#1%
   \GenericWarning
       {(multicol)\@spaces\@spaces}%
       {Package multicol: #2}%
  \fi
}
%    \end{macrocode}
%  \end{macro}
%
% \end{multicols}
%
%
% \begin{multicols}{2}[\section{R\'eglage de
%                        \cs{columnwidth}}]
%
%  \begin{macro}{\@footnotetext}
% \changes{v1.5o}{1997/11/16}{Red\'efinition ajout\'ee. rapport pr/2664.}
% \changes{v1.5r}{1998/08/17}{Utilisation de \cs{@footnotetext}
%                             mais avec un changement local de \cs{columnwidth}.}
%  \begin{macro}{\mult@footnotetext}
% \changes{v1.5r}{1998/08/17}{Macro de nouveau supprim\'ee.}
% Si nous stockons la largeur de la colonne courante dans
% |\columnwidth|, nous devons red\'efinir la macro interne
% |\@footnotetext| pour qu'elle utilise la largeur des notes
% de bas de page, plut\^ot que la d\'efinition originale.
%
% Depuis la v1.5r, cela est maintenant r\'ealis\'e de telle
% fa\c{c}on que la d\'efinition originale soit toujours
% utilis\'ee, sauf que, localement, |\columnwidth| est fix\'e
% \`a |\textwidth|.
%
% Cela r\'esoud deux probl\`emes : premi\`erement, les
% red\'efinitions de |\@footnotetext| r\'ealis\'ees par une
% classe seront toujours valides ; deuxi\`emement, si
% \mc{} est utilis\'e dans un environnement minipage, la
% d\'efinition sp\'eciale de |\@footnotetext| dans cet
% environnement sera prise et non celle du texte
% principal en cours de traitement (ce dernier devrait entra\^\i{}ner
% la perte de toutes les notes de bas de page dans ce cas).
%
% Voir la d\'efinition de la commande |\multicols| pour le
% code exact.
%  \end{macro}
%  \end{macro}
% \end{multicols}
%
%
% \begin{multicols*}{2}[\section{Extensions suppl\'ementaires}]
%
% Cette section contient le code pour les extensions
% ajout\'ees \`a ce package. Toutes peuvent \^etre
% actives, certaines peuvent \^etre dormantes et en
% attente de leur activation dans de futures versions.
%
% \subsection{Non-\'equilibrage des colonnes}
%
% C'est assez facile \`a impl\'ementer. Nous avons juste \`a
% inactiver la routine de sortie d'\'equilibrage et \`a la
% remplacer par celle qui exp\'edie les autres pages.
%
%  \begin{macro}{\multicols*}
% \changes{v1.5q}{1998/01/19}{Macro ajout\'e}
% Le code pour cet environnement a \'et\'e propos\'e par Matthias~\bsc{Clasen}.
%    \begin{macrocode}
%<*nobalance>
 \@namedef{multicols*}{%
%    \end{macrocode}
% Si nous sommes dans le texte principal en cours de traitement,
% \emph{i. e.} dans une bo\^\i{}te de cette sorte, cette
% approche ne fonctionnera pas, puisque nous n'avons pas de
% dimension verticale, il est donc pr\'ef\'erable d'avertir
% l'utilisateur plut\^ot que de r\'ealiser l'\'equilibrage.
%    \begin{macrocode}
   \ifinner
     \PackageWarning{multicol}%
       {multicols* inside a box does
        not make sense.\MessageBreak
        Going to balance anyway}%
   \else
     \let\balance@columns@out
         \multi@column@out
   \fi
   \begin{multicols}
}
%    \end{macrocode}
%  \end{macro}
%
%  \begin{macro}{\endmulticols*}
% Quand l'environnement est fini, nous fermons simplement
% l'environnement \texttt{multicols} int\'erieur, sauf qu'il
% vaut mieux, \'egalement, \^etre dans un glue \'etirable
% verticalement, pour que la derni\`ere colonne, qui contient
% toujours le texte, ne soit pas \'etir\'ee verticalement de
% mani\`ere exag\'er\'ee.
% \changes{v1.5q}{1998/01/19}{Macro ajout\'ee}
%    \begin{macrocode}
 \@namedef{endmulticols*}{\vfill
   \end{multicols}}
%</nobalance>
%    \end{macrocode}
%  \end{macro}
%
%
% \subsection{Rupture manuelle des colonnes} \label{sec:colbreak}
%
% Le probl\`eme, avec les ruptures manuelles de page dans
% \mc{} est que, pendant la collecte du mat\'eriel pour
% toutes les colonnes, une p\'enalit\'e de <<~for\c{c}age de
% page~>> (\emph{i. e.} -10000 ou plus) pourrait arr\^eter la
% passe de collecte, ce qui n'est pas tout \`a fait ce qui est
% d\'esir\'e. D'un autre c\^ot\'e, l'utilisation d'une
% p\'enalit\'e telle que -9999 signifierait qu'il y aurait
% des cas pour lesquels les op\'erations |\vsplit| dans
% \mc{} ignorerait une telle p\'enalit\'e et choisirait
% encore un point de rupture diff\'erent.
%
% Pour cela, l'impl\'ementation actuelle utilise une approche
% totalement diff\'erente. En un mot, elle \'etend le
% traitement de la routine de sortie de \LaTeX{} en
% introduisant un drapeau de p\'enalit\'e suppl\'ementaire
% (\emph{i. e.} un p\'enalit\'e qui est forc\'ee mais
% sup\'erieure \`a -10000, pour que la routine de sortie
% puisse consid\'erer cette valeur et donc, savoir pourquoi
% elle a \'et\'e appel\'ee).
%
% \`A l'int\'erieur de la routine de sortie, nous testons
% cette valeur et si elle appara\^\i{}t, nous faisons deux
% choses : sauvegarde du texte en cours de traitement \`a partir de ce point
% dans une bo\^\i{}te sp\'eciale pour une utilisation
% ult\'erieure et r\'eduction de |\vsize| de la hauteur du
% mat\'eriel vu. De cette fa\c{c}on, la p\'enalit\'e est
% maintenant cach\'ee dans cette bo\^\i{}te et nous pouvons
% red\'emarrer le processus de collecte pour les colonnes
% restantes (cela est r\'ealis\'e par |\speci@ls| ci-dessus).
%
% Dans les routines de sortie qui appliquent |\vsplit| soit
% pour l'\'equilibrage soit pour une page enti\`ere, nous
% combinons simplement la bo\^\i{}te~255 et la bo\^\i{}te
% sauvegard\'ee, ce qui donne donc une bo\^\i{}te unique de
% s\'eparation, qui contient maintenant les ruptures
% forc\'ees \`a la bonne place.
%
%
%  \begin{macro}{\columnbreak}
% \changes{v1.5u}{1999/05/25}{Macro ajout\'ee}
% |\columnbreak| est construit d'apr\`es |\pagebreak|, sauf
% que nous produisons une p\'enalit\'e de -10005.
%    \begin{macrocode}
%<*colbreak>
\mathchardef\@Mv=10005
\def\columnbreak{%
%    \end{macrocode}
% Nous devons nous assurer qu'elle est utilis\'ee
% seulement \`a l'int\'erieur d'un environnement \mc{}
% car si nous avons une p\'enalit\'e pouvant \^etre
% consid\'er\'ee par la routine de sortie inchang\'ee de
% \LaTeX{}, des choses \'etranges pourraient appara\^\i{}tre.
%    \begin{macrocode}
 \ifnum\col@number<\tw@
  \PackageError{multicol}%
   {\noexpand\columnbreak outside multicols}%
   {This command can only be used within
    a multicols or multicols* environment.}%
 \else
  \ifvmode
    \penalty -\@Mv\relax
  \else
    \@bsphack
    \vadjust{\penalty -\@Mv\relax}%
    \@esphack
  \fi
 \fi}
%    \end{macrocode}
%  \end{macro}
%
%
%  \begin{macro}{\colbreak@box}
% \changes{v1.5u}{1999/05/25}{Macro ajout\'ee}
% N\'ecessite une bo\^\i{}te pour collecter le texte en cours de traitement \`a
% partir de la rupture de colonne.
%    \begin{macrocode}
\newbox\colbreak@box
%</colbreak>
%</package>
%    \end{macrocode}
%  \end{macro}
%
% \end{multicols*}
%
% \Finale
%
\endinput
