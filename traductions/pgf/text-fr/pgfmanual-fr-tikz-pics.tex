% Copyright 2013 by Till Tantau
%
% This file may be distributed and/or modified
%
% 1. under the LaTeX Project Public License and/or
% 2. under the GNU Free Documentation License.
%
% See the file doc/generic/pgf/licenses/LICENSE for more details.

\section{Pics: Small Pictures on Paths}

\label{section-pics}

\subsection{Overview}

A ``pic'' is a ``short picture'' (hence the short name\dots) that can
be inserted anywhere in \tikzname\ picture where you could also insert
a node. Similarly to nodes, pics have a ``shape'' (called \emph{type}
to avoid confusion) that someone has defined. Each time a pic of a
specified type is used, the type's code is executed, resulting in some
drawings to be added to the current picture. The syntax for adding
nodes and adding pics to a picture are also very similar. The core
difference is that pics are typically more complex than nodes and may
consist of a whole bunch of nodes themselves together with complex
paths joining them.

As a very simple example, suppose we want to define a pic type
|seagull| that just draw ``two bumps.'' The code for this definition
is quite easy:
\begin{codeexample}[code only]
\tikzset{ 
  seagull/.pic={
    % Code for a "seagull". Do you see it?...
    \draw (-3mm,0) to [bend left] (0,0) to [bend left] (3mm,0);
  }
}
\end{codeexample}

The first line just tells \TeX\ that you set some \tikzname\ options
for the current scope (which is the whole document); you could put
|seagull/.pic=...| anywhere else where \tikzname\ options are allowed
(which is just about anywhere). We have now defined a |seagull| pic
type and can use it as follows:

\tikzset{ 
  seagull/.pic={
    % Code for a "seagull". Do you see it?...
    \draw (-3mm,0) to [bend left] (0,0) to [bend left] (3mm,0);
  }
}
\begin{codeexample}[]
\tikz \fill [fill=blue!20]
     (1,1)
  -- (2,2) pic             {seagull}
  -- (3,2) pic             {seagull}
  -- (3,1) pic [rotate=30] {seagull}
  -- (2,1) pic [red]       {seagull};
\end{codeexample}

As can be see, defining new types of pics is much easier than defining
new shapes for nodes; but see Section~\ref{section-new-pic-types}
for the fine details.

Since defining new pics types is easier than defining new node shapes
and since using pics is as easy as using nodes, why should you use
nodes at all? There are chiefly two reasons:

\begin{enumerate}
\item Unlike nodes, pics cannot be referenced later on. You \emph{can}
  reference nodes that are inside a pic, but not ``the pic itself.''
  In particular, you cannot draw lines between pics the way you
  can draw them between nodes. In general, whenever it makes sense
  that some drawing could conceivably be connected to other
  node-like-things, then a node is better than a pic. 
\item If pics are used to emulate the full power of a node (which is
  possible, in principle), they will be slower to construct and take
  up more memory than a node achieving the same effect.
\end{enumerate}

Despite these drawbacks, pics are an excellent choice for creating
highly configurable reusable pieces of drawings that can be inserted
into larger contexts.


\subsection{The Pic Syntax}

\begin{command}{\pic}
  Inside |{tikzpicture}| this is an abbreviation for |\path pic|.
\end{command}

The syntax for adding a pic to a picture is very similar to the syntax
used for nodes (indeed, internally the same parser code is used). The
main difference is that instead of a node contents you provide the
picture's type between the  braces:

\begin{pathoperation}{pic}{\opt{\meta{foreach statements}}%
      \opt{|[|\meta{options}|]|}\opt{|(|\meta{prefix}|)|}%
    \opt{|at(|\meta{coordinate}|)|}\opt{\marg{pic type}}}
  
  Adds a pic to the current \tikzname\ picture of the specified
  \meta{pic type}. The effect is, basically, that some code associated
  with the \meta{pic type} is executed (how this works, exactly, is explained
  later). This code can consist of arbitrary \tikzname\ code. As for
  nodes, the current path will not be modified by this path command,
  all drawings produced by the code are ``external'' to the path the
  same way neither a node nor its border are part of the path on which
  they are specified. 
  
  Just like the |node| command, this path operation is somewhat
  complex and we go over it step by step.

  \medskip
  \textbf{Order of the parts of the specification.}
  Just like for nodes, everything between ``|pic|'' and the opening
  brace of the \meta{pic type} is optional and can be given in any
  order. If there are \meta{foreach statements}, they must come
  first, directly following ``|pic|.'' As for nodes, the ``end'' of
  the pic specification is normally detected by the presence of the
  opening brace. You can, however, use the |pic type|
  option to specify the pic type as an option.
  
  \begin{key}{/tikz/pic type=\meta{pic type}}
    This key sets the pic type of the current~|pic|. When this option
    is used inside an option block of a |pic|, the parsing of the
    |pic| ends immediately and no pic type in braces is expected. (In
    other words, this option behaves exactly like the |node contents|
    option and, indeed, the two are interchangeable.) 
\begin{codeexample}[]
\tikz {
  \path (0,0) pic [pic type = seagull]
        (1,0) pic                      {seagull};
}
\end{codeexample}
  \end{key}

  \medskip
  \textbf{The location of a pic.}
  Just like nodes, pics are placed at the last position mentioned on
  the path or, when |at| is used, at a specified position. ``Placing''
  a pic somewhere actually means that the coordinate system is
  translated (shifted) to this last position. This means that inside
  of the pic type's code any mentioning of the origin refers to the
  last position used on the path or to the specified |at|.

\begin{codeexample}[]
\tikz { % different ways of placing pics
  \draw [help lines] (0,0) grid (3,2);
  \pic  at (1,0)     {seagull};
  \path (2,1)    pic {seagull};
  \pic  [at={(3,2)}] {seagull};
}
\end{codeexample}

  As for nodes, except for the described shifting, the coordinate
  system of a pic is reset prior to executing the pic type's
  code. This can be changed using the |transform shape| option, which
  has the same effect as for nodes: The ``outer'' transformation gets
  applied to the node:
\begin{codeexample}[]
\tikz [scale=2] { 
  \pic at (0,0)                   {seagull};
  \pic at (1,0) [transform shape] {seagull};
}
\end{codeexample}

  When the \meta{options} contain transformation commands like |scale|
  or |rotate|, these transformations always apply to the pic:
\begin{codeexample}[]
\tikz [rotate=30] { 
  \pic at (0,0)             {seagull};
  \pic at (1,0) [rotate=90] {seagull};
}
\end{codeexample}

  Just like nodes, pics can also be positioned implicitly and,
  somewhat unsurprisingly, the same rules concerning positioning and
  sloping apply:
\begin{codeexample}[]
\tikz \draw
  (0,0) to [bend left] 
           pic [near start]       {seagull} 
           pic                    {seagull}
           pic [sloped, near end] {seagull} (4,0);
\end{codeexample}
  
  \medskip
  \textbf{The options of a node.}
  As always, any given \meta{options} apply only to the pic and have
  no effect outside. As for nodes, most ``outside'' options also apply
  to the pics, but not the ``action'' options like |draw| or
  |fill|. These must be given in the \meta{options} of the pic.

  \medskip
  \textbf{The code of a pic.}
  As stated earlier, the main job of a pic is to execute some code in
  a scope that is shifted according to the last point on the path or
  to the |at| position specified in the pic. It was also claimed that
  this code is specified by the \meta{pic type}. However, this
  specification is somewhat indirect. What really happens is the
  following: When a |pic| is encountered, the current path is
  suspended and a new internal scope is started. The \meta{options}
  are executed and also the \meta{pic type} (as explained
  in a moment). After all this is done, the code stored in the following key
  gets executed:

  \begin{key}{/tikz/pics/code=\meta{code}}
    This key stores the \meta{code} that should be drawn in the
    current pic. Normally, setting this key is done by the \meta{pic
      type}, but you can also set it in the \meta{options} and leave
    the \meta{pic type} empty:
\begin{codeexample}[]
\tikz \pic [pics/code={\draw (-3mm,0) to[bend left] (0,0)
                                      to[bend left] (3mm,0);}]
      {}; % no pic type specified
\end{codeexample}
  \end{key}
  
  Now, how does the \meta{pic type} set |pics/code|? It turns out that
  the \meta{pic type} is actually just a list of keys that are
  executed with the prefix |/tikz/pics/|. In the above examples, this
  ``list of keys'' just consisted of the single key ``|seagull|'' that
  did not take any arguments, but, in principle, you could provide any
  arbitrary text understood by |\pgfkeys| here. This means that when
  we write |pic{seagull}|, \tikzname\ will execute the key
  |/tikz/pics/seagull|. It turns out, see
  Section~\ref{section-new-pic-types}, that this key is just a style
  set to |code={\draw(-3mm,0)...;}|. Thus, |pic{seagull}| will cause
  the |pics/code| key to be set to the text needed to draw the
  seagull.
  
  Indeed, you can also use the \meta{pic type} simply to set the
  |code| of the pic. This is useful for cases when you have some code
  that you ``just want to execute, but do not want to define a new pic
  type.'' Here is a typical example where we use pics to add some
  markings to a path:
\begin{codeexample}[]
\tikz \draw (0,0) .. controls(1,0) and (2,1) .. (3,1)
  foreach \t in {0, 0.1, ..., 1} {
    pic [pos=\t] {code={\draw circle [radius=2pt];}}
  };
\end{codeexample}

  
  In our seagull example, we always explicitly used |\draw| to draw
  the seagull. This implies that when a user writes something
  |pic[fill]{seagull}| in the hope of having a ``filled'' seagull,
  nothing special will happen: The |\draw| inside the pic explicitly
  states that the path should be drawn, not filled, and the fact that
  in the surrounding scope the |fill| option is set has no effect.
  The following key can be used to change this:
  \begin{key}{/tikz/pic actions}
    This key is a style that can be used (only) inside the code of a
    pic. There, it will set the ``action'' keys set inside the
    \meta{options} of the pic (``actions'' are drawing, filling,
    shading, and clipping or any combination thereof).

    To see how this key works, let us define the following pic:
\begin{codeexample}[code only]
\tikzset{
  my pic/.pic = {
    \path [pic actions] (0,0) circle[radius=3mm];
    \draw (-3mm,-3mm) rectangle (3mm,3mm);
  }
}
\end{codeexample}
    In the code, whether or not the circle gets drawn/filled/shaded
    depends on which options where given to the |pic| command when it
    is used. In contrast, the rectangle will always (just) be drawn.
\tikzset{
  my pic/.pic = {
    \path [pic actions] (0,0) circle[radius=3mm];
    \draw (-3mm,-3mm) rectangle (3mm,3mm);
  }
}
\begin{codeexample}[width=6cm]
\tikz \pic                      {my pic}; \space
\tikz \pic [red]                {my pic}; \space
\tikz \pic [draw]               {my pic}; \space
\tikz \pic [draw=red]           {my pic}; \space
\tikz \pic [draw, shading=ball] {my pic}; \space
\tikz \pic [fill=red!50]        {my pic};
\end{codeexample}
  \end{key}
  
  \medskip
  \textbf{Code executed behind or in front of the path.}
  As for nodes, a pic can be ``behind'' the current path or ``in front
  of it'' and, just as for nodes, the two options |behind path| and
  |in front of path| are used to specify which is meant. In detail, if
  |node| and |pic| are both used repeatedly on a path, in the
  resulting picture we first see all nodes and pics with the
  |behind path| option set in the order they appear on the path (nodes
  and pics are interchangeable in this regard), then comes the path,
  and then come all nodes and pics that are in front of the path in
  the order they appeared.
\begin{codeexample}[]
\tikz \fill [fill=blue!20]
     (1,1)
  -- (2,2) pic [behind path]      {seagull}
  -- (3,2) pic                    {seagull}
  -- (3,1) pic [rotate=30]        {seagull}
  -- (2,1) pic [red, behind path] {seagull};
\end{codeexample}
  In contrast to nodes, a pic need not only be completely behind
  the path or in front of the path as specified by the user. Instead,
  a pic type may also specify that a certain part of the drawing
  should always be behind the path and it may specify that a certain
  other part should always be before the path. For this, the values of
  the following keys are relevant:
  
  \begin{key}{/tikz/pics/foreground code=\meta{code}}
    This key stores \meta{code} that will always be drawn in front of
    the current path, even when |behind path| is used. If
    |behind path| is not used and |code| is (also) set, the code of
    |code| is 
    drawn first, following by the foreground \meta{code}.
  \end{key}
  
  \begin{key}{/tikz/pics/background code=\meta{code}}
    Like |foreground code|, only that the \meta{code} is always put
    behind the path, never in front of it.
  \end{key}
  
  \medskip
  \textbf{The foreach statement for pics.}
  As for nodes, a pic specification may start with |foreach|. The
  effect and semantics are the same as for nodes.
\begin{codeexample}[]
\tikz \pic foreach \x in {1,2,3} at (\x,0) {seagull};
\end{codeexample}

  \medskip
  \textbf{Styles for pics.}
  The following styles influence how nodes are rendered:
  \begin{stylekey}{/tikz/every pic (initially \normalfont empty)}
    This style is installed at the beginning of every pic.
\begin{codeexample}[]
\begin{tikzpicture}[every node/.style={draw}]
  \draw (0,0) node {A} -- (1,1) node {B};
\end{tikzpicture}
\end{codeexample}
  \end{stylekey}

  
  \medskip
  \textbf{Name scopes.}
  You can specify a \meta{name} for a pic using the key
  |name=|\meta{name} or by giving the name in parenthesis inside the
  pic's specification. The effect of this is, for once, quite
  different from what happens for nodes: All that happens is that
  |name prefix| is set to \meta{name} at the beginning of the pic.
  
  The |name prefix| key was already introduced in the description of
  the |node| command: It allows you to set some text that is prefixed
  to all nodes in a scope. For pics this makes particular sense: All
  nodes defined by a pic's code can be referenced from outside the pic
  with the prefix provided.

  To see how this works, let us add some nodes to the code of the
  seagull:
\begin{codeexample}[code only]
\tikzset{ 
  seagull/.pic={
    % Code for a "seagull". Do you see it?...
    \coordinate (-left wing) at (-3mm,0);
    \coordinate (-head)      at (0,0);
    \coordinate (-right wing) at (3mm,0);
    
    \draw (-left wing) to [bend left] (0,0) (-head) to [bend left] (-right wing);
  }
}
\end{codeexample}
\tikzset{ 
  seagull/.pic={
    % Code for a "seagull". Do you see it?...
    \coordinate (-left wing) at (-3mm,0);
    \coordinate (-head)      at (0,0);
    \coordinate (-right wing) at (3mm,0);
    
    \draw (-left wing) to [bend left] (0,0) (-head) to [bend left] (-right wing);
  }
}

  Now, we can use it as follows:
\begin{codeexample}[code only]
\tikz {
  \pic (Emma)               {seagull};
  \pic (Alexandra) at (0,1) {seagull};

  \draw (Emma-left wing) -- (Alexandra-right wing);
}
\end{codeexample}

  Sometimes, you may also wish your pic to access nodes outside the
  pic (typically, because they are given as parameters). In this case,
  the name prefix gets in the way since the nodes outside the picture
  do not have this prefix. The trick is to locally reset the name
  prefix to the value it had outside the picture, which is achieved
  using the following style:

  \begin{key}{/tikz/name prefix ..}
    This key is available only inside the code of a pic. There, it
    (locally) changes the name prefix to the value it had outside the
    pic. This allows you to access nodes outside the current pic.
  \end{key}

\end{pathoperation}


There are two general purpose keys that pics may find useful:

\begin{key}{/tikz/pic text=\meta{text}}
  This macro stores the \meta{text} in the macro |\tikzpictext|, which
  is |\let| to |\relax| by default. Setting the |pic text| to some
  value is the ``preferred'' way of communicating a (single) piece of
  text that should become part of a pic (typically of a node). In
  particular, the |quotes| library maps quoted parameters to this key.
\end{key}

\begin{key}{/tikz/pic text options=\meta{options}}
  This macro stores the \meta{options} in the macro
  |\tikzpictextoptions|, which is |\let| to the empty string by
  default. The |quotes| library maps options for quoted parameters to
  this key. 
\end{key}



\subsubsection{The Quotes Syntax}
\label{section-pic-quotes}

When you load the |quotes| library, you can use the ``quotes syntax''
inside the options of a pic. Recall that for nodes this syntax is used
to add a label to a node. For pics, the quotes syntax is used to set
the |pic text| key. Whether or not the pic type's code takes this key
into consideration is, however, up to the key.

In detail, when the |quotes| library is loaded, each time a key--value 
pair in a list of options passed to an |pic| starts with |"|, the
key--value pair must actually be a string of the following form:
\begin{quote}
  |"|\meta{text}|"|\opt{|'|}\opt{\meta{options}}
\end{quote}
This string is transformed into the following:
\begin{quote}
  |every pic quotes/.try,pic text=|\meta{text}|, pic text options={|\meta{options}|}|
\end{quote}

As example of a pic type that takes these values into account is the
|angle| pic type:
\begin{codeexample}[]
\tikz \draw (3,0) coordinate (A)
         -- (0,1) coordinate (B)
         -- (1,2) coordinate (C)
            pic [draw, "$\alpha$"] {angle};
\end{codeexample}

As described in Section~\ref{section-label-quotes}, the apostrophe
becomes part of the \meta{options}, when present. As can be seen
above, the following style is executed:
\begin{stylekey}{/tikz/every pic quotes (initially \normalfont empty)}
\end{stylekey}

\subsection{Defining New Pic Types}
\label{section-new-pic-types}

As explained in the description of the |pic| command, in order to
define a new pic type you need to
\begin{enumerate}
\item define a key with the path prefix |/tikz/pics| that
\item sets the key |/tikz/pics/code| to the code of the pic. 
\end{enumerate}

It turns out that this is easy enough to achieve using styles:

\begin{codeexample}[code only]
\tikzset{
  pics/seagull/.style ={
  	% Ok, this is the key that should, when
    % executed, set the code key:
    code = { %
      \draw (...) ... ;
    }
  }
}
\end{codeexample}

Even though the above pattern is easy enough, there is a special
so-called key handler that allows us to write even simpler code,
namely:

\begin{codeexample}[code only]
\tikzset{
  seagull/.pic = {
    \draw (...) ... ;
  }
}
\end{codeexample}

\begin{handler}{{.pic}|=|\meta{some code}}
  This handler can only be used with a key with the prefix |/tikz/|,
  so just should normally use it only as an option to a \tikzname\
  command or to the |\tikzset| command. It takes the \meta{key}'s path
  and, inside that path, it replaces |/tikz/| by |/tikz/pics/| (so,
  basically, it adds the ``missing'' |pics| part of the path). Then,
  it sets up things so that the resulting name to key is a style that
  executes |code=some code|.
\end{handler}

In almost all cases, the |.pic| key handler will suffice to setup
keys. However, there are cases where you really need to use the first
version using  |.style| and |code=|:
\begin{itemize}
\item Whenever your pic type needs to set the foreground or the
  background code.
\item In case of complicated arguments given to the keys.
\end{itemize}

As an example, let us define a simple pic that draws a filled circle
behind the path. Furthermore, we make the circle's radius a parameter
of the pic:

\begin{codeexample}[]
\tikzset{
  pics/my circle/.style = {
    background code = { \fill circle [radius=#1]; }
  }
}
\tikz [fill=blue!30]
  \draw (0,0) pic {my circle=2mm} -- (1,1) pic {my circle=5mm};
\end{codeexample}
