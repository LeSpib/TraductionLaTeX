%\iffalse
%<*package>
%% \CharacterTable
%%  {Upper-case    \A\B\C\D\E\F\G\H\I\J\K\L\M\N\O\P\Q\R\S\T\U\V\W\X\Y\Z
%%   Lower-case    \a\b\c\d\e\f\g\h\i\j\k\l\m\n\o\p\q\r\s\t\u\v\w\x\y\z
%%   Digits        \0\1\2\3\4\5\6\7\8\9
%%   Exclamation   \!     Double quote  \"     Hash (number) \#
%%   Dollar        \$     Percent       \%     Ampersand     \&
%%   Acute accent  \'     Left paren    \(     Right paren   \)
%%   Asterisk      \*     Plus          \+     Comma         \,
%%   Minus         \-     Point         \.     Solidus       \/
%%   Colon         \:     Semicolon     \;     Less than     \<
%%   Equals        \=     Greater than  \>     Question mark \?
%%   Commercial at \@     Left bracket  \[     Backslash     \\
%%   Right bracket \]     Circumflex    \^     Underscore    \_
%%   Grave accent  \`     Left brace    \{     Vertical bar  \|
%%   Right brace   \}     Tilde         \~}
%</package>
%\fi
% \iffalse
% Doc-Source file to use with LaTeX2e
% Copyright (C) 2015 Nicola Talbot, all rights reserved.
% (New maintainer add relevant lines here.)
% \fi
% \iffalse
%<*driver>
\documentclass{ltxdoc}
\usepackage[babel]{translatex-fr}
\usepackage{metalogo}
\usepackage{alltt}
\usepackage{graphicx}
\usepackage[utf8]{inputenc}
\usepackage[T1]{fontenc}
\usepackage[colorlinks,
            bookmarks,
            hyperindex=false,
            pdfauthor={Nicola L.C. Talbot},
            pdftitle={datetime2.sty French Module}]{hyperref}


\CheckSum{333}

\renewcommand*{\usage}[1]{\hyperpage{#1}}
\renewcommand*{\main}[1]{\hyperpage{#1}}
\IndexPrologue{\section*{\indexname}\markboth{\indexname}{\indexname}}
\setcounter{IndexColumns}{2}

\newcommand*{\sty}[1]{\textsf{#1}}
\newcommand*{\opt}[1]{\texttt{#1}\index{#1=\texttt{#1}|main}}

\RecordChanges
\PageIndex
\CodelineNumbered

\begin{document}
\DocInput{datetime2-french-fr.dtx}
\end{document}
%</driver>
%\fi
%
%\MakeShortVerb{"}
%
%\title{Le module \textsf{French} de l'extension \textsf{datetime2}}
%\author{Nicola L. C. Talbot (inactive)}
%\date{2015-03-27 (v1.0)}
%\maketitle
%
% Ce module est actuellement non maintenu et peut être amené à changer. Si vous
% souhaitez vous porter volontaire pour prendre en charge la maintenance, 
% contactez-moi par le biais de :
%\url{http://www.dickimaw-books.com/contact.html}
%
%\begin{abstract}
% Ce module traite la langue française pour l'extension \sty{datetime2}. Si
% vous voulez en utiliser les réglages, vous devez l'installer en complément
% de l'installation de \sty{datetime2}. Si vous utilisez \sty{babel} ou 
% \sty{polyglossia}, vous aurez besoin de ce module pour éviter qu'ils ne 
% redéfinissent \cs{today}. L'option \opt{useregional} de \sty{datetime2}
% doit être mise à "text" ou "numeric" pour que les styles des langues soient
% utilisés.
% Vous pouvez sinon définir le style dans un document en utilisant 
% \cs{DTMsetstyle}, mais ceci peut être changé par \cs{date}\meta{language}
% selon la valeur d'option \opt{useregional}.
%\end{abstract}
%
% J'ai copié le style de date de la commande \cs{today} de 
% \texttt{babel-french}. 
%
% Je ne sais pas si ces définitions sont correctes. En particulier, je ne sais
% pas si le style d'heure \og français \fg{} est correct. Actuellement, il
% utilise le style d'heure \og par défaut \fg{}. Gardez bien en tête que cela
% peut changer : la personne prenant en charge la maintenance de ce module peut
% le modifier pour le corriger si besoin est.
%
% Le nouveau mainteneur doit ajouter la ligne\footnote{N.D.T. : ligne donnée
% ici en version originale et qui se traduit par : \og Le mainteneur actuel de
% ce projet est ... \fg{}.}
%\begin{verbatim}
% The Current Maintainer of this work is Name.
%\end{verbatim}
% dans la préambule de \texttt{datetime2-french.ins}, où \emph{Name} est le nom du
% ou des mainteneurs et il doit également remplacer le status \og 
% \emph{inactive} \fg{} par \og \emph{maintained} \fg{}.
% 
% Actuellement, il n'y a qu'un style sans variante régionale. Les nouveaux
% mainteneurs peuvent ajouter des variantes telles, par exemple, 
% "fr-FR"\slash"fr-FR-numeric" ou "fr-BE"\slash"fr-BE-numeric". Ce style ne
% vérifie actuellement pas la valeur de l'option \opt{showdow}.
%
%\StopEventually{%
%\clearpage
%\phantomsection
%\addcontentsline{toc}{section}{Change History}%
%\PrintChanges
%\addcontentsline{toc}{section}{\indexname}%
%\PrintIndex}
%\section{Implémentation}
%\iffalse
%    \begin{macrocode}
%<*datetime2-french-utf8.ldf>
%    \end{macrocode}
%\fi
%\subsection{UTF-8}
% Ce fichier contient les réglages pour utiliser les caractères UTF-8. Il est
% chargé si \XeLaTeX\ ou \LuaLaTeX\ sont utilisés. Vérifiez bien que votre
% éditeur de texte est bien en UTF-8 si vous souhaitez voir ce code.
%\changes{1.0}{2015-03-27}{Initial release}
% Le module est d'abord identifié
%    \begin{macrocode}
\ProvidesDateTimeModule{french-utf8}[2015/03/27 v1.0]
%    \end{macrocode}
%\begin{macro}{\DTMfrenchordinal}
%    \begin{macrocode}
\ifdef\ier
{%
  \newcommand*{\DTMfrenchordinal}[1]{%
    \number#1
    \ifnum#1=1\DTMtexorpdfstring{\protect\ier}{er}\fi
  }
}%
{%
  \newcommand*{\DTMfrenchordinal}[1]{%
    \number#1
    \ifnum#1=1\DTMtexorpdfstring{\protect\textsuperscript{er}}{er}\fi
  }
}%
%    \end{macrocode}
%\end{macro}
%
% \tradini
%\begin{macro}{\DTMfrenchmonthname}
% French month names.
%    \begin{macrocode}
\newcommand*{\DTMfrenchmonthname}[1]{%
  \ifcase#1
  \or
  janvier%
  \or
  février%
  \or
  mars%
  \or
  avril%
  \or
  mai%
  \or
  juin%
  \or
  juillet%
  \or
  août%
  \or
  septembre%
  \or
  octobre%
  \or
  novembre%
  \or
  décembre%
  \fi
}
%    \end{macrocode}
%\end{macro}
%
%\begin{macro}{\DTMfrenchMonthname}
% As above but start with a capital.
%    \begin{macrocode}
\newcommand*{\DTMfrenchMonthname}[1]{%
  \ifcase#1
  \or
  Janvier%
  \or
  Février%
  \or
  Mars%
  \or
  Avril%
  \or
  Mai%
  \or
  Juin%
  \or
  Juillet%
  \or
  Août%
  \or
  Septembre%
  \or
  Octobre%
  \or
  Novembre%
  \or
  Décembre%
  \fi
}
%    \end{macrocode}
%\end{macro}
%
%If abbreviated dates are supported, short month names should be
%likewise provided.
%
%\begin{macro}{\DTMfrenchweekdayname}
%These are provided here but not currently used in the date format.
%    \begin{macrocode}
\newcommand*{\DTMfrenchweekdayname}[1]{%
  \ifcase#1
  lundi%
  \or
  mardi%
  \or
  mercredi%
  \or
  jeudi%
  \or
  vendredi%
  \or
  samedi%
  \or
  dimanche%
  \fi
}
%    \end{macrocode}
%\end{macro}
%
%\begin{macro}{\DTMfrenchWeekdayname}
%As above but start with a capital.
%    \begin{macrocode}
\newcommand*{\DTMfrenchWeekdayname}[1]{%
  \ifcase#1
  Lundi%
  \or
  Mardi%
  \or
  Mercredi%
  \or
  Jeudi%
  \or
  Vendredi%
  \or
  Samedi%
  \or
  Dimanche%
  \fi
}
%    \end{macrocode}
%\end{macro}
%
%\iffalse
%    \begin{macrocode}
%</datetime2-french-utf8.ldf>
%    \end{macrocode}
%\fi
%\iffalse
%    \begin{macrocode}
%<*datetime2-french-ascii.ldf>
%    \end{macrocode}
%\fi
%\subsection{ASCII}
%This file contains the settings that use \LaTeX\ commands for
%non-ASCII characters. This should be input if neither XeLaTeX nor
%LuaLaTeX are used. Even if the user has loaded \sty{inputenc} with
%"utf8", this file should still be used not the
%\texttt{datetime2-french-utf8.ldf} file as the non-ASCII
%characters are made active in that situation and would need
%protecting against expansion.
%\changes{1.0}{2015-03-27}{Initial release}
% Identify module
%    \begin{macrocode}
\ProvidesDateTimeModule{french-ascii}[2015/03/27 v1.0]
%    \end{macrocode}
%
%If abbreviated dates are supported, short month names should be
%likewise provided.
%\begin{macro}{\DTMfrenchordinal}
%    \begin{macrocode}
\ifdef\ier
{%
  \newcommand*{\DTMfrenchordinal}[1]{%
    \number#1
    \ifnum#1=1\DTMtexorpdfstring{\protect\ier}{er}\fi
  }
}%
{%
  \newcommand*{\DTMfrenchordinal}[1]{%
    \number#1
    \ifnum#1=1\DTMtexorpdfstring{\protect\textsuperscript{er}}{er}\fi
  }
}%
%    \end{macrocode}
%\end{macro}
%
%\begin{macro}{\DTMfrenchmonthname}
% French month names.
%    \begin{macrocode}
\newcommand*{\DTMfrenchmonthname}[1]{%
  \ifcase#1
  \or
  janvier%
  \or
  f\protect\'evrier%
  \or
  mars%
  \or
  avril%
  \or
  mai%
  \or
  juin%
  \or
  juillet%
  \or
  ao\protect\^ut%
  \or
  septembre%
  \or
  octobre%
  \or
  novembre%
  \or
  d\protect\'ecembre%
  \fi
}
%    \end{macrocode}
%\end{macro}
%
%\begin{macro}{\DTMfrenchMonthname}
% As above but start with a capital.
%    \begin{macrocode}
\newcommand*{\DTMfrenchMonthname}[1]{%
  \ifcase#1
  \or
  Janvier%
  \or
  F\protect\'evrier%
  \or
  Mars%
  \or
  Avril%
  \or
  Mai%
  \or
  Juin%
  \or
  Juillet%
  \or
  Ao\protect\^ut%
  \or
  Septembre%
  \or
  Octobre%
  \or
  Novembre%
  \or
  D\protect\'ecembre%
  \fi
}
%    \end{macrocode}
%\end{macro}
%
%\begin{macro}{\DTMfrenchweekdayname}
%These are provided here but not currently used in the date format.
%    \begin{macrocode}
\newcommand*{\DTMfrenchweekdayname}[1]{%
  \ifcase#1
  lundi%
  \or
  mardi%
  \or
  mercredi%
  \or
  jeudi%
  \or
  vendredi%
  \or
  samedi%
  \or
  dimanche%
  \fi
}
%    \end{macrocode}
%\end{macro}
%
%\begin{macro}{\DTMfrenchWeekdayname}
%As above but start with a capital.
%    \begin{macrocode}
\newcommand*{\DTMfrenchWeekdayname}[1]{%
  \ifcase#1
  Lundi%
  \or
  Mardi%
  \or
  Mercredi%
  \or
  Jeudi%
  \or
  Vendredi%
  \or
  Samedi%
  \or
  Dimanche%
  \fi
}
%    \end{macrocode}
%\end{macro}
%
%
%\iffalse
%    \begin{macrocode}
%</datetime2-french-ascii.ldf>
%    \end{macrocode}
%\fi
%
%\subsection{Main French Module (\texttt{datetime2-french.ldf})}
%\changes{1.0}{2015-03-27}{Initial release}
%
%\iffalse
%    \begin{macrocode}
%<*datetime2-french.ldf>
%    \end{macrocode}
%\fi
%
% Identify Module
%    \begin{macrocode}
\ProvidesDateTimeModule{french}[2015/03/27 v1.0]
%    \end{macrocode}
% Need to find out if XeTeX or LuaTeX are being used.
%    \begin{macrocode}
\RequirePackage{ifxetex,ifluatex}
%    \end{macrocode}
% XeTeX and LuaTeX natively support UTF-8, so load
% \texttt{french-utf8} if either of those engines are used
% otherwise load \texttt{french-ascii}.
%    \begin{macrocode}
\ifxetex
 \RequireDateTimeModule{french-utf8}
\else
 \ifluatex
   \RequireDateTimeModule{french-utf8}
 \else
   \RequireDateTimeModule{french-ascii}
 \fi
\fi
%    \end{macrocode}
%
% Define the \texttt{french} style.
% The time style is the same as the "default" style
% provided by \sty{datetime2}. This may need correcting. 
%
% Allow the user a way of configuring the "french" and
% "french-numeric" styles. This doesn't use the package wide
% separators such as
% \cs{dtm@datetimesep} in case other date formats are also required.
%\begin{macro}{\DTMfrenchdaymonthsep}
% The separator between the day and month for the text format.
%    \begin{macrocode}
\newcommand*{\DTMfrenchdaymonthsep}{\space}
%    \end{macrocode}
%\end{macro}
%
%\begin{macro}{\DTMfrenchmonthyearsep}
% The separator between the month and year for the text format.
%    \begin{macrocode}
\newcommand*{\DTMfrenchmonthyearsep}{\space}
%    \end{macrocode}
%\end{macro}
%
%\begin{macro}{\DTMfrenchdatetimesep}
% The separator between the date and time blocks in the full format
% (either text or numeric).
%    \begin{macrocode}
\newcommand*{\DTMfrenchdatetimesep}{\space}
%    \end{macrocode}
%\end{macro}
%
%\begin{macro}{\DTMfrenchtimezonesep}
% The separator between the time and zone blocks in the full format
% (either text or numeric).
%    \begin{macrocode}
\newcommand*{\DTMfrenchtimezonesep}{\space}
%    \end{macrocode}
%\end{macro}
%
%\begin{macro}{\DTMfrenchdatesep}
% The separator for the numeric date format.
%    \begin{macrocode}
\newcommand*{\DTMfrenchdatesep}{/}
%    \end{macrocode}
%\end{macro}
%
%\begin{macro}{\DTMfrenchtimesep}
% The separator for the numeric time format.
%    \begin{macrocode}
\newcommand*{\DTMfrenchtimesep}{:}
%    \end{macrocode}
%\end{macro}
%
%Provide keys that can be used in \cs{DTMlangsetup} to set these
%separators.
%    \begin{macrocode}
\DTMdefkey{french}{daymonthsep}{\renewcommand*{\DTMfrenchdaymonthsep}{#1}}
\DTMdefkey{french}{monthyearsep}{\renewcommand*{\DTMfrenchmonthyearsep}{#1}}
\DTMdefkey{french}{datetimesep}{\renewcommand*{\DTMfrenchdatetimesep}{#1}}
\DTMdefkey{french}{timezonesep}{\renewcommand*{\DTMfrenchtimezonesep}{#1}}
\DTMdefkey{french}{datesep}{\renewcommand*{\DTMfrenchdatesep}{#1}}
\DTMdefkey{french}{timesep}{\renewcommand*{\DTMfrenchtimesep}{#1}}
%    \end{macrocode}
%
% TODO: provide a boolean key to switch between full and abbreviated
% formats if appropriate. (I don't know how the date should be
% abbreviated.)
%
% Define a boolean key that determines if the time zone mappings
% should be used.
%    \begin{macrocode}
\DTMdefboolkey{french}{mapzone}[true]{}
%    \end{macrocode}
% The default is to use mappings.
%    \begin{macrocode}
\DTMsetbool{french}{mapzone}{true}
%    \end{macrocode}
%
% Define a boolean key that determines if the day of month should be
% displayed.
%    \begin{macrocode}
\DTMdefboolkey{french}{showdayofmonth}[true]{}
%    \end{macrocode}
% The default is to show the day of month.
%    \begin{macrocode}
\DTMsetbool{french}{showdayofmonth}{true}
%    \end{macrocode}
%
% Define a boolean key that determines if the year should be
% displayed.
%    \begin{macrocode}
\DTMdefboolkey{french}{showyear}[true]{}
%    \end{macrocode}
% The default is to show the year.
%    \begin{macrocode}
\DTMsetbool{french}{showyear}{true}
%    \end{macrocode}
%
% Define the "french" style. (TODO: implement day of week?)
%    \begin{macrocode}
\DTMnewstyle
 {french}% label
 {% date style
   \renewcommand*\DTMdisplaydate[4]{%
     \DTMifbool{french}{showdayofmonth}
     {\DTMfrenchordinal{##3}\DTMfrenchdaymonthsep}%
     {}%
     \DTMfrenchmonthname{##2}%
     \DTMifbool{french}{showyear}%
     {%
       \DTMfrenchmonthyearsep
       \number##1 % space intended
     }%
     {}%
   }%
   \renewcommand*\DTMDisplaydate[4]{%
     \DTMifbool{french}{showdayofmonth}
     {%
        \DTMfrenchordinal{##3}\DTMfrenchdaymonthsep
        \DTMfrenchmonthname{##2}%
     }%
     {\DTMfrenchMonthname{##2}}%
     \DTMifbool{french}{showyear}%
     {%
       \DTMfrenchmonthyearsep
       \number##1 % space intended
     }%
     {}%
   }%
 }%
 {% time style (use default)
   \DTMsettimestyle{default}%
 }%
 {% zone style
   \DTMresetzones
   \DTMfrenchzonemaps
   \renewcommand*{\DTMdisplayzone}[2]{%
     \DTMifbool{french}{mapzone}%
     {\DTMusezonemapordefault{##1}{##2}}%
     {%
       \ifnum##1<0\else+\fi\DTMtwodigits{##1}%
       \ifDTMshowzoneminutes\DTMfrenchtimesep\DTMtwodigits{##2}\fi
     }%
   }%
 }%
 {% full style
   \renewcommand*{\DTMdisplay}[9]{%
    \ifDTMshowdate
     \DTMdisplaydate{##1}{##2}{##3}{##4}%
     \DTMfrenchdatetimesep
    \fi
    \DTMdisplaytime{##5}{##6}{##7}%
    \ifDTMshowzone
     \DTMfrenchtimezonesep
     \DTMdisplayzone{##8}{##9}%
    \fi
   }%
   \renewcommand*{\DTMDisplay}[9]{%
    \ifDTMshowdate
     \DTMDisplaydate{##1}{##2}{##3}{##4}%
     \DTMfrenchdatetimesep
    \fi
    \DTMdisplaytime{##5}{##6}{##7}%
    \ifDTMshowzone
     \DTMfrenchtimezonesep
     \DTMdisplayzone{##8}{##9}%
    \fi
   }%
 }%
%    \end{macrocode}
%
% Define numeric style.
%    \begin{macrocode}
\DTMnewstyle
 {french-numeric}% label
 {% date style
    \renewcommand*\DTMdisplaydate[4]{%
      \DTMifbool{french}{showdayofmonth}%
      {%
        \number##3 % space intended
        \DTMfrenchdatesep
      }%
      {}%
      \number##2 % space intended
      \DTMifbool{french}{showyear}%
      {%
        \DTMfrenchdatesep
        \number##1 % space intended
      }%
      {}%
    }%
    \renewcommand*{\DTMDisplaydate}[4]{\DTMdisplaydate{##1}{##2}{##3}{##4}}%
 }%
 {% time style
    \renewcommand*\DTMdisplaytime[3]{%
      \number##1
      \DTMfrenchtimesep\DTMtwodigits{##2}%
      \ifDTMshowseconds\DTMfrenchtimesep\DTMtwodigits{##3}\fi
    }%
 }%
 {% zone style
   \DTMresetzones
   \DTMfrenchzonemaps
   \renewcommand*{\DTMdisplayzone}[2]{%
     \DTMifbool{french}{mapzone}%
     {\DTMusezonemapordefault{##1}{##2}}%
     {%
       \ifnum##1<0\else+\fi\DTMtwodigits{##1}%
       \ifDTMshowzoneminutes\DTMfrenchtimesep\DTMtwodigits{##2}\fi
     }%
   }%
 }%
 {% full style
   \renewcommand*{\DTMdisplay}[9]{%
    \ifDTMshowdate
     \DTMdisplaydate{##1}{##2}{##3}{##4}%
     \DTMfrenchdatetimesep
    \fi
    \DTMdisplaytime{##5}{##6}{##7}%
    \ifDTMshowzone
     \DTMfrenchtimezonesep
     \DTMdisplayzone{##8}{##9}%
    \fi
   }%
   \renewcommand*{\DTMDisplay}{\DTMdisplay}%
 }
%    \end{macrocode}
%
%\begin{macro}{\DTMfrenchzonemaps}
% The time zone mappings are set through this command, which can be
% redefined if extra mappings are required or mappings need to be
% removed.
%    \begin{macrocode}
\newcommand*{\DTMfrenchzonemaps}{%
  \DTMdefzonemap{01}{00}{CET}%
  \DTMdefzonemap{02}{00}{CEST}%
}
%    \end{macrocode}
%\end{macro}

% Switch style according to the \opt{useregional} setting.
%    \begin{macrocode}
\DTMifcaseregional
{}% do nothing
{\DTMsetstyle{french}}
{\DTMsetstyle{french-numeric}}
%    \end{macrocode}
%
% Redefine \cs{datefrench} (or \cs{date}\meta{dialect}) to prevent
% \sty{babel} from resetting \cs{today}. (For this to work,
% \sty{babel} must already have been loaded if it's required.)
%    \begin{macrocode}
\ifcsundef{date\CurrentTrackedDialect}
{%
  \ifundef\datefrench
  {% do nothing
  }%
  {%
    \def\datefrench{%
      \DTMifcaseregional
      {}% do nothing
      {\DTMsetstyle{french}}%
      {\DTMsetstyle{french-numeric}}%
    }%
  }%
}%
{%
  \csdef{date\CurrentTrackedDialect}{%
    \DTMifcaseregional
    {}% do nothing
    {\DTMsetstyle{french}}%
    {\DTMsetstyle{french-numeric}}
  }%
}%
%    \end{macrocode}
%\iffalse
%    \begin{macrocode}
%</datetime2-french.ldf>
%    \end{macrocode}
%\fi
%\Finale
\endinput
