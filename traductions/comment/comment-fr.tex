\documentclass[parskip=false, DIV=8, headings=normal, pagesize=auto]{scrartcl}

\usepackage[inputenc,babel]{translatex-fr}
\usepackage{fixltx2e}
\usepackage{etex}
\usepackage{xspace}
\usepackage{lmodern}
\usepackage[T1]{fontenc}
\usepackage{textcomp}
\usepackage{microtype}
\usepackage[unicode=true,colorlinks]{hyperref}

\newenvironment*{noverb}{%
  \flushleft
  \vskip\parskip
  \parskip=0pt\relax
}{%
  \endflushleft
}

\newcommand*{\mail}[1]{\href{mailto:#1}{\texttt{#1}}}
\newcommand*{\pkg}[1]{\textsf{#1}}
\newcommand*{\cs}[1]{\texttt{\textbackslash#1}}
\makeatletter
\newcommand*{\cmd}[1]{\cs{\expandafter\@gobble\string#1}}
\makeatother
\newcommand*{\env}[1]{\texttt{#1}}
\newcommand*{\meta}[1]{\textlangle\textsl{#1}\textrangle}
\newcommand*{\marg}[1]{\texttt{\{}\meta{#1}\texttt{\}}}

\addtokomafont{title}{\rmfamily}

\title{L'extension \pkg{comment}\thanks{Ce manuel corresponds à la version \pkg{comment}~v3.6, datée d'octobre 1999.}}
\author{Victor Eijkhout\\\mail{victor@eijkhout.net}}
\date{Octobre 1999}


\begin{document}

\maketitle

\section{Objectif}

Cette extension inclut ou exclut de façon sélective des blocs de texte dans la
restitution finale. L'utilisateur peut définir de nouvelles versions de 
commandes de commentaire, chacune pouvant être contrôlée séparément.
Des commentaires spéciaux peuvent être définis : l'utilisateur y spécifie
l'action qui doit être faite pour chacune des lignes mises en commentaire.

Cette extension peut être utilisée avec \textsc{plain} \TeX\ ou \LaTeX, comme
avec la plupart des extensions existantes.

\section{Utilisation}

Tout le texte placé entre les commandes
%
\begin{verbatim}
\comment ... \endcomment
\end{verbatim}
%
ou
%
\begin{verbatim}
\begin{comment} ... \end{comment}
\end{verbatim}
%
est ignoré. 

Les commandes de début et de fin doivent apparaître chacune sur une ligne à
part. Il ne doit pas y avoir d'espace et d'éléments avant ou après la commande
sur ces lignes. 
Cet environnement devrait pouvoir fonctionner avec une quantité de commentaire
arbitraire et le commentaire peut être un texte quelconque.

D'autres environnements à l'image de \env{comment} peuvent être définis et
sélectionnés et désélectionnés avec 
%
\begin{verbatim}
\includecomment{versiona}
\excludecoment{versionb}
\end{verbatim}
%
Ces environnements sont utilisables avec
%
\begin{verbatim}
\versiona ... \endversiona
\end{verbatim}
%
ou
%
\begin{verbatim}
\begin{versiona} ... \end{versiona}
\end{verbatim}
%
avec, à nouveau, les commandes de commentaires de début et de fin sur des
lignes à part.

\pagebreak[1]

Note pour les utilisateurs de \LaTeX\ : pour un commentaire inclus, les
lignes \cmd{\begin} et \cmd{\end} agissent comme si elles n'existaient
pas. En particulier, elles n'impliquent pas de groupe, aussi les 
instanciations et autres ne sont pas locales.

\pagebreak[2]

Les commentaires spéciaux sont définis par
%
\begin{noverb}
\cmd{\specialcomment}\marg{nom}\marg{commandes avant}\marg{commandes après}
\end{noverb}
%
où le deuxième argument et le troisième argument sont exécutés respectivement
avant et après chaque bloc de commentaire. Vous pouvez utiliser ceci pour des
commandes de mise en forme globales.
Pour garder ces réglages locaux, vous pouvez inclure un \cmd{\begingroup} dans
les \og \meta{commandes avant} \fg et \cmd{\endgroup} dans les \og 
\meta{commandes après} \fg. Par exemple :
%
\begin{verbatim}
\specialcomment{smalltt}
    {\begingroup\ttfamily\footnotesize}{\endgroup}
\end{verbatim}
%
\tradini
You do \emph{not} have to do an additional
%
\begin{verbatim}
\includecomment{smalltt}
\end{verbatim}
%
To remove `\env{smalltt}' blocks, give \verb+\excludecomment{smalltt}+
after the definition.

Processing comments can apply processing to each line.
%
\begin{noverb}
\cmd{\processcomment}\marg{name}\marg{each-line commands}\marg{before commands}\marg{after commands}
\end{noverb}
%
By defining a control sequence
%
\begin{verbatim}
\def\Thiscomment##1{...}
\end{verbatim}
%
in the before commands the user can
specify what is to be done with each comment line.
BUG this does not work quite yet BUG

Trick for short in/exclude macros (such as \verb+\maybe{this snippet}+):
%
\begin{verbatim}
\includecomment{cond}
\newcommand{\maybe}[1]{}
\begin{cond}
\renewcommand{\maybe}[1]{#1}
\end{cond}
\end{verbatim}


\section{Basic approach of the implementation:}

to comment something out, scoop up  every line in verbatim mode
as macro argument, then throw it away.
For inclusions, in \LaTeX\ the block is written out to
a file \cmd{\CommentCutFile} (default ``\texttt{comment.cut}''), which is
then included.
In plain \TeX\ (and other formats) both the opening and
closing comands are defined as noop.


\section{Changes in version 3.1}

\begin{itemize}
\item updated author's address
\item cleaned up some code
\item trailing contents on \cmd{\begin}\marg{env} line is always discarded
    even if you've done \cmd{\includecomment}\marg{env}
\item comments no longer define grouping!! you can even
  % 
\begin{verbatim}
\includecomment{env}
\begin{env}
\begin{itemize}
\end{env}
\end{verbatim}
  % 
  Isn't that something\ldots
\item included comments are written to file and input again.
\end{itemize}


\section{Changes in 3.2}

\begin{itemize}
\item \cmd{\specialcomment} brought up to date (thanks to Ivo Welch).
\end{itemize}


\section{Changes in 3.3}

\begin{itemize}
\item updated author's address again
\item parametrised \cmd{\CommentCutFile}
\end{itemize}


\section{Changes in 3.4}

\begin{itemize}
\item added GNU public license
\item added \cmd{\processcomment}, because Ivo's fix (above) brought an
  inconsistency to light.
\end{itemize}

  
\section{Changes in 3.5}

\begin{itemize}
\item corrected typo in header.
\item changed author email
\item corrected \cmd{\specialcomment} yet again.
\item fixed excludecomment of an earlier defined environment.
\end{itemize}


\section{Changes in 3.6}

\begin{itemize}
\item The `cut' file is now written more verbatim, using \cmd{\meaning};
  some people reported having trouble with ISO~latin~1, or \texttt{umlaute.sty}.
\item removed some \cmd{\newif} statements.
  Has this suddenly become \cmd{\outer} again?
\end{itemize}


\section{Known bugs:}

\begin{itemize}
\item \texttt{excludecomment} leads to one superfluous space
\item \texttt{processcomment} leads to a superfluous line break
\end{itemize}

\end{document}
