\documentclass[parskip=false, DIV=8, headings=normal, pagesize=auto]{scrartcl}

\usepackage[inputenc,babel]{translatex-fr}
\usepackage{fixltx2e}
\usepackage{etex}
\usepackage{xspace}
\usepackage{lmodern}
\usepackage[T1]{fontenc}
\usepackage{textcomp}
\usepackage{microtype}
\usepackage[unicode=true,colorlinks]{hyperref}

\newenvironment*{noverb}{%
  \flushleft
  \vskip\parskip
  \parskip=0pt\relax
}{%
  \endflushleft
}

\newcommand*{\mail}[1]{\href{mailto:#1}{\texttt{#1}}}
\newcommand*{\pkg}[1]{\textsf{#1}}
\newcommand*{\cs}[1]{\texttt{\textbackslash#1}}
\makeatletter
\newcommand*{\cmd}[1]{\cs{\expandafter\@gobble\string#1}}
\makeatother
\newcommand*{\env}[1]{\texttt{#1}}
\newcommand*{\meta}[1]{\textlangle\textsl{#1}\textrangle}
\newcommand*{\marg}[1]{\texttt{\{}\meta{#1}\texttt{\}}}

\addtokomafont{title}{\rmfamily}

\title{L'extension \pkg{comment}\thanks{Ce manuel corresponds à la version \pkg{comment}~v3.6, datée d'octobre 1999.}}
\author{Victor Eijkhout\\\mail{victor@eijkhout.net}}
\date{Octobre 1999}


\begin{document}

\maketitle

\section{Objectif}

Cette extension inclut ou exclut de façon sélective des blocs de texte dans la
restitution finale. L'utilisateur peut définir de nouvelles versions de 
commandes de commentaire, chacune pouvant être contrôlée séparément.
Des commentaires spéciaux peuvent être définis : l'utilisateur y spécifie
l'action qui doit être faite pour chacune des lignes mises en commentaire.

Cette extension peut être utilisée avec \textsc{plain} \TeX\ ou \LaTeX, comme
avec la plupart des extensions existantes.

\section{Utilisation}

Tout le texte placé entre les commandes
%
\begin{verbatim}
\comment ... \endcomment
\end{verbatim}
%
ou
%
\begin{verbatim}
\begin{comment} ... \end{comment}
\end{verbatim}
%
est ignoré. 

Les commandes de début et de fin doivent apparaître chacune sur une ligne à
part. Il ne doit pas y avoir d'espace et d'éléments avant ou après la commande
sur ces lignes. 
Cet environnement devrait pouvoir fonctionner avec une quantité de commentaire
arbitraire et le commentaire peut être un texte quelconque.

D'autres environnements à l'image de \env{comment} peuvent être définis et
sélectionnés et désélectionnés avec 
%
\begin{verbatim}
\includecomment{versiona}
\excludecoment{versionb}
\end{verbatim}
%
Ces environnements sont utilisables avec
%
\begin{verbatim}
\versiona ... \endversiona
\end{verbatim}
%
ou
%
\begin{verbatim}
\begin{versiona} ... \end{versiona}
\end{verbatim}
%
avec, à nouveau, les commandes de commentaires de début et de fin sur des
lignes à part.

\pagebreak[1]

Note pour les utilisateurs de \LaTeX\ : pour un commentaire inclus, les
lignes \cmd{\begin} et \cmd{\end} agissent comme si elles n'existaient
pas. En particulier, elles n'impliquent pas de groupe, aussi les 
instanciations et autres ne sont pas locales.

\pagebreak[2]

Les commentaires spéciaux sont définis par
%
\begin{noverb}
\cmd{\specialcomment}\marg{nom}\marg{commandes avant}\marg{commandes après}
\end{noverb}
%
où le deuxième argument et le troisième argument sont exécutés respectivement
avant et après chaque bloc de commentaire. Vous pouvez utiliser ceci pour des
commandes de mise en forme globales.
Pour garder ces réglages locaux, vous pouvez inclure un \cmd{\begingroup} dans
les \og \meta{commandes avant} \fg et \cmd{\endgroup} dans les \og 
\meta{commandes après} \fg. Par exemple :
%
\begin{verbatim}
\specialcomment{smalltt}
    {\begingroup\ttfamily\footnotesize}{\endgroup}
\end{verbatim}
%
Vous \emph{n'avez pas} besoin d'ajouter en complément : 
%
\begin{verbatim}
\includecomment{smalltt}
\end{verbatim}
%
Pour retirer les blocs \og \env{smalltt} \fg, indiquez 
\verb+\excludecomment{smalltt}+ après la définition.

Le traitement des commentaires peut appliquer un traitement à chaque ligne.
%
\begin{noverb}
\cmd{\processcomment}\marg{nom}\marg{commandes par ligne}\marg{commandes 
avant}\marg{commandes après}
\end{noverb}
%
En définissant la commande 
%
\begin{verbatim}
\def\Thiscomment##1{...}
\end{verbatim}
%
dans les \meta{commandes avant}, l'utilisateur peut indiquer ce qui doit être
fait pour chaque ligne de commentaire. Ceci ne fonctionne pas bien pour le 
moment.

Voici un exemple de code pour de courtes commandes de commentaires (telle que
\verb+\masque{ce petit texte}+):
%
\begin{verbatim}
\includecomment{cond}
\newcommand{\masque}[1]{}
\begin{cond}
\renewcommand{\masque}[1]{#1}
\end{cond}
\end{verbatim}


\section{Approche simplifiée de l'implémentation}

Pour commenter quelque chose, ramasser chaque ligne en mode verbatim comme 
argument de commande et le jeter. 
Pour les inclusions, avec \LaTeX, le bloc est placé dans un fichier 
\cmd{\CommentCutFile} (par défaut \og \texttt{comment.cut} \fg) qui est alors
inclus.
En \textsc{plain} TeX (et d'autres formats), les commandes d'entrée et de
sortie sont définies comme des commandes sans aucun effet.

\section{Changements en version 3.1}

\begin{itemize}
\item mise à jour de l'adresse de l'auteur ;
\item nettoyage d'une partie du code ;
\item les éléments suivants \cmd{\begin}\marg{env} sur sa ligne sont toujours 
ignorés même si vous avez indiqué \cmd{\includecomment}\marg{env}
\item Les commentaires ne définissent plus de groupe ! Vous pouvez même
saisir 
  % 
\begin{verbatim}
\includecomment{env}
\begin{env}
\begin{itemize}
\end{env}
\end{verbatim}
  % 
 Plutôt impressionnant\ldots
\item les commentaires inclus sont écrits dans un fichier puis réinsérés.
\end{itemize}


\section{Changements en 3.2}

\begin{itemize}
\item mise à jour de \cmd{\specialcomment} (merci à Ivo Welch).
\end{itemize}


\section{Changements en 3.3}

\begin{itemize}
\item mise à jour de l'adresse de l'auteur une nouvelle fois ;
\item la commande \cmd{\CommentCutFile} devient paramétrable.
\end{itemize}


\section{Changements en 3.4}

\begin{itemize}
\item ajout de la licence publique GNU ;
\item ajout de la commande \cmd{\processcomment} car la correction d'Ivo 
(ci-dessus) a fait ressortir un manque de cohérence.
\end{itemize}

  
\section{Changements en 3.5}

\begin{itemize}
\item correction d'une erreur d'orthographe dans le titre ;
\item mise à jour de l'adresse de l'auteur ;
\item nouvelle correction de la commande \cmd{\specialcomment} ;
\item correction de l'\cmd{\excludecomment} d'un environnement défini plus tôt.
\end{itemize}

\section{Changements en 3.6}

\begin{itemize}
\item le fichier \og coupé \fg est maintenant écrit un peu plus en mode
verbatim, en utilisant \cmd{\meaning}; certaines personnes avaient indiqué
avoir rencontré des problèmes avec l'encodage ISO~latin~1 ou l'extension 
\textsf{umlaute}.
\item retrait de conditions avec \cmd{\newif}. Cette dernière serait-elle 
soudain redevenue une commande \cmd{\outer} ?
\end{itemize}


\section{Anomalies connues}

\begin{itemize}
\item la commande \cmd{\excludecomment} conduit à une espace superflue.
\item la commande \cmd{\processcomment} conduit à un saut de ligne superflu.
\end{itemize}

\end{document}
