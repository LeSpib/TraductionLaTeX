% \iffalse meta-comment
%
% Copyright 1993 1994 1995 1996 1997 1998 1999
% The LaTeX3 Project and any individual authors listed elsewhere
% in this file.
%
% This file is part of the Standard LaTeX `Tools Bundle'.
% -------------------------------------------------------
%
% It may be distributed and/or modified under the
% conditions of the LaTeX Project Public License, either version 1.1
% of this license or (at your option) any later version.
% The latest version of this license is in
%    http://www.latex-project.org/lppl.txt
% and version 1.1 or later is part of all distributions of LaTeX
% version 1999/06/01 or later.
%
% The list of all files belonging to the LaTeX `Tools Bundle' is
% given in the file `manifest.txt'.
%
% \fi
% \iffalse
%% File: longtable.dtx Copyright (C) 1990-1998 David Carlisle
%
%<*dtx>
          \ProvidesFile{f-longtable.dtx}
%</dtx>
%<package>\NeedsTeXFormat{LaTeX2e}[1995/06/01]
%<package>\ProvidesPackage{longtable}
%<driver> \ProvidesFile{longtable.drv}
% \fi
%         \ProvidesFile{f-longtable.dtx}
          [1998/05/13 v4.09 Multi-page Table package (DPC)]
%
% \iffalse
%<*driver>
\def\traductionchanges{oui}
\documentclass{ltxdoc}
\usepackage[T1]{fontenc}
\usepackage[latin1]{inputenc}
\usepackage[frenchb]{babel}
\usepackage{longtable}
\begin{document}
\DocInput{f-longtable.dtx}
\end{document}
%</driver>
% \fi
%
% \GetFileInfo{f-longtable.dtx}
% \title{Le package \textsf{longtable} \thanks{Ce fichier
%        a le num\'ero de version \fileversion, r\'evis\'e
%        le \filedate.}}
% \author{David Carlisle\thanks{Le nouvel algorithme pour aligner les morceaux (<<~chunks~>>)
% d'une table, utilis\'e dans la version 4 de ce package a \'et\'e
% r\'evis\'e, impl\'ement\'e et document\'e par David Kastrup,
% \texttt{dak@neuroinformatik.ruhr-uni-bochum.de}.}\\ traduction de Jean-Pierre Drucbert
% \thanks{Traductions additionnelles : Fran\c{c}oise Marre-Fournier
% <marre-fournier@unilim.fr> ; relecture Benjamin Bayart <bayartb@edgard.fdn.fr>}}
% \date{\filedate}
%
% \let\package\textsf
% \let\env\textsf
% \providecommand\finalclearpage{\clearpage}
%
%  \maketitle
%
% \begin{abstract}
% Ce package d\'efinit l'environnement \env{longtable}, une version
% <<~multi-page~>> de \env{tabular}.
% \end{abstract}
%
% \DeleteShortVerb{\|}
% \MakeShortVerb{\"}
%
% \changes{v0.00}{1989/11/06}
%    {`Version 0' distribu\'ee en longtab.sty
%      utilisant toujours un seul morceau pour la table enti\`ere}
%
% \changes{v1.00}{1990/12/20}
%    {Version initiale}
%
% \changes{v2.00}{1991/06/17}
%    {Prise en compte de NFSS et array.sty}
%
% \changes{v3.00}{1992/03/16}
%    {Nouvelle impl\'ementation. Les tables peuvent maintenant commencer n'importe
%     o\`u dans la page.}
%
% \changes{v3.01}{1992/04/06}
%    {(Michel Goossens) Si un morceau finit une ligne dans
%     laquelle la premi\`ere entr\'ee est vide, il appara\^\i\ des
%     erreurs. Ajout d\'ebut et fin de groupe <<~bizarres~>>, 
%     (\cs{ifnum}0 ) comme expliqu\'e dans l'Annexe D.}
%
% \changes{v3.02}{1992/04/09}
%    {(Michel Goossens) Longtable \'echoue si le compteur de table
%     est remis \`a z\'ero durant la compilation. Maintenant utilise
%     un compteur interne, mais tout de m\^eme incr\'ement de \texttt{table}
%     pour que \cs{caption} et \cs{label} fonctionnent correctement.}
%
% \changes{v3.03}{1992/06/25}
%    {Ajout de \cs{@ifundefined\{reset@font\}} pour que la
%     documentation puisse �tre r\'ealis\'ee avec d'anciennes versions de
%     \LaTeX.}
%
% \changes{v3.03}{1992/06/25}
%    {Modification du traitement de \cs{d@llar} pour correspondre
%     aux nouvelles versions de array.sty de Mittelbach(array.sty
%     v2.0h)}
%
% \changes{v3.04}{1992/11/12}
%    {(Jean-Pierre Drucbert) Longtable \'echoue quand est utilis\'ee avec
%     \cs{includeonly}.}
%
% \changes{v3.04}{1992/11/12}
%    {Correction du bug qui emp\`echa\^it le fonctionnement correct de
%     \cs{kill} dans les en-t�tes.}
%
% \changes{v3.04}{1992/11/12}
%    {(Graham Gough) \'Ecrire \cs{setlongtables} provoque un
%      message d'alerte.}
%
% \changes{v3.04}{1992/11/12}
%    {(Sebastian Rahtz) Longtable d\'epasse les capacit\'es
%     m\'emoire pour les tables extr\`emement longues. Un autre bug
%     introduit dans la V3.}
%
% \changes{v3.05}{1992/11/20}
%   {(Juergen Peus) Les tables sont correctement cod\'ees
%    quant aux l\'egendes, maintenant la l\'egende utilise
%    \cs{fnum@table}, donc \cs{tablename} sera utilis\'e.}
%
% \changes{v3.06}{1993/01/21}
%    {(Ingo Hoffmann) Longtable \'echoue avec la classe
%     letter. Le compteur de table n'est pas d\'efini, et de la
%     m�me fa�on, le fichier .aux n'est pas utilis\'e. Cette
%     version utilisera un fichier a .lta pour le style letter.}
%
% \changes{v3.07}{1993/06/09}
%    {Autorise la syntaxe \LaTeX\ \cs{setcounter}\{LTchunksize\}\{10\}}
%
% \changes{v3.08}{1993/06/09}
%     {Mise \`a jour pour \LaTeXe}
%
% \changes{v3.09}{1994/03/15}
%     {Nouveau style ltxdoc}
%
% \changes{v3.11}{1994/05/22}
%      {Ajout d'option de traitement, nouveaux styles de
%      messages d'erreur et d'alerte}
%
% \changes{v3.12}{1994/06/30}
%      {Suppression du traitement sp\'ecial de la classe
%      letter de la v3.06, inutile pour la nouvelle classe letter}
%
% \changes{v4.00}{1996/04/08}
%      {(DK) Nouvel algorithme pour aligner les morceaux de
%       table s\'epar\'es et cod\'e par David Kastrup}
%
% \changes{v4.02}{1996/04/16}
%      {(DPC/DK) Am\'elioration de la documentation}
%
% \changes{v4.06}{1997/06/28}
%      {(DK) nouvelle adresse e-mail}
%
% \CheckSum{1037}
%
%
% \makeatletter
% \def\@oddfoot{\normalfont\rmfamily\dotfill Page \thepage\dotfill}
% \def\@oddhead{\dotfill{\normalfont\ttfamily longtable.sty}\dotfill}
% \def\ps@titlepage{\let\@oddhead\@empty}
% \makeatother
%
%
% \setlength\LTleft\parindent
% \setlength\LTright\fill
% \setcounter{LTchunksize}{10}
%
% \def\v{\char`}
%
% ^^A \vbox to 100pt makes the page breaks the same on the first run.
% \changes{v3.08}{1993/06/09}
%     {Inutile d'utiliser \cs{vbox} avec les minipages \LaTeXe\ }
% \noindent\begin{minipage}[t][130pt]{\textwidth}
% \listoftables
% \end{minipage}
%
% \section{Introduction}
%
% Ce package d\'efinit un nouvel environnement, \DescribeEnv{longtable}
% \env{longtable}, qui
% poss\`ede la plupart des possibilit\'es de l'environnement
% \env{tabular}, mais produit des tables pouvant d\'epasser
% la hauteur d'une page, c'est-\`a-dire qui peuvent \^etre
% coup\'ees par l'algorithme standard de rupture de page de
% \TeX. Il partage aussi certaines possibilit\'es de
% l'environnement \env{table}. En particulier, il utilise le
% m\^eme compteur, {\tt table}, et a une commande
% "\caption" similaire. La commande "\listoftables"
% liste les tables produites par les environnements \env{table} et \env{longtable}.
%
% L'exemple suivant utilise la plupart des caract\'eristiques
% de l'environnement \env{longtable}. Un listing source de
% cet exemple est donn\'e dans la Section~\ref{listing}.
%
% \textbf{Note :} Plusieurs parties de la table suivante
% \emph{ne} seront \emph{pas} align\'ees correctement
% jusqu'\`a ce que ce document soit compil\'e plusieurs fois
% par \LaTeX{}. C'est un trait caract\'eristique de ce
% package, comme d\'ecrit ci-dessous.



% \begin{longtable}{@{*}r||p{1in}@{*}}
% KILLED & LINE!!!! \kill
% \caption
% [Un caption optionnel (pour la liste des tables)]
% {Une longue table\label{long}}\\
% \hline\hline
% \multicolumn{2}{@{*}c@{*}}%
% {Cette partie appara\^{\i}t au sommet de la table}\\
% \textsc{Premi\`ere}&\textsc{Seconde}\\
% \hline\hline
% \endfirsthead
% \caption[]{(suite)}\\
% \hline\hline
% \multicolumn{2}{@{*}c@{*}}%
%       {Cette partie appara\^{\i}t en haut de chacune des autres pages}\\
% \textbf{Premi\`ere}&\textbf{Seconde}\\
% \hline\hline
% \endhead
% \hline
% Ceci va en&bas.\\
% \hline
% \endfoot
% \hline
% Ces lignes&appara\^{\i}tront\\
% \`a la place du&bas habituel\\
% \`a la fin&de la table\\
% \hline
% \endlastfoot
% Les colonnes de \env{longtable} sont sp\'ecifi\'ees& de la\\
% m\^eme mani\`ere que dans l'environnement& \env{tabular}.\\
% "@{*}r||p{1in}@{*}"& dans ce cas.\\
% Chaque ligne se termine par une& commande "\\".\\
% La commande "\\"  a un& argument\\
% optionnel, tout comme dans l'environnement& \env{tabular}.\\[10pt]
% Voyez vous l'effet de "\\[10pt]"&?\\
% Des tas de lignes& comme celle-ci.\\
% Des tas de lignes& comme celle-ci.\\
% Des tas de lignes& comme celle-ci.\\
% Des tas de lignes& comme celle-ci.\\
% De m\^eme "\hline" peut \^etre utilis\'ee,& comme dans \env{tabular}.\\
% \hline
% Ceci \'etait une "\hline"&.\\
% \hline\hline
% Ceci \'etait une "\hline\hline"&.\\
% \multicolumn{2}{||c||}%
% {Ceci est une \ttfamily\v\\multicolumn\v{2\v}\v{||c||\v}}\\
% Si une rupture de page a lieu sur une "\hline" alors& une ligne est trac\'ee\\
% au bas d'une page et au& sommet de la suivante.\\
% \hline
% L'argument "[t] [b] [c]" de \env{tabular}& ne peut pas \^etre utilis\'e.\\
% L'argument optionnel doit \^etre choisi parmi& "[l] [r] [c]"\\
% pour sp\'ecifier si la table doit \^etre& ajust\'ee\\
% sur la gauche, sur la droite& ou centr\'ee.\\
% \hline\hline
% Des tas de lignes& comme celle-ci.\\
% Des tas de lignes& comme celle-ci.\\
% Des tas de lignes& comme celle-ci.\\
% Des tas de lignes& comme celle-ci.\\
% Des tas de lignes& comme celle-ci.\\
% Des tas de lignes& comme celle-ci.\\
% Des tas de lignes& comme celle-ci.\\
% Des tas de lignes& comme celle-ci.\\
% Des tas de lignes& comme celle-ci.\\
% Des tas de lignes& comme celle-ci.\\
% Des tas de lignes& comme celle-ci.\\
% Des tas de lignes& comme celle-ci.\\
% Des tas de lignes& comme celle-ci.\\
% Des tas de lignes& comme celle-ci.\\
% Des tas de lignes& comme celle-ci.\\
% Des tas de lignes& comme celle-ci.\\
% Des tas de lignes& comme celle-ci.\\
% Des tas de lignes& comme celle-ci.\\
% Des tas de lignes& comme celle-ci.\\
% Des tas de lignes& comme celle-ci.\\
%   Certaines lignes prennent beaucoup de place, comme celle-ci: &
%    \raggedleft Cette derni\`ere colonne est une colonne <<~p~>> si bien que cette
%   <<~ligne~>> de la table peut occuper plusieurs lignes. Notez cependant que \TeX\ ne coupera
%   jamais une page \`a l'int\'erieur d'une telle ligne. Les ruptures de page ne se produisent
%   qu'entre les rang\'ees de la table ou sur des commandes "\hline".\\
%    \tabularnewline
% Des tas de lignes& comme celle-ci.\\
% Des tas de lignes& comme celle-ci.\\
% Des tas de lignes& comme celle-ci.\\
% Des tas de lignes& comme celle-ci.\\
% Des tas de lignes& comme celle-ci.\\
% Des tas de lignes& comme celle-ci.\\
% Des tas de lignes& comme celle-ci.\\
% \hline
% Des tas\footnote{Ceci est une note} de lignes& comme celle-ci.\\
% Des tas de lignes& comme celle-ci\footnote{\env{longtable} prend des pr\'ecautions sp\'eciales
% et des notes de bas de page peuvent aussi \^etre uti\-li\-s\'ees (sans abus!) dans les colonnes `p'.}\\
% \hline
% Des tas de lignes& comme celle-ci.\\
% Des tas de lignes& comme celle-ci.\\
% \end{longtable}
%
%
\section{Taille du pas ou <<~chunk size~>>}
%
% \DescribeMacro{LTchunksize}
% Pour pouvoir traiter par \TeX\ des tables faisant plusieurs pages, il
% est n\'eces\-saire de partager la table en plus petits
% morceaux (les <<~chunks~>>, litt\'e\-ra\-le\-ment
% <<~bouch\'ees~>>, NdT), pour que \TeX\ ne soit pas oblig\'e de
% tout garder \`a la fois. Par d\'efaut, \env{longtable}
% utilise 20 lignes par pas, mais l'usager peut choisir la
% valeur qu'il d\'esire, avec par exemple,
% "\setcounter{LTchunksize}={10}".\footnote{On peut aussi
% utiliser la syntaxe plain \TeX\ {\ttfamily\v\\LTchunksize=10.}}
% Ces pas ou <<~chunks~>> n'affectent pas la rupture des pages,
% donc si vous utilisez \TeX\ avec beaucoup de m\'emoire,
% vous pouvez donner \`a "LTchunksize" l'\'equivalent
% de plusieurs pages de la table. \TeX\ s'ex\'ecutera plus
% rapidement avec un grand "LTchunksize". Cependant si
% cela est n\'eces\-saire \env{longtable} peut travailler
% avec "LTchunksize=1", auquel cas la m\'emoire
% utilis\'ee sera n\'egligeable.
% Notez que si vous utilisez les commandes pour placer un
% en-t\^ete en haut de la page (voir plus loin), la valeur de
% "LTchunksize" doit \^etre au moins \'egale au nombre
% de lignes de l'en-t\^ete ou du pied. (En pratique, une
% valeur entre~10 et~50 semble raisonnable. NdT).

% Ce document sp\'ecifie "\setcounter{LTchunksize}{10}". Si
% vous pouviez regarder la table pr\'ec\'edente apr\`es la
% {\em premi\`ere\/} ex\'ecution de \LaTeX\, vous verriez que
% les deux moiti\'es ne collent pas\,\footnote{En fait, vous
% ne voyez rien d'\'etrange ici, car au second passage de
% \LaTeX, tout est remis en place. Et ce document ne s'est
% pas fait en un seul passage (ni m\^eme en seul un
% jour). NdT}. De plus \LaTeX\ a imprim\'e un avertissement
% comme quoi la largeur des colonnes a chang\'e.
% \env{longtable} \'ecrit des informations dans le fichier
% ".aux", pour pouvoir aligner les diff\'erents pas ou
% << chunks >>. Avant la version~4, il n'utilise ces informations
% que si une commande "\setlongtables" est donn\'ee avant que
% la table commence. Cependant, maintenant, ces informations
% sont toujours utilis\'ees, grace \`a un nouvel
% algorithme\footnote{D\^u \`a David Kastrup} et donc
% "\setlongtables" n'est plus utile. Il est d\'efini (mais ne
% fait rien) pour que les anciens documents qui l'emploient
% puissent \^etre utilis\'e.

% \iffalse
% [Il est pr\'ef\'erable de ne pas
% utiliser cette commande avant que le document soit presque
% achev\'e, car sinon {\tt longtable} laissera assez de place
% dans chaque colonne pour l'entr\'ee la plus large {\em dans
% toutes les versions pr\'ec\'edentes}. (Au moins depuis la
% derni\`ere destruction du fichier {\tt.aux}). Par exemple
% vous pouvez mettre une ligne \verb|%\setlongtables| dans le
% pr\'eambule du document. Pour les derniers essais et
% l'\'epreuve finale, supprimez le \verb|%|: les tables
% s'alignent parfaitement apr\`es une nouvelle ex\'ecution de
% \LaTeX\ (deux si le fichier {\tt.aux} a \'et\'e
% d\'etruit).NdT]
% \fi
%
% \begin{table}
% \centering
% \begin{tabular}{||l|l|l||}
% \hline\hline
% Un &environnement& \env{tabular}\\
% \hline
% dans & une \env{table}&flottante\\
% \hline\hline
% \end{tabular}
% \caption{Une table flottante}
% \end{table}
%
% \section{L\'egendes et en-t\^etes}
%
% Au d\'ebut de la table vous pouvez sp\'ecifier des lignes
% qui devront appara\^{\i}tre en haut de chaque page (en
% dessous de la ligne de t\^ete, mais avant les autres lignes
% de la table).
%
% \DescribeMacro{\endhead}
% Ces lignes sont entr\'ees normalement, mais la derni\`ere
% commande "\\" est remplac\'ee par une commande "\endhead".
%
% \DescribeMacro{\endfirsthead}
% Si la premi\`ere page doit avoir une en-t\^ete
% diff\'erente, alors celle-ci doit \^etre donn\'ee en
% premier, et termin\'ee par la commande "\endfirsthead".
% \iffalse
% [Dans ce cas les deux en-t\^etes devraient avoir des
% hauteurs \`a peu pr\`es \'egales, car \env{longtable}
% laisse la m\^eme quantit\'e d'espace sur chaque page pour
% l'en-t\^ete. NdT]
% \fi
% La valeur de "\LTchunksize" doit \^etre
% choisie au moins \'egale au nombre de lignes de
% l'en-t\^ete.
% \DescribeMacro{\endfoot}
% Il y a aussi des commandes "\endfoot" et "\endlastfoot"
% \DescribeMacro{\endlastfoot}
% qui sont utilis\'ees de la m\^eme fa\c{c}on (au {\em
% d\'ebut\/} de la table) pour sp\'ecifier des lignes (ou une
% "\hline") devant appara\^{\i}tre en bas de chaque page.
% Dans certaines situations, vous pourrez vouloir placer des
% lignes qui appartiennent logiquement au corps de la table
% \`a la fin de \env{firsthead}, ou au d\'ebut de
% \env{lastfoot}. Ceci peut aider \`a contr\^oler quelles
% lignes apparaissent sur la premi\`ere  et la derni\`ere
% page de la table.
%
% \DescribeMacro{\caption}
% La commande "\caption{...}" est essentiellement
% \'equivalente \`a\\
% "\multicolumn{n}{c}{\parbox{\LTcapwidth}{...}}"  o\`u {\tt
% n} est le nombre de colonnes de la table. Vous pouvez
% forcer la largeur de la l\'egende avec une commande telle que
% "\setlength{LTcapwidth}{2in}" dans le pr\'eambule de votre
% document. La valeur par d\'efaut est de quatre pouces.
% "\caption" \'ecrit aussi les informations pour produire une
% entr\'ee dans la liste des tables. Comme pour la commande
% "\caption" des environnements \env{figure} et \env{table},
% un argument optionnel sp\'ecifie le texte devant
% appara\^{\i}tre dans la liste des tables si ce texte est
% diff\'erent de celui apparaissant dans la l\'egende
% elle-m\^eme. Donc la l\'egende pour la table~\ref{long} a
% \'et\'e sp\'ecifi\'es par {\ttfamily  "\caption"[Une l\'egende
% optionnelle (pour la liste des tables)]\v{Une longue
% table"\label{long}"\v}}
% Vous pouvez d\'esirer que la l\'egende sur les pages
% suivantes soit diff\'erente de celle de la premi\`ere page.
% Dans ce cas placez la commande "\caption" dans le premier
% en-t\^ete, et une autre l\'egende dans une commande
% "\caption[]" dans l'en-t\^ete principal. Si l'argument
% optionnel de la commande "\caption" est vide, aucune
% entr\'ee n'est ajout\'ee dans la liste des tables. De plus,
% si vous ne voulez pas que le num\'ero de table soit
% imprim\'e chaque fois, utilisez la commande "\caption*".
%
% Le l\'egendes sont formatt\'es suivant le code de la classe
% \package{article}. Si vous avez red\'efini la commande
% standard "\@makecaption" pour produire un format
% diff\'erent de l\'egende, vous serez peut-\^etre amen\'e
% \`a r\'ealiser des changements similaires pour le
% "\LT@makecaption" de la version de \package{longtable}.
% Voir le code pour plus de d\'etails. Une m\'etode plus
% pratique pour personnaliser les l\'egendes donn\'ee par le
% package \package{caption2}, qui fournit des commandes de
% l\'egendes personnalis\'es, et qui adapte les l\'egendes
% dans les environnements standards. De nombreux autres
% environnements fournis par des packages (incluant le
% package \package{longtable}) sont modifi\'es de fa\c{c}on
% compatible.
%
%
% Vous pouvez utiliser la commande "\label" pour faire des
% r\'ef\'erences crois\'es aux longues tables avec "\ref" (ou
% "\pageref", NdT). Notez dependant que la commande "\label"
% ne doit pas \^etre utilis\'ee dans un en-t\^ete qui peut
% appara\^\i{}tre plus d'une fois. Placez la, soit dans le
% \env{fisrthead} ou dans le corps de la table. Elle ne doit
% \^etre la \emph{premi\`ere} commande de n'importe quelle
% entr\'ee. [Les r\`egles exactes sur les endroits o\`u il
% est possible de placer la commande "\label" sont assez
% \TeX{}niques, mais il est toujours possible et inoffensif
% de la placer {\em \`a l'int\'erieur\/} de l'argument d'une
% commande "\caption". NdT].
%
% \section{Entr\'ees multicolonnes}
%
% La commande "\multicolumn" peut \^etre utlis\'ee exactement
% de la m\^eme mani\`ere que dans \env{tabular}. Vous pouvez
% vouloir sauter cette section, qui est plut\^ot technique,
% cependant prendre en compte "\multicolumn" est un des
% principaux probl\`emes pour un environnement tel que
% \env{longtable}. Le principal effet que l'utilisateur verra
% est que certaines combinaisons des entr\'ees "\multicolumn"
% auront comme r\'esultat que le document devra \^etre
% plusieurs fois compil\'e par \LaTeX\ avant que les
% diff\'erents morceaux (<<~chunks~>>) d'une table ne soient
% align\'es.
%
% Les exemples de cette section sont r\'ealis\'es en
% utilisant un "LTchunksize" fix\'e \`a la valeur minimum,
% pour montrer les effects lorsque les entr\'es
% "\multicolumn" se produisent dans diff\'erents morceaux.

% \begin{table}[!htp]
% \begin{center}
% \LTchunksize=1
%  \makeatletter
%
%  \global\let\LT@save@row\relax
%  \let\LT@warn\@gobble
%  \let\LT@final@warn\relax
%
%  \newcommand\ltexample[1]{
%  \stepcounter{LT@tables}
%  \expandafter\let\csname LT@\romannumeral\c@LT@tables\endcsname
%                   \LT@save@row
%  \addtocounter{LT@tables}{-1}
%  \begin{longtable}{|c|c|c|}
%  \caption{Une combinaison difficile de {\cs{multicolumn}} :
%                                pass #1\label{pass#1}}\\
%    \hline
%    1&2&3\\
%    \multicolumn{3}{|c|}{une multicolonne tr\'es large 1--3}\\
%    \multicolumn{2}{|c|}{une multicolonne 1--2}&3\\
%    wide 1&2&3\\
%    \hline
% \end{longtable}
% }
%
% \ltexample{1}
%
% \ltexample{2}
%
% \ltexample{3}
%
% \ltexample{4}
%
% \end{center}
% \end{table}
%
% Consid\'erons la Table~\ref{pass1}. Dans le deuxi\`eme
% morceau, \env{longtable} voit l'entr\'e large de la
% multicolonne. \`A ce point, il consid\`ere que les deux
% premi\`eres colonnes sont tr\'es \'etroites. Toute la
% largeur de l'entr\'ee multicolonne est suppos\'ee \^etre
% dans la troisi\`eme colonne (c'est une
% <<~caract\'eristique~>> de la commande originelle "\halign"
% de \TeX). Ensuite \env{longtable} prend en compte
% l'information selon laquelle il y a une troisi\`eme colonne
% large pour le dernier morceau. Le r\'esultat est que la
% premi\`ere <<~passe~>> est trop <<~large~>>. Si les
% <<~lignes sauv\'ees~>> \`a partir de cette premi\`ere passe
% sont r\'eins\'er\'ees dans la table \`a la passe suivante,
% la table sera align\'ee en deux passes mais pourra \^etre
% beaucoup trop large.
%
% \DescribeMacro{\kill}%
% La solution de ce probl\`eme, employ\'ee dans les versions
% 1~et~2, est d'utiliser une <<~"\kill" line~>>
% (litt\'eralement, une ligne tu\'ee, NdT). Si une ligne est
% tu\'ee ("\kill"ed), par l'utilisation de "\kill", au lieu
% de "\\", \`a la fin de la ligne, cela est utilis\'e dans le
% calcul des largeurs de colonne, mais enlev\'e de la table
% finale. Donc, l'entr\'ee de copies tu\'ees ("\kill"ed
% copies) des deux derni\`eres lignes, avant l'entr\'ee
% enti\`ere de la colonne, devrait signifier que "\halign"
% <<~voyait~>> l'entr\'ee compl\`ete dans les deux premi\`eres
% colonnes, et donc, \'elargissait la troisi\`eme colonne de
% fa\c{c}on \`a faire la place \`a l'entr\'ee multicolonne.
%
% Dans la version~3, une nouvelle solution a \'et\'e
% introduite. Si la ligne sauv\'ee dans le fichier ".aux" n'a
% pas \'et\'e utilis\'ee, \env{longtable} utilisait une forme
% <<~brouillon~>> de "\multicolumn"~; cela modifiait la
% d\'efinition, de fa\c{c}on \`a ce que l'entr\'ee \'etendue ne
% soit pas plus large que la colonne \'elargie.
% Donc, apr\`es la premi\`ere passe, le fichier ".aux"
% stockait la plus large entr\'ee normale pour chaque
% colonne, aucune colonne n'\'etant \'elargie \`a cause des
% colonnes "\span"\'ees. Par d\'efaut, \env{longtable}
% ignorait le fichier ".aux", et donc chaque compilation par
% \LaTeX\ \'etait consid\'er\'ee comme la premi\`ere passe.
% Une fois la d\'eclaration de "\setlongtables" faite, la
% ligne sauv\'ee dans le fichier ".aux" et la d\'efinition
% correcte de "\multicolumn" \'etait utilis\'ees.
% Si une entr\'ee "\multicolumn" entra\^\i{}nait l'\'elargissement
% d'une colonne, cette information ne pouvait \^etre
% ignor\'ee pour le morceau suivant, et donc la table ne
% pouvait pas \^etre correctement align\'ee jusqu'\`a la
% troisi\`eme passe. Cette algorithme convergeait toujours en
% trois passes, comme d\'ecrit ci-dessus, mais dans les
% exemples tels qu'un de ceux des tables
% \ref{pass1}--\ref{pass4}, les largeurs finales n'\'etaient
% pas optimales pour celle de la colonne~2 : en effet, cette
% largeur est d\'etermin\'ee par une entr\'ee "\multicolumn"
% qui n'est pas connue quand la largeur finale de la
% colonne~3 a \'et\'e d\'etermin\'ee car les \emph{deux}
% commandes "\multicolumn" sont interverties entre les modes
% <<~brouillon~>> et <<~normal~>> en m\^eme temps.
%
% La version~4 all\`ege consid\'erablement le probl\`eme. La
% premi\`ere compilation de la table aura en effet une
% troisi\`eme colonne beaucoup trop large. Cependant, \`a la
% passe suivante, \env{longtable} prendra en compte l'erreur
% et r\'eduira la largeur de colonne de fa\c{c}on conforme.
% Si cela doit d\'eplacer le morceau avant celui de
% "\multicolumn", une passe additionnelle, bien s\^ur, sera
% n\'ecessaire.
%
% Il est possible de construire des tables dans lesquelles
% cette d\'etermination en cascade des bonnes largeurs
% n\'ecessite plusieurs compilation pour <<~converger~>> et
% produire une table avec tous ses morceaux align\'es.
% Cependant, pour avoir besoin de plusieurs compilations, il
% faut construire une table avec de nombreuses entr\'ees
% "\multicolumn" se chevauchant, toutes \'etant plus larges
% que la largeur naturelle des colonnes \`a \'etendre et
% toutes se produisant dans des parties de table
% diff\'erentes.
%
% Dans le cas le plus courant, l'algorithme convergera
% apr\`es deux ou trois passes ; on aura l'avantage de ne pas
% r\'e\'editer le document avant la compilation finale pour
% ajouter "\setlongtables" et le meilleur choix de la largeur
% finale des colonnes, en cas de plusieurs entr\'ees
% "\multicolumn" sera, heureusement, plus ais\'e que les
% passes suppl\'ementaires qui seraient n\'ecessaires.
%
% Donc, la table~\ref{pass1} converge apr\`es 4~passes comme
% on peut le voir table~\ref{pass4}.
%
% Vous pouvez encore acc\'el\'erer la convergence en
% introduisant judicieusement des lignes "\kill"\'ees, s'il
% vous arrive d'avoir des tables morcell\'ees, comme
% ci-dessus.
%
% Et si vous ne d\'esirez pas compiler un fichier deux fois,
% vous devez, \`a la premi\`ere ligne de chaque
% \env{longtable}, saisir une ligne "\kill"\'ee qui contient
% l'entr\'ee la plus large qui sera utilis\'ee dans chaque
% colonne. Tous les morceaux seront align\'es \`a la
% premi\`ere compilation.
%
% \section{Ajustement}
%
% L'argument optionnel de \env{longtable} contr\^ole
% l'alignement horizontal de la table. Les options possibles
% sont "[c]", % "[r]" et "[l]", pour centrer, aligner \`a
% droite et aligner \`a gauche, respectivement.
% \DescribeMacro{\LTleft}
% Normalement, le centrage est l'option par d\'efaut, mais ce
% document sp\'ecifie
% \DescribeMacro{\LTright}
% \begin{verbatim}
% \setlenght\LTleft\parindent
% \setlenght\LTright\fill
% \end{verbatim}
% dans son pr\'eambule, ce qui signifie que les tables sont
% cadr\'ees \`a gauche, mais indent\'ees comme pour un
% paragraphe normal. Des longueurs quelconques peuvent \^etre
% sp\'ecifi\'ees pour ces param\`etres, mais au moins l'une
% d'elles devrait \^etre \'elastique pour pouvoir remplir la
% largeur de la page, \`a moins que des longueurs
% \'elastiques soient ajout\'ees entre les colonnes en
% utilisant la commande "\extracolumnsep".
% Par exemple
% \begin{verbatim}
% \begin{tabular*}
% {\textwidth}{@{\extracolsep{...}}...}
% \end{verbatim}
% produit un table de pleine largeur, pour produire un effet
% similaire avec "longtable":
% \begin{verbatim}
% \setlength\LTleft{0pt}
% \setlength\LTright{0pt}
% \begin{longtable}{@{\extracolsep{...}}...}
% \end{verbatim}
%
% \section{Changements}
%
% Cette section pr\'esente les principaux changements depuis
% la version~2. Un \'etat plus d\'etaill\'e des changements
% peut \^etre produit \`a la fin du listing source si le
% fichier "ltxdoc.cfg" sp\'ecifie
% \begin{verbatim}
% \AtBeginDocument{\RecordChanges}
% \AtEndDocument{\PrintChanges}
% \end{verbatim}
%
% Changements effectu\'es entre les versions 2 et 3 :
%
% \begin{itemize} %
% \item Le m\'ecanisme d'ajout des en-t\^ete et pied de table a
% \'et\'e compl\`etement r\'e\'ecrit. Avec ce nouveau code,
% \env{longtable} ne n\'ecessite pas de "\clearpage" au
% d\'ebut de la page, ainsi une table peut commencer au
% milieu d'une page. De plus, la commande "\endlastfoot", qui
% n'\'etait pas impl\'ement\'ee de fa\c{c}on solide dans
% l'ancien code, a \'et\'e ajout\'ee ;
% \item \env{longtable} provoque maintenant une erreur s'il
% d\'emarre dans un contexte "\twocolumn" ou dans
% l'environnement \env{multicols} ;
% \item le fichier s\'epar\'e de documentation "longtable.tex"
% a \'et\'e fusionn\'e avec le fichier d'extension,
% "longtable.dtx" en utilisant le package \package{doc} de
% Mittelbach ;
% \item la prise en compte des notes de bas de page a \'et\'e
% ajout\'ee. Cependant, il est \`a noter que "\footnote" ne
% fonctionnera pas dans les sections d'<<~en-t\^ete~>> ou de
% <<~pied~>> de table. Pour pouvoir utiliser une note de bas
% de page dans ces sections (par exemple, dans une
% l\'egende), il faut utiliser "\footnotemark" au point
% d'insertion et "\footnotetext", n'importe o\`u dans le
% \emph{corps} de la table, qui appara\^\i{}tra sur la m\^eme
% page ;
% \item le traitement de "\multicolumn" a \'et\'e modifi\'e,
% rendant inutile de <<~tuer~>> ("\kill") les lignes, au
% prix, quelquefois, de r\'ealiser une troisi\'eme
% compilation par \LaTeX\ ;
% \item la commande "\newpage" fonctionne maintenant dans
% l'environnement \env{longtable}.
% \end{itemize}
%
% Changements effectu\'es entre les versions 3 et 4 :
%
% \begin{itemize}
% \item Un nouvel algorithme est utilis\'e pour aligner les
% morceaux. En plus de la plus grande largeur de chaque
% colonne, \package{longtable} se souvient de quel morceau
% produisait ce maximum. Cela lui permet de v\'erifier que le
% maximum est encore atteint dans les derni\'eres
% compilations. Comme \package{longtable} peut maintenant
% traiter les colonnes retr\'ecies pendant que le fichier est
% \'edit\'e, la commande "\setlongtables" n'est plus
% n\'ecessaire et est d\'esactiv\'ee.
% \item Un b\'en\'efice suppl\'ementaire de la nouvelle
% capacit\'e de l'algorithme \`a traiter les colonnes
% r\'etr\'ecies est qu'il produit de meilleures (par rapport
% l'algorithme pr\'ec\'edent) largeurs de colonne (plus
% \'etroites) dans le cas d'entr\'ees de "\multicolumn" se
% chevauchant dans diff\'erents morceaux ;
% \item le mode <<~brouillon~>> de "\multicolumn" a \'et\'e
% enlev\'e suivant les commandes asoci\'ees telles que
% "\LTmulticolumn" ;
% \item l'inconv\'enient du nouvel algorithme est qu'il peut
% faire plus de passes. Le maximum th\'eorique est
% approximativement deux fois la longueur d'un
% <<~encha\^\i{}nement~>> de colonnes avec des entr\'ees
% "\multicolumn" se chevauchant, bien qu'en pratique il
% converge habituellement plus vite que l'ancienne version
% (qui converge toujours en trois passes une fois la commande
% "\setlongtables" activ\'ee) ;
% \item les commandes "\\*" et "\nopagebreak" peuvent \^etre
% utilis\'ees pour contr\^oler les sauts de page.
% \end{itemize}
%
%
% \section{R\'esum\'e}
%
% ^^A Allow the table to stick into the left margin.
% \setlength{\LTleft}{0pt plus 1fill minus 1fill}
% \setlength{\LTright}{0pt}
%
% \begin{longtable}{@{}l@{\hspace{10pt}}p{.8\linewidth}@{}}
% \caption[R\'esum\'e des commandes \env{longtable}]%
%         {\normalsize R\'esum\'e des commandes \env{longtable}}\\
% \multicolumn{2}{c}{\textbf{Param\`etres}}\\*
% \hline
% "\LTleft"&
%     Glu sur la gauche de la table.           \hfill("\fill")\\
% "\LTright"&
%     Glu sur la droite de la table.          \hfill("\fill")\\
% "\LTpre"&
%     Glu avant la table.             \hfill("\bigskipamount")\\
% "\LTpost"&
%     Glu apr\`es la table.       \hfill("\bigskipamount")\\
% "\LTcapwidth"&
%     La largeur de la <<~parbox~>> contenant la l\'egende.\hfill(4in)\\
% "LTchunksize"&
%     Le nombre de ligne par pas ou <<~chunk~>>.                 \hfill(20)\\[5pt]
% \multicolumn{2}{c}{\textbf{Arguments
%      optionnels de} \ttfamily\v\\begin\v{longtable\v}}\\*
% \hline
% \it aucun& Position sp\'ecifi\'es par "\LTleft" et "\LTright".\\
% "[c]"&   Centre la table.\\
% "[l]"&   Place la table sur la gauche.\\
% "[r]"&   Place la table sur la droite.\\[5pt]
% \pagebreak[2]
% \multicolumn{2}{c}{\textbf{Commandes
%                     \`a la fin des lignes de la table}}\\*
% \hline
% "\\"&
%     Sp\'ecifie la fin d'une ligne\\
% "\\"\oarg{dim}& Finit la ligne, puis ajoute un espace vertical
%     (comme dans l'environnement \env{tabular}).\\
% "\\*"&
%     Identique \`a "\\" mais interdit un saut de page apr\`es la ligne.\\
% "\tabularnewline"&
%     \'Equivalent \`a "\\". \`A utiliser dans un contexte "\raggedright" (ou
%     similaire) qui red\'efinit "\\".\\
% "\kill"&
%     La ligne est effac\'ee mais est utilis\'ee dans le calcul des largeurs.\\
% "\endhead"&
%     Sp\'ecifie les lignes qui doivent appara\^\i{}tre en haut de chaque page.\\
% "\endfirsthead"&
%     Sp\'ecifie les lignes qui doivent appara\^\i{}tre en haut de la premi\`ere page.\\
% "\endfoot"&
%     Sp\'ecifie les lignes qui doivent appara\^\i{}tre en bas de chaque page.\\
% "\endlastfoot"&
%     Sp\'ecifie les lignes qui doivent appara\^\i{}tre en bas de la derni\`ere page.\\[5pt]
% \multicolumn{2}{c}{Les commandes de l\'egende de \textbf{\env{longtable}}}\\*
% \hline
% "\caption"\marg{caption}&
%     L\'egende `Table ?: \meta{caption}', et une entr\'ee `\meta{caption}'
%     dans la liste des tables.\\
% "\caption"\oarg{lot}\marg{caption}&
%     L\'egende `Table ?: \meta{caption}', et une entr\'ee `\meta{lot}'
%     dans la liste des tables.\\
% "\caption[]"\marg{caption}&
%     L\'egende `Table ?: \meta{caption}',
%     mais pas d'entr\'ee dans la liste des tables.\\
% "\caption*"\marg{caption}&
%     L\'egende `\meta{caption}', mais pas d'entr\'ee dans la liste des tables.\\[5pt]
% \multicolumn{2}{c}{%^^A
%        \textbf{Commandes disponibles au d\'ebut d'une ligne}}\\*
% \hline
% "\pagebreak"&
%     Force un saut de page.\\*
% "\pagebreak"\oarg{val}& Un poids (<<~hint~>>) entre 0 et 4
%  pour autoriser un saut.\\
% "\nopagebreak"& Interdit un saut de page .\\*
% "\nopagebreak"\oarg{val}& Un poids (<<~hint~>>) entre 0 et 4 pour interdire
%    un saut.\\
% "\newpage"&
%     Force un saut de page.\\[5pt]
% \multicolumn{2}{c}{\textbf{Commandes de note de bas de page
%                      disponible dans \env{longtable}}}\\*
% \hline
% "\footnote"&
%     Note de bas de page, mais qui ne doit pa \^etre utilis\'ee dans les t\^etes et pieds de tables.\\*
% "\footnotemark"&
%     Marque de la note de bas de page, peut \^etre utilis\'ee dans les t\^etes et pieds de tables.\\*
% "\footnotetext"&
%     Texte de la note de bas de page, \`a utiliser dans le corps de la page.\\[5pt]
% \multicolumn{2}{c}{\textbf{Setlongtables}}\\
% \hline
% "\setlongtables"&  Commande obsol\`ete. Ne fait plus rien.
% \end{longtable}
%
%
% \finalclearpage
% \section{Source de la Table \protect\ref{long}}
% \label{listing}
% \begingroup\catcode`\/=0
% \begin{verbatim}
% \begin{longtable}{@{*}r||p{1in}@{*}}
% KILLED & LINE!!!! \kill
% \caption
% [Un caption optionnel (pour la liste des tables)]
% {Une longue table\label{long}}\\
% \hline\hline
% \multicolumn{2}{@{*}c@{*}}%
% {Cette partie appara\^{\i}t au sommet de la table}}\\
% \textsc{Premi\`ere}&\textsc{Seconde}\\
% \hline\hline
% \endfirsthead
% \caption[]{(suite)}\\
% \hline\hline
% \multicolumn{2}{@{*}c@{*}}%
%       {Cette partie appara\^{\i}t en haut de chacune des autres pages}\\
% \textbf{Premi\`ere}&\textbf{Seconde}\\
% \hline\hline
% \endhead
% \hline
% Ceci va en&bas.\\
% \hline
% \endfoot
% \hline
% Ces lignes&appara\^{\i}tront\\
% \`a la place du&bas habituel\\
% \`a la fin&de la table\\
% \hline
% \endlastfoot
% Les colonnes de \env{longtable} sont sp\'ecifi\'ees& de la\\
% m\^eme mani\`ere que dans l'environnement& \env{tabular}.\\
% ...
% \multicolumn{2}{||c||}%
% {Ceci est une ...}\\
% ...
%   Certaines lignes prennent beaucoup de place, comme celle-ci: &
%    \raggedleft Cette derni\`ere colonne est une colonne <<~p~>> \\
%    si bien que cette <<~ligne~>> de la table peut occuper \\
%    plusieurs lignes. Notez cependant que \TeX\ ne coupera \\
%    jamais une page \`a l'int\'erieur d'une telle ligne. \\
%    Les ruptures de page ne se produisent qu'entre les rang\'ees \\
%    de la table ou sur des commandes "\hline".\\
%    \tabularnewline
% ...
% Des tas de lignes& comme celle-ci.\\
% ...
% \hline
% Des tas\footnote{...} de lignes& comme celle-ci.\\
% Des tas de lignes& comme celle-ci\footnote{...}\\
% \hline
% Des tas de lignes& comme celle-ci.\\
% ...
% \end{longtable}
% \end{verbatim}
% \endgroup
%
% \StopEventually{}
%
% \finalclearpage
% \section{Les macrocommandes}
%
%    \begin{macrocode}
%<*package>
%    \end{macrocode}
%
% \subsection{Code initial}
%
% Avant de d\'eclarer les options du package, nous devons
% d\'efinir les valeurs par d\'efaut ici.
%
% \begin{macro}{\LT@err}
% La commande de g\'en\'eration d'erreur
%    \begin{macrocode}
\def\LT@err{\PackageError{longtable}}
%    \end{macrocode}
%  \end{macro}
%
%  \begin{macro}{\LT@warn}
% La commande de g\'en\'eration des alertes
%    \begin{macrocode}
\def\LT@warn{\PackageWarning{longtable}}
%    \end{macrocode}
%  \end{macro}
%
%  \begin{macro}{\LT@final@warn}
% \changes{v4.04}{1996/05/24}
%      {Ajout de macro}
% Si des \env{longtable}s ne sont pas align\'ees, g\'en\`ere une alerte
% \`a la fin de la compilation, "\AtEndDocument".
%    \begin{macrocode}
\def\LT@final@warn{%
  \AtEndDocument{%
    \LT@warn{Table \@width s have changed. Rerun LaTeX.\@gobbletwo}}%
  \global\let\LT@final@warn\relax}
%    \end{macrocode}
%  \end{macro}
%
% \subsection{Options}
%
% Les deux premi\`res options traitent le maniement des erreurs.
% Elles sont compatibles avec les options du package \texttt{tracefnt}
%
%  \begin{macro}{errorshow}
% Affiche \emph{seulement} les erreurs  \`a l'\'ecran. Les
% messages d'alertes apparaissent uniquement dans le fichier log.
% \changes{v3.14}{1995/04/25}
%      {Changement de la cha\^\i{}ne de LT \`a longtable}
%    \begin{macrocode}
\DeclareOption{errorshow}{%
  \def\LT@warn{\PackageInfo{longtable}}}
%    \end{macrocode}
%  \end{macro}
%
%  \begin{macro}{pausing}
% \changes{v3.14}{1995/04/25}
%      {Changement de la cha\^\i{}ne de LT \`a longtable}
% \changes{v3.15}{1995/06/15}
%      {Utilisation d'un seul hash, corrige latex/1557}
% Dans une erreur, tous les messages d'alerte stoppent la compilation par \TeX{}.
% Peut \^etre utile pour le debuggage.
%    \begin{macrocode}
\DeclareOption{pausing}{%
  \def\LT@warn#1{%
    \LT@err{#1}{This is not really an error}}}
%    \end{macrocode}
%  \end{macro}
%
%  \begin{macro}{set}
%  \begin{macro}{final}
% \changes{v4.01}{1996/04/11}
%      {(DPC) make set and final options no op}
% Les options suivantes pr\'esentent seulement une syntaxe
% alternative pour la d\'eclaration de "\setlongtables".
%    \begin{macrocode}
\DeclareOption{set}{}
\DeclareOption{final}{}
%    \end{macrocode}
%  \end{macro}
%  \end{macro}
%
%
%    \begin{macrocode}
\ProcessOptions
%    \end{macrocode}
%
% \subsection{Param\`etres utilisateur}
%
% \begin{macro}{\LTleft}
% \begin{macro}{\LTright}
% Glu sur la gauche ou la droite de la table, "\fill" par
d\'efaut (\emph{i.e.} centr\'ee).
%    \begin{macrocode}
\newskip\LTleft       \LTleft=\fill
\newskip\LTright      \LTright=\fill
%    \end{macrocode}
% \end{macro}
% \end{macro}
%
% \begin{macro}{\LTpre}
% \begin{macro}{\LTpost}
% Glu avant et apr\`es \env{longtable}. Par d\'efaut, "\bigskip".
%    \begin{macrocode}
\newskip\LTpre        \LTpre=\bigskipamount
\newskip\LTpost       \LTpost=\bigskipamount
%    \end{macrocode}
% \end{macro}
% \end{macro}
%
% \begin{macro}{\LTchunksize}
% Taille du pas (<<~chunk~>>) : le nombre de ligne donn\'e par "\halign", 20 par d\'efaut.
%    \begin{macrocode}
\newcount\LTchunksize \LTchunksize=20
%    \end{macrocode}
% \end{macro}
%
% \begin{macro}{\c@LTchunksize}
% Ajout\'e dans la V3.07 pour autoriser la syntaxe \LaTeX\
% "\setcounter{LTchunksize}{10}".
%    \begin{macrocode}
\let\c@LTchunksize\LTchunksize
%    \end{macrocode}
% \end{macro}
%
% \begin{macro}{\LTcapwidth}
% Largeur de la "\parbox" contenant la l\'egende, 4in par d\'efaut.
%    \begin{macrocode}
\newdimen\LTcapwidth  \LTcapwidth=4in
%    \end{macrocode}
% \end{macro}
%
% \subsection{Param\`etres internes}
%
% \begin{macro}{\LT@head}
% \begin{macro}{\LT@firsthead}
% \begin{macro}{\LT@foot}
% \begin{macro}{\LT@lastfoot}
% Bo\^\i{}tes pour les t\^ete et pied de table.
%    \begin{macrocode}
\newbox\LT@head
\newbox\LT@firsthead
\newbox\LT@foot
\newbox\LT@lastfoot
%    \end{macrocode}
% \end{macro}
% \end{macro}
% \end{macro}
% \end{macro}
%
% \begin{macro}{\LT@cols}
% Compteur de num\'ero des colonnes.
%    \begin{macrocode}
\newcount\LT@cols
%    \end{macrocode}
% \end{macro}
%
% \begin{macro}{\LT@rows}
% Compteur du pas (<<~chunksize~>>)
%    \begin{macrocode}
\newcount\LT@rows
%    \end{macrocode}
% \end{macro}
%
% \begin{macro}{\c@LT@tables}
% Compteur de table, ajout\'e dans la V3.02. Les versions
% pr\'ec\'edentes utilisaient seulement de compteur "table"
% de \LaTeX\, mais cela \'echouait si "table" \'etait remis
% \`a z\'ero pendant la compilation ; par exemple, la classe
% \package{report} le remet \`a z\'ero dans chaque chapitre.
%
% Cela a \'et\'e modifi\'e, dans la V3.04, par l'utilisation de
% "\newcount\LT@tables". Les compteurs \LaTeX\ sont
% correctement pr\'eserv\'es lorsque "\includeonly" est
% utilis\'e. Dans le reste du fichier, "\LT@tables" a \'et\'e
% remplac\'e par "\c@LT@tables" sans autre commentaire
% suppl\'ementaire.
%    \begin{macrocode}
\newcounter{LT@tables}
%    \end{macrocode}
% \end{macro}
%
% \begin{macro}{\c@LT@chunks}
% \changes{v4.00}{1996/04/08}
%      {(DK) Ajout de compteur de pas}
% Nous avons besoin de compter les pas d'une table depuis la Version~4.
%    \begin{macrocode}
\newcounter{LT@chunks}[LT@tables]
%    \end{macrocode}
% \end{macro}
%
% \begin{macro}{\c@table}
% \begin{macro}{\fnum@table}
% \begin{macro}{\tablename}
% Si le compteur "table" n'est pas d\'efini (par exemple dans le style "letter"),
% cette macro le d\'efinit (Ajout\'e dans la V3.06).
%    \begin{macrocode}
\ifx\c@table\undefined
  \newcounter{table}
  \def\fnum@table{\tablename~\thetable}
\fi
\ifx\tablename\undefined
  \def\tablename{Table}
\fi
%    \end{macrocode}
% \end{macro}
% \end{macro}
% \end{macro}
%
% \begin{macro}{\LT@out}
% Dans les styles standards, "longtable" utilise le fichier
% ".aux" pour enregistrer les largeurs de colonne. Avec
% "letter.sty", le package utilise un fichier s\'epar\'e
% ".lta". (Ajout\'e dans la V3.06).
%
% Inutile pour la nouvelle classe "letter".
% \begin{verbatim}
%\ifx\startlabels\undefined
%  \let\@auxout\@auxout
%\else
%  {\@input{\jobname.lta}}%
%  \newwrite\@auxout
%  \immediate\openout\@auxout=\jobname.lta
%\fi
%\end{verbatim}
% \end{macro}
%
% \begin{macro}{\LT@p@ftn}
% Stockage temporaire du texte de la note de bas de page dans un colonne <<~p~>>.
%    \begin{macrocode}
\newtoks\LT@p@ftn
%    \end{macrocode}
% \end{macro}
%
% \begin{macro}{\LT@end@pen}
% Poids (<<~penalty~>>) sp\'ecial pour la fin de la table.
% R\'ealis\'e comme \c{c}a pour \'epargner un compteur.
%    \begin{macrocode}
\mathchardef\LT@end@pen=30000
%    \end{macrocode}
% \end{macro}
%
% \subsection{L'environnement \env{longtable}}
%
% \begin{macro}{\longtable}
% Appel\'e par "\begin{longtable}". Cette impl\'ementation
% ne fonctionne pas dans un format \`a colonnes multiples. "\par" ajout\'e dans la V3.04.
%    \begin{macrocode}
\def\longtable{%
  \par
  \ifx\multicols\@undefined
  \else
     \ifnum\col@number>\@ne
       \@twocolumntrue
     \fi
  \fi
  \if@twocolumn
    \LT@err{longtable not in 1-column mode}\@ehc
  \fi
  \begingroup
%    \end{macrocode}
% V\'erifie la pr\'esence d'un argument optionnel.
%    \begin{macrocode}
  \@ifnextchar[\LT@array{\LT@array[x]}}
%    \end{macrocode}
% \end{macro}
%
% \begin{macro}{\LT@array}
% Commence l'ajustement de l'alignement. Bas\'e sur
%"\@array" du noyau \LaTeX\ et sur le package \package{array}
%
% Depuis la Version 3.02, \package{longtable} utilise le
% compteur interne "\c@LT@tables". Le compteur "table" de
% \LaTeX\ est encore incr\'ement\'e pour que "\caption"
% fonctionne correctement.
%    \begin{macrocode}
\def\LT@array[#1]#2{%
  \refstepcounter{table}\stepcounter{LT@tables}%
%    \end{macrocode}
% Cr\'ee la justification (<<~glu~>>) autour de la table,
% si un argument optionnel est donn\'e.
%    \begin{macrocode}
  \if l#1%
    \LTleft\z@ \LTright\fill
  \else\if r#1%
    \LTleft\fill \LTright\z@
  \else\if c#1%
    \LTleft\fill \LTright\fill
  \fi\fi\fi
%    \end{macrocode}
% Fixe ces commandes internes pour \env{longtable}.
% \changes{v3.13}{1994/12/08}
%      {ajout \cs{tabularnewline}}
% \changes{v3.14}{1995/04/25}
%      {(Mike Van Geest) renommage de  \cs{LT@mc} en
%         \cs{LT@mcol} pour autoriser 1100 tables}
% \changes{v4.01}{1996/04/11}
%      {(DPC) message d'alerte multicolumn inutile}
%\begin{verbatim}
%  \global\let\LT@mcw@rn\relax
%\end{verbatim}
%    \begin{macrocode}
  \let\LT@mcol\multicolumn
%    \end{macrocode}
% \changes{v3.17}{1996/01/31}
%      {Remise \`a z\'ero de \cs{hline} et \cs{multicolumn}
%      dans les environnements \env{tabular} et \env{array}
%      encapsul\'es, corrige tools/2068}
% Maintenant red\'efinit "\@tabarray", pour r\'etablir "\hline" et
% "\multicolumn", pour que array et tabular
% encapsul\'es dans longtable (ou dans l'en-t\^ete de
% page d'un longtable) donnent le r\'esultat
% escompt\'e. Les d\'efinitions originales sont ici
% sauvegard\'ees pour que vous puissiez charger le package
% \package{array} avant ou apr\`es \package{longtable}.
%    \begin{macrocode}
  \let\LT@@tabarray\@tabarray
  \let\LT@@hl\hline
  \def\@tabarray{%
    \let\hline\LT@@hl
%    \end{macrocode}
%\begin{verbatim}
%    \let\multicolumn\LT@mcol
%\end{verbatim}
%    \begin{macrocode}
    \LT@@tabarray}%
  \let\\\LT@tabularcr\let\tabularnewline\\%
  \def\newpage{\noalign{\break}}%
%    \end{macrocode}
% \changes{v4.05}{1996/11/12}
%      {Ajout de \cs{nopagebreak} et \cs{pagebreak}}
% D\'efinitions plus ou moins standards, mais commencent d'abord un "\noalign".
%    \begin{macrocode}
  \def\pagebreak{\noalign{\ifnum`}=0\fi\@testopt{\LT@no@pgbk-}4}%
  \def\nopagebreak{\noalign{\ifnum`}=0\fi\@testopt\LT@no@pgbk4}%
%    \end{macrocode}
%
%    \begin{macrocode}
  \let\hline\LT@hline \let\kill\LT@kill\let\caption\LT@caption
  \@tempdima\ht\strutbox
%    \end{macrocode}
%
% \changes{v4.08}{1998/01/20}
%      {Placement plus pr\'ecoce de la d\'efinition de \cs{@endpbox} et d\'efinition de \cs{@@endpbox}
%      et \cs{@@startpbox} pour le mode non-array. tools/2736}
%    \begin{macrocode}
  \let\@endpbox\LT@endpbox
%    \end{macrocode}

% Commandes internes \'etablies en accord avec celles de Lamport ou
% celles de Mittelbach.
%    \begin{macrocode}
  \ifx\extrarowheight\@undefined
%    \end{macrocode}
% Initialise ces commandes comme dans \env{tabular} du noyau \LaTeX.
%    \begin{macrocode}
    \let\@acol\@tabacol
    \let\@classz\@tabclassz \let\@classiv\@tabclassiv
    \def\@startpbox{\vtop\LT@startpbox}%
    \let\@@startpbox\@startpbox
    \let\@@endpbox\@endpbox
    \let\LT@LL@FM@cr\@tabularcr
  \else
%    \end{macrocode}
% Initialise ces commandes comme dans \package{array}.
% "\d@llar"remplac\'e par "\d@llarbegin" "\d@llarend"  dans
% la V3.03 pour correspondre \`a \package{array} V2.0h. Nous
% n'avons pas besoin de fixer la valeur de "\d@llarbegin" et
% "\d@llarend" puisque le package \package{array} leur
% donne leurs valeurs correctes au premier niveau.
%    \begin{macrocode}
    \advance\@tempdima\extrarowheight
    \col@sep\tabcolsep
    \let\@startpbox\LT@startpbox\let\LT@LL@FM@cr\@arraycr
  \fi
%    \end{macrocode}
% La reste de cette macro est principalement bas\'e sur le
% package \package{array}, mais devrait aussi fonctionner
% pour l'environnement \env{tabular} standard.
%    \begin{macrocode}
  \setbox\@arstrutbox\hbox{\vrule
    \@height \arraystretch \@tempdima
    \@depth \arraystretch \dp \strutbox
    \@width \z@}%
  \let\@sharp##\let\protect\relax
%    \end{macrocode}
% Interpr\'ete l'argument du pr\'eambule.
%    \begin{macrocode}
   \begingroup
    \@mkpream{#2}%
%    \end{macrocode}
% Nous avons besoin de renommer ici "\@preamble" puisque le
% syst\`eme de F. M. utilise "\global", et nous avons peut
% \^etre besoin d'encapsuler "\@mkpream", par exemple pour
% "\multicolumn" ou pour un \env{array}. Nous n'avons pas
% besoin de nous pr\'eoccuper des \env{longtable}s
% encapsul\'ees.
%    \begin{macrocode}
    \xdef\LT@bchunk{%
%    \end{macrocode}
% \changes{v4.00}{1996/04/08}
%      {(DK) Incr\'ementation du compteur de pas}
%    \begin{macrocode}
       \global\advance\c@LT@chunks\@ne
       \global\LT@rows\z@\setbox\z@\vbox\bgroup
%    \end{macrocode}
% \changes{v4.05}{1996/11/12}
%      {Ajout de \cs{LT@setprevdepth}}
% La ligne suivante a \'et\'e ajout\'ee dans la v4.05.
% Pour que "\penalties" soit pris en compte aux limites des
% morceaux, il est n\'ecessaire de se soucier de o\`u et
% quand "\lineskip" est ajout\'e. La commande suivante ne
% fait rien en haut de la table, ni dans les pas d'en-t\^ete,
% mais dans les morceaux du corps de la table, cela fixe
% "\prevdepth" (\`a 0pt, mais n'importe quelle valeur
% convient) pour que "\lineskip" soit ajout\'ee. La chose
% importante \`a noter est que la glu sera ajout\'ee
% \emph{apr\`es} n'importe quelle donn\'ee verticale
% provenant de "\noalign".
%    \begin{macrocode}
       \LT@setprevdepth
%    \end{macrocode}
%
%    \begin{macrocode}
       \tabskip\LTleft\halign to\hsize\bgroup
      \tabskip\z@ \@arstrut \@preamble \tabskip\LTright \cr}%
  \endgroup
%    \end{macrocode}
% D\'ecouvre combien nous avons de colonnes (stock\'e dans "\LT@cols").
%    \begin{macrocode}
  \expandafter\LT@nofcols\LT@bchunk&\LT@nofcols
%    \end{macrocode}
% Obtient les lignes sauv\'ees \`a partir de "\LT@i"\ldots"\LT@ix"
% (\`a partir du fichier".aux"), ou ins\`ere une nouvelle ligne vide.
%    \begin{macrocode}
  \LT@make@row
%    \end{macrocode}
% Quelques commandes internes suppl\'ementaires pour \env{longtable}.
%    \begin{macrocode}
  \m@th\let\par\@empty
  \everycr{}\lineskip\z@\baselineskip\z@
%    \end{macrocode}
% D\'ebute le premier morceau.
%    \begin{macrocode}
  \LT@bchunk}
%    \end{macrocode}
% \end{macro}
%
% \begin{macro}{\LT@no@pgbk}
% \changes{v4.05}{1996/11/12}
%      {Ajout d'une macro}
% Peut simplifier le "\@no@pgbk" standard puisqu'il est
% seulement en vmode, mais n\'ecessite alors de terminer "\noalign".
%    \begin{macrocode}
\def\LT@no@pgbk#1[#2]{\penalty #1\@getpen{#2}\ifnum`{=0\fi}}
%    \end{macrocode}
% \end{macro}
%
% \begin{macro}{\LT@start}
% Cette macro d\'ebute le proc\'ed\'e de placement de la
% table dans la page courante. Il n'est pas appel\'e
% jusqu'\`a ce qu'une commande "\\" ou "\endlongtable"
% termine un morceau, puisque nous ne connaissons pas encore
% laquelle des quatre sections possibles d'en-t\^ete
% ou de pied a \'et\'e d\'efinie.
%
% Elle commence par se red\'efinir elle-m\^eme, de
% fa\c{c}on \`a ce que la table soit commenc\'ee une seule
% fois ! Dans la V3.04, elle \'etait red\'efinie en "\relax",
% maintenant, elle utilise "\endgraf" pour forcer
% l'activation de saut de page.
%    \begin{macrocode}
\def\LT@start{%
  \let\LT@start\endgraf
  \endgraf\penalty\z@\vskip\LTpre
%    \end{macrocode}
% Commence une nouvelle page s'il n'y a pas assez de place
% pour l'en-t\^ete et le pied de page et pour une ligne
% suppl\'ementaire.
%    \begin{macrocode}
  \dimen@\pagetotal
  \advance\dimen@ \ht\ifvoid\LT@firsthead\LT@head\else\LT@firsthead\fi
  \advance\dimen@ \dp\ifvoid\LT@firsthead\LT@head\else\LT@firsthead\fi
  \advance\dimen@ \ht\LT@foot
%    \end{macrocode}
% \changes{v3.16}{1995/11/09}
%      {Mesure de la premi\`ere ligne de la table}
% \`A ce point, j'avais l'habitude d'ajouter "\ht\@arstrutbox" et
% "\dp\@arstrutbox" pour mesurer la taille d'une ligne.
% Cependant, cela peut \'echouer de fa\c{c}on spectaculaire
% pour une colonne "p" qui peut \^etre beaucoup plus large.
% Les versions pr\'ec\'edentes pouvaient finir par cr\'eer une table
% commen\c{c}ant par un pied de table, puis un saut de page
% puis un en-t\^ete puis, \emph{enfin} le <<~premier
% en-t\^ete~>> ! Donc, maintenant, on mesure la premi\`ere
% ligne de la table, pr\'ecis\'ement, en la s\'eparant
% verticalement ("\vsplit"ting) du premier morceau.
%    \begin{macrocode}
  \dimen@ii\vfuzz
  \vfuzz\maxdimen
    \setbox\tw@\copy\z@
    \setbox\tw@\vsplit\tw@ to \ht\@arstrutbox
    \setbox\tw@\vbox{\unvbox\tw@}%
  \vfuzz\dimen@ii
  \advance\dimen@ \ht
        \ifdim\ht\@arstrutbox>\ht\tw@\@arstrutbox\else\tw@\fi
  \advance\dimen@\dp
        \ifdim\dp\@arstrutbox>\dp\tw@\@arstrutbox\else\tw@\fi
%    \end{macrocode}
%
%    \begin{macrocode}
  \advance\dimen@ -\pagegoal
  \ifdim \dimen@>\z@\vfil\break\fi
%    \end{macrocode}
% Stockage de la hauteur de la page moins le pied de table dans "\@colroom".
% \changes{v3.14}{1995/05/02}
%      {D\'efinition de \cs{@colroom}, pour tools/1584}
%    \begin{macrocode}
      \global\@colroom\@colht
%    \end{macrocode}
% Si le pied n'est pas vide, r\'eduction de fa\c{c}on
% homog\`ene de "\vsize" et "\@colroom".
%    \begin{macrocode}
  \ifvoid\LT@foot\else
    \advance\vsize-\ht\LT@foot
    \global\advance\@colroom-\ht\LT@foot
    \dimen@\pagegoal\advance\dimen@-\ht\LT@foot\pagegoal\dimen@
    \maxdepth\z@
  \fi
%    \end{macrocode}
% Place l'en-t\^ete de table sur la page puis passe \`a le
% nouvelle routine de sortie.
%    \begin{macrocode}
  \ifvoid\LT@firsthead\copy\LT@head\else\box\LT@firsthead\fi
  \output{\LT@output}}
%    \end{macrocode}
% \end{macro}
%
% \begin{macro}{\endlongtable}
% Appel\'e par "\end{longtable}".
%    \begin{macrocode}
\def\endlongtable{%
%    \end{macrocode}
% Ajoute essentiellement un "\\" final. Mais comme nous
% connaissons le nombre r\'eel de morceaux, nous pouvons
% enlever toutes les entr\'ees se r\'ef\'erant \`a une entr\'ee
% maximale hors de la table (cela se produit seulement si la
% table a \'et\'e raccourcie ou que la num\'erotation de
% table a disparu). Dans ce cas, nous devons, au moins,
% commencer par r\'ecup\'erer la nouvelle information
% correcte pour le dernier morceau de cette table, en
% enlevant la contrainte de largeur.
% \changes{v4.01}{1996/04/11}
%      {(DPC) utilisation de \cs{noalign} pour l'ins\'erer dans \cs{LT@entry@chop}}
%    \begin{macrocode}
  \crcr
  \noalign{%
    \let\LT@entry\LT@entry@chop
    \xdef\LT@save@row{\LT@save@row}}%
  \LT@echunk
  \LT@start
  \unvbox\z@
  \LT@get@widths
%    \end{macrocode}
% \'Ecrit la colonne mod\`ele dans le fichier".aux".
% Depuis la V3.06, utilisation de fichier ".lta" pour "letter.sty".
% \changes{v3.12}{1994/06/30}
%      {N'\'ecrit pas si \cs{nofiles} est activ\'e.}
%    \begin{macrocode}
  \if@filesw
    {\let\LT@entry\LT@entry@write\immediate\write\@auxout{%
%    \end{macrocode}
% Depuis la Version 3.02, \package{longtable} utilise le
% compteur interne "\c@LT@tables" plut\^ot que le compteur
% \textsf{table} de \LaTeX{}. Cette information semble
% totalement diff\'erente de celle de la version 3.
% Cependant, nous n'avons pas besoin de renommer la macro car
% le code le plus r\'ecent consid\`erera qu'il n'y a pas de
% colonne et donc effacera les anciennes donn\'ees.
%    \begin{macrocode}
      \gdef\expandafter\noexpand
        \csname LT@\romannumeral\c@LT@tables\endcsname
          {\LT@save@row}}}%
  \fi
%    \end{macrocode}
% \`A ce point, \'emission d'un message d'alerte si une "\multicolumn"
% a \'et\'e fix\'ee en mode brouillon.
% \changes{v4.01}{1996/04/11}
%      {(DPC) Alertes multicolumn inutiles}
%\begin{verbatim}
%  \LT@mcw@rn
%\end{verbatim}
% Si le dernier morceau a des largeurs diff\'erentes par rapport au premier,
% alerte de l'utilisateur. De plus d\'eclenchement d'une alerte pour recompiler avec \LaTeX\
% \`a la fin du document.
% \changes{v4.04}{1996/05/24}
%      {Utilisation de \cs{LT@final@warn}}
%    \begin{macrocode}
  \ifx\LT@save@row\LT@@save@row
  \else
    \LT@warn{Column \@width s have changed\MessageBreak
             in table \thetable}%
    \LT@final@warn
  \fi
%    \end{macrocode}
% Force one more go with the \env{longtable} output routine.
%    \begin{macrocode}
  \endgraf\penalty -\LT@end@pen
%    \end{macrocode}
% Maintenant, oblige le groupe \`a revenir \`a une routine standard.
%    \begin{macrocode}
  \endgroup
%    \end{macrocode}
% Remise \`a z\'ero de "\@mparbottom" pour que les marges
% de paragraphes soient proches de la fin de la
% table.\footnote{Cela peut ne pas \^etre correct. Cependant,
% si c'est oubli\'e, il y a un probl\`eme avec les marges de
% paragraphe, [par exemple, page 3 de la documentation en V.
% O., NdT]. Y a-t'il des gourous des routine de sortie dans
% la salle ?}
%    \begin{macrocode}
  \global\@mparbottom\z@
  \pagegoal\vsize
  \endgraf\penalty\z@\addvspace\LTpost
%    \end{macrocode}
% Note de bas de page, comme cela est r\'ealis\'e dans le package \package{multicol}.
%    \begin{macrocode}
  \ifvoid\footins\else\insert\footins{}\fi}
%    \end{macrocode}
% \end{macro}
%
% \subsection{Comptage des colonnes}
%
% Les colonnes sont compt\'ees par l'inspection de "\@preamble",
% plut\^ot que, simplement, par l'obtention de
% "\@mkpream", pour incr\'ementer le compteur. De cette
% fa\c{c}on, le pr\'eambule est construit pour que le package
% soit compatible avec la plupart de ceux qui ajoutent une
% d\'efinition suppl\'ementaire de colonne au standard \LaTeX.
%
% La version 1 comptait les "\@sharp" pour calculer le
% nombre de colonnes, cela a \'et\'e chang\'e dans la
% version~2 puisque cela ne fonctionnait pas avec NFSS.
% Maintenant, comptage des "&" ["lfonts.new" (et maintenant
% la d\'efinition selon le Standard \LaTeX) d\'efinit
% "\@tabclassz" pour que "\@sharp" soit \`a l'int\`erieur
% d'un groupe.]
%
% \begin{macro}{\LT@nofcols}
% Trouve le "&" suivant, puis regarde plus avant pour voir quel est le suivant.
%    \begin{macrocode}
\def\LT@nofcols#1&{%
  \futurelet\@let@token\LT@n@fcols}
%    \end{macrocode}
% \end{macro}
%
% \begin{macro}{\LT@n@fcols}
% Ajoute un, puis stoppe au niveau d'un "\LT@nofcols" ou
% cherche le "&" suivant. L'astuce "\expandafter" a \'et\'e
% ajout\'ee et le nom de "\@LT@nofcols" a \'et\'e modifi\'e
% pour conserver la convention d'appellation, "\LT@".
%    \begin{macrocode}
\def\LT@n@fcols{%
  \advance\LT@cols\@ne
  \ifx\@let@token\LT@nofcols
    \expandafter\@gobble
  \else
    \expandafter\LT@nofcols
  \fi}
%    \end{macrocode}
% \end{macro}
%
% \subsection{Les commandes {\ttfamily\bslash\bslash} et \cs{kill}}
%
% \begin{macro}{\LT@tabularcr}
% \changes{v4.05}{1996/11/12}
%      {Code r\'e-organis\'e pour le traitement de la forme *.}
% La d\'efinition interne de "\\".
% Dans la forme "*", ins\`ere un "\nobreak" apr\`es le "\cr" (ou "\crcr") suivant.
%
% Le traitement de la forme <<~\'etoile~>> a finalement
% \'et\'e ajout\'e dans la v4.05. Durant les six ou sept
% derni\`eres ann\'ees, le commentaire de ce point disait
% \begin{quote}\small
% Cette d\'efinition accepte aussi "\\*" qui agit de le
% m\^eme fa\c{c}on que "\\". \env{tabular} fait cela mais
% \env{longtable} devra probablement d\'efinir "\\*" pour
% pr\'evenir les sauts de page.
% \end{quote}
%
% "{\ifnum0=`}\fi" ajout\'e dans la version 3.01, requis
% si la premi\`ere entr\'ee est vide.
% En fait, c'est insuffisant, puisque avec le package
% \package{array}, un groupe "{}" peut \^etre introduit en
% mode math\'ematique.
% Donc, utilise la modification suivante. Ajout\'ee dans la v3.14
% \changes{v3.14}{1995/04/25}
%      {Meilleur r\'esultat avec \cs{ifnum} cf tools/1571}
%    \begin{macrocode}
\def\LT@tabularcr{%
  \relax\iffalse{\fi\ifnum0=`}\fi
  \@ifstar
    {\def\crcr{\LT@crcr\noalign{\nobreak}}\let\cr\crcr
     \LT@t@bularcr}%
    {\LT@t@bularcr}}
%    \end{macrocode}
% \end{macro}
%
% \begin{macro}{\LT@crcr}
% \changes{v4.05}{1996/11/12}
%      {Macro ajout\'ee}
%    \begin{macrocode}
\let\LT@crcr\crcr
%    \end{macrocode}
% \end{macro}
%
% \begin{macro}{\LT@setprevdepth}
% \changes{v4.05}{1996/11/12}
%      {Macro ajout\'ee}
% Cela sera red\'efini pour fixer "\prevdepth" au d\'ebut
% d'un morceau.
%    \begin{macrocode}
\let\LT@setprevdepth\relax
%    \end{macrocode}
% \end{macro}
%
% \begin{macro}{\LT@t@bularcr}
%    \begin{macrocode}
\def\LT@t@bularcr{%
%    \end{macrocode}
% Incr\'emente le compteur et ins\`ere le "\\" de
% \env{tabular} ou termine le morceau.\\ L'astuce
% "\expandafter" a \'et\'e ajout\'e dans la version~3. Fixe "\prevdepth"
% au d\'ebut d'un nouveau morceau. (C'est fait ici et pas
% dans le morceau en t\^ete).
%    \begin{macrocode}
  \global\advance\LT@rows\@ne
  \ifnum\LT@rows=\LTchunksize
    \gdef\LT@setprevdepth{%
      \prevdepth\z@\global
      \global\let\LT@setprevdepth\relax}%
    \expandafter\LT@xtabularcr
  \else
    \ifnum0=`{}\fi
    \expandafter\LT@LL@FM@cr
  \fi}
%    \end{macrocode}
% \end{macro}
%
% \begin{macro}{\LT@xtabularcr}
% Cela cherche juste un argument optionnel.
%    \begin{macrocode}
\def\LT@xtabularcr{%
  \@ifnextchar[\LT@argtabularcr\LT@ntabularcr}
%    \end{macrocode}
% \end{macro}
%
% \begin{macro}{\LT@ntabularcr}
% La version sans argument optionnel.
% Ajout\'ee dans la version 3.01. Modifi\'ee dans la v3.14.
%    \begin{macrocode}
\def\LT@ntabularcr{%
  \ifnum0=`{}\fi
  \LT@echunk
  \LT@start
  \unvbox\z@
  \LT@get@widths
  \LT@bchunk}
%    \end{macrocode}
% \end{macro}
%
% \begin{macro}{\LT@argtabularcr}
% La version avec un argument optionnel.
% "\ifnum0=`{\fi}" dans la version 3.01. Modifi\'e dans la  v3.14.
%    \begin{macrocode}
\def\LT@argtabularcr[#1]{%
  \ifnum0=`{}\fi
  \ifdim #1>\z@
    \unskip\@xargarraycr{#1}%
  \else
    \@yargarraycr{#1}%
  \fi
%    \end{macrocode}
% Ajoute la ligne mod\`ele et termine "\halign".
%    \begin{macrocode}
  \LT@echunk
  \LT@start
  \unvbox\z@
  \LT@get@widths
  \LT@bchunk}
%    \end{macrocode}
% \end{macro}
%
% \begin{macro}{\LT@echunk}
% \changes{v4.05}{1996/11/12}
%      {\cs{unskip} ajout\'e pour prendre en compte les sauts de page}
% Cela finit le morceau en cours et efface la ligne mod\`ele.
%    \begin{macrocode}
\def\LT@echunk{%
  \crcr\LT@save@row\cr\egroup
  \global\setbox\@ne\lastbox
%    \end{macrocode}
% La ligne suivante a \'et\'e ajout\'ee dans la v4.05. \package{longtable}
% d\'epend du param\`etre "\lineskip" (qui est 0pt) pour
% fournir des points de saut entre chaque ligne pour que la
% table puisse \^etre divis\'ee en plusieurs pages.
%
% Les versions pr\'ec\'edentes laissaient le param\`etre
% "\lineskip" \`a la fin de chaque morceau ajout\'e quand la
% ligne mod\`ele est ajout\'ee. Il n'y avait pas de glu au
% d\'ebut du morceau suivant puisque normalement \TeX\ ne met
% pas de glu "\lineskip" en haut d'une bo\^\i{}te. Cela
% signifiait que les morceaux \'etaient parfaitement
% ajust\'es, cependant "\noalign" \`a la limite d'un morceau
% venait avant la premi\`ere ligne du morceau suivant ou
% apr\`es l'interligne \`a la fin de ce morceau. C'est une
% mauvaise place, par exemple, cela signifie m\^eme que "\penalty10000"
% n'interdit pas un saut puisque la glu "\lineskip", dans
% l'item pr\'ec\'edent de la liste, donne un point de saut
% autoris\'e. Donc, maintenant, on enl\`eve la glu "\lineskip"
% qui \'etait avant la ligne mod\`ele et on introduit "\LT@setprevdepth"
% pour fixer "\prevdepth" au d\'ebut du prochain morceau,
% pour s'assurer que la glu "\lineskip" est ajout\'ee plus tard.
%    \begin{macrocode}
    \unskip
  \egroup}
%    \end{macrocode}
% \end{macro}
%
% \begin{macro}{\LT@entry}
% \changes{v4.00}{1996/04/08}
%      {(DK) Ajout de macro}
% \changes{v4.01}{1996/04/11}
%      {(DPC) Utilisation de l'astuce \cs{ifhmode} pour d\'eterminer la premi\`ere entr\'ee}
% Nous donnons ici la d\'efinition \'el\'ementaire de "\LT@entry",
% c'est \`a dire celle utilis\'ee dans les mod\`eles
% d'alignement. Il ajoute un "\kern" seulement si le maximum est
% impos\'e par un morceau diff\'erent.
%
% Le test "\ifhmode" donne la premi\`ere entr\'ee quand nous
% ne voulons pas ajouter un "&".
%    \begin{macrocode}
\def\LT@entry#1#2{%
  \ifhmode\@firstofone{&}\fi\omit
  \ifnum#1=\c@LT@chunks
  \else
    \kern#2\relax
  \fi}
%    \end{macrocode}
% \end{macro}
%
% \begin{macro}{\LT@entry@chop}
% \changes{v4.00}{1996/04/08}
%      {(DK) Ajout de macro}
% Cette d\'efinition pour l'argument de "\LT@save@row" est
% utilis\'ee pour mettre de c\^ot\'e tous ces maxima qui ne
% peuvent \^etre v\'erifi\'es parce qu'ils apparaissent
% apr\`es la fin de la table. Cela peut se produire seulement
% si une table a \'et\'e raccourcie (ou si l'ordre des lignes
% a \'et\'e bouscul\'e depuis la compilation pr\'ec\'edente).
% Il est \`a noter que cela est pr\'ematur\'e : le dernier
% morceau est toujours \`a fixer et avec les limites
% tronqu\'ees.
%    \begin{macrocode}
\def\LT@entry@chop#1#2{%
  \noexpand\LT@entry
    {\ifnum#1>\c@LT@chunks
       1}{0pt%
     \else
       #1}{#2%
     \fi}}
%    \end{macrocode}
% \end{macro}
%
% \begin{macro}{\LT@entry@write}
% \changes{v4.07}{1997/10/16}
%      {\'Evite l'utilisation de percent. tools/2631}
% Pour \'ecrire une entr\'ee dans le fichier "aux", nous
% utilisons une d\'efinition, l\'eg\`erement surprenante, qui
% a pour seul but d'\'eviter des lignes d\'ebordantes (qui
% pourraient d\'epasser les limites de \TeX\ lors de la
% lecture du fichier "aux" ; vous aurez probablement besoin
% de quelques centaines de colonnes avant que cela ne se
% produise, mais\ldots).
%    \begin{macrocode}
\def\LT@entry@write{%
  \noexpand\LT@entry^^J%
  \@spaces}
%    \end{macrocode}
% \end{macro}
%
% \begin{macro}{\LT@kill}
% Cela termine le morceau courant comme ci-dessus, mais
% supprime deux lignes, la ligne mod\`ele et la ligne tu\'ee,
% avant de prendre en compte le morceau suivant. Depuis la
% V3.04, l'ancien morceau est <<~remis en bo\^\i{}te~>> au
% d\'ebut de la bo\^\i{}te contenant le morceau suivant. Cela
% permet d'utiliser "\kill" dans les en-t\^ete (de table)
% puisque cette commande doit \^etre trait\'ee dans une seule
% bo\^\i{}te.
%    \begin{macrocode}
\def\LT@kill{%
  \LT@echunk
  \LT@get@widths
  \expandafter\LT@rebox\LT@bchunk}
%    \end{macrocode}
% \end{macro}
%
% \begin{macro}{\LT@rebox}
% Renvoie l'ancien morceau (box0) au d\'ebut du nouveau
% morceau, ce qui supprime la ligne tu\'ee. Cette macro a
% \'et\'e ajout\'ee dans la V3.04.
%    \begin{macrocode}
\def\LT@rebox#1\bgroup{%
  #1\bgroup
  \unvbox\z@
  \unskip
  \setbox\z@\lastbox}
%    \end{macrocode}
% \end{macro}
%
% \subsection{La Ligne mod\`ele}
%
% La ligne mod\`ele est prise en compte par "\LT@save@row".
%
% \begin{macro}{\LT@blank@row}
% \begin{macro}{\LT@build@blank}
% \changes{v4.00}{1996/04/08}
%      {(DK) Macro ajout\'ee}
% \changes{v4.01}{1996/04/11}
%      {(DPC) Ne m\'elange pas \cs{multicolumn} dans les passes brouillon}
% Cr\'ee une ligne blanche si nous n'avons pas utililis\'e les informations
% du fichier ".aux".
%    \begin{macrocode}
\def\LT@blank@row{%
  \xdef\LT@save@row{\expandafter\LT@build@blank
    \romannumeral\number\LT@cols 001 }}
%    \end{macrocode}
% Oups ! Qu'est-ce que c'est suppos\'e \^etre ? Un
% remplacement de la premi\`ere tache de l'Appendice~D du \TeX book.
% "\romannumeral" affecte \`a "\LT@cols" un "m" par unit\'e suivi de
% "i"\footnote{NdT: si le compteur vaut 12,
% \texttt{\char92number\char92LT@cols~001} vaut 12001, qui en chiffres
% romains donne \texttt{mmmmmmmmmmmmi}, soit un \texttt{m} par unit\'e.}.
% La macro ci-dessous remplace donc les "m" par les entr\'ees ad\'equates.
%    \begin{macrocode}
\def\LT@build@blank#1{%
  \if#1m%
    \noexpand\LT@entry{1}{0pt}%
    \expandafter\LT@build@blank
  \fi}
%    \end{macrocode}
% \end{macro}
% \end{macro}
%
% \begin{macro}{\LT@make@row}
% \changes{v4.00}{1996/04/08}
%      {(DK) Nouvelle impl\'ementation}
% \changes{v4.01}{1996/04/11}
%      {(DPC) Rend ce comportement automatique, ne n\'essitant pas  \cs{setlongtables}}
%
% Avant la version 4, par d\'efaut on n'utilisait pas
% l'information du fichier ".aux" mais maintenant nous
% pouvons d\'efinir "\LT@make@row" pour que la commande
% utilise le fichier ".aux", m\^eme pour les passes brouillon.
%    \begin{macrocode}
\def\LT@make@row{%
  \global\expandafter\let\expandafter\LT@save@row
    \csname LT@\romannumeral\c@LT@tables\endcsname
  \ifx\LT@save@row\relax
    \LT@blank@row
%    \end{macrocode}
% L\`a, c'est un cas l\'eg\`erement difficile. Avant de
% d\'ecider de faire un mod\`ele \`a partir du fichier
% ".aux", nous v\'erifions que le nombre de champs est
% toujours le m\^eme. Si ce n'est pas le cas, soit le format
% de la table a chang\'e, soit nous avons une table
% totalement mauvaise. Dans les deux cas, nous avons
% d\'ecid\'e de supprimer toutes informations accumul\'ees et de red\'emarrer.
% \changes{v4.01}{1996/04/11}
%      {(DPC) Utilisation du test \cs{if} test au lieu de \cs{ifx}\cs{@empty}}
%
% Le d\'eveloppement entre les"!"\ldots"!" ci-dessous sera
% vide si le nombre de macros "\LT@entry" incluant les
% arguments dans "\LT@save@row" est \'egal \`a "\LT@cols".
% S'il n'est pas vide, nous supprimons la ligne et nous repartons de z\'ero.
%    \begin{macrocode}
  \else
    {\let\LT@entry\or
     \if!%
         \ifcase\expandafter\expandafter\expandafter\LT@cols
         \expandafter\@gobble\LT@save@row
         \or
         \else
           \relax
         \fi
        !%
     \else
       \aftergroup\LT@blank@row
     \fi}%
  \fi}
%    \end{macrocode}
% \end{macro}
%
% \begin{macro}{\setlongtables}
% Red\'efinit la macro "\LT@make@row" pour utiliser
% l'information du fichier ".aux", s'il y a une ligne
% sauvegard\'ee pour cette table avec le bon nombre de colonne.
%
% Depuis la Version 3.02 \package{longtable} utilise le
% compteur interne "\c@LT@tables" au lieu du compteur
% \textsf{table} de \LaTeX\ . Le message d'alerte a \'et\'e
% ajout\'e dans la V3.04, ainsi que le "\global", pour \'eviter
% de faire d\'eborder la pile.
%
% Depuis la version 4.01, "\setlongtables" ne fait rien
% puisqu'il n'est plus n\'ecessaire, mais est d\'efini comme "\relax"
% au b\'en\'efice des anciens documents.
% \changes{v3.12}{1994/06/30}
%      {Modification des messages d'alerte pour faire une ligne \`a l'\'ecran.}
% \changes{v4.01}{1996/04/11}
%      {(DPC) make into no op}
%    \begin{macrocode}
\let\setlongtables\relax
%    \end{macrocode}
% \end{macro}
%
% \begin{macro}{\LT@get@widths}
% C'est le c\oe{}ur de \package{longtable}. S'il n'y a pas
% de t\^ete et pied de table, cette macro, associ\'ee avec la
% commande "\\" modifi\'ee, pourrait tout \`a fait constituer
% une base pour un petit package simple pour les tables
% longues. Cette macro est tr\'es inspir\'ee de la macro "\endvrulealign"
% pr\'esent\'ee en annexe D du \TeX book.
%    \begin{macrocode}
\def\LT@get@widths{%
%    \end{macrocode}
% "\global" ajout\'e par V3.04, pour \'eviter de faire d\'eborder la
% pile.
%    \begin{macrocode}
%    \end{macrocode}
% Fait une boucle \`a travers la derni\`ere ligne, supprimant
% la justification et sauvegardant les largeurs de bo\^\i{}te.
% Pour la V3.04, changement de la scratch box \`a 2 puisque
% la nouvelle commande "\kill" demande que "\box0" soit conserv\'e.
%    \begin{macrocode}
  \setbox\tw@\hbox{%
    \unhbox\@ne
    \let\LT@old@row\LT@save@row
    \global\let\LT@save@row\@empty
    \count@\LT@cols
    \loop
      \unskip
      \setbox\tw@\lastbox
    \ifhbox\tw@
      \LT@def@row
      \advance\count@\m@ne
    \repeat}%
%    \end{macrocode}
% Rappelle la largeur si nous sommes dans le premier morceau.
%    \begin{macrocode}
  \ifx\LT@@save@row\@undefined
    \let\LT@@save@row\LT@save@row
  \fi}
%    \end{macrocode}
% \end{macro}
%
% \begin{macro}{\LT@def@row}
% \changes{v4.00}{1996/04/08}
%      {(DK) Nouvelle impl\'ementation}
% Ajoute une colonne \`a la ligne mod\`ele. Nom chang\'e
% (\'etait "\defLT@save@row") dans la Version~3, pour
% conserver la convention de d\'enommination "\LT@".
%    \begin{macrocode}
\def\LT@def@row{%
%    \end{macrocode}
% Nous commen\c{c}ons par choisir les entr\'ees respectives
% de nos anciennes lignes. Ces red\'efinitions de "\LT@entry"
% sont limit\'ees au groupe d\'ebut\'e en "\LT@get@widths".
%    \begin{macrocode}
  \let\LT@entry\or
  \edef\@tempa{%
    \ifcase\expandafter\count@\LT@old@row
    \else
      {1}{0pt}%
    \fi}%
%    \end{macrocode}
% Maintenant, nous attachons la bonne d\'efinition devant "\LT@save@row" :
%    \begin{macrocode}
  \let\LT@entry\relax
  \xdef\LT@save@row{%
    \LT@entry
    \expandafter\LT@max@sel\@tempa
    \LT@save@row}}
%    \end{macrocode}
% \end{macro}
%
% \begin{macro}{\LT@max@sel}
% \changes{v4.00}{1996/04/08}
%      {(DK) macro ajout\'ee}
% \changes{v4.01}{1996/04/11}
%      {(DPC) Modifications mineures}
% Comment s\'electionner la bonne combinaison. Noter que nous
% donnons l'information de l'ancien maximum, seulement si la
% taille ne change dans \emph{aucune} direction. Si la taille
% a augment\'e, nous aurons, bien s\^ur, un nouveau maximum.
% Si la taille a diminu\'e, l'ancien maximum (qui \'etait
% explicitement impos\'e \`a cause de l'existence du morceau
% courant) est non valide, et nous commen\c{c}ons avec ce
% morceau comme \'etant la nouvelle taille. Noter que m\^eme
% en cas d'\'egalit\'e, nous \emph{devons} utiliser la
% construction "\the\wd\tw@" au lieu de "#2", parce que "#2"
% pourrait \^etre lu \`a partir du fichier, et donc il
% pourrait exister des versions "\catcode"~11 de "p" et "t"
% dans lesquelles nous voulons les remplacer par leur
% versions "\catcode"~12 correctes.
%    \begin{macrocode}
\def\LT@max@sel#1#2{%
  {\ifdim#2=\wd\tw@
     #1%
   \else
     \number\c@LT@chunks
   \fi}%
  {\the\wd\tw@}}
%    \end{macrocode}
% \end{macro}
%
% \subsection{La commande \cs{hline}}
%
% \begin{macro}{\LT@hline}
% "\hline" et "\hline\hline" produisent tous deux
% \emph{deux} lignes, la seule diff\'erence \'etant la
% justification et les <<~penalties~>>. C'est pourquoi un
% saut de page au niveau d'une "\hline" produit une ligne sur
% les deux pages \`a la fois.\footnote{\env{longtable} a
% toujours r\'ealis\'e cela, mais peut-\^etre ce serait
% mieux si les hlines \'etaient \emph{omises} au niveau du
% saut de page, puisque habituellement, les t\^ete et pied de table
% mettent une hline ici, de toutes fa\c{c}ns}. De plus, cette
% "\hline" est plus comparable \`a "\cline{1-\LT@cols}". Les "\hline"
% de \env{tabular} devraient tracer des lignes de la largeur
% totale de la page.
%    \begin{macrocode}
\def\LT@hline{%
  \noalign{\ifnum0=`}\fi
    \penalty\@M
    \futurelet\@let@token\LT@@hline}
%    \end{macrocode}
% \end{macro}
%
% \begin{macro}{\LT@@hline}
% Ce code est bas\'e sur celui de "\cline". Deux copies de la
% ligne sont produites, comme d\'ecrit ci-dessus.
%    \begin{macrocode}
\def\LT@@hline{%
  \ifx\@let@token\hline
    \global\let\@gtempa\@gobble
    \gdef\LT@sep{\penalty-\@medpenalty\vskip\doublerulesep}%
  \else
    \global\let\@gtempa\@empty
    \gdef\LT@sep{\penalty-\@lowpenalty\vskip-\arrayrulewidth}%
  \fi
  \ifnum0=`{\fi}%
  \multispan\LT@cols
     \unskip\leaders\hrule\@height\arrayrulewidth\hfill\cr
  \noalign{\LT@sep}%
  \multispan\LT@cols
     \unskip\leaders\hrule\@height\arrayrulewidth\hfill\cr
  \noalign{\penalty\@M}%
  \@gtempa}
%    \end{macrocode}
% \end{macro}
%
% \subsection{L\'egendes}
%
% \begin{macro}{\LT@caption}
% La l\'egende est donn\'ee par "\multicolumn{\LT@cols}{c}{"\meta{une parbox avec la
% l\'egende de la table}"}"
%    \begin{macrocode}
\def\LT@caption{%
  \noalign\bgroup
    \@ifnextchar[{\egroup\LT@c@ption\@firstofone}\LT@capti@n}
%    \end{macrocode}
% \end{macro}
%
% \begin{macro}{\LT@c@ption}
% Commande de l\'egende (avec [argument optionnel]). "\protect"
% ajout\'e dans la version 3. "\fnum@table" ajout\'e dans la V3.05.
% \changes{v3.14}{1995/05/02}
%      {Appel de \cs{LT@makecaption} et non de \cs{LT@mkcaption}}
% \changes{v3.14}{1995/05/24}
%      {Ajout d'un nouvel argument de contr\^ole}
%    \begin{macrocode}
\def\LT@c@ption#1[#2]#3{%
  \LT@makecaption#1\fnum@table{#3}%
  \def\@tempa{#2}%
  \ifx\@tempa\@empty\else
     {\let\\\space
     \addcontentsline{lot}{table}{\protect\numberline{\thetable}{#2}}}%
  \fi}
%    \end{macrocode}
% \end{macro}
%
% \begin{macro}{\LT@capti@n}
% Commande de l\'egende (sans [argument optionnel])
% \changes{v3.14}{1995/05/02}
%      {Appel de \cs{LT@makecaptionx} et non de \cs{LT@mkcaption}}
% \changes{v3.14}{1995/05/24}
%      {Appel de \cs{LT@makecaption} avec \cs{@gobble} ou \cs{@firstofone}}
%    \begin{macrocode}
\def\LT@capti@n{%
  \@ifstar
    {\egroup\LT@c@ption\@gobble[]}%
    {\egroup\@xdblarg{\LT@c@ption\@firstofone}}}
%    \end{macrocode}
% \end{macro}
%
%
% \begin{macro}{\LT@makecaption}
% Met la l\'egende dans une bo\^\i{}te de largeur 0pt, pour
% que \c{c}a n'affecte jamais les largeurs de colonne.
% La l\'egende comprend une "\parbox" de largeut "\LTcapwidth".
% \changes{v3.14}{1995/05/02}
%      {renomm\'ee \`a partir de \cs{LT@mkcaption}, et
% modifi\'ee pour appeler \cs{@makecaption}}
% \changes{v3.14}{1995/05/02}
%      {Utilise le premier argument pour enlever le compteur de la forme "*"}
%    \begin{macrocode}
\def\LT@makecaption#1#2#3{%
  \LT@mcol\LT@cols c{\hbox to\z@{\hss\parbox[t]\LTcapwidth{%
%    \end{macrocode}
% Bas\'ee sur "\@makecaption" de la classe article, "#1" est "\@gobble"
% dans la forme "*", et "\@firstofone" autrement.
%    \begin{macrocode}
    \sbox\@tempboxa{#1{#2: }#3}%
    \ifdim\wd\@tempboxa>\hsize
      #1{#2: }#3%
    \else
      \hbox to\hsize{\hfil\box\@tempboxa\hfil}%
    \fi
    \endgraf\vskip\baselineskip}%
  \hss}}}
%    \end{macrocode}
% \end{macro}
%
%
% \subsection{La routine de sortie}
%
% La m\'ethode utilis\'ee ici pour relier une routine
% de sortie, utilis\'ee dans un but sp\'ecial, \`a la routine standard de
% \LaTeX\ provient en droite ligne du package \package{multicol} de F.~Mittelbach
%
% \begin{macro}{\LT@output}
% De fait, ce n'est pas si terrible, FM ayant d\'ej\`a d\'egag\'e le terrain.
%    \begin{macrocode}
\def\LT@output{%
  \ifnum\outputpenalty <-\@Mi
    \ifnum\outputpenalty > -\LT@end@pen
%    \end{macrocode}
%     Si c'\'etait un flottant ou une marge de paragraphe que nous r\'eclamons.
%    \begin{macrocode}
      \LT@err{floats and marginpars not allowed in a longtable}\@ehc
    \else
%    \end{macrocode}
% Nous atteignons la fin de la table, au moins par d\'efilement,
%    \begin{macrocode}
      \setbox\z@\vbox{\unvbox\@cclv}%
      \ifdim \ht\LT@lastfoot>\ht\LT@foot
%    \end{macrocode}
% Le dernier pied peut ne pas correspondre, donc :
% \footnote{Une alternative pourra\^\i{}t \^etre la suppression d'un petit
% bout du dernier morceau, de fa\c{c}on \`a ce que la derni\`ere page n'ait
% pas de sections de t\^ete et de pied, mais il est difficile de r\'ealiser
% cela d'une mani\`ere robuste.}
%    \begin{macrocode}
        \dimen@\pagegoal
        \advance\dimen@-\ht\LT@lastfoot
        \ifdim\dimen@<\ht\z@
          \setbox\@cclv\vbox{\unvbox\z@\copy\LT@foot\vss}%
          \@makecol
          \@outputpage
          \setbox\z@\vbox{\box\LT@head}%
%    \end{macrocode}
% Fin de "\ifdim\dimen@<\ht\z@"\footnote{NdT: La documentation originale
% dit ici <<~End of \texttt{\char92ifdim\char92dimen@<\char92ht\char92@cclc}~>>
% qui semble \^etre une faute, puisque le \texttt{\char92if} effectivement
% ferm\'e porte sur la hauteur de la bo\^ite 0 et non pas 255.}.
%    \begin{macrocode}
        \fi
%    \end{macrocode}
% Fin de "\ifdim \ht\LT@lastfoot > \ht\LT@foot".
%    \begin{macrocode}
      \fi
%    \end{macrocode}
% R\'einitialisation de "\@colroom".
% \changes{v3.14}{1995/05/02}
%      {R\'einitialisation de \cs{@colroom}, corrige tools/1584}
%    \begin{macrocode}
      \global\@colroom\@colht
      \global\vsize\@colht
%    \end{macrocode}
% Met la derni\`ere page de la table dans la liste verticale principale.
%    \begin{macrocode}
      \vbox
        {\unvbox\z@\box\ifvoid\LT@lastfoot\LT@foot\else\LT@lastfoot\fi}%
%    \end{macrocode}
% Fin de "\ifnum\outputpenalty > -\LT@end@pen".
%    \begin{macrocode}
    \fi
%    \end{macrocode}
% Fin de "\outputpenalty > -\@Mi".
%    \begin{macrocode}
  \else
%    \end{macrocode}
% Si nous n'avons pas atteint la fin de la table,
%    \begin{macrocode}
    \setbox\@cclv\vbox{\unvbox\@cclv\copy\LT@foot\vss}%
    \@makecol
    \@outputpage
%    \end{macrocode}
% R\'einitialisation de "\vsize".
% \changes{v3.14}{1995/05/02}
%      {R\'einitialisation de "\vsize", corrige tools/1584}
%    \begin{macrocode}
      \global\vsize\@colroom
%    \end{macrocode}
% Met la t\^ete de table en haut de la page suivante.
%    \begin{macrocode}
    \copy\LT@head
%    \end{macrocode}
% Fin de "\ifnum\outputpenalty <-\@Mi".
%    \begin{macrocode}
  \fi}
%    \end{macrocode}
% \end{macro}
%
% \subsection{Commande pour les en-t\^ete et pied de table}
%
% \begin{macro}{\LT@end@hd@ft}
% Le c\oe{}ur de "\endhead" et compagnie. Stocke le morceau courant
% dans la bo\^\i{}te sp\'ecifi\'ee par "#1". Produit une erreur si
% la table est d\'ej\`a commenc\'ee. Puis, commence un nouveau morceau.
%    \begin{macrocode}
\def\LT@end@hd@ft#1{%
  \LT@echunk
%    \end{macrocode}
% Changement de "\relax" en "\endgraf" pour V3.04, \emph{cf} "\LT@start".
%    \begin{macrocode}
  \ifx\LT@start\endgraf
    \LT@err
     {Longtable head or foot not at start of table}%
     {Increase LTchunksize}%
  \fi
  \setbox#1\box\z@
  \LT@get@widths
  \LT@bchunk}
%    \end{macrocode}
% \end{macro}
%
% \begin{macro}{\endfirsthead}
% \begin{macro}{\endhead}
% \begin{macro}{\endfoot}
% \begin{macro}{\endlastfoot}
% Appelle "\LT@end@hd@ft" avec la bo\^\i{}te ad\'equate.
%    \begin{macrocode}
\def\endfirsthead{\LT@end@hd@ft\LT@firsthead}
\def\endhead{\LT@end@hd@ft\LT@head}
\def\endfoot{\LT@end@hd@ft\LT@foot}
\def\endlastfoot{\LT@end@hd@ft\LT@lastfoot}
%    \end{macrocode}
% \end{macro}
% \end{macro}
% \end{macro}
% \end{macro}
%
% \subsection{La commande \cs{multicolumn}}
%
% Les plus anciennes versions n\'ecessitaient une forme brouillon
% sp\'eciale de "\multicolumn". Cela n'est plus n\'ecessaire dans
% la version 4, et donc ces commandes ont \'et\'e supprim\'ees.
%
% \begin{macro}{\LTmulticolumn}
% \changes{v4.01}{1996/04/11}
%      {(DPC) macro supprim\'ee}
% \end{macro}
%
% \begin{macro}{\LT@mcwarn}
% \changes{v4.01}{1996/04/11}
%      {(DPC) macro supprim\'ee}
% \end{macro}
%
% \subsection{Notes de bas de page}
%
% La commande "\footnote" standard fonctionne dans une colonne "c",
% mais nous avons besoin de modifier cette d\'efinition pour une
% colonne "p", pour passer outre le niveau suppl\'ementaire
% d'embo\^\i{}tement. Ces macros sont bas\'ees sur le package
% \package{array}, mais devraient \^etre OK pour l'environnement \env{tabular}
% standard.
%
% \begin{macro}{\LT@startpbox}
% \changes{v4.09}{1998/05/13}
%      {Utilise \cs{setlength} pour que les extensions de
%      calcul soient applicables. tools/2793}
% Ajoute un code suppl\'ementaire pour changer la
% d\'efinition de "\@footnotetext".
%    \begin{macrocode}
\def\LT@startpbox#1{%
  \bgroup
    \let\@footnotetext\LT@p@ftntext
    \setlength\hsize{#1}%
    \@arrayparboxrestore
    \vrule \@height \ht\@arstrutbox \@width \z@}
%    \end{macrocode}
% \end{macro}
%
% \begin{macro}{\LT@endpbox}
% Apr\`es la fermeture de la parbox, d\'eveloppement de "\LT@p@ftn"
%qui ex\'ecutera une s\'erie de\\
% "\footnotetext["\meta{num}"]{"\meta{note}"}"\\
% Apr\`es avoir \'et\'e enlev\'ees de la parbox, elles
% peuvent aller \`a leur place \`a partir d'ici.
% \changes{v3.10}{1994/05/15}
%     {Utilise \cs{@finalstrut}}
% \changes{v4.03}{1996/05/07}
%     {Utilise \cs{hfil} pour tools/2120}
%    \begin{macrocode}
\def\LT@endpbox{%
  \@finalstrut\@arstrutbox
  \egroup
  \the\LT@p@ftn
  \global\LT@p@ftn{}%
  \hfil}
%    \end{macrocode}
% \end{macro}
%
% \begin{macro}{\LT@p@ftntext}
% Dans la colonne "p", sauvegarde juste le texte de la note
% de bas de page dans registre \`a jeton.
%    \begin{macrocode}
\def\LT@p@ftntext#1{%
  \edef\@tempa{\the\LT@p@ftn\noexpand\footnotetext[\the\c@footnote]}%
  \global\LT@p@ftn\expandafter{\@tempa{#1}}}%
%    \end{macrocode}
% \end{macro}
%
%    \begin{macrocode}
%</package>
%    \end{macrocode}
%
% \Finale
\endinput
