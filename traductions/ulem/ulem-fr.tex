%
%  ulem.ltx   Manual for ulem.sty
%
%
%  Copyright (c) 1989-2011 by Donald Arseneau
%
%  Version date 2011/03/18
%
\documentclass[12pt]{ltxdoc}
\usepackage[inputenc,fontenc,babel]{translatex-fr}
\addtolength{\textwidth}{1cm}
%\makeshortverb`\|
\usepackage{ulem}
\def\baselinestretch{1.06}
\setlength\parskip{2pt}
\hyphenation{normalem uwforbf ulforem}

\title{L'extension \textsf{ulem}\thanks{Ce fichier a pour numéro de
        version \fileversion\ et a été mis à jour le \filedatefr. Son 
        titre original est \og \emph{The \textsf{ulem} package:
        underlining for emphasis} \fg.}\\ 
        Soulignement pour marquer l'importance}
\author{Donald Arseneau\\ asnd@triumf.ca}
\date{18/03/2011}
\begin{document}

\maketitle
\sloppy

\begin{abstract}
  \tradini
  The ulem package provides various types of underlining that can stretch
  between words and be broken across lines.  Use it with \LaTeX\ or plain \TeX.

  In \LaTeX, ulem normally replaces italics with underlining in text 
  emphasized by \cs{emph}, and to some extent by \cs{em}. A declaration 
  of |\normalem| or the |\usepackage| option |[normalem]| disables this 
  feature.

   The following commands are defined for general use:\\[5pt]
  \indent \begin{tabular}{l@{\quad}l}\hline\noalign{\vskip2pt}
   |\uline{important}|  & underlined text like \uline{important}\\[1pt]
   |\uuline{urgent}|    & double-underlined text like  \uuline{urgent}\\[1pt]
   |\uwave{boat}|       & wavy underline like {\let\ULleaders\cleaders\uwave{boat}}\\[1pt]
   |\sout{wrong}|       & line struck through word like \sout{wrong}\\[1pt]
   |\xout{removed}|     & marked over like \xout{removed} \\[1pt]
   |\dashuline{dashing}|& dashed underline like \dashuline{dashing}\\[1pt]
   |\dotuline{dotty}|   & dotted underline like \dotuline{dotty}\\[3pt]\hline
  \end{tabular}\\[6pt]
   Other similar commands can be defined with relative ease by utilizing the
   \cs{markoverwith} command provided by ulem.
\end{abstract}


\begin{footnotesize}
\noindent \tradini The ulem package is Copyright \copyright\ 1989--2011 by 
Donald Arseneau (Vancouver, Canada).\\
The package (ulem.sty) and this documentation (ulem.ltx, ulem.pdf) may be 
freely transmitted, reproduced, or modified for any purpose provided that the
copyright notice is left intact.
(Small excerpts may of course be taken and used without any restriction.)\par
\end{footnotesize}

\section{Basic Use}
\tradini
Ulem is a package for \LaTeX\ or plain \TeX\ which provides various types of
underlining that can stretch between words and be broken across lines.
Such underlining is given by the \cs{uline} command, leaving the original
\cs{underline} command available for math mode.
To load this package in plain \TeX, use `|\input ulem.sty|'.

In \LaTeX\ ulem replaces italics with underlining in text emphasized
by \cs{em} or \cs{emph} -- but only if the text is delimited by 
braces.  Unlike regular \cs{emph} emphasis, nested ulem emphasis generates
multiple underlining; it does not alternate on and off.
To use \cs{uline} for underlining, but have \cs{em} and \cs{emph} still 
produce normal italics, load ulem with |\usepackage[normalem]{ulem}|, 
or declare \cs{normalem} in the preamble.

Unlike regular underlining, ulem allows line breaks, and manual
hyphenation, within the underlined text; but it is far from perfect. It is
most suitable for simple text like {\em \LaTeX: A document preparation
system\/}  that may need to be underlined in a manuscript submitted for
publication.  Again, ulem can only give underlined text for \cs{em} when 
the text is delimited by explicit braces.

The thickness of the underline rule is given by the command macro
\cs{ULthickness}; use \cs{renewcommand} (not the usual 
\cs{setlength}) to change it.\footnote
{Users of plain \TeX\ should use \textbackslash def wherever these
instructions recommend \textbackslash(re)newcommand, and dimen
assignments wherever \textbackslash setlength is mentioned.}\,
The depth of the underline is controlled 
by the length \cs{ULdepth}.  The default value is a special flag 
(\cs{maxdimen}) which lets the depth vary depending on the current font.
You can set a particular value to \cs{ULdepth} (using \cs{setlength}) to 
force a particular depth, either locally for a special purpose, or 
for the document as a whole.  (See the definition of \cs{sout} for an
example.)

Other types of underlining are defined as well:
a wavey underline with \cs{uwave} (\uwave{under-wave}), double underlines 
using \cs{uuline} (\uuline{two lines under this}), dashed \cs{dashuline}
(\dashuline{dashes underneath}) or dotted \cs{dotuline} (\dotuline{dots 
below}) underlines.  Non-underlines are: a line to strike out text \cs{sout}
(\sout{strike out}), and text crossed-out with hatching \cs{xout} 
(\xout{cross out}).  See them tabulated in the abstract.

Alternative package: soul.

\section{Defining new commands}

You can define your own styles of overprinting or underlining by using
the \cs{markoverwith} command in the definition of your new command.
The definition should be something like:\\[3pt]
\indent
 \cs{newcommand}\cs{cmd}|{|\cs{bgroup} \cs{markoverwith}|{|\meta{something}|}|\cs{ULon}|}|\\[3pt]
The `\meta{something}' can be as simple as a single character, or as complex as
you can keep track of; it will likely contain some repositioning commands, perhaps
\cs{raisebox}.

Producing a colored underline or strike-through is not supported by
regular |\uline| or |\sout|, but it is quite easy to colorize using 
the \cs{markoverwith} mechanism: just put |\textcolor{|\dots|}| in 
the \meta{something}, such as this definition:
\begin{verbatim}
    \newcommand\reduline{\bgroup\markoverwith
      {\textcolor{red}{\rule[-0.5ex]{2pt}{0.4pt}}}\ULon}
\end{verbatim}
If you really feel the need to make a new command with a truly 
flexible rule, then look in ulem.sty and copy from the definitions
of \cs{uline} and \cs{sout}.

Any type of underlining can be substituted for any font-selection command
by issuing a proper \cs{useunder} declaration:\\[3pt]
 \cs{useunder}|{|\meta{underlinecommand}|}{|\meta{fontdeclaration}|}{|\meta{fontcommand}|}|\\[3pt]
e.g., |\useunder{\uuline}{\bfseries}{\textbf}| gives a double underline
instead of bold face in \LaTeX. 

The commands \cs{normalem} and \cs{ULforem} respectively disable and enable
underlining for \cs{em}/\cs{emph}, and so do the \cs{usepackage}
options [normalem] and [ULforem]. There is also the \cs{usepackage} 
option [UWforbf] to replace boldface with a wavey underline.  These features
use the \cs{useunder} command internally.  UWforbf does handle bold in 
math mode, in a limited way, but it doesn't work in section titles,
unfortunately, because the titles are not delimited by explicit braces when
printed by the \cs{section} command.  Currently under UWforbf the \cs{bfseries}
command still produces bold face, but \cs{bf} makes an 
under-wave\footnote{%% no verbatim in footnotes
To get under-waved section titles (in ordinary \LaTeX\ classes) you could define:\\
\texttt{\string\renewcommand\string\@seccntformat[1]\string{\string\uwave
\string{\string\csname\ the\#1\string\endcsname\string}\string\hskip 1em\string}}\\
and later specify 
\texttt{\string\section[...]\string{\string\uwave\string{...\string}\string}}. 
But you don't want to enter that swamp.}
(if |\bf| is defined at all).
In plain \TeX\ there is |\bf| but no \cs{textbf} so you could say |\useunder{\UWave}{\bf}{}|.

Some commands, such as |\\| and \cs{hskip} are given special treatment to
work within uline, but others are not.  Support for others can sometimes 
be added by assigning special meanings in the token register \cs{UL@hook}.  
(In \LaTeX\ do 
|\addto@hook\UL@hook{\let\cmd\|\textsl{UL-version-of-cmd}|}|.)  The UL 
versions of commands should be modelled on |\UL@hskip| or |\UL@cr|, and 
should include the test `|\ifx\ \LA@space|'.  For example, support for 
\cs{marginpar} is added through the hook mechanism.

All the underlining commands are robust (self-protecting) in \LaTeX.


\section{Complications}
The various underlining commands are essentially textual, and will not
work quite the same in math mode.  But since some font commands,
in the old-\LaTeX\ style (`oldlfont') serve both for text and math, 
math mode is handled (in an approximate way).  Generally, you should
avoid using ulem's commands within math, but math may appear in the 
text argument to ulem's commands.

Every word is typeset in an underlined box, so automatic hyphenation is
disabled, but explicit discretionary hyphens (|\-|) will still be
obeyed.  Several text-formatting commands are specially supported within
the underlining:  |\-|, |\ |, |~|, |\\|, |\newline|, |\linebreak|, 
|\nolinebreak|, |\penalty|, |\hskip|, |\hspace|, |\hfil|, |\hfill|, |\hss|.  
Displayed math is not supported.  The special commands 
do have a problem: they end a group so any local assignments are lost.  

The underlines continue between words, and stretch just like ordinary
spaces do.  Since spaces delimit words, there may be some difficulty
with syntactical spaces (e.g. `|2.3 pt|'). Some effort is made to handle
such cases, but sometimes (such as |\let\x= y|) the space is interpreted
incorrectly. You can usually solve the problem by enclosing the offending
command in braces or in a macro (like |\newcommand\xeqy{\let\x= y}|), 
but\,\dots

One important incompatibility with braces and macro replacement:
\textbf{All the text in braces or coming from a macro is typeset in a box}
(as if in \cs{mbox}). Consequently,
braces will suppress stretching and line-breaking in the text they
enclose.  Moreover, the specially-handled commands |\-|, |\\|, |\newline|
and |\linebreak| are usually ignored if they appear inside extra braces.
They operate only when the braces delimit a command parameter without
introducing a level of grouping.  (Even though braces delimiting command
parameters do not normally imply grouping, many commands will add their
own grouping.)  Thus, you should try to limit inner braces to short bits of
text or for delimiting parameters to commands. For emergency repairs, see
the sadistic `Marat/Sade' example below.  Syntactical spaces inside braces 
never cause a problem, nor do spaces in math mode.

Text produced by expansion of a command (macro) is boxed too, but |\\|, |\ |,
and |\-| still work properly in the expansion text so that while\\[2pt]
\indent  |\newcommand\iff{if and only if} ... \uline{\iff}|\\[2pt]
prevents stretching and line-breaking between words, the alternative\\[2pt]
\indent  |\newcommand\iff{if\ and\ only\ if} ... \uline{\iff}|\\[2pt]
allows stretching and line-breaking.  There is a remaining problem though:
the |\ | (backslash-space) between words closes a group and any local 
assignments will be lost, in particular, font changes and color changes.

This loss of local assignments will break some other standard commands,
(e.g., \cs{cite}) which produce multiple `words' using local assignments.
The way to protect such commands is to bury them in an \cs{mbox}:
\hskip 5pt plus 2in
|\emph{every\-one agrees~\mbox{\cite{you,me}}.}|

With ULforem in effect,
nested \cs{em} or \cs{emph} commands produce multiple underlining, 
but heed the warnings
about braces above.  To get italics without underlining, use \cs{it}, 
\cs{itshape}, or \cs{textit}.  Nesting
of other types of underline is also possible, but the `underlines' may
overlap.

\medskip
Here is a simple example (highlighting all invented words):\\[4pt]
\begin{minipage}[t]{.58\linewidth}\hbadness=9999
\ttfamily \rightskip=0pt plus 2cm
|'Twas| |\emph{brillig}| |and| |the| |\emph{slithy~toves}|
|did| |\emph{gyre}| |and| |\emph{gim\-ble}| |in| |the| |\emph{wabe,\\| |}|
|All| |\emph{mim\-sey}| |were| |the| |\emph{boro\-goves}| |and|
|the| |\emph{mome| |raths| |outgrabe}.|
\end{minipage}\hfill
\begin{minipage}[t]{.4\linewidth}\hbadness=9999
'Twas \emph{brillig} and the \emph{slithy~toves}
did \emph{gyre} and \emph{gim\-ble} in the \emph{wabe,\\ }
All \emph{mim\-sey} were the \emph{boro\-goves} and
the \emph{mome raths out\-grabe}.\par\mbox{}
\end{minipage}\vspace{6pt}
Note use
of explicit hyphenation in `gimble' and `borogoves', the tie (|~|) 
that prevents a line break in `slithy toves', but stretches like a 
usual space, the `|\\|' that gives a proper linebreak, and the regular 
(unforced) linebreak in `mome raths outgrabe'.

\medskip
Here is an ugly example showing how nested uline (\cs{emph}) needs
to be broken up to allow line-breaks\\[2pt]
\begin{minipage}[t]{.5\linewidth}
\ttfamily \rightskip=0pt plus 1.5cm \hbadness=9999
|No,| |I| |did| |{\em| |not}| |act| |in| |the| |movie| |{\em| |\emph{The}|
|\emph{Persecution}| |\emph{and}| |\emph{Assassin}\-\emph{ation}| |\emph{of}|
|\emph{Jean-Paul}| |\emph{Marat},| |as| |Per\-formed| |by| |the| |Inmates|
|of| |the| |Asylum| |of| |Charenton| |Under| |the| |Direc\-tion| |of| |the|
|Marquis| |de~Sade!}| |But| |I| |{\em| |did}| |see| |it.|\\\mbox{}
\end{minipage}\hfill
\begin{minipage}[t]{.43\linewidth}\hbadness=9999
No, I did {\em not} act in the movie {\em \emph{The} 
\emph{Persecution} \emph{and} \emph{Assassin}\-\emph{ation} \emph{of}
\emph{Jean-Paul} \emph{Marat}, as Per\-formed by the Inmates
of the Asylum of Charenton Under the Direc\-tion of the
Marquis de~Sade!} But I {\em did} see it.
\end{minipage}\\
In the nested emphasis, \cs{emph} had to be given for each word separately
so the spaces between could stretch and break into lines.  Even the
discretionary hyphen (|\-|) in `Assassination' had to be outside the braces,
but the hyphen in `Direction' was just fine because it was not in nested
braces. The same applies to other special commands like |\ | and |~|. Also,
the spaces are printed with only a single underline because they are
outside the nested \cs{emph} commands.  This example really illustrates that
ulem does not handle nested emphasis very well!  It should be reserved for
simpler cases where it performs well without effort.

\end{document}
