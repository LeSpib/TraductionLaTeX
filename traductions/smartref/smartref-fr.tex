\documentclass[pagesize=auto, parskip=half, headings=normal]{scrartcl}

\usepackage{fixltx2e}
\usepackage{etex}
\usepackage{lmodern}
\usepackage[T1]{fontenc}
\usepackage[inputenc,babel]{translatex-fr}
\usepackage{textcomp}
\usepackage{hologo}
\usepackage{microtype}
\usepackage{hyperref}

\newcommand*{\mail}[1]{\href{mailto:#1}{\texttt{#1}}}
\newcommand*{\pkg}[1]{\textsf{#1}}
\newcommand*{\cls}[1]{\textsf{#1}}
\newcommand*{\cs}[1]{\texttt{\textbackslash#1}}
\makeatletter
\newcommand*{\cmd}[1]{\cs{\expandafter\@gobble\string#1}}
\makeatother
\newcommand*{\opt}[1]{\texttt{#1}}
\newcommand*{\meta}[1]{\textlangle\textsl{#1}\textrangle}
\newcommand*{\marg}[1]{\texttt{\{}\meta{#1}\texttt{\}}}

\addtokomafont{title}{\rmfamily}

\title{L'extension \pkg{smartref}\thanks{Ce fichier a pour numéro de
        version v1.9 et a été mis à jour le 28/02/2002. Son 
        titre original est \og \emph{The smartref package} \fg.}}
\author{Giuseppe Bilotta\thanks{\mail{bourbaki@bigfoot.com}}}
\date{28/02/2002}


\begin{document}

\maketitle

Le but de cette extension est d'étendre les possibilités de la commande
\cmd{\ref} ; plus précisément, à chaque fois qu'une étiquette est placée,
l'extension enregistre, en plus de l'étiquette, la valeur d'autres
compteurs (qui peuvent être sélectionnés par l'utilisateur) ; par la suite, 
la valeur de ces compteurs peut être rappelée avec une commande similaire
de \cmd{\pageref} ; par ailleurs, cette extension ajoute une commande 
(\cmd{\s\meta{nom}ref}), pour chaque compteur ajouté, qui affiche quelque
chose seulement si la valeur du compteur \meta{name} a évolué depuis 
le moment où l'étiquette a été définie. Par exemple, suppsons que nous
utilisons la classe \cls{amsbook} et que nous numérotons les théorèmes
au sein des sections ; quand nous voulons faire référence à un théorème
d'une autre section, dans le même chapitre, nous indiquons juste \og 
\verb|comme dans le théorème \ref{zorn}| \fg ; mais si nous voulons faire
référence à un théorème dans un autre chapitre, nous devons mettre en place
une étiquette pour ce chapitre et dire alors \og 
\verb|comme dans le théorème \ref{zorn}, du chapitre \ref{chaptitredezorn}|
\fg ; si, plus tard, nous déplaçons le théorème ainsi référencé (ou la ligne
qui y fait référence, nous pourrions avoir à mettre à jour les références ;
avec cette extension, il n'y a pas besoin de le faire : vous avez à placer
en début de document : 
%
\begin{verbatim}
    \addtoreflist{chapter}
\end{verbatim}
%
puis, chaque \cmd{\label} est enregistré avec le numéro du chapitre où elle
est placée. Quand nous faisons référence au théorème, nous pouvons
maintenant indiquer 
%
\begin{verbatim}
    théorème \ref{zorn}, chapitre \chapterref{zorn}
\end{verbatim}
%
si nous sommes sûrs qu'il sera toujours dans un autre chapitre ou
%
\begin{verbatim}
    théorème \ref{zorn}\schapterref{zorn}
\end{verbatim}
%
où la seconde commande n'écrit rien si le chapitre est le même que le 
chapitre courant ou écrit \og \verb|, chap. \chapterref{zorn}| \fg si 
le chapitre a changé. Dans la mesure où cette séquence de commandes est 
supposée être utilisée assez fréquemment, elle peut être à peine raccourcie
par \cmd{\srefchapterref} (\cmd{\sref\meta{nom}ref}, en général).

Voici une liste complète des commandes mises à disposition par cette
extension.
%
\begin{quote}
  \cmd{\addtoreflist}\marg{nom}
\end{quote}
%
Il s'agit de la première commande appelée : elle indique que le compteur
\meta{nom} doit être enregistré dans toutes les étiquettes (qui suivent) ;
il convient donc de la placer dans le document \emph{avant} les différentes
étiquettes car le résultat n'est pas maîtrisé si la liste des références
ainsi établie change pendant l'utilisation (rien de particulier ne devrait
survenir de toute manière ; et cela devrait même fonctionner correctement). 

La commande précédente définit quelques commandes dont le nom dépend du 
compteur choisi. Ces commandes sont :
%
\begin{quote}
  \cmd{\sget\meta{nom}val}\texttt{\{}\cs{\meta{variable}}\texttt{\}}\marg{test}
\end{quote}
%
Cette commande récupère la valeur du compteur \meta{nom} enregistré avec
l'étiquette \meta{test} et la place dans la \cs{\meta{variable}} choisie.

\begin{quote}
  \cs{\meta{nom}ref}\marg{test}
\end{quote}
%
Cette commande écrit la valeur du compteur \meta{nom} enregistré avec 
l'étiquette \meta{test}.

\begin{quote}
  \cs{if\meta{nom}changed}
\end{quote}
%

Construction équivalente à un \og if \fg pour les deux commandes suivantes :
%
\begin{quote}
  \cmd{\is\meta{nom}changed}\marg{test}
\end{quote}
%
Vérifie si le compteur \meta{nom} a changé par rapport à la valeur
enregistré avec l'étiquette \meta{test} et définit \cs{\meta{name}changed}
en conséquence. 

\begin{quote}
  \cmd{\s\meta{name}ref}\marg{test}
\end{quote}
%
La commande n'écrit rien si la valeur de \meta{nom} n'est pas différente
et écrit sinon 
%
\begin{quote}
  \ttfamily
  , \cmd{\short\meta{nom}name} \cs{\meta{nom}ref}\marg{test}
\end{quote}
%
La commande \cmd{\short\meta{nom}name} doit être définie par l'utilisateur.
Elle est interrogée en utilisant la commande précédente.

\begin{quote}
  \cmd{\sref\meta{nom}ref}\marg{test}
\end{quote}
%
\'{E}quivalent de \og \cmd{\ref}\marg{test}\cmd{\s\meta{nom}ref}\marg{test}
\fg.

\pagebreak[2]

Les seules options acceptées, pour le moment, sont \opt{chapter} et \opt{part} ;
L'option \opt{chapter} fait les traitements suivants :
%
\begin{itemize}
\item ajout du compteur \texttt{chapter} dans les références conservées ;
\item définition de \cmd{\shortchaptername} comme valant \og Cap. \fg si 
\pkg{babel} est chargé avec l'option \opt{italian} et comme valant \og Chap. 
\fg sinon ;
\item definition de \cmd{\smartref} comme étant \cmd{\srefchapterref}.
\end{itemize}

L'option \opt{part} fait les traitements suivants :
\tradini
%
\begin{itemize}
\item adds the \texttt{part} counter to the smart list
\item defines \cmd{\shortpartname} to be `Parte' if \pkg{babel} is loaded with
  option \opt{italian}, `Part' otherwise
\item defines \cmd{\smartref} to be \cmd{\srefpartref} if previously undefined,
  or adds \cmd{\spartref} to previous definition of \cmd{\smartref} (saved in
  \cmd{\nopart@smartref}) otherwise (this is mainly to be used after the
  option \opt{chapter}).
\end{itemize}

As of version~1.6, the package comes with a small style file:
%
\begin{quote}
  \texttt{byname.sty};
\end{quote}
%
it can be used alone, or within \pkg{smartref}, when the option
\opt{byname} is used; it adds the command
%
\begin{quote}
  \cmd{\byname}\marg{label}
\end{quote}
%
that can be used when referencing to a section (subsection, etc) by
name instead of number. As of version~1.8, the same style file also
provides the command \cmd{\byshortname} that does the same as \cmd{\byname},
except that it uses the ``short'' name (the one provided in square
brackets).

Note that the references create by \pkg{smartref} and \pkg{byname} are \emph{not}
hyperlink, when \pkg{hyperref} is active. Maybe in the future \dots

(You can still have ``named'' links with \pkg{hyperref} using the \pkg{nameref}
style file provided with \pkg{hyperref} itself).

\medskip\centerline{$***$}\medskip

Please send me any bugs, comments, suggestions, hacks, etc etc
etc \dots

Giuseppe Bilotta

Email: \mail{bourbaki@bigfoot.com}

Heartily thanks to James Kilfiger (\mail{mapdn@csv.warwick.ac.uk}) whose
consulence has been essential (let me say vital) to the birth of
this small package in its first revision (v0.1); it was him who
gave me the idea to hack the \cmd{\label} command, and then encouraged
me to put the newborn code in a package (and telling me how to do
it!); it was also him that pointed out a potential bug in
documents where chapters don't start on a page on their own
(seems that the bug is not present, anyway \dots).

\medskip\centerline{$***$}


\section*{TODOs:}

\begin{itemize}
\item \pkg{Babel}: instead of \cmd{\chaptername} (for example), this package uses
  \cmd{\short\-chapter\-name}, and this is not yet changed by \pkg{Babel}; you can
  set it up to your short form of chapter. Please send me the short
  names, so that I can implement them.

\item Maybe put an option (\opt{long}) to make \cmd{\short\meta{name}name} an alias for
  \cs{\meta{name}name}

\item Options: If the option is not recognized, check if it's a counter,
  and add it; otherwise, do nothing

\item Smartrefs: check for counters that get reset by other counters
  (section in chapter, etc); maybe link to the \cmd{\@addtoreset}
  internal command \dots

\item \pkg{ByName}: refine, and add commands to do both references with a single
  command
\end{itemize}

\medskip\centerline{$***$}


\section*{History:}

\begin{labeling}{v1.9}
\item[v0.1] First release (only did chapter and part).
\item[v1.0] First customizable release.
\item[v1.1] Made it work with \pkg{HyperRef}.
\item[v1.5] Added \pkg{nameref.sty}: reference by name.
\item[v1.6] Changed name to \pkg{byname.sty}, because of name conflict with style
  file from the \pkg{HyperRef} bundle.
\item[v1.8] Fixed bugs in \pkg{smartref}, changed behaviour for unknown options,
  fixed bug in \pkg{byname} that actually prevented it from functioning
  properly.
\item[v1.9] Fixed limitation with \hologo{AmSLaTeX}: would not handle \cmd{\labels} within
  equation because \hologo{AmSLaTeX} redefines the \cmd{\label} command for
  equations.\\
  Made \cmd{\smartref} robust.	 
\end{labeling}

\end{document}
