\documentclass[pagesize=auto, parskip=half, headings=normal]{scrartcl}

\usepackage{fixltx2e}
\usepackage{etex}
\usepackage{lmodern}
\usepackage[T1]{fontenc}
\usepackage[inputenc,babel]{translatex-fr}
\usepackage{textcomp}
\usepackage{hologo}
\usepackage{microtype}
\usepackage{hyperref}

\newcommand*{\mail}[1]{\href{mailto:#1}{\texttt{#1}}}
\newcommand*{\pkg}[1]{\textsf{#1}}
\newcommand*{\cls}[1]{\textsf{#1}}
\newcommand*{\cs}[1]{\texttt{\textbackslash#1}}
\makeatletter
\newcommand*{\cmd}[1]{\cs{\expandafter\@gobble\string#1}}
\makeatother
\newcommand*{\opt}[1]{\texttt{#1}}
\newcommand*{\meta}[1]{\textlangle\textsl{#1}\textrangle}
\newcommand*{\marg}[1]{\texttt{\{}\meta{#1}\texttt{\}}}

\addtokomafont{title}{\rmfamily}

\title{L'extension \pkg{smartref}\thanks{Ce fichier a pour numéro de
        version v1.9 et a été mis à jour le 28/02/2002. Son 
        titre original est \og \emph{The smartref package} \fg.}}
\author{Giuseppe Bilotta\thanks{\mail{bourbaki@bigfoot.com}}}
\date{28/02/2002}


\begin{document}

\maketitle

Le but de cette extension est d'étendre les possibilités de la commande
\cmd{\ref} ; plus précisément, à chaque fois qu'une étiquette est placée,
l'extension enregistre, en plus de l'étiquette, la valeur d'autres
compteurs (qui peuvent être sélectionnés par l'utilisateur) ; par la suite, 
la valeur de ces compteurs peut être rappelée avec une commande similaire
de \cmd{\pageref} ; par ailleurs, cette extension ajoute une commande 
(\cmd{\s\meta{nom}ref}), pour chaque compteur ajouté, qui affiche quelque
chose seulement si la valeur du compteur \meta{name} a évolué depuis 
le moment où l'étiquette a été définie. Par exemple, suppsons que nous
utilisons la classe \cls{amsbook} et que nous numérotons les théorèmes
au sein des sections ; quand nous voulons faire référence à un théorème
d'une autre section, dans le même chapitre, nous indiquons juste \og 
\verb|comme dans le théorème \ref{zorn}| \fg ; mais si nous voulons faire
référence à un théorème dans un autre chapitre, nous devons mettre en place
une étiquette pour ce chapitre et dire alors \og 
\verb|comme dans le théorème \ref{zorn}, du chapitre \ref{chaptitredezorn}|
\fg ; si, plus tard, nous déplaçons le théorème ainsi référencé (ou la ligne
qui y fait référence, nous pourrions avoir à mettre à jour les références ;
avec cette extension, il n'y a pas besoin de le faire : vous avez à placer
en début de document : 
%
\begin{verbatim}
    \addtoreflist{chapter}
\end{verbatim}
%
puis, chaque \cmd{\label} est enregistré avec le numéro du chapitre où elle
est placée. Quand nous faisons référence au théorème, nous pouvons
maintenant indiquer 
%
\begin{verbatim}
    théorème \ref{zorn}, chapitre \chapterref{zorn}
\end{verbatim}
%
si nous sommes sûrs qu'il sera toujours dans un autre chapitre ou
%
\begin{verbatim}
    théorème \ref{zorn}\schapterref{zorn}
\end{verbatim}
%
où la seconde commande n'écrit rien si le chapitre est le même que le 
chapitre courant ou écrit \og \verb|, chap. \chapterref{zorn}| \fg si 
le chapitre a changé. Dans la mesure où cette séquence de commandes est 
supposée être utilisée assez fréquemment, elle peut être à peine raccourcie
par \cmd{\srefchapterref} (\cmd{\sref\meta{nom}ref}, en général).

Voici une liste complète des commandes mises à disposition par cette
extension.
%
\begin{quote}
  \cmd{\addtoreflist}\marg{nom}
\end{quote}
%
Il s'agit de la première commande appelée : elle indique que le compteur
\meta{nom} doit être enregistré dans toutes les étiquettes (qui suivent) ;
il convient donc de la placer dans le document \emph{avant} les différentes
étiquettes car le résultat n'est pas maîtrisé si la liste des références
ainsi établie change pendant l'utilisation (rien de particulier ne devrait
survenir de toute manière ; et cela devrait même fonctionner correctement). 

La commande précédente définit quelques commandes dont le nom dépend du 
compteur choisi. Ces commandes sont :
%
\begin{quote}
  \cmd{\sget\meta{nom}val}\texttt{\{}\cs{\meta{variable}}\texttt{\}}\marg{test}
\end{quote}
%
Cette commande récupère la valeur du compteur \meta{nom} enregistré avec
l'étiquette \meta{test} et la place dans la \cs{\meta{variable}} choisie.

\begin{quote}
  \cs{\meta{nom}ref}\marg{test}
\end{quote}
%
Cette commande écrit la valeur du compteur \meta{nom} enregistré avec 
l'étiquette \meta{test}.

\begin{quote}
  \cs{if\meta{nom}changed}
\end{quote}
%

Construction équivalente à un \og if \fg pour les deux commandes suivantes :
%
\begin{quote}
  \cmd{\is\meta{nom}changed}\marg{test}
\end{quote}
%
Vérifie si le compteur \meta{nom} a changé par rapport à la valeur
enregistré avec l'étiquette \meta{test} et définit \cs{\meta{name}changed}
en conséquence. 

\begin{quote}
  \cmd{\s\meta{name}ref}\marg{test}
\end{quote}
%
La commande n'écrit rien si la valeur de \meta{nom} n'est pas différente
et écrit sinon 
%
\begin{quote}
  \ttfamily
  , \cmd{\short\meta{nom}name} \cs{\meta{nom}ref}\marg{test}
\end{quote}
%
La commande \cmd{\short\meta{nom}name} doit être définie par l'utilisateur.
Elle est interrogée en utilisant la commande précédente.

\begin{quote}
  \cmd{\sref\meta{nom}ref}\marg{test}
\end{quote}
%
\'{E}quivalent de \og \cmd{\ref}\marg{test}\cmd{\s\meta{nom}ref}\marg{test}
\fg.

\pagebreak[2]

Les seules options acceptées, pour le moment, sont \opt{chapter} et \opt{part} ;
L'option \opt{chapter} fait les traitements suivants :
%
\begin{itemize}
\item ajout du compteur \texttt{chapter} dans les références conservées ;
\item définition de \cmd{\shortchaptername} comme valant \og Cap. \fg si 
\pkg{babel} est chargé avec l'option \opt{italian} et comme valant \og Chap. 
\fg sinon ;
\item définition de \cmd{\smartref} comme étant \cmd{\srefchapterref}.
\end{itemize}

L'option \opt{part} fait les traitements suivants :
%
\begin{itemize}
\item ajout du compteur \texttt{part} dans les références conservées ;
\item définition de \cmd{\shortpartname} comme valant \og Parte \fg si 
\pkg{babel} est chargé avec l'option \opt{italian} et comme valant \og Part 
\fg sinon ;
\item définition de \cmd{\smartref} comme étant \cmd{\srefchapterref} s'il n'est
pas défini auparavant; Ajout de \cmd{\spartref} à la définition
précédente (sauvegardée dans \cmd{\nopart@smartref}) sinon (ceci est
principalement pensé pour être utilisé après l'option \opt{chapter}).
\end{itemize}

Avec la version~1.6, l'extension est fournie avec un petit fichier d'extension :
%
\begin{quote}
  \texttt{byname.sty};
\end{quote}
%
Cette extension peut être utilisée seul ou avec \pkg{smartref}, quand l'option \opt{byname} est choisie; elle ajoute la commande
%
\begin{quote}
  \cmd{\byname}\marg{étiquette}
\end{quote}
%
qui sert à référencer une section (sous-section ou autre) avec son nom au lieu
de son numéro. Avec la version~1.8, le fichier fournit en plus la commande
\cmd{\byshortname} qui fait la même chose que \cmd{\byname} mais en utilisant
le \og nom court \fg : celui figurant en argument optionnel entre crochets des
commandes de sectionnement.

Notez que les références créées par \pkg{smartref} et \pkg{byname} \emph{ne sont pas} des hyperliens lorsque l'extension \pkg{hyperref} est active. Ce point sera peut être vu plus tard\dots (vous pouvez toujours vous servir des liens \og nommés \fg avec \pkg{hyperref} en utilisant l'extension \pkg{nameref} qui 
accompagne \pkg{hyperref} elle-même.)

\medskip\centerline{$***$}\medskip

N'hésitez pas à m'envoyer vos rapports d'anomalies, commentaires, suggestions, astuces, etc.\dots

Giuseppe Bilotta

Email: \mail{bourbaki@bigfoot.com}

Remerciements \trad{heartily} à James Kilfiger (\mail{mapdn@csv.warwick.ac.uk}) dont la \trad{consulence} a été essentielle (pour ne pas dire vitale) pour la création de la première version (v0.1) de cette petite extension ; c'est lui qui
m'a donné l'idée d'analyser et modifier la commande \cmd{\label} et qui m'a
ensuite encouragé à faire du code obtenu une extension (en m'expliquant comment
le faire !) ; c'est également lui qui m'a indiqué une erreur potentielle pour
les documents où les chapitres ne commencent pas sur une nouvelle page (il
semble que l'erreur ne soit pas présente, cependant\dots).

\medskip\centerline{$***$}

\section*{À faire}

\begin{itemize}
\item lien avec \pkg{babel}: au lieu de \cmd{\chaptername} (par exemple), cette 
  extension utilise \cmd{\short\-chapter\-name}, et ceci n'est pas modifié
  par \pkg{Babel}; vous pouvez la configurer pour votre propre forme de nom 
  court pour les chapitres. N'hésitez pas à m'envoyer les noms courts afin
  que je puisse les implémenter ;

\item ajout éventuel d'une option (\opt{long}) pour faire de 
  \cmd{\short\meta{name}name} un synonyme de \cs{\meta{name}name} ;

\item traitement des options: si une option n'est pas reconnue, vérifier s'il
  s'agit d'un compteur et alors l'ajouter ; sinon, ne rien faire ;
  
\item vérification des compteurs qui sont réinitialisés par d'autres compteurs
  (la section par le chapitre, par exemple) ; ajout éventuel d'un lien à la
  commande interne \cmd{\@addtoreset}\dots

\item évolution de l'extension \pkg{byname}: l'affiner et ajouter des 
  commandes pour se servir des deux systèmes de référence avec une seule
  commande.
\end{itemize}

\medskip\centerline{$***$}

\section*{Historique}

\begin{labeling}{v1.9}
\item[v0.1] Première version (ne traitait que les chapitres et parties).
\item[v1.0] Première version paramétrable.
\item[v1.1] Version qui fonctionne avec \pkg{hyperref}.
\item[v1.5] Ajout de \pkg{nameref}: référence par nom.
\item[v1.6] Changement du nom pour \pkg{byname}, du fait d'un conflit de nom
  avec un fichier de l'ensemble \pkg{hyperref}.
\item[v1.8] Correction d'erreurs dans \pkg{smartref}, changement du
  comportement pour les options inconnues, correction d'une anomalie dans 
  \pkg{byname} qui l'empêchait de fonctionner correctement.
\item[v1.9] Limites fixées avec \hologo{AmSLaTeX}: cette extension ne 
  gérera pas la commande \cmd{\label} dans les équations car \hologo{AmSLaTeX}
   la redéfinit pour les équations.\\
  La commande \cmd{\smartref} est rendue robuste.	 
\end{labeling}

\end{document}
