% \iffalse meta-comment
%
% Copyright 1993 1994 1995 1996 1997 1998 1999
% The LaTeX3 Project and any individual authors listed elsewhere
% in this file.
%
% This file is part of the Standard LaTeX `Tools Bundle'.
% -------------------------------------------------------
%
% It may be distributed and/or modified under the
% conditions of the LaTeX Project Public License, either version 1.1
% of this license or (at your option) any later version.
% The latest version of this license is in
%    http://www.latex-project.org/lppl.txt
% and version 1.1 or later is part of all distributions of LaTeX
% version 1999/06/01 or later.
%
% The list of all files belonging to the LaTeX `Tools Bundle' is
% given in the file `manifest.txt'.
%
% \fi
% \iffalse
%% File: tabularx.dtx Copyright (C) 1991-1999 David Carlisle
%
%<*dtx>
          \ProvidesFile{f-tabularx.dtx}
%</dtx>
%<package>\NeedsTeXFormat{LaTeX2e}
%<package>\ProvidesPackage{tabularx}
%<driver> \ProvidesFile{tabularx.drv}
% \fi
%         \ProvidesFile{f-tabularx.dtx}
          [1999/01/07 v2.07 `tabularx' package (DPC)]
% \iffalse
%<*driver>
\documentclass{ltxdoc}
\usepackage[T1]{fontenc}
\usepackage[latin1]{inputenc}
\usepackage[frenchb]{babel}
\usepackage[infoshow]{tabularx}
\begin{document}
 \DocInput{f-tabularx.dtx}
\end{document}
%</driver>
% \fi
%
% \changes{v1.00}{1992/01/30}{Version initiale}
% \changes{v1.01}{1992/07/07}{R\'e-\'edition pour la nouvelle doc et <<~docstrip~>>.}
% \changes{v1.02}{1992/07/17}{Ajout de support pour \cmd\verb}
% \changes{v1.03}{1992/08/17}
%    {Ajout des crochets \cs{ifnum0}!=`\{\cs{fi}\} brackets apr\`es rapport de
%     Andreas Maassen}
% \changes{v1.04}{1992/09/02}
%    {fixation de \cmd\verb, et prise en compte des <<~footnotes~>>.}
% \changes{v1.05}{1992/11/06}
%    {conservation de tous les compteurs \LaTeX{}}
% \changes{v1.06}{1993/08/02}
%    {(Martin Schroeder) Prise en compte de l'argument optionnel [t] ou [b].
%     Donc fonctionne maintenent avec delarray.sty.}
% \changes{v1.07}{1993/08/27}
%    {Modifications pour rendre the style compatible avec calc.sty.}
% \changes{v2.00}{1994/03/14}
%    {Mise \`a jour pour \LaTeX2e}
% \changes{v2.01}{1994/05/22}
%    {Nouveau format de tra\c{c}age.}
%
%
% %%%%%%%%%%%%%%%%%%%%%%%%%%%%%%%%%%%%%%%%%%%%%%%%%%%%%%%%%%%%%%%%%%%%%%
%
%
% \CheckSum{456}
%
% \GetFileInfo{f-tabularx.dtx}
% \title{Le package \textsf{tabularx}\thanks{version \fileversion, date
% \filedate.}}
% \author{David~\bsc{Carlisle}\thanks{traduction fran\c{c}aise du
\ 3~D\'ecembre~1999 par Jean-Pierre~\bsc{Drucbert} <Jean-Pierre.Drucbert@onecert.fr>.
% Titre original <<~The \textsf{tabularx} package~>>.}}
% \date{\filedate}
% \maketitle
% \DeleteShortVerb{\|}
% \MakeShortVerb{\"}
%
% \begin{abstract}
% Ce package d\'efinit un nouvel environnement, {\tt
% tabularx}, qui accepte les m\^emes arguments que {\tt
% tabular*}, mais modifie les largeurs de certaines colonnes,
% plut\^ot que l'espacement entre les colonnes, pour composer
% une table avec la largeur totale demand\'ee. Les colonnes
% qui peuvent s'\'etirer sont marqu\'ees par le nouvel
% indicateur \verb|X| dans l'argument-pr\'eambule.
%
% Ce package utilise (automatiquement) le package {\tt
% array}.
% \end{abstract}
%
%
% \section{Introduction}
% Ce package d\'efinit une version de l'environnement {\tt
% tabular} dans laquelle les largeurs de certaines colonnes
% sont calcul\'ees pour que la table ait une largeur
% sp\'ecifi\'ee. Il semble qu'un tel environnement soit
% souvent demand\'e sur \texttt{comp.text.tex}.

% \DescribeEnv{tabularx}
% "\begin{tabularx}{"\meta{width}"}{"\meta{preamble}"}"\\
% Les arguments de "tabularx" sont essentiellement les
% m\^emes que ceux de l'environnement "tabular*" classique.
% Cependant, plut\^ot que d'ajouter un espace entre les
% colonnes pour atteindre la largeur d\'esir\'ee, il ajuste
% la largeur de certaines colonnes. Les colonnes qui sont
% affect\'ees par l'environnement {\tt tabularx} doivent
% \^etre marqu\'ees par la lettre {\tt X} dans l'argument
% pr\'eambule. La sp\'ecification de colonne {\tt X} sera
% convertie en "p{"\meta{dimension}"}" une fois que la
% largeur correcte pour la colonne aura \'et\'e calcul\'ee.
%
% \section{Exemples}
%
% La table suivante est compos\'ee avec
%
% "\begin{tabularx}{250pt}{|c|X|c|X|} ...".
%
% \begin{center}
% \begin{tabularx}{250pt}{|c|X|c|X|}
% \hline
% \multicolumn{2}{|c|}{Entr\'ee multicolonne !}&
% TROIS&
% QUATRE\\
% \hline
% un&
% \raggedright\arraybackslash La largeur de cette colonne
% d\'epend de la largeur de la table.\footnote{Vous pouvez
% maintenant utiliser {\ttfamily \bslash footnote} dans
% l'environnement {\ttfamily tabularx}!}&
% trois&
% \raggedright\arraybackslash La colonne quatre sera trait\'ee comme
% la colonne deux, avec la m\^eme largeur.\\
% \hline
% \end{tabularx}
% \end{center}
%
% Si nous changeons la premi\`ere ligne en "\begin{tabularx}{300pt}{|c|X|c|X|}"
% nous obtenons :
% \begin{center}
% \begin{tabularx}{300pt}{|c|X|c|X|}
% \hline
% \multicolumn{2}{|c|}{Entr\'ee multicolonne !}&
% TROIS&
% QUATRE\\
% \hline
% un&
% \raggedright\arraybackslash La largeur de cette colonne
% d\'epend de la largeur de la table.&
% trois&
% \raggedright\arraybackslash La colonne quatre sera trait\'ee comme
% la colonne deux, avec la m\^eme largeur.\\
% \hline
% \end{tabularx}
% \end{center}
%
% \edef\mytt{\expandafter\noexpand\csname
%             mdseries\endcsname\noexpand\ttfamily}
% \section{Diff\'erences entre {\tt tabularx} et {\tt tabular*}}
% Ces deux environnements acceptent les m\^emes arguments,
% pour produire une table de la largeur sp\'ecifi\'ee. Les
% principales diff\'erences entre elles sont :
% \begin{itemize}
% \item {\tt tabularx} modifie  les largeurs des {\em colonnes},
% tandis que {\tt tabular*} modifie les largeurs des {\em
% espaces\/} entre les colonnes ;
% \item les environnements {\tt tabular} et {\tt tabular*} peuvent
% \^etre imbriqu\'es sans restriction, mais si un
% environnement {\tt tabularx} se trouve \`a l'int\'erieur
% d'un autre, alors celui \`a l'int\'erieur de l'autre {\em
% doit\/} \^etre plac\'e entre \verb|{ }| ;
% \item le corps de l'environnement {\tt tabularx} est utilis\'e en
% r\'ealit\'e comme argument d'une commande, et donc
% certaines constructions qui ne sont pas permises dans les
% arguments de commandes (comme \verb|\verb|) ne peuvent pas
% \^etre utilis\'ees\footnote{en fait, depuis la version
% 1.02, {\tt $\backslash$verb} peut \^etre utilis\'ee, mais d'une part
% les espaces ne seront pas correctement repr\'e\-sen\-t\'es,
% d'autre part, il {\bf faut} que les accolades soient
% appa\-ri\'ees \`a l'int\'erieur de son argument, ce qui en
% limite vraiment l'int\'er\^et} ;
% \item {\tt tabular*} utilise une possibilit\'e primitive de \TeX\
% pour modifier l'espacement entre les colonnes d'un
% alignement. {\tt tabularx} doit composer plusieurs fois la
% table dans sa recherche des meilleures largeurs de
% colonnes, et est donc bien plus lent. De plus, le fait que
% le corps soit expans\'e plusieurs fois peut mettre en
% d\'efaut certaines constructions~\TeX.
% \end{itemize}



% \section{Adaptation du comportement de {\tt tabularx}}
%
% \subsection{Sorties au terminal}
% \DescribeMacro{\tracingtabularx}
% Si la d\'eclaration est faite, par exemple dans le
% pr\'eambule du document, alors tous les environnements {\tt
% tabularx} ult\'e\-rieurs afficheront des informations sur
% les largeurs de colonnes lors des essais pour trouver les
% largeurs correctes.
%
% Comme alternative, on peut utiliser la d\'eclaration de
% "\tracingtabularx", soit pour les options "infoshow" ou
% "debugshow" qui peuvent \^etre donn\'ees, soit dans la
% commande "\usepackage" qui charge "tabularx", ou comme
% option globale de la commande "\documentclass".
%
%
% \subsection{Environnement utilis\'e pour composer les
% colonnes~{\tt X}}
% Par d\'efaut, la sp\'ecification {\tt X} est transform\'ee
% en "p{"\meta{une valeur}"}". De telles colonnes
% \'etroites exigent souvent un format sp\'ecial, ce qui peut
% \^etre obtenu en utilisant la syntaxe \verb|>| du package
% {\tt array.sty}. Ainsi par exemple vous pouvez sp\'ecifier
% \verb|>{\small}X|. Un autre format utile dans les colonnes
% \'etroites est le <<~d\'echiquet\'e \`a droite~>> (ou
% <<~fer \`a gauche~>>), mais la commande \verb|\raggedright|
% de \LaTeX\ red\'efinit la commande \verb|\\| d'une
% mani\`ere incompatible avec son utilisation dans les
% environnements \verb|array| et \verb|tabular|.
% \DescribeMacro{\arraybackslash}
% Pour cette raison, ce package introduit la commande
% \verb|\arraybackslash|, qui peut \^etre utilis\'ee apr\`es
% une d\'ecla\-ra\-tion \verb|\raggedright|,
% \verb|\raggedleft| ou \verb|\centering|. Donc un
% pr\'eambule de {\tt tabularx} peut sp\'ecifier\\
% \verb|>{\raggedright\arraybackslash}X|.
%
% \DescribeMacro{\newcolumntype}
% Ces sp\'ecifications de pr\'eambule peuvent \'evidemment
% \^etre conserv\'ees en utilisant la commande
% \verb|\newcolumntype| d\'efinie par le package
% \verb|array.sty|. Donc nous pouvons dire\\
% "\newcolumntype{Y}{>{\small\raggedright\arraybackslash}X}"\\
% puis utiliser \verb|Y| dans l'argument pr\'eambule de
% \verb|tabularx|.
%
% \DescribeMacro{\tabularxcolumn}
% Les colonnes {\tt X} sont compos\'ees en utilisant le type
% {\tt p} de colonne qui correspond \`a \verb|\parbox[t]|.
% Vous pouvez vouloir qu'elles soient compos\'ees avec, par
% exemple, le type {\tt m} de colonne, qui correspond \`a
% \verb|\parbox[c]|. Il n'est pas possible de changer le type
% de colonne en utilisant la syntaxe \verb|>|, donc une autre
% m\'ethode est fournie. \verb|\tabularxcolumn| doit \^etre
% d\'efinie comme une commande \`a un seul argument, qui
% s'expanse en la sp\'ecification de pr\'eambule {\tt
% tabular} que vous voulez faire correspondre \`a {\tt X}.
% L'argument sera remplac\'e par la largeur calcul\'ee d'une
% colonne.
%
% Par d\'efaut, c'est
% "\newcommand{\tabularxcolumn}[1]{p{#1}}". Nous pouvons donc
% changer ceci en une commande telle que :\\
% "\renewcommand{\tabularxcolumn}[1]{>{\small}m{#1}}"
%
% \subsection{Largeurs des colonnes}
% Normalement toutes les colonnes {\tt X} dans une m\^eme
% table auront finalement la m\^eme largeur ; cependant il
% est possible de forcer {\tt tabularx} \`a leur donner des
% largeurs diff\'erentes.
% Un argument de pr\'eambule tel que\\
% "{>{\hsize=.5\hsize}X>{\hsize=1.5\hsize}X}" sp\'ecifie deux
% colonnes, la seconde \'etant trois fois plus large que la
% premi\`ere. Mais si vous voulez vous amuser (et vous
% risquer) \`a de telles choses il vous faudra suivre les
% deux r\`egles suivantes :
% \begin{itemize}
% \item Assurez vous que la somme des largeurs de toutes les
% colonnes {\tt X} reste inchang\'ee. (Dans l'exemple
% pr\'ec\'edent, les nouvelles largeurs faisaient un total de
% deux fois la largeur par d\'efaut, comme deux colonnes {\tt
% X} normales) ;
% \item N'utilisez pas d'entr\'ees \verb|\multicolumn| qui
% englobe une ou plusieurs colonnes {\tt X}.
% \end{itemize}
% Comme la plupart des r\`egles, il est possible de passer
% outre, si vous savez ce que vous faites.
%
% \subsection{Si l'algorithme \'echoue\ldots}
% C'est peut-\^etre par ce que les largeurs des colonnes
% <<~normales~>> de la table sont d\'ej\`a plus grandes que la
% largeur totale demand\'ee. \textsf{tabularx} refuse de
% composer des colonnes \texttt{X} de largeur n\'egative,
% alors, dans ce cas, vous aurez un message d'alerte <<~X
% Columns too narrow (table too wide)~>>\footnote{<<~Colonnes
% X trop \'etroites (table trop large)~>>}.
%
% Dans ce cas, les colonnes \texttt{X} devront \^etre
% compos\'ees avec une largeur de 1~em et donc la table
% elle-m\^eme sera plus large que la largeur totale
% demand\'ee et donn\'ee par l'argument d'environnement.
% Ce comportement du package peut \^etre l\'eg\`erement
% modifi\'e, comme d\'ecrit dans le commentaire du code.
%
% \StopEventually{}
%
% \section{Les Macros}
%
%    \begin{macrocode}
%<*package>
%    \end{macrocode}
%
% \changes{v2.00}{1994/02/07}
%    {Traitement de nouvelles options}
%    \begin{macrocode}
\DeclareOption{infoshow}{\AtEndOfPackage\tracingtabularx}
\DeclareOption{debugshow}{\AtEndOfPackage\tracingtabularx}
\ProcessOptions
%    \end{macrocode}
%
% Cela n\'ecessite le package {\ttfamily array.sty}.
% \changes{v2.00}{1994/02/07}
%    {Utilisation de la commande \cmd{\RequirePackage} \LaTeX2e{} pour charger array}
%    \begin{macrocode}
\RequirePackage{array}[1994/02/03]
%    \end{macrocode}
%
% D'abord, quelques registres, etc., dont nous avons besoin.
%    \begin{macrocode}
\newdimen\TX@col@width
\newdimen\TX@old@table
\newdimen\TX@old@col
\newdimen\TX@target
\newdimen\TX@delta
\newcount\TX@cols
\newif\ifTX@
%    \end{macrocode}
%
% Maintenant une astuce pour construire le corps d'un
% environnement dans un identificateur de registre, sans
% faire aucune expansion. Cela ne v\'erifie pas vraiment les
% environnements imbriqu\'es, donc, si vous aviez besoin
% d'imbriquer un environnement {\ttfamily tabularx} dans un
% autre, l'environnement imbriqu\'e \`a l'int\'erieur de
% l'autre doit \^etre entour\'e par "{ }".
%
% \begin{macro}{\tabularx}
% Avant la v1.06, cette macro prenait deux arguments, qui
% \'etaient sauvegard\'es dans des registres s\'epar\'es,
% avant que le corps de la table ne soit sauvegard\'e par
% "\TX@get@body". Malheureusement, cela d\'esactivait
% l'argument optionnel "[t]". Maintenant, on sauvegarde
% seulement et de fa\c{c}on s\'epar\'ee la sp\'ecification de
% largeur, puis on efface l'identificateur de registre "\toks@".
% Finalement, on appelle "\TX@get@body" pour commencer \`a
% sauvegarder le corps de la table. "{\ifnum0=`}\fi" a
% \'et\'e rajout\'e dans la v1.03 pour permettre \`a "tabularx"
% d'appara\^\i{}tre dans un "\halign".
% \setbox0=\hbox{\footnotesize"\iffalse{\fi\ifnum0=`}\fi"}^^A
% \setbox2=\hbox{\footnotesize"\ifnum0=`{}\fi"}^^A
% \footnote{Cela ajoute un niveau suppl\'ementaire de
% groupement qui n'est pas vraiment n\'ecessaire. \`A la
% place, je (D. \bsc{Carlisle}, NdT) pourrais utiliser \box0\ ici, et \box2\
% en-dessous, cependant ici le code devrait \^etre
% d\'eplac\'e apr\`es la premi\`ere ligne, \`a cause de la
% note du bas de la page 386 du \TeX{}Book, et je ne sais pas
% si je devrais \'ecrire du code si obscur puisque comment\'e
% dans une note de bas de page d'un appendice appel\'e
% <<~Dirty Tricks~>> !}
%
% Ce mecanisme de saisie du corps d'un environnement a
% l'inconv\'enient (partag\'e avec les environnements
% d'alignement de l'AMS) d'emp\^echer de faire des
% environnement d'extension comme
%\begin{verbatim}
%\newenvironment{foo}{\begin{tabularx}{XX}}{\end{tabularx}}
%\end{verbatim}
% puisque le code cherche une cha\^\i{}ne litt\'erale "\end{tabularx}"
% pour s'arr\^eter de scruter. Depuis la version 2.02, on
% peut \'eviter ce probl\`eme en utilisant "\tabularx" et
% "\endtabularx" directement dans la d\'efinition :
%\begin{verbatim}
%\newenvironment{foo}{\tabularx{XX}}{\endtabularx}
%\end{verbatim}
% Maintenant, le scruteur cherche la fin de l'environnement
% courant ("foo" dans cet exemple). Il y a quelques
% limitations \`a cette utilisation, la principale \'etant que
% "\endtabularx" est le \emph{premier} identificateur de la
% <<~fin du code~>> de l'environnement.
%    \begin{macrocode}
\def\tabularx#1{%
%    \end{macrocode}
% \changes{v2.02}{1995/03/20}
%    {Nouvelle d\'efinition locale de \cs{TX@}}
% \changes{v4.09}{1998/05/13}
%      {Utilisation \cs{setlength}, pour pouvoir appliquer les extensions
%      calc extensions. rapport : tools/2793}
% Permet d'utiliser "\tabularx" "\endtabularx" (mais pas\\
% "\begin{tabularx}" "\end{tabularx}") dans les d\'efinitions
% "\newenvironment".
%    \begin{macrocode}
\edef\TX@{\@currenvir}%
  {\ifnum0=`}\fi
%    \end{macrocode}
% "\relax" a \'et\'e ajout\'e dans la v1.05 pour que les
% identificateurs de longueur non-expansibles, comme
% "\textwidth", ne g\'en\`erent pas d'espace
% suppl\'ementaire et un d\'ebordement de bo\^\i{}te.
% "\relax" a \'et\'e de nouveau supprim\'e dans la
% v4.09\footnote{Il s'agit d'une mise \`a jour majeure de la s\'erie
% tools et non de la version en cours de tabularx. NtD}
% remplac\'e par "\setlength" pour que vous puissiez
% utiliser une largeur de "(\textwidth-12pt)/2" si vous
% employez le package calc.
%    \begin{macrocode}
  \setlength\TX@target{#1}%
  \TX@typeout{Target width: #1 = \the\TX@target.}%
  \toks@{}\TX@get@body}
%    \end{macrocode}
% \end{macro}
%
%  \begin{macro}{\endtabularx}
% Ne fait vraiment pas grand chose\ldots
% \changes{v2.02}{1995/03/20}
%    {Ajout de macro}
%    \begin{macrocode}
\let\endtabularx\relax
%    \end{macrocode}
%  \end{macro}
%
% \begin{macro}{\TX@get@body}
% Place tous les identificateurs aussi loin que possible de "\end"
% dans un registre d'identificateur. Puis appelle "\TX@find@end"
% pour voir si nous sommes au "\end{tabularx}".
%    \begin{macrocode}
\long\def\TX@get@body#1\end
  {\toks@\expandafter{\the\toks@#1}\TX@find@end}
%    \end{macrocode}
% \end{macro}
%
% \begin{macro}{\TX@find@end}
% Si nous sommes au "\end{tabularx}", appelle "\TX@endtabularx",
% sinon ajoute "\end{...}" \`a l'identificateur et appelle
% une nouvelle fois "\TX@get@body".
%    \begin{macrocode}
\def\TX@find@end#1{%
  \def\@tempa{#1}%
  \ifx\@tempa\TX@\expandafter\TX@endtabularx
  \else\toks@\expandafter
    {\the\toks@\end{#1}}\expandafter\TX@get@body\fi}
%    \end{macrocode}
% \end{macro}
%
% \begin{macro}{\TX@}
% La cha\^\i{}ne {\ttfamily tabularx} est consid\'er\'ee
% comme une macro pour \^etre test\'ee avec "\ifx".
%    \begin{macrocode}
\def\TX@{tabularx}
%    \end{macrocode}
% \end{macro}
%
% Maintenant, toutes les parties de sp\'ecification de
% table sont stock\'ees dans les registres, nous pouvons
% commencer le travail de composition de la table.
%
% L'algorithme de d\'ecouverte des bonnes largeurs de
% colonne est le suivant. Tout d'abord, composition de la
% table avec chaque largeur de colonne {\ttfamily X}
% identique \`a la largeur de la table finale. En supposant
% qu'il y a au moins une colonne {\ttfamily X}, cela produira
% une table trop large. Puis, divise la largeur en exc\`es
% par le nombre de colonnes {\ttfamily X} et r\'eduit la
% largeur de colonne de cette valeur. Puis recompose la
% table. Si la table n'a pas encore la bonne largeur, une
% entr\'ee "\multicolumn" doit <<~cacher~>> une des colonnes {\ttfamily
% X} et donc, il y a une colonne {\ttfamily X} de moins qui
% affecte la largeur de la table. Donc, nous r\'eduisons de 1
% le nombre de colonnes X et recommen\c{c}ons le processus.
%
% \begin{macro}{\TX@endtabularx}
% Bien que j'aie\footnote{D. \bsc{Carlisle}, NdT} essay\'e de
% construire un environnement {\ttfamily
% tabularx}, c'est en r\'ealit\'e une commande et tout le
% travail est fait par cette macro.
%    \begin{macrocode}
\def\TX@endtabularx{%
%    \end{macrocode}
% D\'efinit la colonne {\ttfamily X} avec une version
% interne de la commande "\newcolumntype". Les commandes "\expandafter"
% activent "\NC@newcol" pour consid\'erer l'{\em expansion}
% de "\tabularxcolumn{\TX@col@width}" comme son argument. Ce
% sera la d\'efinition d'une colonne {\ttfamily X}, comme
% \'evoqu\'e dans la section 4.
%    \begin{macrocode}
  \expandafter\TX@newcol\expandafter{\tabularxcolumn{\TX@col@width}}%
%    \end{macrocode}
% Initialise la largeur de colonne et le nombre de colonnes {\ttfamily
% X}. Le nombre de colonnes {\ttfamily X} est fix\'e \`a un,
% ce qui signifie que le d\'ecompte initial sera trop haut de
% un, mais cette valeur est d\'ecr\'ement\'ee avant son
% utilisation dans la boucle principale.
%
% Depuis la v1.02, modifie la d\'efinition de "\verb".
%    \begin{macrocode}
  \let\verb\TX@verb
%    \end{macrocode}
% Depuis la v1.05, sauve les valeurs de tous les compteurs
% \LaTeX{}, la liste "\cl@@ckpt" contenant le nom pour tous
% les compteurs \LaTeX\ qui ont d\'ej\`a \'et\'e d\'efinis.
% Nous expansons "\setcounter" \`a ce moment-l\`a, puisqu'il
% r\'esulte de moins d'identificateurs stock\'es dans"\TX@ckpt",
% mais la v\'eritable r\'e-initialisation des compteurs a
% lieu quand "\TX@ckpt" est expans\'e apr\`es chaque passe
% d'essai.
% R\'eellement, depuis la v1.07, on utilise quelque chose
% d'\'equivalent \`a la d\'efinition originale de
% "\setcounter", pour que "tabularx" soit compatible avec "calc.sty".
%    \begin{macrocode}
  \def\@elt##1{\global\value{##1}\the\value{##1}\relax}%
  \edef\TX@ckpt{\cl@@ckpt}%
  \let\@elt\relax
  \TX@old@table\maxdimen
  \TX@col@width\TX@target
  \global\TX@cols\@ne
%    \end{macrocode}
% Sortie de certains en-t\^etes (\`a moins qu'ils ne soient
% d\'esactiv\'es).
%    \begin{macrocode}
  \TX@typeout@
    {\@spaces Table Width\@spaces Column Width\@spaces X Columns}%
%    \end{macrocode}%
% Pemier essai. Modifie la d\'efinition de {\ttfamily X} pour
% compter les colonnes {\ttfamily X}.
%    \begin{macrocode}
  \TX@trial{\def\NC@rewrite@X{%
          \global\advance\TX@cols\@ne\NC@find p{\TX@col@width}}}%
%    \end{macrocode}
% Diminution r\'ep\'etitive de la largeur de colonne
% jusqu'\`a ce que la table ait la bonne largeur, ou stoppe la
% diminution sinon la colonne devient trop \'etroite. S'il
% n'y a pas d'entr�e multicolonne, il n'y aura qu'un essai.
%    \begin{macrocode}
  \loop
    \TX@arith
    \ifTX@
    \TX@trial{}%
  \repeat
%    \end{macrocode}
% Une derni\`ere fois, avec des messages d'alerte (voir
% Appendice D), utilise {\ttfamily tabular*} pour le mettre
% dans une bo\^\i{}te de dimension correcte, dans le cas ou
% l'algorithme ne r\'eussit pas \`a trouver la bonne
% dimension.
%
% Depuis la v1.04, sauvegarde localement l'argument de "\footnotetext"
% dans un identificateur de registre.
% Depuis la v1.06, "\toks@" contient la sp\'ecification de
% pr\'eambule et l'argument optionnel possible, ainsi que la
% corps de la table.
%    \begin{macrocode}
  {\let\@footnotetext\TX@ftntext\let\@xfootnotenext\TX@xftntext
    \csname tabular*\expandafter\endcsname\expandafter\TX@target
      \the\toks@
    \csname endtabular*\endcsname}%
%    \end{macrocode}
% Maintenant, l'alignement est termin\'e et "}" a
% restaur\'e la signification originale de "\@footnotetext"
% qui expanse le registre "\TX@ftn" qui ex\'ecutera une
% s\'erie de commandes\\
% "\footnotetext["\meta{num}"]{"\meta{note}"}".\\
% Nous avons besoin de faire attention \`a l'effacement du
% registre car nous pouvons \^etre dans un {\ttfamily
% tabularx} imbriqu\'e.
%    \begin{macrocode}
  \global\TX@ftn\expandafter{\expandafter}\the\TX@ftn
%    \end{macrocode}
% Maintenant, finit l'environnement {\ttfamily tabularx}. Il
% est \`a noter que nous avons besoin ici de "\end{tabularx}"
% puisque "\end{tabularx}" dans le fichier utilisateur n'est
% jamais expans\'e. Maintenant utilise "\TX@" plut\^ot que
% "tabularx".
% \changes{v2.02}{1995/03/20}
%    {Fermeture de l'environnement \cs{TX@} plut\^ot que <<~tabularx~>>}
%
% Nous avons \'egalement besoin de terminer le groupe
% commenc\'e par \\ "{\ifnum0=`}\fi" dans la macro "\tabularx".
%    \begin{macrocode}
  \ifnum0=`{\fi}%
  \expandafter\end\expandafter{\TX@}}
%    \end{macrocode}
% \end{macro}
%
%
% \begin{macro}{\TX@arith}
% Calcule la largeur de colonne pour l'essai suivant, fixant
% l'indicateur "\ifTX@" pour qu'il \'echoue si la boucle doit
% \^etre interrompue.
%    \begin{macrocode}
\def\TX@arith{%
  \TX@false
  \ifdim\TX@old@table=\wd\@tempboxa
%    \end{macrocode}
% Si nous avons r\'eduit la largeur de colonne, mais si la
% largeur de la table n'a pas \'et\'e modifi\'ee, nous
% arr\^etons la boucle et composons la table (ce qui causera
% un alignement d\'ebordant) avec la valeur pr\'ec\'edente de
% "\TX@col@width".
%    \begin{macrocode}
    \TX@col@width\TX@old@col
    \TX@typeout@{Reached minimum width, backing up.}%
  \else
%    \end{macrocode}
% Autrement, calcule le d\'epassement de largeur de la table
% courante.
%    \begin{macrocode}
    \dimen@\wd\@tempboxa
    \advance\dimen@ -\TX@target
    \ifdim\dimen@<\TX@delta
%    \end{macrocode}
% Si le d\'epassement est inf\'erieur \`a "\TX@delta", stoppe.
% ("\TX@delta" doit \^etre diff\'erent de z\'ero, sinon nous
% pouvons oublier la cible, \`a cause de l'erreur d'arrondi).
%    \begin{macrocode}
      \TX@typeout@{Reached target.}%
    \else
%    \end{macrocode}
% Diminue de 1 le nombre de colonnes {\ttfamily X} valides.
% (v\'erifiant que nous n'avons pas un 0, ce qui pourrait
% produire une future erreur). Puis divise la largeur en
% exc\`es par le nombre de colonnes valides, et calcule la
% nouvelle largeur de colonne. Stocke temporairement cette
% valeur (multipli\'ee par $-1$) dans "\dimen@".
%    \begin{macrocode}
      \ifnum\TX@cols>\@ne
        \advance\TX@cols\m@ne
      \fi
      \divide\dimen@\TX@cols
      \advance\dimen@ -\TX@col@width
      \ifdim \dimen@ >\z@
%    \end{macrocode}
% Si la nouvelle largeur devait \^etre trop \'etroite, la
% boucle est interrompue. Actuellement, trop \'etroite
% signifie moins de 0\,pt !
%
% Avant la v2.03, si la boucle s'arr\^etait ici, les colonnes
% X \'etaient laiss\'ees avec la largeur de la passe
% pr\'ec\'edente, mais cela peut construire une table beaucoup trop
% large puisque les estimations initiales sont toujours trop
% grossi\`eres. Maintenant, force "\TX@error@width" \`a avoir
% la valeur par d\'efaut de 1~em. Si vous voulez conserver
% l'ancien comportement, mettez\\
%  "\renewcommand\TX@error@width{\TX@col@width}"\\
% dans un package charg\'e apr\`es \textsf{tabularx}.
% \changes{v2.03}{1997/02/20}{Am\'elioration des messages d'alerte
%                            et force la valeur par d\'efaut \`a 1em.}
%    \begin{macrocode}
        \PackageWarning{tabularx}%
           {X Columns too narrow (table too wide)\MessageBreak}%
        \TX@col@width\TX@error@width\relax
      \else
%    \end{macrocode}
% Autrement, sauvegarde les anciens r\'eglages et fixe la
% nouvelle largeur de colonne. Fixe l'indicateur � vrai pour
% que la table soit fix\'ee et que la boucle puisse de
% nouveau s'ex\'ecuter.
%    \begin{macrocode}
        \TX@old@col\TX@col@width
        \TX@old@table\wd\@tempboxa
        \TX@col@width-\dimen@
        \TX@true
      \fi
    \fi
  \fi}
%    \end{macrocode}
% \end{macro}
%
% \begin{macro}{\TX@error@width}
% \changes{v2.03}{1997/02/20}{Ajout de macro.}
% \changes{v2.04}{1997/02/26}{suppression des fausses entretoises.}
% Si le calcul de largeur donne un r\'esultat n\'egatif, on
% utilise cela, \`a la place.
%    \begin{macrocode}
\def\TX@error@width{1em}
%    \end{macrocode}
% \end{macro}
%
% \begin{macro}{\TX@delta}
% Accepte une table qui est dans un "\hfuzz" avec une largeur
% correcte.
%    \begin{macrocode}
\TX@delta\hfuzz
%    \end{macrocode}
% \end{macro}
%
% Initialise la colonne {\ttfamily X}. Ici, la d\'efinition
% peut \^etre vide puisqu'elle est fix\'ee par chaque
% environnement {\ttfamily tabularx}.
%    \begin{macrocode}
\newcolumntype{X}{}
%    \end{macrocode}
%
% \begin{macro}{\tabularxcolumn}
% La d\'efinition par d\'efaut de {\ttfamily X} est "p{#1}".
%    \begin{macrocode}
\def\tabularxcolumn#1{p{#1}}
%    \end{macrocode}
% \end{macro}
%
% \begin{macro}{\TX@newcol}
% Une petite macro juste utilis\'ee pour limiter le nombre de
% commandes "\expandafter" n\'ecessaires.
%    \begin{macrocode}
\def\TX@newcol{\newcol@{X}[0]}
%    \end{macrocode}
% \end{macro}
%
% \begin{macro}{\TX@trial}
% Fait un test.
%    \begin{macrocode}
\def\TX@trial#1{%
  \setbox\@tempboxa\hbox{%
%    \end{macrocode}
% Toute commande suppl\'ementaire. Cela est utilis\'e au
% premier passage pour compter le nombre de colonnes {\ttfamily
%  X}.
%    \begin{macrocode}
    #1\relax
%    \end{macrocode}
% Depuis la v1.04, "\footnotetext" englobe ses arguments.
% Nettoie aussi localement "\TX@vwarn" pour que le message
% d'alerte soit g\'en\'er\'e par la passe {\ttfamily finale}
% et n'appara\^\i{}sse pas au milieu de la table si "\tracingtabularx"
% est utilis\'e.
%
%    \begin{macrocode}
  \let\@footnotetext\TX@trial@ftn
  \let\TX@vwarn\@empty
%    \end{macrocode}
% N'imbrique pas les environnements {\ttfamily tabularx}
% pendant les passes d'essai. Cela peut gaspiller du temps et
% la fixation globale de "\TX@cols" peut arr\^eter
% l'algorithme.
%    \begin{macrocode}
   \expandafter\let\expandafter\tabularx\csname tabular*\endcsname
   \expandafter\let\expandafter\endtabularx\csname endtabular*\endcsname
%    \end{macrocode}
% Ajout\'e dans la v1.05 : d\'esactive "\write" pendant une
% passe d'essai. Cette astuce provient du
% \TeX{}Book.\footnote{L'astuce du \TeX{}Book ne fonctionne
% pas correctement, donc cela a \'et\'e modifi\'e dans la
% v2.05}
% \changes{v2.05}{1997/09/18}
%    {Nouvelle astuce \cs{write}. rapport : tools/2607}
% \changes{v2.07}{1999/01/07}
%    {Astuce \cs{write} plus r\'ecente. rapport : tools/2792}
%    \begin{macrocode}
   \def\write{\begingroup
     \def\let{\afterassignment\endgroup\toks@}%
        \afterassignment\let\count@}%
%    \end{macrocode}
% D\'esactive les messages d'alerte (\emph{cf} appendice D).
% \'Egalement, previent leur activation intempestive, en
% imposant l'enregistrement des noms de param\`etres.
%    \begin{macrocode}
    \hbadness\@M
    \hfuzz\maxdimen
    \let\hbadness\@tempcnta
    \let\hfuzz\@tempdima
%    \end{macrocode}
% R\'ealise la table et termine la <<~hbox~>>. Depuis la
% v1.06, "\toks@" contient la sp\'ecification de pr\'eambule
% et son argument optionnel possible, ainsi que le corps de
% la table.
%    \begin{macrocode}
    \expandafter\tabular\the\toks@
    \endtabular}%
%    \end{macrocode}
% Depuis la v1.05, r\'e-initialise tous les compteurs \LaTeX\
% en ex\'ecutant "\TX@ckpt".
%    \begin{macrocode}
  \TX@ckpt
%    \end{macrocode}
% Imprime certaines statistiques.
% Ajout de "\TX@align" dans la v1.05, pour aligner les
% colonnes.
%    \begin{macrocode}
  \TX@typeout@{\@spaces
     \expandafter\TX@align
        \the\wd\@tempboxa\space\space\space\space\space\@@
     \expandafter\TX@align
        \the\TX@col@width\space\space\space\space\space\@@
     \@spaces\the\TX@cols}}
%    \end{macrocode}
% \end{macro}
%
% \begin{macro}{\TX@align}
% Macro ajout\'ee dans la v1.05 pour am\'eliorer l'impression
% des informations de tra\c{c}age.
%    \begin{macrocode}
\def\TX@align#1.#2#3#4#5#6#7#8#9\@@{%
  \ifnum#1<10 \space\fi
  \ifnum#1<100 \space\fi
  \ifnum#1<\@m\space\fi
  \ifnum#1<\@M\space\fi
  #1.#2#3#4#5#6#7#8\space\space}
%    \end{macrocode}
% \end{macro}
%
% \begin{macro}{\arraybackslash}
% Rupture "\\".
%    \begin{macrocode}
\def\arraybackslash{\let\\\@arraycr}
%    \end{macrocode}
% \end{macro}
%
% \begin{macro}{\tracingtabularx}
% Imprime des statistiques pour les largeurs de colonne et de
% table.
%    \begin{macrocode}
\def\tracingtabularx{%
  \def\TX@typeout{\PackageWarningNoLine{tabularx}}%
  \def\TX@typeout@##1{\typeout{(tabularx) ##1}}}
%    \end{macrocode}
% \end{macro}
%
% \begin{macro}{\TX@typeout}
% Par d\'efaut, est silencieux.
%    \begin{macrocode}
\let\TX@typeout\@gobble
\let\TX@typeout@\@gobble
%    \end{macrocode}
% \end{macro}
%
% \begin{macro}{\TX@ftn}
% Un identificateur de registre pour le texte des notes de
% bas de page.
%    \begin{macrocode}
\newtoks\TX@ftn
%    \end{macrocode}
% \end{macro}
%
% \begin{macro}{\TX@ftntext}
% \begin{macro}{\TX@xftntext}
% \`A l'int\'erieur, sauvegarde juste le texte de note de
% bas de page dans un identificateur de registre.
%    \begin{macrocode}
\long\def\TX@ftntext#1{%
  \edef\@tempa{\the\TX@ftn\noexpand\footnotetext
                    [\the\csname c@\@mpfn\endcsname]}%
  \global\TX@ftn\expandafter{\@tempa{#1}}}%
\long\def\TX@xftntext[#1]#2{%
  \global\TX@ftn\expandafter{\the\TX@ftn\footnotetext[#1]{#2}}}
%    \end{macrocode}
% \end{macro}
% \end{macro}
%
% \begin{macro}{\TX@trial@ftn}
% Dans les passes d'essai, englobe les textes des notes de
% bas de page.
%    \begin{macrocode}
\long\def\TX@trial@ftn#1{}
%    \end{macrocode}
% \end{macro}
%
% Cette derni\`ere section a \'et\'e ajout\'e dans la Version
% 1.02. Les versions pr\'ec\'edentes indiquaient que "\verb"
% ne fonctionnait pas \`a l'int\'erieur de {\ttfamily
% tabularx}, mais \c{c}a n'emp\^echait pas les gens de
% l'utiliser ! Cela entra\^\i{}ne \LaTeX\ dans une erreur
% irr\'ecup\'erable, avec un message d'erreur qui ne
% mentionne pas la cause de l'erreur. Le "\verb" <<~du
% pauvre~>> (et "\verb*") d\'efini ici, est bas\'e sur les
% indications de la page 382 du \TeX{}Book. Comme il est
% expliqu\'e, faire un verbatim de cette fa\c{c}on signifie
% que les espaces ne sont pas trait\'es correctement et donc
% que "\verb*" serait bien inutile ; cependant, je
% consid\`ere que cette section du code est une correction
% d'erreur, plut\^ot qu'une v\'eritable impl\'ementation de
% verbatim.
%
% Le m\'ecanisme est totalement global et toute macro qui
% veut autoriser l'utilisation d'une forme de "\verb" \`a
% l'int\'erieur de son argument peut \^etre \\ "\let\verb=\TX@verb"
% (tout en s'assurant de restaurer la vraie d\'efinition
% ensuite !).
%
% "\verb" et "\verb*" ont les restrictions suivantes :
% \begin{enumerate}
% \item Les espaces dans l'argument ne sont pas lus tels que,
% mais peuvent \^etre ignor\'es, en accord avec les r\`egles
% habituelles de \TeX ;
% \item les espaces doivent \^etre ajout\'es en sortie
% apr\`es le contr\^ole des mots, m\^eme s'ils sont absents
% en entr\'ee ;
% \item sauf si l'argument est un espace isol\'e, tout espace
% tra\^\i{}nant, s'il est dans l'argument original, ou
% ajout\'e comme en (2), sera omit ;
% \item l'argument ne doit pas finir par un "\", puisque
% "\verb|\|" n'est pas permis, et, de toute fa\c{c}on, \`a
% cause de (3), "\verb|\ |" donne "\" ;
% \item l'argument doit \^etre \'equilibr\'e par "{" et "}".
% Donc, "\verb|{|" n'est pas permis ;
% \item un caract\`ere de commentaire comme "%"
% n'appara\^\i{}tra pas tel quel. Il agira comme d'habitude,
% commentant le reste de la ligne en entr\'ee !
% \item les combinaisons "?`" et "!`" appa\^\i{}tront comme
% {\ttfamily?`} et {\ttfamily!`} si la fonte {\ttfamily cmtt} est utilis\'ee.
% \end{enumerate}
%
% \begin{macro}{\TX@verb}
% La d\'efinition interne de "\verb". Les espaces seront
% remplac\'es par "~", pour que dans la forme <<~\'etoile~>>,
% "\let" "~" soit \verb*| |, que nous obtenons par "\uppercase{*}".
% L'utilisation de "{\ifnum0=`}\fi" plut\^ot que "\bgroup"
% permet que "&" appara\^\i{}sse dans l'argument.
%    \begin{macrocode}
{\uccode`\*=`\ %
\uppercase{\gdef\TX@verb{%
  \leavevmode\null\TX@vwarn
  {\ifnum0=`}\fi\ttfamily\let\\\ignorespaces
  \@ifstar{\let~*\TX@vb}{\TX@vb}}}}
%    \end{macrocode}
% \end{macro}
%
% \begin{macro}{\TX@vb}
% Donne le <<~presque verbatim~>> par "\meaning". <<~"!"~>>
% est ajout\'e au d\'ebut du texte fourni, pour s'assurer que
% l'argument entier ne consiste pas en un groupe "{ }"
% isol\'e. \TeX\ enl\`evera les accolades ext\'erieures d'un
% tel groupe. <<~"!"~>> sera enlev\'e plus tard.
%
% \`A l'origine, je suivais \bsc{Knuth} et j'avais "\def\@tempa{##1}",
% cependant cela ne permettait pas \`a "#" d'appara\^\i{}tre
% dans l'argument. Donc, dans la v1.04, j'ai chang\'e \c{c}a
% et utilis\'e un identificateur de registre et "\edef". Cela
% permet \`a "#" d'appara\^\i{}tre mais chacun appara\^\i{}t
% deux fois !, donc plus tard je fais passer une boucle
% rempla\c{c}ant "##" par "#".
%
%    \begin{macrocode}
\def\TX@vb#1{\def\@tempa##1#1{\toks@{##1}\edef\@tempa{\the\toks@}%
    \expandafter\TX@v\meaning\@tempa\\ \\\ifnum0=`{\fi}}\@tempa!}
%    \end{macrocode}
% \end{macro}
%
% \begin{macro}{\TX@v}
% Enl\`eve le segment initial de "\meaning", incluant
% <<~"!"~>> ajout\'e plus t\^ot.
%    \begin{macrocode}
\def\TX@v#1!{\afterassignment\TX@vfirst\let\@tempa= }
%    \end{macrocode}
% \end{macro}
%
% Comme expliqu\'e ci-dessus, nous allons remplacer les
% paires "##" par "#". Pour faire cela, nous avons besoin
% d'un identificateur "#" non sp\'ecifique. Fait un "*" dans
% un identificateur de param\`etre pour que nous puissions
% d\'efinir des macro avec des arguments. La signification
% normale sera restaur\'ee par le dernier "\endgroup".
%    \begin{macrocode}
\begingroup
\catcode`\*=\catcode`\#
\catcode`\#=12
%    \end{macrocode}
%
% \begin{macro}{\TX@vfirst}
% Comme un cas sp\'ecial, emp\^eche que le premier
% caract\`ere soit \'elimin\'e. Alors "\verb*| |" produit \verb*| |.
% Puis, appelle "\TX@v@". Cela est l\'eg\`erement astucieux
% depuis la v1.04, puisque je me suis assur\'e qu'un vrai "#"
% plut\^ot qu'une commande "\let" to "#" est pris en compte
% si le premier caract\`ere est "#".
%    \begin{macrocode}
\gdef\TX@vfirst{%
  \if\@tempa#%
    \def\@tempb{\TX@v@#}%
  \else
    \let\@tempb\TX@v@
    \if\@tempa\space~\else\@tempa\fi
  \fi
  \@tempb}
%    \end{macrocode}
% \end{macro}
%
% \begin{macro}{\TX@v@}
% Boucle \`a travers "\meaning", rempla\c{c}ant tous les
% espaces par "~". Si le dernier caract\`ere est un espace,
% il sera \'elimin\'e, puisque "\verb*|\LaTeX|" produit
% "\LaTeX" et non \verb*|\LaTeX |. Les identificateurs
% r\'e-\'ecrits sont ensuite trait\'es pour remplacer les
% paires "##".
%    \begin{macrocode}
\gdef\TX@v@*1 *2{%
  \TX@v@hash*1##\relax\if*2\\\else~\expandafter\TX@v@\fi*2}
%    \end{macrocode}
% \end{macro}
%
% \begin{macro}{\TX@v@hash}
% La boucle interne, rempla\c{c}ant "##" par "#".
%    \begin{macrocode}
\gdef\TX@v@hash*1##*2{*1\ifx*2\relax\else#\expandafter\TX@v@hash\fi*2}
%    \end{macrocode}
% \end{macro}
%
% Comme promis, nous restaurons la signification normale de
% "#" et "*".
%    \begin{macrocode}
\endgroup
%    \end{macrocode}
%
% \begin{macro}{\TX@vwarn}
Avertit l'utilisateur la premi\`ere fois que "\verb" est
utilis\'e.
%    \begin{macrocode}
\def\TX@vwarn{%
  \@warning{\noexpand\verb may be unreliable inside tabularx}%
  \global\let\TX@vwarn\@empty}
%    \end{macrocode}
% \end{macro}
%
%    \begin{macrocode}
%</package>
%    \end{macrocode}
%
% \Finale
\endinput
